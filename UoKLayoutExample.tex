% Documentation:
%
% The UoK class file extends the standard report style to follow the Registry
% guidelines for laying out a thesis. It sets the margins, interline spacing,
% the page, figure and table numbering style, and disallows page breaks at
% hyphens. The class file consists of setting one and an half line spacing text
% with a 4cm left margin, at least a 2.5cm right margin, approximately 2cm top
% and bottom margin, on A4 paper.
% 
% The class the following options, in addition to those of the standard report
% class.
%     mini - Toggles the thesis in to mini-thesis mode. This adds "mini" to the
%            title and appends a nocite(*) at the end for an automatic output of
%            your complete bibliography.
%     draftmark - Puts a DRAFT' watermark on every page of the document along
%                 with the draft statement on the title page. Additionaly, it
%                 is used as a switch for the UoKExtentions package.
%     draft - Puts the entire document into draft mode. Applies all the effect
%             of draftmark above, but also propergates to other packages used.
%     copyright - Adds a copyright page between the title page and the preface.
%     nofig - Disables output of the list of figures in the preface.
%     notab - Disables output of the list of tables in the preface.
%     noalgo- Disables output of the list of algorithms in the preface.
% All options passed to UoKthesis will be passed along to included packages:
%    natbib, draftwatermark, setspace, hyperref, lmodern
%
% The cover page and optional copyright page are implicitly added before the
% start of the preface section. Use the following commands to populate the 
% cover page/copyright page information:
%     \title{thesis title}
%     \author{author's name} 
%     \degree{Master of Science, Doctor of Philosophy, etc.} 
%     \subject{author's department}
%          - Computer Science if omitted 
%     \submitdate{month year in which submitted}
%          - dated by LaTeX if omitted 
%     \copyrightyear{year degree conferred (next year if submitted in Dec.)}
%          - assumes current year (or next year, in December) if omitted 
% 
% The preface environment allows for the use of sections that precede the main
% document; such as Abstract and  Acknowlegements. These sections should be
% defined using \section{Preface Section Title}. The contents page (and list of
% figures and tables if in use) will be automatically inserted at the end of the
% preface environment.
%
% The thesis style invokes the setspace package to set the commands:
%     \doublespace
%     \onehalfspace
%     \singlespace
% for spacing. By default one and an half spacing is used which resembles the
% UKC Typewriter requirement. Singlespace can be used for letterpress
% appearance. If you want to use true double space, for some reason, place the
% \doublespace command where you want to start using double spacing. Just call
% the appropriate spacing command at where you want to use them.
% 
% In the figure and table environments, single spacing is used. If you want to
% use any other size rather than one and an half spacing, then do:
% 	\renewcommand{\baselinestretch}{1.6} (or whatever you want instead of 1.6)
% This command won't take effect unless it comes before the \begin{document} or
% is triggered by a font change (after something like \small \normalsize).
%
% The example below shows the 12pt thesis style being used. This seems to give
% acceptable looking results, but it may be omitted to get 10pt. Alternatively,
% the 11pt option can be used.
%
% This version differs from old_ukcthesis.sty in the following ways:
% 1. Removed the doublespace package (now uses setspace).
% 2. Merged the phantom section for correct PDF links into the bibliography
%    generating function. 
% 3. Added thesis type options (mini, draft).
% 4. Kent Harvard is used for referencing and citation, this is supported by the
%    natbib package.
% 5. PsFig macro removed.
% 6. Now comes as two files, UoKthesis.cls, which defines purely stylistic layout,
%    and UoKextentions.sty, that provideds some additional functionality.

\documentclass[12pt,mini]{UoKthesis}

% Note: The UoKextentions package includes the xcolor package with the [usenames]
% options. If you need to add further options, these can be given to UoKextentions
% to be propogated through.
\usepackage{UoKextentions}

% Used to generate some sample text.
\usepackage{lipsum}

% Used to configure the algorithm enviroment
\usepackage[ruled,vlined,linesnumbered,commentsnumbered,resetcount]{algorithm2e}

% Used to generate sample image.
\let\oldincludegraphics\includegraphics\renewcommand{\includegraphics}[1]{\IfFileExists{#1}{\oldincludegraphics{#1}}{\fbox{\phantom{\rule{150pt}{100pt}}}}}

% Used in sample table.
\usepackage{multirow}

\begin{document}
\normalsize
%Cover page information
\title{University of Kent\\Thesis Template} 
\author{James F. Smith} 
\subject{computer science} 
\degree{PhD} 

%thesis preface
\begin{preface}
\section{Abstract}
This mini-thesis tells you all you need to know about...
\section{Acknowledgements}
I would like to thank...
\end{preface}

%main content
\chapter{Introduction}

\lipsum[1]

\begin{figure}
\centering
\includegraphics{sample}
\caption{An example figure.}
\end{figure}

\begin{algorithm}
	\SetAlgoLined
	\DontPrintSemicolon	
	\While{Not Terminated}{
		DoSomeWork() \;
		SomeMoreWork() \tcp{A Comment}
	}	
	\Return{TheSolution} \;
	
	\caption{An Example algorithm.}
\end{algorithm}

\lipsum[2-5]

\chapter{The Method}

\lipsum[6-8]

\section{Section A}

\lipsum[9]

\section{Section B}

\lipsum[10-11]

\chapter{Conclusions}

\begin{table}
\centering
\caption{Example table}
\begin{tabular}{ll|c|c|c}
\cline{3-4}
                                              &       & \multicolumn{2}{c|}{Prediction} &                         \\ \cline{3-4}
                                              &       & True           & False          &                         \\ \hline
\multicolumn{1}{|c|}{\multirow{2}{*}{Actual}} & True  & 20             & 30             & \multicolumn{1}{l|}{50} \\ \cline{2-5} 
\multicolumn{1}{|c|}{}                        & False & 45             & 5              & \multicolumn{1}{l|}{50} \\ \hline
                                              &       & 65             & 35             &                         \\ \cline{3-4}
\end{tabular}
\end{table}

\lipsum[12-14]

%Sample bibliography
\bibliography{biblio}

\appendix 
\chapter{Long Proof}
\lipsum[15-24]
\end{document}