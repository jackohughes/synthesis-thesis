So far we have motivated the use of distributive laws, and
demonstrated that they are useful in practice when programming in
languages with linear and graded modal types.
The same methodology we have been discussing can also be used to
derive other useful generic combinators for programming with linear
and graded modal types. In this section, we consider two structural
combinators, \granin{drop} and \granin{copyShape}, in Granule as well
as related type classes for dropping, copying, and moving resources in Linear Haskell.

\subsection{A Combinator for Weakening (``drop'')}
\label{subsec:drop}

%First, we consider a combinator for ``dropping,'' or consuming, values in
%Granule.
%
The built-in type constants of Granule can be split into those
which permit structural weakening $C^{\mathsf{w}}$ such as
\granin{Int}, \granin{Char}, \granin{String}, and those which do not
$C^{\mathsf{l}}$ such as \granin{Handle} (file handles) and
\granin{Chan} (concurrent channels). Those that
permit weakening contain non-abstract values that
can in theory be systematically inspected in order to consume them.
Granule provides a built-in implementation of \granin{drop}
for $C^{\mathsf{w}}$ types, which is then used by the following derivation
procedure to derive weakening on compound types:

\begin{align*}
    \deriv{drop}{C^w}^\Sigma z & = \text{\granin{drop} $z$} \\
    \deriv{drop}{1}^\Sigma z & = [[ case z of () -> () ]] \\
    \deriv{drop}{X}^\Sigma z & = \Sigma(X) z \\
    \deriv{drop}{A \oplus B}^\Sigma z & =
    \derCaseTwoShort{z}{\mathsf{inl}\ x}{\deriv{drop}{A}(x)}{\mathsf{inr}\ y}
                                {\deriv{drop}{B}(y)}\\
    \deriv{drop}{A \otimes B}^\Sigma z & =
    \derCaseOne{z}{(x, y)}
       {\derCaseOne{\deriv{drop}{A}(x)}{()}
         {\derCaseOne{\deriv{drop}{B}(y)}{()}{()}}}
    \\
    \deriv{drop}{\mu X . A}^\Sigma & z =
                                     \derLetRec{f}{\deriv{drop}{A}^{\Sigma,
                                     X \mapsto f : A \multimap 1}}{f\ z}
    \end{align*}

Note we cannot use this procedure in a polymorphic context (over type
variables $\alpha$) since type polymorphism ranges over all types,
including those which cannot be dropped like $C^{\mathsf{l}}$.

%\dnote{An interesting thing is that a function $[[ A -o B ]]$ is
%  droppable if you can 'pick' an element $x \in [[ A ]]$ (i.e., you
%  have a canonical way of generating one) and then that $[[ B ]]$ is
%  droppable. Not sure whether we can/should get into this here though?}

\subsection{A Combinator for Copying ``shape''}

The ``shape'' of values for a parametric data types $\tcF$ can be determined
by a function $\mathit{shape} : \tcF A \rightarrow \tcF 1$, usually
derived when $\tcF$ is a functor by mapping with $A \rightarrow 1$
(dropping elements)~\cite{jay1994shapely}. This provides a way of
capturing the size, shape, and form of a data structure.
Often when programming with data structures which must be
used linearly, we may wish to reason about properties of the data structure
(such as the length or ``shape'' of the structure) but we may not
be able to drop the contained values. Instead, we wish to extract
the shape but without consuming the original data structure itself.

This can be accomplished with a function which copies
the data structure exactly, returning this duplicate along with a data
structure of the same shape, but with the
terminal nodes replaced with values of the unit type $1$  (the `spine'). For example, consider a
pair of integers: \granin{(1, 2)}. Then applying \granin{copyShape} to this pair
would yield \granin{(((), ()), (1, 2))}. The original input pair is duplicated
and returned on the right of the pair, while the left value contains
a pair with the same structure as the input, but with values replaced with
\granin{()}. This is useful, as it allows us to use the left value of the
resulting pair to reason about the structure of the input (e.g., its
depth / size), while preserving the original input. This is
particularly useful for deriving size and length combinators for
collection-like data structures.
\noindent
As with ``drop'', we can derive such a
function automatically:
\begin{align*}
\deriv{copyShape}{\tcF \alpha} : \tcF \alpha \multimap \tcF 1 \otimes \tcF \alpha
\end{align*}
defined by
$\deriv{copyShape}{A} = \lambda z . \deriv{copyShape}{A}^\emptyset z$
by an intermediate derivation $\deriv{copyShape}{A}^\Sigma$:
%
\begin{align*}
\deriv{copyShape}{C^w}^\Sigma z & = ((),\ z) \\
\deriv{copyShape}{1}^\Sigma z & = \derCaseOne{z}{()}{((),\ ())}\\
\deriv{copyShape}{\alpha}^\Sigma z & = ((), z) \\
\deriv{copyShape}{X}^\Sigma z & = \Sigma(X) z \\
\deriv{copyShape}{A \oplus B}^\Sigma z & =
                                         \derCaseTwo{z}
             {\mathsf{inl}\ x}
                {\textbf{case}\ {\llbracket A \rrbracket}^\Sigma_\mathsf{copyShape}(x)\ \textbf{of}\ (s,\ x') \rightarrow (\mathsf{inl}\ s,\ \mathsf{inr}\ x')} 
             {\mathsf{inr}\ y}
                {\textbf{case}\ {\llbracket B \rrbracket}^\Sigma_\mathsf{copyShape}(y)\ \textbf{of}\ (s,\ y') \rightarrow (\mathsf{inl}\ s,\ \mathsf{inr}\ y')} \\
\deriv{copyShape}{A \otimes B}^\Sigma z & =
\derCaseOne{z}{(x, y)}
   {
                                          \derCaseOne{\llbracket A \rrbracket^\Sigma_\mathsf{copyShape}(x)}{ ( [[ s ]],\ [[x']] )}{
                                          \\ & \qquad  \derCaseOne{\llbracket B \rrbracket^\Sigma_\mathsf{copyShape}(y)}{ ( [[ s' ]],\ [[y']] )}{
                                          (([[s]],\ [[s']]),\ ([[x']],\ [[y']]))}
                                          }}
\\
\deriv{copyShape}{\mu X . A}^\Sigma & z =
                                 \derLetRec{f}{\deriv{copyShape}{A}^{\Sigma,
                                 X \mapsto f : A \multimap 1 \otimes A}}{f\ z}
\end{align*}
%
The implementation recursively follows the structure of the type,
replicating the constructors, reaching the crucial case where a
polymorphically type $z : \alpha$ is mapped to $( (),\ z )$ in
the third equation.

Granule implements both these derived combinators in a similar
way to \emph{push}/\emph{pull} providing \granin{copyShape}
and \granin{drop} which can be derived for a type \granin{T} via type
application, e.g. \granin{drop @T : T -> ()} if it can be derived.
Otherwise, the type checker produces an error,
explaining why \granin{drop} is not derivable at type \granin{T}.

% \subsection{Other Combinators in Linear Haskell}

% As previously covered in Section~\ref{sec:linhaskell}, we demonstrated that the
% \emph{push} and \emph{pull} combinators derived from distributed laws are useful
% in Haskell with its linear types extension, and we demonstrated that they can be
% automatically derived using compile-time meta-programming with Template Haskell.

% To the best of our knowledge, nothing comparable to the \granin{Pushable} and
% \granin{Pullable} type classes proposed here has been previously discussed in
% the literature on Linear Haskell. However, several other type classes have been
% proposed for inclusion in the standard library to deal with common use cases
% when programming with linear function types.\footnote{See the
% \texttt{linear-base} library: \url{https://github.com/tweag/linear-base}.} One
% of these classes, \haskin{Consumable}, roughly corresponds to the
% \granin{drop} combinator above, while the other two, \haskin{Dupable} and
% \haskin{Movable}, are for when a programmer wants to allow a data type to be
% duplicated or moved in linear code.

% \begin{lstlisting}[language=Haskell]
%     class Consumable a where
%       consume :: a %1-> ()
    
%     class Consumable a => Dupable a where
%       dup2 :: a %1-> (a, a)
    
%     class Dupable a => Movable a where
%       move :: a %1-> Ur a
%     \end{lstlisting}

%     The \haskin{consume} function is a linear function from a value to unit,
%     whereas \haskin{dup} is a linear function from a value to a pair of that
%     same value. The \haskin{move} linear function maps \haskin{a} to
%     \haskin{Ur a}, where \haskin{Ur} is the ``unrestricted'' modality.
%     Thus, \granin{move} can be used to implement both \granin{consume} and \granin{dup2}:
%     \begin{lstlisting}[language=Haskell]
%         case move x of {Ur _ -> ()}     -- consume x
%         case move x of {Ur x -> x}      -- x
%         case move x of {Ur x -> (x, x)} -- dup2 x
%         \end{lstlisting}
%         %
%         A `copy shape' class may also be a useful addition to Linear
%         Haskell in the future.