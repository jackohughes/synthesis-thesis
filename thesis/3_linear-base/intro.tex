We begin our journey into synthesis by designing a synthesis tool for a simple
resourceful type system. The target language of our synthesis tool in this chapter, 
is derived from the linear-base type system we introduced in chapter~\ref{chapter:background}, 
extending this linear $\lambda$-calculus core to a full functional 
programming language.

We then introd

To do this we must overcome the problem of \textit{resource management}. This
issue was touched on in~\ref{chapter:intro}. In
section~\ref{sec:resource-management}, we expand on the problem and provide an
overview of its history. We identify two feasible approaches to resource
management named \textit{additive} and \textit{subtractive}, and implement a
synthesis calculus for both in section~\ref{sec:core-synth-calculi}. Both
approaches follow a similar structure; inverting the typing rules to derive a
a set of synthesis rules, yet they differ significantly in how resources are
managed.

These differences carry implications with regard to performance and
implementation. Both calculi are implemented as part of a synthesis tool for Granule~\cite{}. 
Having outlined these two solutions to the resource-management problem, we then
evaluate the performance of our implementations against a set of benchmarks.

By the end of this chapter, we will outlined the design and implementation of 
fully usable program-synthesis tool in Granule~\footnote{}. While the expressivity of this 
tool is limited, it lays the theoretical groundwork for a more complete synthesis tool which 
we discuss in subsequent chapters. 