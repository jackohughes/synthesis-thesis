Our linear base calculus presented in chapter~\ref{chapter:background} contains
only the absolute esssential core of a functional programming
language with graded modalities. We extend the syntax with multiplicatve
product types $\otimes$, additive coproduct types $\oplus$, and a multiplicative unit
$1$. The syntax for these extensions is given by the following grammar, which
extends the linear-base syntax of section~\ref{sec:linear-base}:
\begin{align*}
\hspace{-0.8em} [[ t ]] ::= \;
       & [[ x ]]
  \mid [[ \x . t ]]
  \mid [[ t1 t2 ]]
  \mid [[ [t] ]]
  \mid [[ let [ x ] = t1 in t2 ]]
\mid [[ pair t1 t2 ]]
  \mid [[ letpair x1 x2 = t1 in t2 ]] \\
\hspace{-0.9em}  \mid \; & () \mid [[ let () = t1 in t2 ]]
\mid [[ inl t ]] \mid [[ inr t ]] \mid \textbf{case} \ t_{1}\ \textbf{of}\ \textbf{inl}\ x_{1} \rightarrow t_{2};\ \textbf{inr}\ x_{2} \rightarrow t_{3}
{\small{\tag{terms}}}
\end{align*}
We use the syntax $()$ for the inhabitant of  the
multiplicative unit $1$. Pattern matching via a $\textbf{let}$
is used to eliminate products and unit types; for sum types,
$\textbf{case}$ is used to distinguish the constructors.

\begin{figure}[H]
\begin{align*}
\hspace{-0.5em}
  \begin{array}{c}
\inferrule*[right = Pair]
  {[[ G1 |- t1 : A ]] \\ [[ G2 |- t2 : B ]]}
  {[[ G1 + G2 |- pair t1 t2 : Tup A B]]}
\\[0.9em]
\inferrule*[right = LetPair]
  {[[ G1  |- t1 : Tup A B ]] \;\; [[ G2, x1 : A, x2 : B |- t2 : C ]]}
  {[[ G1 + G2 |- letpair x1 x2 = t1 in t2 : C  ]]}
\\[0.9em]
\inferrule*[right = Inl]
  {[[ G |- t : A ]]}
  {[[ G |- inl t : Sum A B ]]}
\;\;\;
\inferrule*[right = Inr]
  {[[ G |- t : B ]]}
  {[[ G |- inr t : Sum A B]]}
\\[0.9em]
\inferrule*[right = Case]
  {[[ G1 |- t1 : Sum A B ]] \\ [[ G2, x1 : A |- t2 : C]] \\ [[ G3, x2 : B |- t3 : C]]}
    {\Gamma + (\Gamma_{2} \sqcup \Gamma_{3}) \vdash\ \textbf{case} \ t_{1}\ \textbf{of}\ \textbf{inl}\ x_{1} \rightarrow t_{2};\ \textbf{inr}\ x_{2} \rightarrow t_{3} : C }
\\[0.9em]
\inferrule*[right = 1]
 {\quad}{[[ . |- () : Unit ]]}
\;\;\;
\inferrule*[right = Let$1$]
 {[[G1 |- t1 : Unit ]] \quad [[ G2 |- t2 : A ]]}
 {[[ G1 + G2 |- let () = t1 in t2 : A ]]}
\end{array}
\end{align*}
\vspace{-1.25em}
  \caption{Typing rules of for $\otimes$, $\oplus$, and $1$}
\label{fig:typing-prod-sum-unit}
 \end{figure}

Figure~\ref{fig:typing-prod-sum-unit} gives the typing rules.  Rules
for multiplicative products (pairs) and additive coproducts (sums) are routine, where
pair introduction ($\textsc{Pair}$)
adds the contexts used to type the pair's constituent subterms. Pair
elimination ($\textsc{LetPair}$) binds a pair's components to two
linear variables in the scope of the body $[[t2]]$. The
$\textsc{Inl}$ and $\textsc{Inr}$ rules handle the typing of
constructors for the sum type $[[Sum A B]]$. Elimination of sums
($\textsc{Case}$) takes the least upper bound (defined above) of the contexts used to
type the two branches of the case.

In the typing of $\mathbf{case}$ expressions, the \emph{least-upper
  bound} of the two contexts used to type each branch is used, defined:

\begin{definition}[Partial least-upper bounds of
  contexts]\label{def:context-lub}
For all $[[ G1 ]]$, $[[ G2 ]]$:
\begin{align*}
\label{def:lub}
[[G1]] \sqcup [[G2]] =
%%
\left\{\begin{matrix}
\begin{array}{lll}
%% Both empty case
\emptyset
  & [[ G1 ]] = \emptyset & \wedge \; [[ G2 ]] = \emptyset
\\
%
%% Left empty
(\emptyset \sqcup [[ G2' ]]), [[ x : [ A ] {lub 0 s} ]]
  & [[ G1 ]] = \emptyset & \wedge \; [[G2]] = [[ G2',x : [A] s]]
\\
%
%% Left is left linear
([[G1']] \sqcup [[(G2',G2'')]]), [[x : A]]
 & [[G1]] = [[{G1', x : A} ]] & \wedge \; [[ G2 ]] = [[ {G2', x : A},, G2'' ]]
\\
%
%% Left is graded
([[G1']] \sqcup [[(G2',G2'')]]), [[x : [A] {lub r s}]]\;\;
 & [[G1]] = [[ G1',x : [A] r]] & \wedge \; [[ G2 ]] = [[{G2', x : [A] s}, G2'']]
\end{array}
\end{matrix}\right.
\end{align*}
where $r\!\sqcup\!s$ is the least-upper bound of grades $[[r]]$
and $[[s]]$ if it exists, derived from $\sqsubseteq$.
\end{definition}
%
As an example of the partiality of $\sqcup$, if one branch of a \textbf{case} uses a linear variable,
then the other branch must also use it to maintain linearity overall,
otherwise the upper-bound of the two contexts for these branches is not defined.

 With these extensions in place, we now have the
 capacity to write more idiomatic functional programs in our target language.
 As a demonstration of this, and to showcase how graded modalities interact with
 these new type extensions, we provide two further examples of different graded
 modalities which complement these new types.

\begin{example}%[Intervals]
\label{exm:or3}
Exact usage analysis is less useful when control-flow is involved, e.g.,
eliminating sum types where each control-flow branch
uses variables differently. The above $\mathbb{N}$-semiring can be imbued with
a notion of \emph{approximation} via less-than-equal ordering, providing
upper bounds. A more expressive semiring is that of
natural number intervals~\cite{DBLP:journals/pacmpl/OrchardLE19}, given by pairs
$\mathbb{N} \times \mathbb{N}$ written $[[ Intrv r s ]]$ here for
the lower-bound $r \in \mathbb{N}$ and upper-bound usage $s \in
\mathbb{N}$ with $0 = [[ Intrv 0 0 ]]$ and $1 = [[ Intrv 1 1 ]]$,
addition and multiplication defined pointwise,
and ordering $[[ Intrv r s ]] \sqsubseteq [[ Intrv r' s' ]] = [[ r' ]] \leq [[ r ]]
\wedge [[ s ]] \leq [[ s' ]]$. Thus a coproduct elimination function
can be written and typed:
%
\begin{align*}
\oplus_e & : [[ {[] {Intrv 0 1} (A -o C)} -o {{[] {Intrv 0 1} (B -o C)} -o {(Sum A B) -o C}} ]] \\
\oplus_e & =
\lambda x' . \lambda y' . \lambda z. \textbf{let}\ [x] = x'\ \textbf {in}\
\textbf{let}\ [y] = y'\ \textbf{in}\ \textbf{case}\ z\ \textbf{of}\
\textbf{inl}\ u \rightarrow x\ u\ |\ \textbf{inr}\ v\ \rightarrow y\ v
\end{align*}
\end{example}

\begin{example}%[Information flow]
\label{exm:security}
%
Graded modalities can capture a form of information-flow
security, tracking the flow of labelled data through a
program~\cite{DBLP:journals/pacmpl/OrchardLE19},
with a lattice-based semiring on
$\mathcal{R} = \{[[ Irrelevant ]] \sqsubseteq [[ Private ]] \sqsubseteq  [[ Public ]]\}$
where $0 = [[ Irrelevant ]]$, $1 = [[ Private ]]$, $+ = \sqcup$ and
if $r = [[ Irrelevant ]]$ or $s = [[ Irrelevant ]]$ then $[[r * s ]] = [[
Irrelevant ]]$ otherwise $[[ r * s ]] = \sqcup$. This
allows the following well-typed program, eliminating a pair of
$[[ Public ]]$ and $[[ Private ]]$ security values, picking the left
one to pass to a continuation expecting a $[[ Public ]]$ input:
%
\begin{align*}
\textit{noLeak} & : [[ (Tup {[] Public A} {[] Private A}) -> {({[]
                  Public (Tup A Unit)} -> B) -> B} ]] \\
\textit{noLeak} & = [[ \z . {\z' . {letpair x' y' = z in {let [x] = x' in
                  {let [y] = y' in {z' [ pair x () ]}}}}} ]]
\end{align*}
\end{example}
