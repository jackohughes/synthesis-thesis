We have now laid out the foundations for a full program synthesis tool for
Granule, exploring in particular the challenges presented by the treatment of
data as resources with usage constraints, and how the synhtesis tool must manage
these resources. Two schemes were proposed as a solution to this problem:
additive and subtractive, with additive generally outperforming subtractive.

Going forward, we focus purely on the additive resource management scheme. The
reasoning behind this is that additive produces smaller and simpler theorems
than subtractive. In chapter~\ref{}, where we consider a fully
graded typing calculus, theorems become even more prevalent in synthesis,
and the speed at which they can be solved is even more relevant than here.

While the tool presented in this chapter allows users to synthesise a
considerable subset of Granule programs, it is quite limited in its
expressivity. Data types comprise only product, sum, and unit types, while
synthesis of recursive function defintions or functions which make use of other
in-scope values such as top-level definitions are not permitted. These features
would all make theoretically feasible additions to the current calculi. Indeed, 
all these features are included in...  

However, the most notable limitation of our calculi is the inability to
synthesise programs which perform a deep pattern match over a graded data type.
A clear example of this can be found in the synthesis of programs which
distribute a graded modality over a data type. Consider the archetypal
distributive program \textit{push}:
\begin{align*}
  push: \Box_r(A \otimes B) \multimap \Box_r A \otimes \Box_r B
\end{align*}
which takes a data type graded by $r$ (in this case the product type $A \otimes
B$), and distributes $r$ over the constituent elements of the product
$A$ and $B$. Given this goal type, how would we go about synthesising a program
in our tool? 

We instatiate the \addAbsName\ rule at this type,
building a partial synthesis tree. Note that although we use the additive scheme
for this example, the same situation arises in the subtractive.
\begin{align*}
    \inferrule*[right=\addUnboxName]
      { x_2 :_r A \otimes B \vdash \Box_r A \otimes \Box_r B \Rightarrow\ ?\ |\ ? }
      {\inferrule*[right=\addAbsName] {x_1 : \Box_r (A \otimes B) \vdash \Box_r A \otimes \Box_r B \Rightarrow\ \textbf{let}\ [x_2] = x_1\ \textbf{in}\ ?  \ |\ ?} 
        {\emptyset \vdash \Box_r(A \otimes B) \multimap \Box_r A \otimes \Box_r B \Rightarrow \lambda x_1 . ? \ |\ ? }}
\end{align*}

After applying \addAbsName\ followed by \addUnboxName, we now have the graded
assumption $x_2 :_r A \otimes B$ in our context which we must use to construct a
term of type $\Box_r A \otimes \Box_r B$. We might expect that the path
synthesis should take now would be to break $x_2$ down into two graded
assumptions with types $A$ and $B$, promote these graded assumptions using the
\addBoxName\ rule, before finally peforming a pair introduction to yield $\Box_r
A \otimes \Box_r B$. However, in order to apply the pair elimination rule
\addPairElimName\  
and break our graded assumption into two, we must perform a dereliction on
$x_2$, to yield a linear copy: 
\begin{align*}
  \inferrule*[right=\addDerName]
      {x_2 :_r A \otimes B, x_3 : A \otimes B \vdash \Box_r A \otimes \Box_r B \Rightarrow\ ?\ }
      { x_2 :_r A \otimes B \vdash \Box_r A \otimes \Box_r B \Rightarrow\ ?\ |\ ? }
\end{align*}
Clearly, this cannot lead us to the goal: the \addBoxName\ rule cannot promote
terms using linear assumptions. Therefore, \textit{push} and other types which
exhibit this distributive behaviour are not synthesiseable in our calculus.

One solution would be to extend our calculi with additional rules which
synthesise elimination forms using graded assumptions. The downside to this is
that it greatly increases the complexity of our calculi and introduces 
a considerable degree of non-deterministism.  

In the following chapter, we will present an alternative approach to generating
programs which exhibit this distributive behaviour using a generic programming
methodology. While the limitation described above extends beyond the inability to 
synthesise distributive programs ... 




