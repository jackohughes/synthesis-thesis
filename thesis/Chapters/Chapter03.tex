\chapter{A Core Synthesis Calculus}
\label{chapter:core}
We begin our first exploration into program synthesis with a system for the
graded linear $\lambda$-calculus of Section~\ref{sec:linear-base}. The primary
aim of this chapter is to introduce the core concepts of type-directed program
synthesis in a resourceful setting, in particular, the problem of
\textit{resource management}. We therefore prioritise simplicity over
expressivity for our target language, with the core typing calculus of
Section~\ref{sec:linear-base} forming an ideal candidate.

As mentioned in Chapter~\ref{chapter:intro}, type-directed program synthesis can
be framed as an inversion of type checking. In type checking, we have a
judgement of the form: 
\begin{align*}
    \Gamma \vdash t : A {\small{\tag{type checking}}}
\end{align*}
which states that under some context of assumptions $\Gamma$ we can assign the
program term $t$ the type $A$. Here, $\Gamma$ and $t$ constitute the ``inputs''
to the judgement, while the type $A$ forms the ``output''. Synthesis inverts
this judgement form, leaving us with a \textit{synthesis judgement} form:
\begin{align*}
    \Gamma \vdash A \Rightarrow t {\small{\tag{synthesis}}}
\end{align*}
which states that we can construct a program term $t$ from the type $A$, using
the assumptions in $\Gamma$. As in type checking, $\Gamma$ forms an input to the
judgement. However, $A$ and $t$ exchange roles: the former is now also an input,
while the latter is the judgement's output. Program synthesis then becomes a
task of inductively enumerating programs in a ``bottom-up'' starting from the
goal type $A$: $A$ is broken into sub-goals, from which sub-terms are
synthesised until the goal can not be broken into further sub-goals. At this
point, we either synthesise a usage of a variable from $\Gamma$ if possible, or
synthesis fails. This is the essence of type-directed program synthesis. 

Resourceful types introduce another dimension to synthesis: how do we ensure
that the assumptions in $\Gamma$ are used according to their resource
constraints in the synthesised term $t$? I.e. if $x : A$ is a linear assumption in $\Gamma$
that is used in some way to construct $t$, then the synthesis algorithm must synthesise 
a $t$ which uses $x$ exactly once. Likewise, if $x :_r A$ is a graded assumption, then 
it must be used in $t$ in a way which satisfies its grade $r$.

This problem has been explored before in the context of automated theorem
proving for linear logic, and has been termed the \textit{resource management
problem}. We describe this problem in detail in
Section~\ref{sec:resource-management} and propose two candidate solutions,
basing our approach on the \textit{input-output context management} model
described by~\citet{HODAS1994327}, and further developed by~\citet{CERVESATO2000133}. 

The challenges posed by ensuring the well-resourcedness of synthesised programs
are most exemplified by the inclusion in our target language of multiplicative
conjunction, and additive disjunction. Therefore, prior to fully describing the
problem of resource management and our proposed solutions, we first expand our
target language with multiplicative product ($\otimes$), and unit types ($1$),
as well as disjunctive sum types ($\oplus$). These extensions are detailed in
Section~\ref{sec:linear-base-calculus}, which will be the target language of the
synthesis calculi of this chapter. As well as helping to conceptualise the
challenges posed by program synthesis in a resourceful setting, these have the added benefit
of allowing the synthesis of more expressive programs, without introducing
unnecessary complexity at this stage.

Having outlined both a suitable target language and two approaches to dealing
with the issue of resource management, we then present two synthesis calculi in
Section~\ref{sec:linear-base-synthesis} as augmented inversions of the typing
rules. Each calculus is based on a one of our proposed solutions to the resource
management problem, which we then evaluate and contrast against each other in
Section~\ref{sec:linear-base-evaluation}. 

Both calculi are implemented as part of a synthesis tool for
Granule~\footnote{The exact implementation of the rules as they stand is
deprecated, but may be found in Granule release v0.7.8.0:
https://github.com/granule-project/granule/releases/tag/v0.7.8.0}. The
implementation is a fairly direct translation of the synthesis calculi into
Haskell. We thus elide the details of the implementation, focusing only on an
important optimisation technique in Section~\ref{sec:linear-base-focusing}:
focusing. Focusing removes much of the unnecessary non-determinism 
present in our synthesis rules by fixing an ordering on
the application of rules. We present the two \textit{focused} forms of our original
synthesis calculi which comprise the basis of our Granule implementation.

\section{A Core Target Language}
\label{sec:linear-base-calculus}
The syntax for the full language is given by the following grammar:
\begin{align*}
\hspace{-0.8em} [[ t ]] ::= \;
       & [[ x ]]
  \mid [[ \x . t ]]
  \mid [[ t1 t2 ]]
  \\ \mid \; & [[ [t] ]]
  \mid [[ let [ x ] = t1 in t2 ]]
  \\  \mid \; & [[ pair t1 t2 ]]
  \mid [[ letpair x1 x2 = t1 in t2 ]] \\
  \mid \; & () \mid [[ let () = t1 in t2 ]]
\\ \mid \; & [[ inl t ]] \mid [[ inr t ]] \mid \textbf{case} \ t_{1}\ \textbf{of}\ \textbf{inl}\ x_{1} \rightarrow t_{2};\ \textbf{inr}\ x_{2} \rightarrow t_{3}
{\small{\tag{terms}}}
\end{align*}
We use the syntax $()$ for the inhabitant of  the multiplicative unit $1$.
Pattern matching via a $\textbf{let}$ is used to eliminate products and unit
types; for sum types, $\textbf{case}$ is used to distinguish the constructors.

\begin{figure}[t]
\begin{align*}
  \begin{array}{c}
\inferrule*[right = Pair]
  {[[ G1 |- t1 : A ]] \\ [[ G2 |- t2 : B ]]}
  {[[ G1 + G2 |- pair t1 t2 : Tup A B]]}
\\[1.25em]
\inferrule*[right = LetPair]
  {[[ G1  |- t1 : Tup A B ]] \;\; [[ G2, x1 : A, x2 : B |- t2 : C ]]}
  {[[ G1 + G2 |- letpair x1 x2 = t1 in t2 : C  ]]}
\\[1.25em]
\inferrule*[right = Inl]
  {[[ G |- t : A ]]}
  {[[ G |- inl t : Sum A B ]]}
\;\;\;
\inferrule*[right = Inr]
  {[[ G |- t : B ]]}
  {[[ G |- inr t : Sum A B]]}
\\[1.25em]
\inferrule*[right = Case]
  {[[ G1 |- t1 : Sum A B ]] \\ [[ G2, x1 : A |- t2 : C]] \\ [[ G3, x2 : B |- t3 : C]]}
    {\Gamma + (\Gamma_{2} \sqcup \Gamma_{3}) \vdash\ \textbf{case} \ t_{1}\ \textbf{of}\ \textbf{inl}\ x_{1} \rightarrow t_{2};\ \textbf{inr}\ x_{2} \rightarrow t_{3} : C }
\\[1.25em]
\inferrule*[right = 1]
 {\quad}{[[ . |- () : Unit ]]}
\;\;\;
\inferrule*[right = LetUnit]
 {[[G1 |- t1 : Unit ]] \quad [[ G2 |- t2 : A ]]}
 {[[ G1 + G2 |- let () = t1 in t2 : A ]]}
\end{array}
\end{align*}
\vspace{-1.25em}
  \caption{Typing rules of for $\otimes$, $\oplus$, and $1$}
\label{fig:typing-prod-sum-unit}
 \end{figure}

Figure~\ref{fig:typing-prod-sum-unit} gives the typing rules.  Rules for
multiplicative products (pairs) and additive coproducts (sums) are routine,
where pair introduction ($\textsc{Pair}$) adds the contexts used to type the
pair's constituent sub-terms. Pair elimination ($\textsc{LetPair}$) binds a
pair's components to two linear variables in the scope of the body $[[t2]]$. The
$\textsc{Inl}$ and $\textsc{Inr}$ rules handle the typing of constructors for
the sum type $[[Sum A B]]$. Elimination of sums ($\textsc{Case}$) takes the
least upper bound (defined above) of the contexts used to type the two branches
of the case.

In the typing of $\mathbf{case}$ expressions, the \emph{least-upper bound} of
  the two contexts used to type each branch is used, defined:

\begin{definition}[Partial least-upper bounds of
  contexts]\label{def:context-lub} For all $[[ G1 ]]$, $[[ G2 ]]$, $[[G1]] \sqcup [[G2]] =$
\begin{align*}
\label{def:lub}
%%
\left\{\begin{matrix}
\begin{array}{lll}
%% Both empty case
\emptyset
  & [[ G1 ]] = \emptyset & \wedge \; [[ G2 ]] = \emptyset
\\
%
%% Left empty
(\emptyset \sqcup [[ G2' ]]), [[ x : [ A ] {lub 0 s} ]]
  & [[ G1 ]] = \emptyset & \wedge \; [[G2]] = [[ G2',x : [A] s]]
\\
%
%% Left is left linear
([[G1']] \sqcup [[(G2',G2'')]]), [[x : A]]
 & [[G1]] = [[{G1', x : A} ]] & \wedge \; [[ G2 ]] = [[ {G2', x : A},, G2'' ]]
\\
%
%% Left is graded
([[G1']] \sqcup [[(G2',G2'')]]), [[x : [A] {lub r s}]]\;\;
 & [[G1]] = [[ G1',x : [A] r]] & \wedge \; [[ G2 ]] = [[{G2', x : [A] s}, G2'']]
\end{array}
\end{matrix}\right.
\end{align*}
where $r\!\sqcup\!s$ is the least-upper bound of grades $[[r]]$ and $[[s]]$ if
it exists, derived from $\sqsubseteq$.
\end{definition}
%
As an example of the partiality of $\sqcup$, if one branch of a \textbf{case}
uses a linear variable, then the other branch must also use it to maintain
linearity overall, otherwise the upper-bound of the two contexts for these
branches is not defined.

With these extensions in place, we now have the capacity to write more 
idiomatic functional programs in our target language. As a demonstration of 
this, and to showcase how graded modalities interact with these new type 
extensions, we provide two further examples of different graded modalities 
which complement these new types.

\begin{example}%[Intervals]
\label{exm:or3}
Exact usage analysis is less useful when control-flow is involved, e.g.,
eliminating sum types where each control-flow branch uses variables differently.
The above $\mathbb{N}$-semiring can be imbued with a notion of
\emph{approximation} via less-than-equal ordering, providing upper bounds. A
more expressive semiring is that of natural number
intervals~\cite{DBLP:journals/pacmpl/OrchardLE19}, given by pairs $\mathbb{N}
\times \mathbb{N}$ written $[[ Intrv r s ]]$ here for the lower-bound $r \in
\mathbb{N}$ and upper-bound usage $s \in \mathbb{N}$ with $0 = 0 ... 0$
and $1 = 1 ... 1$, addition and multiplication defined pointwise, and
ordering $[[ Intrv r s ]] \sqsubseteq [[ Intrv r' s' ]] = [[ r' ]] \leq [[ r ]]
\wedge [[ s ]] \leq [[ s' ]]$. Thus a coproduct elimination function can be
written and typed:
%
\begin{align*}
\oplus_e & : [[ {[] {Intrv 0 1} (A -o C)} -o {{[] {Intrv 0 1} (B -o C)} -o {(Sum A B) -o C}} ]] \\
\oplus_e & =
\lambda x' . \lambda y' . \lambda z. \textbf{let}\ [x] = x'\ \textbf {in}\ \\ 
          & \;\;\;\;\;\;\;\;\;\;\;\;\;\;\;\;\;\;\;\;\;\;\;\;\;\;\; \textbf{let}\ [y] = y'\ \textbf{in}\ \\ 
          & \;\;\;\;\;\;\;\;\;\;\;\;\;\;\;\;\;\;\;\;\;\;\;\;\;\;\;\;\;\;\;\; \textbf{case}\ z\ \textbf{of}\
\textbf{inl}\ u \rightarrow x\ u\ |\ \textbf{inr}\ v\ \rightarrow y\ v
\end{align*}
\end{example}

\begin{example}%[Information flow]
\label{exm:security}
%
Graded modalities can capture a form of information-flow security, tracking the
flow of labelled data through a program~\cite{DBLP:journals/pacmpl/OrchardLE19},
with a lattice-based semiring on $\mathcal{R} = \{[[ Irrelevant ]] \sqsubseteq
[[ Private ]] \sqsubseteq  [[ Public ]]\}$ where $0 = [[ Irrelevant ]]$, $1 = [[
Private ]]$, $+ = \sqcup$ and if $r = [[ Irrelevant ]]$ or $s = [[ Irrelevant
]]$ then $[[r * s ]] = [[ Irrelevant ]]$ otherwise $[[ r * s ]] = \sqcup$. This
allows the following well-typed program, eliminating a pair of $[[ Public ]]$
and $[[ Private ]]$ security values, picking the left one to pass to a
continuation expecting a $[[ Public ]]$ input:
%
\begin{align*}
\textit{noLeak} & : [[ (Tup {[] Public A} {[] Private A}) -o {({[]
                  Public (Tup A Unit)} -o B) -o B} ]] \\
\textit{noLeak} & = \lambda z . \lambda u . \textbf{let}\ \textlangle x',\ y' \textrangle\ =\ z\ \textbf{in}\ \\
                & \;\;\;\;\;\;\;\;\;\;\; \textbf{let}\ [x]\ =\ x'\ \textbf{in}\ \\ 
                & \;\;\;\;\;\;\;\;\;\;\;\;\;\;\;\; \textbf{let}\ [y]\ =\ y'\ \textbf{in}\ u\ [(x,\ ())]
\end{align*}
\end{example}

\subsection{Metatheory}
Finally, the admissibility of substitution is a key result that holds
for this language~\citep{DBLP:journals/pacmpl/OrchardLE19}, which is
leveraged in soundness of the synthesis calculi.
%
\begin{restatable}[Admissibility of substitution]{lemma}{linearSubst}
Let $[[ D |- t' : A]]$, then:
\label{lemma:substitution}
\begin{itemize}[leftmargin=1em]
\item (Linear) \hspace{0.04em} If $[[ {G, x : A}
    ,, { G' } |- t : B]]$ then $[[ G + D + G' |-
[ t' / x ] t : B ]]$
\item (Graded) If $[[ {G, x : [A] r} ,, { G' } |- t : B]]$
then $[[ G + (r * D) + G' |- [ t' / x ] t : B ]]$
\end{itemize}
\end{restatable}


\section{The Resource Management Problem}
\label{sec:resource-management}

In Chapter~\ref{chapter:intro} we considered a synthesis rule for pairs and
highlighted how graded types could be use to control the number of times
assumptions are used in the synthesised term. 

Chapter~\ref{chapter:intro} considered (Cartesian) product types
$\times$, but in our target language we use the multiplicative
product of linear types, given in Figure~\ref{fig:typing-prod-sum-unit}%
Each sub-term is typed by a different context $[[ G1 ]]$ and $[[ G2 ]]$ which are
then combined via \emph{disjoint} union: the pair cannot be formed if variables
are shared between $[[ G1 ]]$ and $[[ G2 ]]$. This prevents the structural
behaviour of \emph{contraction} (where a variable appears in multiple sub-terms).
Na\"{i}vely inverting this typing rule into a synthesis rule yields:
%
\begin{align*}
  \inferrule*[Right=$\otimes_{\textsc{Intro}}$]{ \Gamma_1 \vdash [[ A ]] \Rightarrow [[ t1 ]] \\ \Gamma_2 \vdash [[ B ]] \Rightarrow [[ t2 ]]}{ \Gamma_1, \Gamma_2 \vdash [[ A ]] \otimes [[ B ]] \Rightarrow ([[ t1 ]],\ [[ t2 ]] )}
\end{align*}
%
As a declarative specification, the $\otimes_{\textsc{Intro}}$ synthesis rule is
sufficient. However, this rule embeds a considerable amount of non-determinism
when considered from an algorithmic perspective. Reading `clockwise' starting
from the bottom-left, given some context $[[ G ]]$ and a goal $A \otimes B$, we
have to split the context into disjoint subparts $[[ G1 ]]$ and $[[ G2 ]]$ such
that $[[ G ]] = [[ G1 , G2 ]]$ in order to pass the $[[ G1 ]]$ and $[[ G2 ]]$ to
the sub-goals for $A$ and $B$. For a context of size $n$ there are $2^n$ possible
such partitions! This quickly becomes intractable. Instead,~\cite{HODAS1994327} 
developed a technique for linear logic programming, refined
by~\citet{CERVESATO2000133}, where proof search for linear logic
has both an \emph{input context} of available resources and an \emph{output
context} of the remaining resources, which we write as judgements of the form
$[[ G |- A =>- t | G' ]]$ for input context $[[ G ]]$ and output context $[[ G'
]]$. Synthesis for multiplicative products then becomes:
%
\begin{align*}
  \inferrule*[right=$\otimes_{\textsc{Intro}}^{-}$]{\Gamma_1 \vdash [[ A ]] \Rightarrow^- [[t1]] \ |\ \Gamma_{2} \\ \Gamma_{2} \vdash [[ B ]] \Rightarrow^- [[ t2 ]] \ | \ \Gamma_{3} }{ \Gamma_1 \vdash [[ A Prod B ]] \Rightarrow^- [[ pair t1 t2 ]] \ | \ \Gamma_{3}}
\end{align*}
%
where the remaining resources after synthesising for $A$ the first term $[[ t1
]]$ are $[[ G2 ]]$ which are then passed as the resources for synthesising the
second term $B$. There is an ordering implicit here in `threading through' the
contexts between the premises. For example, starting with a context $[[ x : A, y
: B ]]$, then this rule can be instantiated as:
%
\begin{align}
\tag{\footnotesize{example}}
  \inferrule*[right=$\otimes_{\textsc{Intro}}^{-}$]{x : A, y : B \vdash [[ A ]] \Rightarrow^- [[x]] \ |\ y : B \\ y : B \vdash [[ B ]] \Rightarrow^- [[ y ]] \ | \ \emptyset }{ x : A, y : B \vdash [[ A Prod B ]] \Rightarrow^- [[ pair x y ]] \ | \ \emptyset}
\end{align}
%
Thus this approach neatly avoids the problem of having to split the input
context, and facilitates efficient proof search for linear types. We extend this
input-output context management model to graded types to graded types to
facilitate the synthesis of programs in Granule. We  
term the above approach \textit{subtractive} resource management (in a style
similar to \textit{left-over} type checking for linear type
systems~\citep{allais2018typing,zalakain2020pi}). 

Graded type systems, as we consider them here, have typing contexts in which
free-variables are assigned a type, and a grade. In a graded setting, the
subtractive approach is problematic as there is not necessarily a notion of
actual subtraction for grades. Consider a version of the above example for
subtractively synthesising a pair, but now for a context with some grades $r$
and $s$ on the input variables. Using a variable to synthesise a sub-term now
does not result in that variable being left out of the output context. Instead a
new grade must be assigned in the output context that relates to the first by
means of an additional constraint describing that some usage took place:
%
\begin{align}
\tag{\footnotesize{example}}
\!\!\!\!\!\inferrule*[right=$\otimes_{\textsc{Intro}}^{-}$]{
\exists r' . r' + 1 = r \\ \!\!\!\! \exists s' . s' + 1 = s \\
[[ x : [A] r, y : [B] s ]] \vdash [[ A ]] \Rightarrow^-\! [[x]] \ |\ [[ x : [A] r', y : [B] s ]] \\ \!\!\!\! [[ x : [A] r', y : [B] s ]] \vdash [[ B ]] \Rightarrow^-\! [[ y ]] \ | \ [[ x : [A] r', y : [B] s' ]]  }{ [[ x : [A] r, y : [B] s ]]  \vdash [[ A Prod B ]] \Rightarrow^- \![[ pair x y ]] \ | \ [[ x : [A] r', y : [B] s' ]]}\!\!\!
\end{align}
%
In the first synthesis premise, $x$ has grade $r$ in the input context, $x$ is
synthesised for the goal, and thus the output context has some grade $r'$ where
$r' + 1 = r$, denoting that some usage of $x$ occurred (which is represented by
the $1$ element of the semiring in graded systems).

For the natural numbers semiring, with $r = 1$ and $s = 1$ then the constraints
above are satisfied with $r' = 0$ and $s' = 0$. In a general setting, this
subtractive approach to synthesis for graded types requires solving many such
existential equations over semirings, which also introduces a new source of
non-determinism is there is more than one solution. These constraints can be
discharged via an off-the-shelf SMT solver, such as Z3~\citep{z3}. Such calls to an 
external solver are costly, however, and thus efficiency of resource management is 
a key concern. 

We propose a dual approach to the subtractive: the \emph{additive} resource
management scheme. In the additive approach, output contexts describe what was
\emph{used} not what was is \emph{left}. In the case of synthesising a term with
multiple sub-terms (like pairs), the output context from each premise is then
added together using the semiring addition operation applied pointwise on
contexts to produce the final output in the conclusion. For pairs this looks
like:
% %
\begin{align*} \inferrule*[right=$\otimes_{\textsc{Intro}}^{+}$]{\Gamma \vdash
  [[ A ]] \Rightarrow^+ [[t1]] \ |\ \Delta_{1} \\ \Gamma \vdash [[ B ]]
  \Rightarrow^+ [[ t2 ]] \ |\ \Delta_{2} }{ \Gamma \vdash [[ A Prod B ]]
  \Rightarrow^+ [[ pair t1 t2 ]] \ |\ \Delta_{1} + \Delta_{2}} \end{align*}
%
The entirety of $[[G]]$ is used to synthesise both premises. For example, for
a goal of $[[ A Prod A ]]$:
%
\begin{align} \tag{\footnotesize{example}}
\inferrule*[right=$\otimes_{\textsc{Intro}}^{+}$]{ [[ x : [A] r, y : [B] s ]]
\vdash [[ A ]] \Rightarrow^+ [[x]] \ |\ [[ x : [A] 1, y : [B] 0 ]] \\ [[ x :
[A] r, y : [B] s ]] \vdash [[ A ]] \Rightarrow^+ [[ x ]] \ | \ [[ x : [A] 1, y
: [B] 0 ]]  }{ [[ x : [A] r, y : [B] s ]]  \vdash [[ A Prod A ]] \Rightarrow^+
[[ pair x x ]] \ | \ [[ x : [A] {1 + 1} , y : [B] 0 ]]} \end{align}
%
Later checks in synthesis then determine whether the output context describes
usage that is within the grades given by $[[ G ]]$, i.e., that the synthesised
terms are \emph{well-resourced}.

Both the subtractive and additive approaches avoid having to split the incoming
context $[[ G ]]$ into two prior to synthesising sub-terms. 

We adapt the input-output context management model of linear logic synthesis to
graded types, pruning the search space via the quantitative constraints of
grades. We implement synthesis calculi based on both the additive and
subtractive approaches, evaluating their performance on a set of benchmarking
synthesis problems. 
\subsection{Related Work}
Before Hodas and Miller, the problem of resource
non-determinism was first identified by~\citet{harlandpym}. 
Their solution delays splitting of contexts at a multiplicative connective. 
They later explored the implementation details of this approach, proposing a 
solution where proof search is formulated in terms of constraints on 
propositions. Propositions which occur in the conclusion of a multiplicative 
connective are assigned a Boolean expression whose solution Constraints 
generated during the proof search, with a solution to these constituting a 
valid proof. The logic programming language 
Lygon~\citep{lygon} implements this approach. 

Our approach to synthesis implements a \textit{backward} style of proof search:
starting from the goal, recursively search for solutions to sub-goals. In
contrast to this, \textit{forward} reasoning approaches attempt to reach the
goal by building sub-goals from previously proved sub-goals until the overall
goal is proved.~\citet{10.1007/11532231_6,10.1007/11538363_15} consider forward
approaches to proof search in linear logic using the \textit{inverse
method}~\citep{DEGTYAREV2001179} where the issue of resource non-determinism
that is typical to backward approaches is absent.


\section{The Synthesis Calculi}
\label{sec:linear-base-synthesis}

We now present two synthesis calculi based on the subtractive and additive resource
management schemes, respectively. The structure of the synthesis calculi mirrors a
cut-free sequent calculus, with
\textit{left} and \textit{right} rules for each type constructor. Right rules
synthesise an introduction form for the goal type. Left rules
eliminate (deconstruct) assumptions so that they may be
used inductively to synthesise sub-terms. Each type in the
core language has right and left
rules corresponding to its constructors and destructors respectively.


\begin{figure}[H]
\begin{gather*}
{\footnotesize{
  \hspace{-3em}\begin{array}{c}
    \subLinVar
    \;\;\;
    \subGrVar
    \\[1.25em]
    \subAbs
    \\[1.25em]
    \subApp
    \\[1.25em]
    \subDer
    \\[1.25em]
    \subBox
    \\[1.25em]
    \subUnbox
    \\[1.25em]
    \subPairIntro
    \\[1.25em]
    \subPairElim
    \\[1.25em]
    \subSumIntroL
    \;\;\;
    \subSumIntroR
    \\[1.25em]
    \subSumElim
    \\[1.25em]
    \subUnitIntro
    \;\;\;
    \subUnitElim
  \end{array}
}}
\end{gather*}
\caption{Collected rules of the subtractive synthesis calculus}
\label{fig:sub-rules}
  \end{figure}



\subsection{Subtractive Resource Management}
\label{subsec:subtractive}
  Our subtractive approach follows the philosophy of earlier work on
  linear logic proof search~\citep{HODAS1994327,CERVESATO2000133},
  structuring synthesis rules around an input context of the available
  resources and an output context of the remaining resources that
  can be used to synthesise subsequent sub-terms. Synthesis rules
  are read bottom-up, with judgments $\Gamma \vdash A \Rightarrow^{-} t\ |\ \Delta$
  meaning from the \emph{goal type} $[[A]]$ we can synthesise a term $[[t]]$ using
  assumptions in $[[G]]$, with output context $[[D]]$. We describe
  the rules in turn to aid understanding. Figure~\ref{fig:sub-rules} collects the
  rules for reference.
  \subsubsection{Variables}
Variable terms can be synthesised from assumptions in $\Gamma$ by rules:
  \begin{align*}
    \begin{array}{c}
  \hspace{-3em}\subLinVar
  \;\;\;
  \subGrVar
    \end{array}
  \end{align*}
%
On the left, a variable $[[x]]$ may be synthesised for the goal
$[[ A ]]$ if a linear assumption $[[ x : A ]]$ is present
  in the input context. The input context without $[[x]]$ is then returned as
  the output context, since $x$ has been used. On the right,
  we can synthesise a variable $x$ for $A$ we have a graded
  assumption of $x$ matching the type. However, % to synthesise $[[ x ]]$
  %it may be the case that
  %using $x$ violates the constraints placed by the assumption's grade. For example,
  %the input context may contain assumptions graded by 0, either because this was
  %specified by the type or the variable has been used as far as it's grade
  %permits already. For this reason, the ability to synthesise a
  %graded variable $[[x]]$
%  requires that $[[x]]$'s grade $r$ can be factored into some grade
%  $s + 1$
  the grading $[[ r ]]$ must permit $[[ x ]]$ to
  be used once here. Therefore, the premise states that there exists
  some grade $s$ such that grade $r$ approximates $s + 1$. The grade $s$
  represents the use of $x$ in the rest of the synthesised term, and
  thus $[[ x : [ A ] s ]]$ is in the output context. For the natural
  numbers semiring, this constraint is satisfied by $s = r - 1$ whenever $r \neq
  0$, e.g., if $r = 3$ then $s = 2$. For
  intervals, the role of approximation is more apparent: if $r = [[
  Intrv 0 3]]$ then this rule is satisfied by $s = [[ Intrv 0 2 ]]$
  where $s + 1 = [[ Intrv 0 2 ]] + [[ Intrv 1 1 ]] = [[ Intrv 1 3 ]]
  \sqsubseteq [[ Intrv 0 3 ]]$.
  This is captured by the instantiation of a new
  existential variable representing the new grade for $[[x]]$ in the output
  context of the rule. In the natural numbers semiring, this could be done by
  simply subtracting $1$ from the assumption's
  existing grade $r$. However, as not all semirings have an
  additive inverse, this is instead handled via a constraint on the new grade
  $s$, requiring that $ r \sqsupseteq s + 1 $. In the implementation, the constraint is
  discharged via an SMT solver, where an unsatisfiable result terminates
  this branch of synthesis.

  \subsubsection{Functions}
In typing, $\lambda$-abstraction binds linear variables to introduce
  linear functions. Synthesis from a linear function type therefore mirrors typing:
%
  \begin{align*}
\subAbs
    \end{align*}
%
  Thus, $\lambda x . t$ can be synthesised given that
  $t$ can be synthesised from $B$ in the context of $[[G]]$ extended with a fresh linear assumption $[[ x
  : A]]$. To ensure that $[[x]]$ is used linearly
  by $[[t]]$ we must therefore check that it is not present in
  $[[D]]$.

The left-rule for linear function types then synthesises applications
(as in~\cite{HODAS1994327}):
%
  \begin{align*}
    \hspace{-1em}\subApp
    \end{align*}
%
  %The left rule equivalent of $\textsc{R}\multimap^{-}$ for
  %synthesising an abstraction is the rule for
  %synthesising an application $\textsc{L}\multimap^{+}$.
  The rule synthesises a term for type $[[ C ]]$ in a context that
  contains an assumption $[[ x1 : A -o B ]]$.
%
%, we can apply some value
%  of type $[[A]]$ to this to obtain a value of type $[[B]]$ to use in the
  %synthesis of $[[t1]]$ from the goal type $[[C]]$.
  The first premise synthesises a term $[[t1]]$ for $[[C]]$ under the context
  extended with a fresh linear assumption $[[x2
  : B]]$, i.e., assuming the result of $[[ x1 ]]$. This produces an output context $[[D1]]$ that must not contain
  $[[x2]]$, i.e., $[[x2]]$ is used by $[[t1]]$. The remaining
  assumptions $[[D1]]$ provide the input context to
  synthesise $[[t2]]$ of type $[[A]]$: the argument to the function $[[x1]]$. In the conclusion,
  the application $[[x1 t2]]$ is substituted for $[[x2]]$ inside
  $[[t1]]$, and $[[D2]]$ is the output context.

\subsubsection{Dereliction} Note that the above rule synthesises the application of a
function given by a linear assumption. What if we have a graded
assumption of function type? Rather than duplicating every left rule
for both linear and graded assumptions, we mirror the
dereliction typing rule (converting a linear assumption to graded) as:
%
  \begin{gather*}
    {\small{
    \subDer
    }}
    \end{gather*}
%
Dereliction captures the ability to reuse a graded assumption being
considered in a left rule. A fresh linear assumption $[[y]]$ is generated that
represents the graded assumption's use in a left rule, and must be used
linearly in the subsequent synthesis of $[[t]]$. The output context of this premise then contains $[[x]]$ graded by $s'$, which reflects how $[[x]]$ was used in the synthesis of $[[t]]$, i.e. if $[[x]]$ was not used then $s' = s$. The premise $[[ exists s . r >= s + 1 ]]$ constrains the number of times dereliction can be applied so that it does not exceed $x$'s
original grade $r$.

One may observe that the $\subDerName$ rule makes the presence of the
$\subGrVarName$ rule admissible. Synthesising the usage of a graded variable can
instead be achieved through the use of dereliction on the graded assumption,
followed by the $\subLinVarName$. Nonetheless, we find the inclusion of
$\subGrVarName$ useful as an explanatory tool and optimisation in the
implementation of the calculus. 

  \subsubsection{Graded modalities}
  For a graded modal goal type $[[ [] r A ]]$, we synthesise a promotion
$[[ [ t ] ]]$ if we can synthesise the `unpromoted' $[[t]]$ from $[[A]]$:
%Synthesis of promoted values is captured by the
%  rule $\textsc{R}\square^{-}$.
  \begin{align*}
    \subBox
    \end{align*}
%
  A non-graded value $[[t]]$ may be promoted to a graded value using
  the box syntactic construct.
  Recall that typing of a promotion $[[ [ t ] ]]$
  scales all the graded assumptions used to type $[[ t ]]$ by $r$. Therefore,
  to compute the output context we must ``subtract'' $r$-times the use of the variables in $[[
  t ]]$. However, in the subtractive model $[[ D ]]$ tells us what is
  left, rather than what is used. Thus we first compute the
  \textit{context subtraction} of $[[G]]$ and $[[D]]$
  yielding the variable usage information about $[[ t ]]$:
 %
  \begin{definition}[Context subtraction]\label{def:contextSub}
  For all $[[ G1 ]], [[ G2 ]]$ where $ [[G2]] \subseteq [[G1]]$, $[[G1 - G2]] =$
\begin{align*}
\hspace{-2em}\left\{\!\begin{matrix}
\begin{array}{lll}
% Base case
[[G1]]
  & [[G2]] = \emptyset
\\[0.25em]
([[G1']], [[G1'']]) - [[G2']]
  & [[G2]] = [[ G2', x : A]] & \wedge\ [[G1]] = [[G1', x : A]], [[G1'']]
\\[0.25em]
(([[G1']], [[G1'']]) - [[G2']]), [[x : [A] q]]
  & [[ G2]] = [[G2', x : [A] s]] & \wedge\ [[G1]] = [[ G1',x : [A]
                                   r]],[[G1'']] \\[0em]
          & \wedge \ [[ exists q . r >= q + s]]
          & \!\! \wedge \ \maximal{q}{q'}{r}{q' + s}
\end{array}
\end{matrix}\right.
\end{align*}
%
\end{definition}
As in graded variable synthesis, context subtraction existentially quantifies a
variable $q$ to express the relationship
between grades on the right being ``subtracted'' from those on the
left. The last conjunct states
$q$ is the greatest element (wrt.
to the pre-order) satisfying this constraint, i.e., for all
other $q' \in \mathcal{R}$ satisfying the subtraction constraint
then $[[ q >= q']]$ e.g., if $r = [[ Intrv 2 3 ]]$
and $s = [[ Intrv 0 1]]$ then $q = [[ Intrv 2 2 ]]$ instead of, say,
$[[ Intrv 0 1]]$. This \emph{maximality} condition is
important for soundness (that synthesised programs are well-typed).
%, which does not satisfy
%maximality.

%There may be many such existentially quantified variables coming from
%a subtraction, and we sometimes write $\exists \Sigma . [[ G1 - G2 ]]$
%to denote the set $\Sigma$ of existential variables introduced by such
%a context subtraction.

Thus for \subBoxName, $[[ G - D ]]$ is multiplied by the goal type grade $r$ to obtain how these
  variables are used in $[[t]]$ after promotion. This is then subtracted from
  the original input context $[[G]]$ giving an output context
  containing the left-over variables and grades. Context
  multiplication
  requires that $[[G - D]]$ contains only graded variables,
  preventing the incorrect use of linear variables from $[[G]]$ in
  $[[t]]$.

Synthesis of graded modality elimination, is handled by the
  \subUnboxName\ left rule:
  \begin{align*}
    \subUnbox
    \end{align*}
%
  Given an input context comprising $[[ G ]]$ and a linear
  assumption $[[ x1 ]]$ of graded modal type, we can synthesise an unboxing of
  $[[x1]]$ if we can synthesise a term $[[t]]$ under $[[G]]$
  extended with a graded assumption $[[x2 : [A] r]]$. This returns an output
  context that must contain $[[x2]]$ graded by $s$
  with the constraint that $s$ must approximate $0$. This enforces
  that $x_2$ has been used as much as stated by the grade $r$.

\subsubsection{Products}
The right rule for products \subPairIntroName\ behaves similarly to the
\subAppName\ rule, passing the entire input context $[[ G ]]$ to the first
premise. This is in then used to synthesise the first sub-term of the pair
$[[ t1 ]]$, yielding an output context $ [[ D1 ]]$, which is passed to the
second premise. After synthesising the second sub-term $[[ t2 ]]$, the output
context for this premise becomes the output context of the rule's conclusion.

The left rule equivalent \subPairElimName\  binds two assumptions
$[[ x1 : A ]]$ $[[ x2 : B ]]$ in the premise, representing the constituent sides
of the pair. As with \subAppName, we also ensure that these bound assumptions must not
present in the premise's output context $[[ D ]]$.

\begin{align*}
\begin{array}{c}
  \subPairIntro
\\[1.25em]
  \subPairElim
\end{array}
\end{align*}
\subsubsection{Sums}
The introduction rules for sum types, \subSumIntroLname\ and \subSumIntroRname, are straightforward:
\begin{align*}
  \subSumIntroL
\;\;\;
  \subSumIntroR
\end{align*}
The \subSumElimName\ rule synthesises the left and
right branches of a case statement that may use resources
differently:
\begin{align*}
\begin{array}{c}
\hspace{-3em}{\small{\subSumElim}}
\end{array}
\end{align*}
The output context therefore takes the \textit{greatest
lower bound} ($\sqcap$) of $[[ D1 ]]$ and $[[ D2 ]]$, given by definition~\ref{def:context-glb},
\begin{definition}[Partial greatest-lower bounds of
  contexts]\label{def:context-glb}
For all $[[ G1 ]]$, $[[ G2 ]], [[G1]] \sqcap [[G2]] =$
\begin{align*}
\label{def:lub}
\left\{\begin{matrix}
\begin{array}{lll}
%% Both empty case
\emptyset
  & [[ G1 ]] = \emptyset & \wedge \; [[ G2 ]] = \emptyset
\\
%
%% Left empty
(\emptyset \sqcap [[ G2' ]]), [[ x : [ A ] {glb 0 s} ]]
  & [[ G1 ]] = \emptyset & \wedge \; [[G2]] = [[ G2',x : [A] s]]
\\
%
%% Left is left linear
([[G1']] \sqcap [[(G2',G2'')]]), [[x : A]]
 & [[G1]] = [[{G1', x : A} ]] & \wedge \; [[ G2 ]] = [[ {G2', x : A}, G2'' ]]
\\
%
%% Left is graded
([[G1']] \sqcap [[(G2',G2'')]]), [[x : [A] {glb r s}]]\;\;
 & [[G1]] = [[ G1',x : [A] r]] & \wedge \; [[ G2 ]] = [[{G2', x : [A] s}, G2'']]
\end{array}
\end{matrix}\right.
\end{align*}
where $r\!\sqcap\!s$ is the greatest-lower bound of grades $[[r]]$
and $[[s]]$ if it exists, derived from $\sqsubseteq$.
\end{definition}
%
%

As an example of $\sqcap$, consider the semiring of intervals over natural numbers and two
judgements that could be used as premises for the (\subSumElimName) rule:
%
\begin{align*}
& [[ G, y : [A'] Intrv 0 5, x2 : A |- C =>- t1 ; y : [A'] Intrv 2 5 ]] \\
& [[ G, y : [A'] Intrv 0 5, x3 : B |- C =>- t2 ; y : [A'] Intrv 3 4 ]]
\end{align*}
%
where $t_1$ uses $y$ such that there are $2$-$5$ uses remaining
and $t_2$ uses $y$ such that there are $3$-$4$
uses left. To synthesise $\textbf{case} \ x_{1}\ \textbf{of}\ \textbf{inl}\ x_{2} \rightarrow t_{1};\ \textbf{inr}\ x_{3} \rightarrow t_{2}$
the output context must be pessimistic about what resources are left,
thus we take the greatest-lower bound yielding the interval $[
2\dots4 ]$ here: we know $y$ can be used at least twice and at most
$4$ times in the rest of the synthesised program.

\subsubsection{Unit}
The right and left rules for units are then
self-explanatory following the subtractive resource model:
%
\begin{align*}
\begin{array}{c}
  \subUnitIntro
  \;\;
  \subUnitElim
\end{array}
\end{align*}

%

This completes subtractive synthesis. We conclude
with a key result, that synthesised terms are well-typed at the type from which they
were synthesised:
%
\begin{restatable}[Subtractive synthesis soundness]{lemma}{subSynthSound}
\label{lemma:subSynthSound}
For all $[[ G ]]$ and $[[ A ]]$
then:
\begin{align*}
[[ G |- A =>- t ; D ]] \quad \implies \quad [[ G - D |- t : A ]]
\end{align*}
i.e. $[[ t ]]$ has type $[[ A ]]$
under context $[[ G - D ]]$,
that contains just those linear and
graded variables with grades reflecting their use in $[[ t ]]$.
\end{restatable}
The proof of soundness can be found in Section~\ref{proof:linear-sub-sound} of Appendix~\ref{appendix:proofs}


\begin{figure}[H]
{\footnotesize{
\begin{gather*}
\hspace{-3em}\begin{array}{c}
  \addLinVar
  \;\;\;
  \addGrVar
\\[1.25em]
  \addDer
\\[1.25em]
  \addAbs
\\[1.25em]
  \addApp
\\[1.25em]
  \addBox
\\[1.25em]
  \addUnbox
\\[1.25em]
  \addPairIntro
\\[1.25em]
  \addPairElim
\\[1.25em]
  \addSumIntroL
\;\;\;
  \addSumIntroR
\\[1.25em]
  \addSumElim
\\[1.25em]
  \addUnitIntro
\;\;\;
  \addUnitElim
      \end{array}
  \end{gather*}
}}
\caption{Collected rules of the additive synthesis calculus}
\label{fig:add-rules}
  \end{figure}

\subsection{Additive Resource Management}
\label{subsec:additive}
We now present the dual to subtractive resource management --- the
\emph{additive} approach.
Additive synthesis also uses the input-output context approach, but where
output contexts describe exactly which assumptions were used to synthesise
a term, rather than which assumptions are still available. Additive
   synthesis rules are read bottom-up, with $[[G |- A =>+ t; D]]$
  meaning that from the type $[[A]]$ we synthesise a term $[[t]]$ using
  exactly the assumptions $[[D]]$ that originate from the input
  context $[[G]]$.

  \subsubsection{Variables}
  We unpack the rules, starting with variables:
%
\begin{align*}
  \begin{array}{c}
  \hspace{-3em}\addLinVar
  \;\;\;
  \addGrVar
  \end{array}
  \end{align*}
%
For a linear assumption, the output context contains
just the variable that was synthesised. For a graded assumption $[[x : [A] r]]$, the output
context contains the assumption graded by $1$. To synthesise a
variable from a graded assumption, we must check that the use is
compatible with the grade.

\subsubsection{Graded modalities}
The subtractive approach handled the \textsc{GrVar$^{-}$}
by a constraint $[[ exists s . r >= s + 1]]$. Here however, the
point at which we check that a graded assumption has been used
according to the grade takes place in the \addUnboxName\ rule, where graded
assumptions are bound:
%
\begin{align*}
  \addUnbox
  \end{align*}
%
Here, $[[t]]$ is synthesised under a fresh graded assumption
$[[ x2 : [A] r]]$. This produces an output context containing $[[x2]]$ with
some grade $s$ that describes how $[[x2]]$ is used in $[[t]]$. An
additional premise requires that the original grade $r$ approximates either $s$
if $[[x2]]$ appears in $[[D]]$ or $0$ if it does not,
ensuring that $[[x2]]$ has been used correctly. For the
$\mathbb{N}$-semiring with equality as the ordering, this would
ensure that a variable has been used exactly the number of times
specified by the grade.

The synthesis of a promotion is considerably simpler in the additive
approach. In subtractive resource management it was necessary to calculate how
resources were used in the synthesis of $[[t]]$ before then applying the
scalar context multiplication by the grade $r$ and subtracting this from the
original input $[[G]]$. In additive resource management, however, we can simply
apply the multiplication directly to the output context $[[D]]$ to obtain how
our assumptions are used in $[[ [t] ]]$:
%
\begin{align*}
  \addBox
\end{align*}

\subsubsection{Functions}
Synthesis rules for $\multimap$ have a similar shape to the
subtractive calculus:
%
\begin{align*}
\begin{array}{c}
\addAbs
\\[1.25em]
\addApp
\end{array}
\end{align*}
%
Synthesising an abstraction (\addAbsName) requires that $[[x : A]]$ is in
the output context of the premise, ensuring that linearity is preserved.
Likewise for application (\addAppName), the output
context of the first premise must contain the linearly bound $[[x2 :
B]]$ and the final output context must contain the assumption being used in the
application $[[ x1 : A -o B ]]$. This output context computes the \emph{context
addition} (Def.~\ref{def:contextAdd}) of both output contexts of the premises $[[D1 + D2]]$. If $[[D1]]$
describes how assumptions were used in $[[t1]]$ and $[[D2]]$ respectively for
$[[t2]]$, then the addition of these two contexts describes the usage of
assumptions for the entire subprogram. Recall, context addition
ensures that a linear assumption may not appear in both $[[D1]]$ and
$[[D2]]$, preventing us from synthesising terms that violate linearity.



\subsubsection{Dereliction}
As in the subtractive calculus,
%the additive equivalent of dereliction ($\textsc{Der}$) also allows us to
%reuse graded assumptions in a left rule:
we avoid duplicating left rules to
match graded assumptions by giving a synthesising version of dereliction:
\begin{align*}
  \addDer
  \end{align*}
%
The fresh linear assumption $[[ y : A ]]$ must
appear in the output context of the premise, ensuring it is used. The final
context therefore adds to $[[ D ]]$ an assumption of $[[x]]$ graded by
$1$, accounting for this use of $[[ x ]]$ (temporarily renamed to
$y$).
As with the subtractive case, $\addDerName$ makes $\addGrVarName$ admissible. 

\subsubsection{Products}
The right rule for products \addPairIntroName\ follows the same structure as its
subtractive equivalent, however, here $[[ G ]]$ is passed to both premises.
The conclusion's output context is then formed by taking the context addition of
the $[[ D1 ]]$ and $[[ D2 ]]$. The left rule, \addPairElimName\ follows fairly
straightforwardly from the resource scheme.
\begin{align*}
\begin{array}{c}
  \addPairIntro
\\[0.8em]
  \addPairElim
\end{array}
  \end{align*}

\subsubsection{Sums}
In contrast to the subtractive rule, the rule \addSumElimName\ takes the least-upper bound of
the premise's output contexts (see definition~\ref{def:context-lub}). Otherwise,
the right and left rules for synthesising programs from sum types are straightforward.
\begin{align*}
\begin{array}{c}
  \hspace{-2em}\addSumIntroL
  \;\;\;
  \addSumIntroR
\\[1.25em]
\hspace{-2em}{\footnotesize{\!\!\addSumElim}}
\end{array}
  \end{align*}

\subsubsection{Unit}
As in the subtractive approach, the right and left rules for unit types, are
as expected.
\begin{align*}
\begin{array}{c}
  \hspace{-0.5em}\addUnitIntro
  \;\;\;
  \addUnitElim
\end{array}
  \end{align*}


  Thus concludes the rules for additive synthesis. As with subtractive, we
  have prove that this calculus is sound.
  \begin{restatable}[Additive synthesis soundness]{lemma}{addSynthSound}
\label{lemma:addSynthSound} For all $[[ G ]]$ and $[[ A ]]$:
%
\begin{align*}
[[ G |- A =>+ t ; D ]] \quad \implies \quad [[ D |- t : A ]]
\end{align*}
\end{restatable}
Thus, the synthesised term $[[ t ]]$ is well-typed
at $[[ A ]]$ using only the assumptions $[[ D ]]$.
, where $[[D]]$ is a
subset of $[[G]]$.
i.e., synthesised terms are well typed at the type from which they
were synthesised.
The proof of soundness can be found in Section~\ref{proof:linear-add-sound} of
Appendix~\ref{appendix:proofs}

\subsubsection{Additive pruning}
%
As seen above, the additive approach delays checking
whether a variable is used according to its linearity/grade
until it is bound. We hypothesise that this can lead additive synthesis to explore
many ultimately ill-typed (or \emph{ill-resourced})
paths for too long. Subsequently, we define a ``pruning''
variant of any additive rules with multiple sequenced
premises. For \addPairIntroName\ this is:
%
\begin{align*}
  \begin{array}{c}
    \addPrunePairIntro
  \end{array}
\end{align*}
%
Instead of passing $[[G]]$ to both
premises, $[[G]]$ is the input only for
the first premise. This premise outputs context $[[D1]]$ that
is subtracted from $[[G]]$ to give the input context
of the second premise. This provides an opportunity to
terminate the current branch of synthesis early if $[[ G - D1
]]$ does not contain the necessary resources to attempt the
second premise.
The \addAppName\ rule is similarly adjusted:

\begin{align*}
  \begin{array}{c}
    \addPruneApp
  \end{array}
\end{align*}

\begin{restatable}[Additive pruning synthesis soundness]{lemma}{addPruningSynthSound}
\label{lemma:addPruningSynthSound} For all $[[ G ]]$ and $[[ A ]]$:
%
\begin{align*}
[[ G |- A =>+ t ; D ]] \quad \implies \quad [[ D |- t : A ]]
\end{align*}
\end{restatable}
The proof of soundness can be found in Section~\ref{sec:add-pruning-sound} of
Appendix~\ref{appendix:proofs}


\section{Post-Synthesis Refactoring}
\label{sec:linear-base-refactoring}
A synthesised term often contains some artefacts of the fact that it was
constructed automatically. The structure of our synthesis rules means aspects of
our synthesised programs are unrepresentative in some stylistic ways of the kind
of programs functional programmers typically write. We consider two examples
of these below using Granule code, and show how we apply a refactoring procedure to any synthesised
term to rewrite them in a more idiomatic style. 

\subsection{Abstractions}
A function definition synthesised from a function type using the
$\multimap_{\textsc{R}}$ will take the form of a sequence of nested abstractions which
bind the function's arguments, with the sub-term of the innermost abstraction
containing the function body, e.g.

\begin{granule}
pair : forall { a b : Type } . a -> b -> (a, b)
pair = \x -> \y -> (x, y)
\end{granule}

In most cases, a programmer would write a function definition as a series of
equations with the function arguments given as patterns. Our refactoring
procedure collects the outermost abstractions of a synthesised term and
transforms them into equation-level patterns with the innermost abstraction
body forming the equation body:

\begin{granule}
pair : forall { a b : Type } . a -> b -> (a, b)
pair x y = (x, y)
\end{granule}

\subsection{Unboxing}
An unboxing term is synthesised via the $\Box_{\textsc{L}}$ rule as a case statement
which pattern matches over a box pattern, yielding an assumption with the
grade's usage. Such terms can also be refactored both into function equations
and to avoid nested let bindings. For example, we may write the $k$
combinator using an explicit graded modality:

\begin{granule}
k : forall { a b : Type } . a -> b [0] -> a
k x y = let [z] = y in x
\end{granule}
which we can then refactor into
\begin{granule}
k : forall { a b : Type } . a -> b [0] -> a
k x [z] = z
\end{granule}
This procedure included programs which perform a nested unboxing, refactoring:
\begin{granule}
comp : forall {k : Coeffect, n m : k, a b c : Type} 
     . (a [m] -> b) [n] 
     -> (b [n] -> c) 
     -> a [n * m] 
     -> c
comp x y z = let [u] = x in let [v] = z in y [ u [v] ]
\end{granule}
into:
\begin{granule}
comp : forall {k : Coeffect, n m : k, a b c : Type}
     . (a [m] -> b) [n] 
     -> (b [n] -> c) 
     -> a [n * m] 
     -> c
comp [u] y [v] = y [ u [v] ]
\end{granule}

\section{Focusing}
\label{sec:linear-base-focusing}

The additive and subtractive calculi presented in
Sections~\ref{subsec:subtractive} and~\ref{subsec:additive} provide the
foundations for the implementations of a synthesis tool for Granule programs.
Implementing the rules as they currently stand, however, would yield a highly
inefficient tool. In their current form, the rules of both calculi exhibit a
high degree of non-determinism with regard to order in which rules
can be applied. 
% Examples too wide!!! Need a slimmer one
% For example, consider the following scenario in the additive system:
% \begin{align*} 
% %     x : (A \otimes B) \otimes (C \otimes D) \vdash (D \otimes C) \otimes (B \otimes A) \Rightarrow^{+}\ ? \mid\ ?
%     x :_{0 .. 5} A \otimes B \vdash A \multimap (A \otimes B) \Rightarrow^{+}\ ?\ |\ ? 
% \end{align*}
% Here $x$ is a linear assumption containing a pair of pairs. To reach the goal, $x$ needs to be 
% broken down into its constituent parts using the \addPairElimName rule, and then reconstructed 
% using the \addPairIntroName rule. The exact order in which these rules are applied is irrelevant: the 
% final synthesised terms will be behaviourally equivalent: 
% \begin{align*}
% \inferrule*[right=\addPairElimName] 
%   {\inferrule*[right=\addPairElimName]
%     {\inferrule*[right=\addPairElimName]
%       {} 
%       {y_1 : A,\ y_2 : B,\ x_2 : (C \otimes D)}}
%     {x_1 : (A \otimes B),\ x_2 : (C \otimes D) \vdash (D \otimes C) \otimes (B \otimes A) \Rightarrow^{+}\ \textbf{let}\ (x_1,\ x_2) = x\ \textbf{in}\ ? \mid\ ? }}
%   {x : (A \otimes B) \otimes (C \otimes D) \vdash (D \otimes C) \otimes (B \otimes A) \Rightarrow^{+}\ t \mid\ ?}
% \end{align*}
% where $t$ = 
% \begin{align*}
%   \begin{array}{ll}
%   & \textbf{let}\ (x_1,\ x_2) = x\ \textbf{in} \\ 
%   & \;\;\;\;\;\; \textbf{let}\ (y_1,\ y_2) = x_1\ \textbf{in} \\
%   & \;\;\;\;\;\;\;\;\;\;\; \textbf{let}\ (z_1,\ z_2) = x_2\ \textbf{in}\ ((z_2,\ z_1),\ (y_2,\ y_1))
%   \end{array}
% \end{align*}


This leads to us exploring a large number of redundant search branches: something which can
be avoided through the application of a technique from linear logic proof theory
called \textit{focusing}~\citep{focusing}. Focusing is based on the observation
that some of the synthesis rules are invertible, i.e. whenever the conclusion of
the rule is derivable, then so are its premises. In other words, the order in
which we apply invertible rules doesn't matter. By fixing a particular ordering
on the application of these rules, we eliminate much of the non-determinism that
arises from trying branches which differ only in the order in which invertible
rules are applied. 

We take our both of our calculi and apply this focusing technique to them,
yielding two \textit{focusing} calculi. To do so, we augment our previous
synthesis judgement with an additional input context $\Omega$:
\begin{align*}
\Gamma ; \Omega \vdash [[ A ]] \Rightarrow [[t ]]\ |\ \Delta
\end{align*}
Unlike $\Gamma$ and $\Delta$, $\Omega$ is an \textit{ordered} context, which 
behaves like a stack.

Using the terminology of~\citet{focusing}, we refer to rules that are invertible
as \textit{asynchronous} and rules that are not as \textit{synchronous}. The
intuition is that of asynchronous communication: asynchronous rules can be
applied eagerly, while the non-invertible synchronous rules require us to
\textit{focus} on a particular part of the judgement: either on the assumption
(if we are in an elimination rule) or on the goal (for an introduction rule).
When focusing we apply a chain of synchronous rules, until we either reach a
position where no rules may be applied (at which point the branch terminates),
we have synthesised a term for our goal, or we have exposed an asynchronous
connective at which point we switch back to applying asynchronous rules.

We divide our synthesis rules into four categories, each with their own
judgement form, which refines the focusing judgement above with an arrow
indicating which part of the judgement is currently in focus. An $\Uparrow$
indicates an asynchronous phase, while a $\Downarrow$ indicates a synchronous
(focused) phase. The location of the arrow in the judgement indicates whether we
are focusing on the left or right:
\begin{enumerate}
  \item Right Async: $\multimap_{\textsc{R}}$ rule with the judgement:
        \begin{align*}\Gamma ; \Omega \vdash A \Uparrow\ \Rightarrow t\ |\ \Delta \end{align*}
        \item Left Async:  $\otimes_{\textsc{L}}$, $\oplus_{\textsc{L}}$, $1_{\textsc{L}}$, $\textsc{Der}$, and $\Box_{\textsc{L}}$ rules with the judgement:
        \begin{align*}\Gamma ; \Omega \Uparrow\ \vdash A \Rightarrow t\ |\ \Delta \end{align*}
        \item Right Sync:  $\otimes_{\textsc{R}}$, $\oplus1_{\textsc{R}}$, $\oplus2_{\textsc{R}}$, $1_{\textsc{R}}$, and $\Box_{\textsc{R}}$ rules with the judgement:
        \begin{align*}\Gamma ; \Omega \vdash A \Downarrow\ \Rightarrow t\ |\ \Delta \end{align*}
        \item Left Sync:   $\multimap_{\textsc{L}}$ rule with the judgement:
        \begin{align*}\Gamma ; \Omega \Downarrow\ \vdash A \Rightarrow t\ |\ \Delta \end{align*}
\end{enumerate}

The complete calculi of focusing synthesis rules are given in
Figures~\ref{fig:focus-sub-right-async}-\ref{fig:focus-sub-left-sync} for the
subtractive calculus,
and~\ref{fig:focus-add-right-async}-\ref{fig:focus-add-left-sync} for the
additive, divided into focusing phases. The focusing rules for the additive pruning 
calculus are identical to the additive calculus, save for the $\otimes^{+}_{\textsc{R}}$ and 
$\multimap^{+}_{\textsc{L}}$ rules, which are given in Figure~\ref{fig:focus-add-pruning}.

For the most part, the translation from
non-focused to focused rules is straightforward. The most notable change occurs
in rules in which assumptions are bound. In the cases where a fresh assumption's
type falls into the Left Async category (i.e. $\otimes$, $\oplus$, etc.), then it is bound in the
ordered context $\Omega$ instead of $\Gamma$. Left Async rules 
operate on assumptions in $\Omega$, rather than $\Gamma$. This results in invertible elimination rules
being applied as fully as possible before \textit{focusing} on non-invertible
rules when $\Omega$ is empty.

In addition to the focused forms of the original synthesis calculi, each
calculus has a set of rules which determine which part of the synthesis
judgement will be focused on: the \textsc{Focus} rules. These rules are given by
Figures~\ref{fig:focus-sub-focus}, and~\ref{fig:focus-add-focus} for the
subtractive and additive calculi, respectively.


\begin{figure}[H]
  \begin{align*}
    {\footnotesize{
\begin{array}{c}
  \fSubAbsRuleNoLabel
  \\[1.25em]
  \fSubRAsyncTransitionRule
  \end{array}
    }}
  \end{align*}
  \caption{Right Async rules of the focused subtractive synthesis calculus}
  \label{fig:focus-sub-right-async}
\end{figure}

\begin{figure}[H]
  \begin{align*}
    {\footnotesize{
\hspace{-3.5em}\begin{array}{c}
  \fSubPairElimRuleNoLabel
  \\[1.25em]
  \fSubSumElimRule
  \\[1.25em]
  \fSubUnboxRule
  \\[1.25em]
  \fSubUnitElimRule
  \\[1.25em]
  \fSubDerRule
  \\[1.25em]
  \fSubLAsyncTransitionRule
  \end{array}
    }}
  \end{align*}
  \caption{Left Async rules of the focused subtractive synthesis calculus}
  \label{fig:focus-sub-left-async}
\end{figure}

\begin{figure}[H]
  \begin{align*}
    {\footnotesize{
\begin{array}{c}
  \fSubFocusRRuleNoLabel
  \;\;\;
  % \\[1.25em]
  \fSubFocusLRule
  \end{array}
    }}
  \end{align*}
  \caption{Focus rules of the focused subtractive synthesis calculus}
  \label{fig:focus-sub-focus}
\end{figure}

\begin{figure}[H]
  \begin{align*}
    {\footnotesize{
\begin{array}{c}
\fSubPairIntroRuleNoLabel
  \\[1.25em]
  \fSubSumIntroRuleR
  \,
  \fSubSumIntroRuleL
  \\[1.25em]
  \fSubBoxRule
  \\[1.25em]
  \fSubUnitIntroRule
  \,
  \fSubRSyncTransitionRule
  \end{array}
    }}
  \end{align*}
  \caption{Right Sync rules of the focused subtractive synthesis calculus}
  \label{fig:focus-sub-right-sync}
\end{figure}

\begin{figure}[H]
  \begin{align*}
    {\footnotesize{
\begin{array}{c}
  \fSubAppRuleNoLabel
  \\[1.25em]
  \fSubLinVarRule
  \,
  \fSubGrVarRule
  \\[1.25em]
  \fSubLSyncTransitionRule
  \end{array}
    }}
  \end{align*}
  \caption{Left Sync and Var rules of the focused subtractive synthesis calculus}
  \label{fig:focus-sub-left-sync}
\end{figure}

\begin{figure}[H]
  \begin{align*}
    {\footnotesize{
\hspace{-2em}\begin{array}{c}
  \fAddAbsRuleNoLabel
  \;\;\;
  \fAddRAsyncTransitionRule
  \end{array}
    }}
  \end{align*}
  \caption{Right Async rules of the focused additive synthesis calculus}
  \label{fig:focus-add-right-async}
\end{figure}

\begin{figure}[H]
  \begin{align*}
    {\footnotesize{
\hspace{-3em}\begin{array}{c}
  \fAddPairElimRuleNoLabel
  \\[1.25em]
  \fAddSumElimRule
  \\[1.25em]
  \fAddUnboxRule
  \\[1.25em]
  \fAddDerRule
  \\[1.25em]
  \fAddUnitElimRule
  \\[1.25em]
  \fAddLAsyncTransitionRule
  \end{array}
    }}
  \end{align*}
  \caption{Left Async rules of the focused additive synthesis calculus}
  \label{fig:focus-add-left-async}
\end{figure}

\begin{figure}[H]
  \begin{align*}
    {\footnotesize{
\begin{array}{c}
  \fAddFocusRRuleNoLabel
  \\[1.25em]
  \fAddFocusLRule
  \end{array}
    }}
  \end{align*}
  \caption{Focus rules of the focused additive synthesis calculus}
  \label{fig:focus-add-focus}
\end{figure}

\begin{figure}[H]
  \begin{align*}
    {\footnotesize{
\begin{array}{c}
  \fAddPairIntroRuleNoLabel
  \\[1.25em]
  \fAddSumIntroRuleL
  \,
  \fAddSumIntroRuleR
  \\[1.25em]
  \fAddBoxRule
  \,
  \fAddUnitIntroRule
  \\[1.25em]
  \fAddRSyncTransitionRule
  \end{array}
    }}
  \end{align*}
  \caption{Right Sync rules of the focused additive synthesis calculus}
  \label{fig:focus-add-right-sync}
\end{figure}

\begin{figure}[H]
  \begin{align*}
    {\footnotesize{
\hspace{-2em}\begin{array}{c}
  \fAddAppRule
  \\[1.25em]
  \fAddLinVarRule
  \;\;\;
  \fAddGrVarRule
  \\[1.25em]
  \fAddLSyncTransitionRule
  \end{array}
    }}
  \end{align*}
  \caption{Left Sync and Var rules of the focused additive synthesis calculus}
  \label{fig:focus-add-left-sync}
\end{figure}

\begin{figure}[H]
  \begin{align*}
    {\footnotesize{
\begin{array}{c}
\fAddAltAppRule
  \\[1.25em]
\fAddAltPairIntroRule
  \end{array}
    }}
  \end{align*}
  \caption{Rules of the focused additive pruning synthesis calculus}
  \label{fig:focus-add-pruning}
\end{figure}

One way to view focusing is in terms of a finite state machine, such as
Figure~\ref{fig:focusingFSM}. States comprise the four phases of focusing, plus
two additional states, \textsc{Focus}, and \textsc{Var}. Edges are then the
synthesis rules that direct the transition between focusing phases. The
transitions between these focusing phases are handled by dedicated focusing
rules for each transition. For the asynchronous phases, the
$\Uparrow_{R}$/$\Uparrow_{L}$ handle the transition between right to left
phases, and left to focusing phases, respectively. Conversely, the
$\Downarrow{R}$ rule deals with the transition from a right synchronous phase
back to a right asynchronous phase, with the $\Downarrow{L}$ rule likewise
transitioning to a left asynchronous phase. Depending on the current phase of
focusing, these rules consider the goal type, the assumption currently being
focused on's type, as well as the size of $\Omega$, to decide when to
transition between focusing phases. 

This focused approach to synthesis ensures that we are restricted to generating
programs in $\beta$-normal form, which eliminates a class of redundant programs
for which behaviourally equivalent $\beta$-normal forms can be synthesised in
less steps.

\tikzset{
->, % makes the edges directed
node distance=5cm, % specifies the minimum distance between two nodes. Change if necessary.
every state/.style={thick, fill=gray!10}, % sets the properties for each ’state’ node
initial text=$ $, % sets the text that appears on the start arrow
}

\begin{figure}[H] % ’ht’ tells LaTeX to place the figure ’here’ or at the top of the page
\centering 

\scalebox{0.80}{
\begin{tikzpicture}[every text node part/.style={align=center}]
\node[state, initial, draw] (RA) {\textsc{Right Async} \\ $ \Gamma ; \Omega \vdash A \Uparrow\ \Rightarrow t\ |\ \Delta $};
\node[state] at (10,  0) (LA) { \textsc{Left Async} \\ $\Gamma ; \Omega \Uparrow\ \vdash A \Rightarrow t\ |\ \Delta $};
\node[state] at (5, -5) (F) { \textsc{Focus} \\ $\Gamma ; \emptyset \Uparrow\ \vdash A \Rightarrow t\ |\ \Delta $};
\node[state] at (0, -10) (RS) { \textsc{Right Sync} \\ $\Gamma ; \emptyset \vdash A\ \Downarrow\ \Rightarrow t\ |\ \Delta $};
\node[state] at (10, -10) (LS) { \textsc{Left Sync} \\ $\Gamma ; [[ x : B ]] \Downarrow\ \vdash A \Rightarrow t\ |\ \Delta $};
\node[state, accepting] at (5, -12) (V) { \textsc{Var} \\ $\Gamma ; [[ x : A ]] \Downarrow\ \vdash A \Rightarrow t\ |\ \Delta $};
\draw (RA) edge[loop above] node{$\multimap_{\textsc{R}}$} (RA)
(RA) edge[above, bend left] node{$\Uparrow_{\textsc{R}}$} (LA)
(LA) edge[loop above] node{$\otimes_{\textsc{L}}$, $\oplus_{\textsc{L}}$, \\ $1_\textsc{L}$, $\textsc{Der}$, \\ $\Box_{\textsc{L}}$, $\Uparrow_{\textsc{L}}$ } (LA)
% (LA) edge[loop right] node{$\Uparrow_{L}$} (LA)
(LA) edge[left, bend right] node{$\otimes_{\textsc{L}}$, $\oplus_{\textsc{L}}$, \ \ \ \\ $1_\textsc{L}$, $\textsc{Der}$, \ \ \ \ \\ $\Box_{\textsc{L}}$, $\Uparrow_\textsc{L}$ \ \ \ \ \\  } (F)
% (LA) edge[right, bend left] node{$\Uparrow_{L}$} (F)
(F) edge[below, bend right] node{\ \ \ \  $\textsc{F}_{\textsc{R}}$} (RS)
(RS) edge[left, bend left] node{$\Downarrow_{\textsc{L}}$} (RA)
(F) edge[below, bend left] node{$\textsc{F}_{\textsc{L}}$\ \ } (LS)
(LS) edge[right, bend right] node{$\Downarrow_{\textsc{L}}$} (LA)
(LS) edge[below, bend right] node{$\multimap_{\textsc{L}}$} (RS)
(LS) edge[loop below] node{$\multimap_{\textsc{L}}$} (LS)
(RS) edge[loop below] node{$\otimes_{\textsc{R}}$, $\oplus1_{\textsc{R}}$, $\oplus2_{\textsc{R}}$, \\ $1_\textsc{R}$, $\Box_{\textsc{R}}$} (RS)
(LS) edge[below, bend left] node{\\[0.1em] \ \ \ $\textsc{LinVar}$, \\ \ \ \ $\textsc{GrVar}$ } (V)
;
\end{tikzpicture}
}
\caption{Focusing State Machine}
\label{fig:focusingFSM}
\end{figure}

We conclude with three key results: that applying focusing is sound for the
subtractive (Lemma~\ref{lemma:fSubSynthSound}) and additive
(Lemma~\ref{lemma:fAddSynthSound}), and additive pruning
(Lemma~\ref{lemma:fAddPruningSynthSound}) synthesis calculi. The proofs are
contained in Sections~\ref{proof:focusSubSound},~\ref{proof:focusAddSound},
and~\ref{proof:focusAddPruningSound} of Appendix~\ref{appendix:proofs},
respectively.

\begin{restatable}[Soundness of focusing for subtractive synthesis]{lemma}{focusSoundSub}
  \label{lemma:fSubSynthSound}
For all contexts $[[ G ]]$, $[[ O ]]$ and types $[[ A ]]$
then:
\begin{align*}
  {\footnotesize{
\begin{array}{lll}
 1.\ Right\ Async: & [[ G ; O |- A async =>- t ; D ]] \quad &\implies \quad [[ G , O |- A =>- t ; D ]]\\
 2.\ Left\ Async: & [[ G ; O async |- C =>- t ; D ]] \quad &\implies \quad [[ G , O |- C =>- t ; D ]]\\
 3.\ Right\ Sync: & [[ G ; . |- A sync =>- t ; D ]] \quad &\implies \quad [[ G |- A =>- t ; D ]]\\
 4.\ Left\ Sync: & [[ G ; {x : A }sync |- C =>- t ; D ]] \quad &\implies \quad [[ G, x : A |- C =>- t ; D ]]\\
 5.\ Focus\ Right: & [[ G ; O async |- C =>- t ; D ]] \quad &\implies \quad [[ G |- C =>- t ; D ]]\\
 6.\ Focus\ Left: & [[ G, x : A ; O async |- C =>- t ; D ]] \quad &\implies \quad [[ G |- C =>- t ; D ]]
\end{array}
  }}
\end{align*}
i.e. $[[ t ]]$ has type $[[ A ]]$
under context $[[ D ]]$,
which contains assumptions with grades reflecting their use in $[[ t ]]$.
\end{restatable}

\begin{restatable}[Soundness of focusing for additive synthesis]{lemma}{focusSoundAdd}
  \label{lemma:fAddSynthSound}
For all contexts $[[ G ]]$, $[[ O ]]$ and types $[[ A ]]$
then:
\begin{align*}
  {\footnotesize{
\begin{array}{lll}
 1.\ Right\ Async: & [[ G ; O |- A async =>+ t ; D ]] \quad &\implies \quad [[ G , O |- A =>+ t ; D ]]\\
 2.\ Left\ Async: & [[ G ; O async |- C =>+ t ; D ]] \quad &\implies \quad [[ G , O |- C =>+ t ; D ]]\\
 3.\ Right\ Sync: & [[ G ; . |- A sync =>+ t ; D ]] \quad &\implies \quad [[ G |- A =>+ t ; D ]]\\
 4.\ Left\ Sync: & [[ G ; {x : A }sync |- C =>+ t ; D ]] \quad &\implies \quad [[ G, x : A |- C =>+ t ; D ]]\\
 5.\ Focus\ Right: & [[ G ; O async |- C =>+ t ; D ]] \quad &\implies \quad [[ G |- C =>+ t ; D ]]\\
 6.\ Focus\ Left: & [[ G, x : A ; O async |- C =>+ t ; D ]] \quad &\implies \quad [[ G |- C =>+ t ; D ]]
\end{array}
  }}
\end{align*}
i.e. $[[ t ]]$ has type $[[ A ]]$
under context $[[ D ]]$,
which contains assumptions with grades reflecting their use in $[[ t ]]$.
\end{restatable}

\begin{restatable}[Soundness of focusing for additive pruning synthesis]{lemma}{focusSoundAddPruning}
\label{lemma:fAddPruningSynthSound}
For all contexts $[[ G ]]$, $[[ O ]]$ and types $[[ A ]]$
then:
\begin{align*}
  {\footnotesize{
\begin{array}{lll}
 1.\ Right\ Async: & [[ G ; O |- A async =>+ t ; D ]] \quad &\implies \quad [[ G , O |- A =>+ t ; D ]]\\
 2.\ Left\ Async: & [[ G ; O async |- C =>+ t ; D ]] \quad &\implies \quad [[ G , O |- C =>+ t ; D ]]\\
 3.\ Right\ Sync: & [[ G ; . |- A sync =>+ t ; D ]] \quad &\implies \quad [[ G |- A =>+ t ; D ]]\\
 4.\ Left\ Sync: & [[ G ; {x : A }sync |- C =>+ t ; D ]] \quad &\implies \quad [[ G, x : A |- C =>+ t ; D ]]\\
 5.\ Focus\ Right: & [[ G ; O async |- C =>+ t ; D ]] \quad &\implies \quad [[ G |- C =>+ t ; D ]]\\
 6.\ Focus\ Left: & [[ G, x : A ; O async |- C =>+ t ; D ]] \quad &\implies \quad [[ G |- C =>+ t ; D ]]
\end{array}
  }}
\end{align*}
i.e. $[[ t ]]$ has type $[[ A ]]$
under context $[[ D ]]$,
which contains assumptions with grades reflecting their use in $[[ t ]]$.
\end{restatable}


\section{Evaluating the Synthesis Calculi}
\label{sec:linear-base-evaluation}

\newcommand{\stderr}[1]{\textcolor{gray}{${#1}$}} % \pm{#1}
\newcommand{\fail}{\textcolor{mypink3}{$\times$}}
\newcommand{\success}{\checkmark}
\newcommand{\highlight}[1]{%
{\setlength{\fboxsep}{0pt}\colorbox{yellow!50}{$\displaystyle#1$}}}

Prior to evaluation, we made the following hypotheses about the
relative performance of the additive versus subtractive approaches:
%
\begin{enumerate}
\item[H1.] (\textbf{Solving; Additive requires less}) Additive synthesis should make fewer calls to the solver, with lower
complexity theorems (fewer quantifiers). Dually,
subtractive synthesis makes more calls to the solver with
higher complexity theorems.

\item[H2.] (\textbf{Paths; Subtractive explores fewer}) For complex problems, additive will
explore more paths as it cannot tell whether a variable is not
well-resourced until closing a binder; additive pruning and subtractive will
explore fewer paths as they can fail sooner.

\item[H3.] (\textbf{Performance; additive faster on simpler examples}) A
corollary of the above two: simple examples will likely be faster in additive
mode, but more complex examples will be faster in subtractive.
\end{enumerate}

\subsection{Methodology}
We implemented our approach as a synthesis tool for
Granule, integrated with its core tool. Granule features
ML-style polymorphism (rank-0 quantification) but we do not address polymorphism here.
Instead, programs are synthesised from type schemes treating universal
type variables as logical atoms. Multiplicative products are
primitive in Granule, although additives coproducts are provided via
ADTs, from which we define a core sum type to use here.

Constraints on resource usage are handled via Granule's existing
symbolic engine, which compiles constraints on grades (for various semirings)
to the SMT-lib format for Z3~\citep{z3}.
%In the case of graded variable synthesis in the subtractive
%scheme, the kind of the assumption's grade (i.e., what semiring it
%belongs to) is inferred using Granule's type
%checker, which is used to generate an existential variable representing
%the remaining available usage of the graded assumption.
We use the LogicT monad for backtracking search~\citep{logict} and the Scrap
Your Reprinter library for splicing synthesised code into syntactic ``holes''
(represented by \granin{?} in Granule), preserving the rest of the program text~\citep{clarke2017scrap}.

To evaluate our synthesis tool we developed a suite of benchmarks comprising
Granule type schemes for a variety of operations using linear and graded modal
types. We divide our benchmarks into several classes of problem:
%
\begin{itemize}[itemsep=0em,leftmargin=1.1em]
\item \textbf{Hilbert}: the Hilbert-style axioms of
  intuitionistic logic (including SKI combinators), with appropriate $\mathbb{N}$ and $\mathbb{N}$-intervals
  grades where needed (see, e.g., $S$ combinator in
  Example~\ref{ex:s-comb} or coproduct elimination in Example~\ref{exm:or3}).

\item \textbf{Comp}: various translations of function composition
into linear logic: multiplicative, call-by-value and
call-by-name using $!$~\citep{girard1987linear}, I/O using $!$~\citep{liang2009focusing},
and coKleisli composition over $\mathbb{N}$ and arbitrary semirings:
e.g. $\forall r, s \in \mathcal{R}$:
%
\begin{equation*}
\textit{comp-}\textit{coK}_{\mathcal{R}} : [[ {[] r ({[] s A} -o B)} -o {({[] r B} -o C) -o {{[] {r * s} A} -o C}} ]]
\end{equation*}
%
\item \textbf{Dist}: distributive laws of various graded
modalities over functions, sums, and products,
e.g., $\forall r \in \mathbb{N}$, or
$\forall r \in \mathcal{R}$ in any semiring, or $r = [[ Intrv 0 Inf ]]$:
%
\begin{equation*}
\textit{pull}_\oplus : [[ (Sum {[] r A} {[] r B}) -o [] r (Sum A B) ]]
\end{equation*}
\begin{equation*}
\textit{push}_\multimap : [[ {[] r (A -o B)} -o {{[] r A} -o [] r B} ]]
\end{equation*}
%
%% data type and vice vera as our second class of programs. In the latter case, we have programs
%which \textit{pull} natural number, !, and general modalities out of sum,
%product, and vector data types. In the former case, we have programs
%which distribute (\textit{push}) natural number, !, and general graded modalities over a
%function type. The ability to synthesise \textit{push} over other data types is
%further work (see section \ref{sec:futurework}).

\item \textbf{Vec}: map operations on
fixed size vectors encoded as products, e.g.:
\begin{align*}
\begin{array}{rl}
\textit{vmap}_5 :& [[ {[] 5 (A -o B)} ]] \\ \multimap & [[(Tup (Tup (Tup (Tup A A) A) A) A) ]] \\ \multimap & [[ (Tup (Tup (Tup (Tup B B) B) B) B) ]]
\end{array}
\end{align*}
%

\item \textbf{Misc}: includes Example~\ref{exm:security} (information-flow
  security) and functions which must share resources between graded
  modalities, e.g.:
%one problem requires two graded values (the first two parameters)
%to be shared between two applications of a function given as the third input:
%
\begin{align*}
\begin{array}{rl}
\textit{share} :& [[ {[] 4 A}]] \\ \multimap & [[ [] 6 A ]] \\ \multimap & [[ [] 2 ((Tup (Tup (Tup (Tup A A) A) A) A) -o B)]] \\ \multimap & [[ (Tup B B) ]]
\end{array}
\end{align*}
%
\end{itemize}

\begin{table}
\scalebox{0.85}{\parbox{.5\linewidth}{%
\begin{align*}
\hspace{-2em}
\setlength{\arraycolsep}{0.2em}
\begin{array}{r|rll}
\textbf{Hilbert} & \\ \hline
\text{$\otimes{}$Intro} & \otimes_i : \forall a , b . & a \multimap b \multimap (a \otimes b) \\
\text{$\otimes{}$Elim} & \otimes_{e1} : \forall a , b . & (a \otimes \Box_0 b) \multimap a \\
                       & \otimes_{e2} : \forall a , b . & () (\Box_0 a \otimes b) \multimap b \\
\text{$\oplus{}$Intro} & \oplus_{i1} : \forall a , b . & a \multimap a \oplus b \\
                       & \oplus_{i2} : \forall a , b . & b \multimap a \oplus b \\
\text{$\oplus{}$Elim}  & \oplus_{e} : \forall a , b, c . & \Box_{0...1}(a \multimap c) \multimap \Box_{0...1}(b \multimap c) \multimap (a \oplus b) \multimap c \\
\text{SKI} & s : \forall a, b, c . & (a \multimap (b \multimap c)) \multimap (a \multimap b) \multimap (\Box_2 a \multimap c) \\
& k : \forall a, b . & a \multimap \Box_0 b \multimap a \\
& i : \forall a . & a \multimap a \\ \hline
\textbf{Comp} & \\ \hline
\text{0/1} & \circ_{I/O} :\forall a, b, c . & \Box (\Box a \multimap \Box b) \multimap \Box (\Box b \multimap \Box c) \multimap \Box (\Box a \multimap c) \\
\text{CBN} & \circ_{\textsc{cbn}} : \forall a, b, c . & \Box (\Box a \multimap b) \multimap \Box (\Box b \multimap c) \multimap \Box a \multimap c \\
\text{CBV} & \circ_{\textsc{cbv}} : \forall a, b, c . & \Box (\Box a \multimap \Box b) \multimap \Box (\Box b \multimap \Box c) \multimap \Box \Box a \multimap \Box c\\
\text{coK-$\mathcal{R}$} & \circ_{\mathcal{R}} :
\forall \mathcal{R}, r, s \in \mathcal{R}, a, b, c . &
      \Box_r (\Box_s a \multimap b) \multimap (\Box_r b \multimap c) \multimap \Box_{r \cdot s} a \multimap c
\\
\text{mult} & \circ : \forall a, b, c . & (a \multimap b) \multimap (b \multimap c) \multimap (a \multimap c) \\
\text{coK-$\mathbb{N}$} & \circ_{\mathbb{N}} :
\forall r, s \in \mathbb{N}, a, b, c . &
      \Box_r (\Box_s a \multimap b) \multimap (\Box_r b \multimap c) \multimap \Box_{r \cdot s} a \multimap c
\\ \hline
\textbf{Dist} & \\ \hline
\text{$\oplus$-$\mathbb{N}$} & \textit{pull}_\oplus : \forall r :
                               \mathbb{N}, a, b . & (\Box_r a \oplus \Box_r b) \multimap \Box_r (a \oplus b) \\
\text{$\oplus$-!} &  \textit{pull}_\oplus : \forall a, b . & (\Box a \oplus \Box b) \multimap \Box (a \oplus b) \\
\text{$\oplus$-$\mathcal{R}$} & \textit{pull}_\oplus : \forall
                                \mathcal{R}, r \in \mathcal{R}, a, b
                                . &  (\Box_r a \oplus \Box_r b) \multimap \Box_r (a \oplus b) \\
\text{$\otimes$-$\mathbb{N}$} & \textit{pull}_\otimes : \forall r :
                                \mathbb{N}, a, b . & (\Box_r a \otimes \Box_r b) \multimap \Box_r (a \otimes b) \\
\text{$\otimes$-!} & \textit{pull}_\otimes : \forall a, b . & (\Box a \otimes \Box b) \multimap \Box (a \otimes b) \\
\text{$\otimes$-$\mathbb{R}$} & \textit{pull}_\otimes : \forall
                                  \mathcal{R}, r,
                                  a, b . & (\Box_r a \otimes \Box_r b) \multimap \Box_r (a \otimes b)  \\
\text{$\multimap$-$\mathbb{N}$} & \textit{push}_\multimap : \forall r
                                  : \mathbb{N}, a, b . & \Box_r (a \multimap b) \multimap \Box_r a \multimap \Box_r b \\
\text{$\multimap$-!} & \textit{push}_\multimap : \forall a, b . & \Box (a \multimap b) \multimap \Box a \multimap \Box b \\
\text{$\multimap$-$\mathcal{R}$} & \textit{push}_\multimap : \forall
                                   \mathcal{R}, r : \mathcal{R}, a, b
                                   .& \Box_r (a \multimap b) \multimap \Box_r a \multimap \Box_r b \\ \hline
\textbf{Vec} \\ \hline
\text{vec5} & \textit{vmap}_5 : \forall a, b . & \Box_5 (a \multimap b) \multimap ((((a \otimes a) \otimes a) \otimes a) \otimes a) \\
& & \multimap ((((b \otimes b) \otimes b) \otimes b) \otimes b) \\
\text{vec10} & \textit{vmap}_{10} : \forall a, b . & \textit{as above but for 10-tuples} \\
\text{vec15} & \textit{vmap}_{15} : \forall a, b . & \textit{as above but for 15-tuples} \\
\text{vec20} & \textit{vmap}_{20} : \forall a, b . & \textit{as above but for 20-tuples}\\ \hline
\textbf{Misc} \\ \hline
\text{split$\oplus$} & \multicolumn{2}{l}{ \textit{split} : \forall a, b, c .
\Box_{2...3} b \multimap (a \oplus c) \multimap (( a \otimes \Box_{2..2} b) \oplus (c \otimes \Box_{3...3} b))}
 \\
\text{split$\otimes$} & \multicolumn{2}{l}{\textit{split} : \forall a, b . 
\Box_{0...2} (a \multimap a \multimap a) \multimap \Box_{10...10} a \multimap (\Box_{2...2} a \otimes \Box_{6...6} a) 
}
\\
share & \multicolumn{2}{l}{\textit{share} : \forall a, b .  \Box_4 a \multimap \Box_6 a \multimap \Box_2 ((((( a \otimes a) \otimes a) \otimes a) \otimes a) \multimap b) \multimap (b \otimes b)}
\\
Exm.~\ref{exm:security}
& \multicolumn{2}{l}{\textit{noLeak} : \forall a, b . (\Box_{[[ Public ]]} a \otimes \Box_{[[ Private]]} a) \multimap (\Box_{[[ Public ]]}(a \otimes 1) \multimap b) \multimap b }
\end{array}
\end{align*}
}}
\caption{List of benchmark synthesis problems}
\label{fig:list-of-types}
\end{table}
% }
%
Table~\ref{fig:list-of-types} provides the complete list of Granule type schemes
used for these synthesis problems (32 in total). Note that these are type
schemes which quantify over type variables ($a$, and $b$), however, we simply
treat each type variable as a logical atom, unifiable only with itself.
Chapter~\ref{chapter:extended} provides a proper treatment of synthesis from
Rank-1 polymorphic type schemes. Note also that $\Box A$ is used as shorthand
for $\square_{0...\infty} A$ (graded modality with indices drawn from intervals
over $\mathbb{N} \cup \infty$). The compete synthesised program code for the
benchmarking problems can be found in Section~\ref{sec:linear-benchmarks} of
Appendix~\ref{appendix:benchmarks}


We found that Z3 is highly variable in its solving time, so timing
measurements are computed as the mean of 20 trials. We used
Z3 version 4.8.8 on a Linux laptop with an Intel i7-8665u @ 4.8 Ghz
and 16 Gb of RAM.


\subsection{Results and Analysis}
%
For each synthesis problem, we recorded whether synthesis
was successful or not (denoted $\success$ or \fail), the mean
total synthesis time ($\mu{T}$), and the number of
calls made to the SMT solver (\textsc{N}).
Table~\ref{tab:results} summarises the results with the fastest case for each
benchmark highlighted.
For all benchmarks that used
the SMT solver, the solver accounted for $91.73\%-99.98\%$
of synthesis time, so we report only the mean
total synthesis time $\mu{T}$. % as this gives a good proxy of the solver time.
We set a timeout of 120 seconds.

\begin{table}[t]
{\small{
\begin{center}
\setlength{\tabcolsep}{0.3em}
\scalebox{0.9}{
% this is wide enough to show 4 modes worth of data
\begin{tabular}{p{2.5em}r|p{0.75em}rr|p{0.5em}rr|p{0.5em}rr}
 & & \multicolumn{3}{c|}{Additive}&\multicolumn{3}{c|}{Additive (pruning)}&\multicolumn{3}{c|}{Subtractive}\\ \hline
\multicolumn{2}{c|}{{Problem}} &  & \multicolumn{1}{c}{$\mu{T}$ (ms)} & \multicolumn{1}{r|}{\textsc{N}} & & \multicolumn{1}{c}{$\mu{T}$ (ms)} & \multicolumn{1}{r|}{\textsc{N}} & & \multicolumn{1}{c}{$\mu{T}$ (ms)} & \multicolumn{1}{r|}{\textsc{N}} \\ \hline\hline
\multirow{5}{*}{{\rotatebox{90}{\textbf{Hilbert}}}}
& $\otimes{}$Intro & \success{} &   {\highlight{$6.69 (\stderr{  0.05})$}} &   2  &    \success{}
                                             &   9.66 (\stderr{  0.23}) &   2
                                                                     &     \success{}   &  10.93 (\stderr{  0.31}) &   2 \\
& $\otimes{}$Elim                     & \success{} &   0.22 (\stderr{  0.01}) &   0       & \success{} &   {\highlight{$0.05 (\stderr{  0.00})$}} &   0       & \success{} &   0.06 (\stderr{  0.00}) &   0      \\
& $\oplus{}$Intro & \success{} &   0.08 (\stderr{  0.00}) &   0    &  \success{}
                                          &   {\highlight{$0.07 (\stderr{  0.00})$}} &   0
                                                                   &    \success{}
                                                                                 &
                                                                                   {\highlight{$0.07
                                                                                   (\stderr{
                                                                                   0.00})$}}
                                                                                                                   &   0 \\
& $\oplus{}$Elim                       & \success{} &   {\highlight{$7.26 (\stderr{  0.30})$}} &   2       & \success{} &  13.25 (\stderr{  0.58}) &   2       & \success{} & 204.50 (\stderr{  8.78}) &  15      \\ %SmtT/T = 93.85304985917998%, 91.73667044352199%, 99.1533306157868%
& SKI & \success{} &   {\highlight{$8.12 (\stderr{  0.25})$}} &   2  &    \success{} &  24.98
                                                                      (\stderr{
                                                                      1.19}) &
                                                                               2
                                                                             &  \success{}
                                                                                 &
                                                                                   41.92
                                                                                   (\stderr{
                                                                                   2.34})
                                                                                                                   &
                                                                                                                     4
                 \\
\hline
\multirow{6}{*}{{\rotatebox{90}{\textbf{Comp}}}}
& 01                        & \success{} &  {\highlight{$28.31 (\stderr{  3.09})$}} &   5       & \success{} &  41.86 (\stderr{  0.38}) &   5  & \fail{} & Timeout & -    \\
& cbn                       & \success{} &  {\highlight{$13.12 (\stderr{  0.84})$}} &   3       & \success{} &  26.24 (\stderr{  0.27}) &   3  & \fail{} & Timeout & -    \\
& cbv                       & \success{} &  {\highlight{$19.68 (\stderr{  0.98})$}} &   5       & \success{} &  34.15 (\stderr{0.98}) &   5   & \fail{} & Timeout & -   \\
& $\circ\textit{coK}_\mathcal{R}$                 & \success{} &  33.37 (\stderr{  2.01}) &   2       & \success{} &  {\highlight{$27.37$}} (\stderr{  0.78}) &   2       & \fail{}  &  92.71 (\stderr{  2.37}) &   8      \\  % SmtT/T = 98.56608024136683%, 98.56541178986083%, 98.70614421865372%
& $\circ\textit{coK}_\mathbb{N}$                 & \success{} &  27.59 (\stderr{  0.67}) &   2       & \success{} &  {\highlight{$21.62$}} (\stderr{  0.59}) &   2       & \fail{}  &  95.94 (\stderr{  2.21}) &   8      \\ % SmtT/T = 98.36019681512148%, 98.35210841341276%, 98.63478123527435%
& mult                      & \success{} &   0.29 (\stderr{  0.02}) &   0       & \success{} &   0.12 (\stderr{  0.00}) &   0       & \success{} &   {\highlight{$0.11 (\stderr{  0.00})$}} &   0      \\     % SmtT/T = 0.0%, 0.0%, 0.0%
\hline
\multirow{9}{*}{{\rotatebox{90}{\textbf{Dist}}}}
& $\otimes$-!                 & \success{} &  {\highlight{$12.96 (\stderr{  0.48})$}} &   2       & \success{} &  32.28 (\stderr{  1.32}) &   2       & \success{} & 10487.92 (\stderr{  4.38}) &   7      \\ % SmtT/T = 96.84305139487837%, 99.16605247629178%, 99.98233353925326%
& $\otimes$-$\mathbb{N}$                  & \success{} &  {\highlight{$24.83 (\stderr{  1.01})$}} &   2       & \fail{}  &  32.18 (\stderr{  0.80}) &   2       & \fail{}  &  31.33 (\stderr{  0.65}) &   2      \\  % SmtT/T = 99.16249211706345%, 98.19366472714205%, 97.68400765522344
& $\otimes$-$\mathcal{R}$                  & \success{} &  {\highlight{$28.17 (\stderr{  1.01})$}} &   2       & \fail{}  &  29.72 (\stderr{  0.90}) &   2       & \fail{}  &  31.91 (\stderr{  1.02}) &   2      \\ % SmtT/T = 99.26176085615197%, 97.2013820814111%, 97.92618319348946%
& $\oplus$-!                & \success{} &   {\highlight{$7.87 (\stderr{  0.23})$}} &   2       & \success{} &  16.54 (\stderr{  0.43}) &   2       & \success{} & 160.65 (\stderr{  2.26}) &   4      \\ % SmtT/T = 96.52232766433309%, 96.54651122326587%, 99.69538508877449
& $\oplus$-$\mathbb{N}$                & \success{} &  {\highlight{$22.13 (\stderr{  0.70})$}} &   2       & \success{} &  30.30 (\stderr{  1.02}) &   2       & \fail{}  &  23.82 (\stderr{  1.13}) &   1      \\  % SmtT/T = 99.00528362933007%, 99.08548651040508%, 98.49944141606836%
& $\oplus$-$\mathcal{R}$                & \success{} &  {\highlight{$22.18 (\stderr{  0.60})$}} &   2       & \success{} &  31.24 (\stderr{  1.40}) &   2       & \fail{}  &  16.34 (\stderr{  0.40}) &   1      \\ % SmtT/T = 99.08179689945528%, 99.16435051551996%, 98.4221935356417%
& $\multimap$-!                & \success{} &   {\highlight{$6.53 (\stderr{  0.16})$}} &   2       & \success{} &  10.01 (\stderr{  0.25}) &   2       & \success{} & 342.52 (\stderr{  2.64}) &   4      \\% SmtT/T = 96.72718832940333%, 96.60077128502638%, 99.698687842732%
& $\multimap$-$\mathbb{N}$                  & \success{} &  29.16 (\stderr{  0.82}) &   2       & \success{} &  {\highlight{$28.71 (\stderr{  0.67})$}} &   2       & \fail{}  &  54.00 (\stderr{  1.53}) &   4      \\% SmtT/T = 99.14821593847735%, 99.13267964321408%, 99.00055370249412%
& $\multimap$-$\mathcal{R}$                  & \success{} &  29.31 (\stderr{  1.84}) &   2       & \success{} &  {\highlight{$27.44 (\stderr{  0.60})$}} &   2       & \fail{}  &  61.33 (\stderr{  2.28}) &   4      \\% SmtT/T = 99.22644411928067%, 99.20991868100477%, 99.01094953081872%
\hline
\multirow{4}{*}{{\rotatebox{90}{\textbf{Vec}}}}
& vec5                      & \success{} &   {\highlight{$4.72 (\stderr{  0.07})$}} &   1       & \success{} &  14.93 (\stderr{  0.21}) &   1       & \success{} &  78.90 (\stderr{  2.25}) &   6      \\% SmtT/T = 80.52034229302626%, 95.98937410635435%, 98.89325740352668%
& vec10                     & \success{} &   {\highlight{$5.51 (\stderr{  0.36})$}} &   1       & \success{} &  20.81 (\stderr{  0.77}) &   1       & \success{} & 142.87 (\stderr{  5.86}) &  11      \\  % SmtT/T = 57.23088868771888%, 91.78964923101526%, 98.76365835388931%
& vec15                     & \success{} &   {\highlight{$9.75 (\stderr{  0.25})$}} &   1       & \success{} &  22.09 (\stderr{  0.24}) &   1       & \success{} & 195.24 (\stderr{  3.20}) &  16      \\ % SmtT/T = 37.28829884651822%, 88.79604744779657%, 98.11179611018657%
& vec20                     & \success{} &  {\highlight{$13.40 (\stderr{  0.46})$}} &   1       & \success{} &  30.18 (\stderr{  0.20}) &   1       & \success{} & 269.52 (\stderr{  4.25}) &  21      \\ % SmtT/T = 25.881885439958207%, 82.374721744386%, 97.48691949245291%
\hline
\multirow{4}{*}{{\rotatebox{90}{\textbf{Misc}}}}
& split$\oplus$            & \success{} &   {\highlight{$3.79 (\stderr{  0.04})$}} &   1       & \success{} &   5.10 (\stderr{  0.16}) &   1       & \success{} & 10732.65 (\stderr{  8.01}) &   6      \\ % SmtT/T = 94.19385210967563%, 94.14708972520134%, 99.97155844397255%
& split$\otimes$                      & \success{} &  {\highlight{$14.07 (\stderr{  1.01})$}} &   3       & \success{} &  46.27 (\stderr{  2.04}) &   3       & \fail{} & Timeout & -                            \\ % SmtT/T = 87.40012646315907%, 97.15716368724911%
& share                     & \success{} & 292.02 (\stderr{ 11.37}) &  44       & \success{} & {\highlight{$100.85 (\stderr{  2.44})$}} &   6       & \success{} & 193.33 (\stderr{  4.46}) &  17      \\ % SmtT/T = 94.3058504587738%, 97.15508002763923%, 99.19701142373343%
& Exm.~\ref{exm:security}                 & \success{} &   {\highlight{$8.09 (\stderr{  0.46})$}} &   2       & \success{} &  26.03 (\stderr{  1.21}) &   2       & \success{} & 284.76 (\stderr{  0.31}) &   3      \\ % SmtT/T = 96.70245619318075%, 99.14227225428877%, 99.83905641669199%
\end{tabular}
}
\end{center}}}

\caption{Results. $\mu{T}$ in \emph{ms} to 2 d.p.
with standard sample error in brackets}
\label{tab:results}
\end{table}



\subsubsection{Additive versus subtractive}
As expected, the additive approach generally synthesises programs faster than
the subtractive. Our first hypothesis (that the additive approach in general
makes fewer calls to the SMT solver) holds for almost all benchmarks, with the
subtractive approach often far exceeding the number made by the additive. This
is explained by the difference in graded variable synthesis between approaches.
In the additive, a constant grade $1$ is given for graded assumptions in the
output context, whereas in the subtractive, a fresh grade variable is created
with a constraint on its usage which is checked immediately. As the total
synthesis time is almost entirely spent in the SMT solver (more than 90\%),
solving constraints is by far the most costly part of synthesis leading to the
additive approach synthesising most examples in a shorter amount of time.

Graded variable synthesis in the subtractive case also results in several
examples failing to synthesise. In some cases, e.g., the first three
\textit{comp} benchmarks, the subtractive approach times-out as synthesis
diverges with constraints growing in size due to the maximality condition and
absorbing behaviour of $[[ Intrv 0 Inf ]]$ interval. In the case of
$\textit{coK-$\mathcal{R}$}$ and $\textit{coK-$\mathbb{N}$}$, the generated
constraints have the form $\forall r. \exists s. r \sqsupseteq s + 1 $ which is
not valid $\forall r \in \mathbb{N}$ (e.g., when $r = 0$), which suggests that
the subtractive approach does not work well for polymorphic grades. As further
work, we are considering an alternate rule for synthesising promotion with
constraints of the form $\exists s . s = s' * r$, i.e., a multiplicative inverse
constraint.
%solving approach cannot tell this holds for all semirings. Indeed, it
%is false for $\ma
%which is not satisfiable for all semirings, including \textsc{N}, causing examples
%with arbitrary \textsc{N} grades to also fail.

In more complex examples we see evidence to support our second hypothesis. The
\textit{share} problem requires a lot of graded variable synthesis which is
problematic for the additive approach, for the reasons described in the second
hypothesis. In contrast, the subtractive approach performs better, with $\mu{T}
= 193.3\textit{ms}$ as opposed to additive's $292.02\textit{ms}$. However,
additive pruning outperforms both.

% Not as important
Notably, on examples which are purely linear such as \textit{andElim} from
Hilbert's axioms or \textit{mult} for function composition, the subtractive
approach generally performs better. Linear programs without graded modalities
can be synthesised without the need to interface with Z3 at all, making the
differences here somewhat negligible as solver time generally makes up for the
vast proportion of total synthesis time.

\subsubsection{Additive pruning}
The pruning variant of additive synthesis (where subtraction
takes place in the premises of multiplicative rules) had mixed results
compared to the default. In simpler examples, the overhead of pruning
(requiring SMT solving) outweighs
the benefits obtained from reducing the space. However, in more
complex examples which involve synthesising many graded variables (e.g. \textit{share}), pruning is
especially powerful, performing better than the subtractive
approach. However, additive pruning failed to synthesis two
 examples which are polymorphic in their grade
 ($\otimes$-$\mathbb{N}$) and in the semiring ($\otimes$-$\mathcal{R}$).


Overall, the additive approach outperforms the subtractive and is successful at
synthesising more examples, including ones polymorphic in grades and even the
semiring itself. Given that the literature on linear logic theorem proving is
typically subtractive, this is an interesting result. Going forward, we will
focus on the additive scheme. 

\section{Conclusion}
\label{sec:linear-base-conclusion}

At this point we have constructed a simple program synthesis tool for Granule,
paramterised by a resource management scheme, which effectively deals with the
problems of treating data as a resource inside a program. Both schemes would be
a reasonable choice for further development of a synthesis tool for our language
based on the graded linear $\lambda$-calculus.   

Going forward, however, we focus primarily on the additive resource management
scheme, using this as the basis for our more feature-rich fully-graded synthesis
calculus in Chapter~\ref{chapter:extended}.  
The evaluation in Section~\ref{sec:linear-base-evaluation} showed that the
additive approach generally yields smaller and simpler theorems than the
subtractive, requiring less time to solve. Theorem proving becomes even more
prevalent in synthesis for a fully graded typing calculus - potentially every
rule introduces new constraints that require solving, thus the speed at which
this can be carried out is especially important.

While the tool presented in this chapter allows users to synthesise a
considerable subset of Granule programs, our language is still limited in its
expressivity. Data types comprise only product, sum, and unit types, while
synthesis of recursive function defintions or functions which make use of other
in-scope values such as top-level definitions is not permitted. One notable
limitation of our typing calculi is the inability to express (and therefore
synthesise) programs which perform a deep pattern match over a graded data type.
A clear example of this can be found in the synthesis of programs which
distribute a graded modality over a data type. Consider a classic example of a
distributive program, \textit{push}:
\begin{align*}
  push: \Box_r(A \otimes B) \multimap \Box_r A \otimes \Box_r B
\end{align*}
which takes a data type graded by $r$ (in this case the product type $A \otimes
B$), and distributes $r$ over the constituent elements of the product
$A$ and $B$. Given this goal type, how would we go about synthesising a program
in our tool? 

We instatiate the \addAbsName\ rule at this type, building a partial synthesis
derivation. Although we use the additive scheme for this example, the exact same
situation arises in the subtractive.
\begin{align*}
    \inferrule*[right=\addUnboxName]
      { x_2 :_r A \otimes B \vdash \Box_r A \otimes \Box_r B \Rightarrow\ ?\ |\ ? }
      {\inferrule*[right=\addAbsName] {x_1 : \Box_r (A \otimes B) \vdash \Box_r A \otimes \Box_r B \Rightarrow\ \textbf{let}\ [x_2] = x_1\ \textbf{in}\ ?  \ |\ ?} 
        {\emptyset \vdash \Box_r(A \otimes B) \multimap \Box_r A \otimes \Box_r B \Rightarrow \lambda x_1 . ? \ |\ ? }}
\end{align*}
After applying \addAbsName\ followed by \addUnboxName, we now have the graded
assumption $x_2 :_r A \otimes B$ in our context which we must use to construct a
term of type $\Box_r A \otimes \Box_r B$. We might expect that the path
synthesis should take now would be to break $x_2$ down into two graded
assumptions with types $A$ and $B$, promote these graded assumptions using the
\addBoxName\ rule, before finally peforming a pair introduction to yield $\Box_r
A \otimes \Box_r B$. However, in order to apply the pair elimination rule
\addPairElimName\  
and break our graded assumption into two, we must perform a dereliction on
$x_2$, to yield a linear copy: 
\begin{align*}
  \inferrule*[right=\addDerName]
      {x_2 :_r A \otimes B, x_3 : A \otimes B \vdash \Box_r A \otimes \Box_r B \Rightarrow\ ?\ }
      { x_2 :_r A \otimes B \vdash \Box_r A \otimes \Box_r B \Rightarrow\ ?\ |\ ? }
\end{align*}
Clearly, this cannot lead us to the goal: the \addBoxName\ rule cannot promote
terms using linear assumptions. Therefore, \textit{push} and other types which
exhibit this distributive behaviour are not derivable in our calculi.

In the following chapter, we present an alternative approach to generating
programs which exhibit this distributive behaviour using a generic programming
methodology. The approach we present in Chapter~\ref{chapter:deriving} is not
type-directed program synthesis, per se. This approach complements the calculi
presented here and in Chapter~\ref{chapter:extended}, providing users with a
means to automatically generate programs purely from a type for a common class
of graded programs. In describing this mechanism, we also begin to enhance our
language with more advanced features such pattern matching, and recursion,
further laying the foundations for Chapter~\ref{chapter:extended}.