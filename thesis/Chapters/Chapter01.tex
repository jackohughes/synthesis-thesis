\chapter{Introduction}
\label{chapter:intro}
Writing programs is difficult. Ensuring they are correct even more so. A
long-standing constant among programmers is the desire to write correct programs
with greater ease. To fulfil this end came \emph{program synthesis}, where the
computer shares the burden of writing the program with the human by writing part
or all of the program for them. The essence of program synthesis is the
automatic generation of program code from some specification of desired program
behaviour. Program synthesis techniques take many forms and have been applied to
almost all paradigms of programming, from early example driven approaches such
as \citet{deadalus}'s \textsc{Deadalus} and \citet{10.1145/321992.322002}'s \textsc{Thesys} systems, to
machine learning approaches such as \textit{inductive logic
programming}~\citep{MUGGLETON1994629}, to synthesising transformations between
spreadsheet columns in Excel~\citep{flashfill}. However, a general observation
one can make is that the task of generating a program automatically becomes
easier the richer the specification provided by the programmer. 

One of the most useful and well-studied verification and specification tools
available to modern programmers is the type system. Not only do type systems
allow many kinds of errors to be caught statically, they also help inform the
design of a program. Many programmers begin writing their programs by first
defining the types, from which the program code follows naturally. This
phenomenon will be familiar to any who have written programs in typed functional
programming languages, and results from the fact that types form a high-level
abstract specification of program behaviour. 

\emph{Type-directed program synthesis} is a well-studied technique for automatically
generating program code from a given type specification - the \textit{goal}
type. This approach has a long history, which is deeply intertwined with
automated theorem proving and proof synthesis, thanks to the Curry-Howard
correspondence~\citep{manna1980deductive,10.5555/1624562.1624585}. 

One lens through which we can view this task is as an inversion of type
checking: we start with a goal type and inductively synthesise well-typed
sub-terms by breaking the goal into sub-goals, pruning the search space of
programs via typing as we go. This approach follows the treatment of program
synthesis as a form of proof search in logic: given a type $[[ A ]]$ we want to
find a program term $[[ t ]]$ which inhabits $[[ A ]]$. We can express this in
terms of a synthesis \textit{judgement} which acts as the inversion of
typing or proof rules:
% of Gentzen's sequent calculus~\cite{Gentzen1935UntersuchungenD}:
%
\begin{align*}
  \Gamma \vdash [[ A ]] \Rightarrow [[ t ]]
\end{align*}
%
meaning that the term $[[ t ]]$ can be synthesised for the goal type $[[ A ]]$
under a context of assumptions $[[ G ]]$. We may construct a calculus of
synthesis \textit{rules} for a programming language, inductively defining the
above synthesis judgement for each type former. For example, we may define a
rule for standard product types in the following way:
\begin{align*}
  \inferrule*[Right=$\times_{\textsc{Intro}}$]{ \Gamma \vdash [[ A ]] \Rightarrow [[ t1 ]] \\ \Gamma \vdash [[ B ]] \Rightarrow [[ t2 ]]}{ \Gamma \vdash [[ A ]] \times [[ B ]] \Rightarrow ([[ t1 ]], [[t2]] )}
\end{align*}
%
Reading `clockwise' from the bottom-left: to synthesise a value of type $A
\times B$, we synthesise a value of type $A$ and then a value of type $B$ and
combine them into a pair in the conclusion. The `ingredients' for synthesising
the sub-terms $[[ t1 ]]$ and $[[ t2 ]]$ come from the free-variable assumptions
contained in $[[ G ]]$ and any constructors of $A$ and $B$.

Depending on the context, there could be many possible combinations of
assumption choices to synthesise such a pair. Consider the following partial
program containing a program \emph{hole}, marked with $?$, specifying a position
at which we wish to perform synthesis:
\begin{align*}
\begin{array}{ll}
    f : A \rightarrow A \rightarrow A \rightarrow A \times A &
    \\
    f\ x\ y\ z =\ ? &
\end{array}
\end{align*}
The function has three parameters all of type $A$ which can be used to
synthesise an expression of the goal type $A \times A$. Expressing this
synthesis problem as an instantiation of the above $\times_{\textsc{Intro}}$
rule:
%
\begin{align*}
  \inferrule*[Right=$\times_{\textsc{Intro}}$]
  { [[ x : A , y : A , z : A ]] \vdash [[ A ]] \Rightarrow [[ t1 ]]
 \\ [[ x : A, y : A, z : A ]] \vdash [[ A ]] \Rightarrow [[ t2 ]]}
  { [[ x : A, y : A, z : A ]] \vdash [[ A ]] \times [[ A ]] \Rightarrow ([[ t1 ]] , [[ t2 ]]) }
\end{align*}
%
Even in this simple setting, the number of possibilities starts to become
unwieldy. For example, we could synthesise $(x, x)$, $(x, y)$, $(y, x)$ etc. In
total, there are nine ($3^2$) possible candidate programs based on combinations
of $x$, $y$ and $z$. Ideally, we would like some way of constraining the number
of choices that are required by the synthesis algorithm. Many systems achieve
this by allowing the user to specify additional information about their desired
program behaviour. For example, recent work has extended type-directed synthesis
to refinement types~\citep{10.1145/2908080.2908093}, cost
specifications~\citep{DBLP:journals/corr/abs-1904-07415}, differential
privacy~\citep{diffprivacysynthesis}, example-guided
synthesis~\citep{10.1145/2813885.2737977,DBLP:conf/cav/AlbarghouthiGK13} or
examples integrated with types~\citep{frankle2016example,oseraMYTH1}, and
ownership information~\citep{fiala2023leveraging}. The general idea is that,
with more information, whether that be richer types, additional examples, or
behavioural specifications, the proof search / program synthesis process can be
pruned and refined, requiring less exploration of the search space of candidate
programs, ideally reducing the overall synthesis time.

This work presents a program synthesis approach that leverages the information
contained in \emph{linear} and \emph{graded type systems} that track and enforce
program properties related to data flow, statically. We refer to these systems
as \emph{resourceful} type systems, since they treat data as though it is a
physical resource, constraining how data can be used by a program and thus
reducing the number of possible synthesis choices that need to be made. Our
hypothesis is that resource-and-type-directed synthesis speeds up type-directed
synthesis, reducing the number of paths that need to be explored and the amount
of additional specification (e.g. input-output examples) required.

\subsection{Graded Type Systems}

Graded type systems trace their roots to linear logic. In linear logic, data is
treated as though it were a finite resource which must be consumed exactly once
with arbitrary copying and discarding disallowed~\citep{girard1987linear}. Since
Girard's original formulation of linear logic, several works have rendered
linear logic as a type system~\citep{DBLP:conf/ifip2/Wadler90,ABRAMSKY19933},
yielding \emph{linear types}. The identify function $\lambda x . x$ is the ideal
linearly typed program: it binds a variable and then uses it exactly once. The
$K$ combinator $\lambda x . \lambda y . x$ from SKI combinatory logic, however,
would not be linearly typed as the second variable $y$ is never used inside the
body. Non-linear use of data is expressed through the $!$ modal operator (the
\emph{exponential modality}). This gives a binary view---a value may either be
used exactly once (i.e. as a resource) or in a completely unconstrained way. For
example, using $!$ the $K$ combinator can be written as $\lambda x. \lambda !y .
x$, where $!$ provides the capability to discard $y$. Bounded Linear Logic (BLL)
refines this view, replacing ! with a family of indexed modal operators where
the index provides an upper bound on usage~\citep{girard1992bounded}, e.g.,
$!_{\leq 4}A$ represents a value $A$ which may be used up to 4 times. In recent
years, various works have generalised BLL, resulting in \textit{graded} type
systems in which these indices are drawn from an arbitrary pre-ordered
semiring~\citep{DBLP:conf/esop/BrunelGMZ14,DBLP:conf/esop/GhicaS14,petricek2014coeffects,DBLP:journals/pacmpl/AbelB20,DBLP:journals/pacmpl/ChoudhuryEEW21,quantitative-type-theory,McBride2016}.
This allows numerous program properties to be tracked and enforced statically,
including various kinds of reuse, privacy and confidentiality, and capabilities.
Graded types are becoming applied and adopted in practical systems and form the
basis of Haskell's \texttt{LinearTypes}
extension~\citep{DBLP:journals/pacmpl/BernardyBNJS18}, Idris
2~\citep{DBLP:journals/corr/abs-2104-00480}, as well as the experimental
language Granule~\citep{DBLP:journals/pacmpl/OrchardLE19}.

Semantically, these semiring indexed $!$-modalities are modelled by \emph{graded
exponential comonads}~\citep{DBLP:conf/icfp/GaboardiKOBU16}. The terminology of
\textit{graded modal types} was proposed
by~\citet{DBLP:journals/pacmpl/OrchardLE19} encompassing both semiring indexed
!-modalities and the notion of \textit{graded
monads}~\citep{DBLP:journals/corr/OrchardPM14,DBLP:conf/popl/Katsumata14,
smirnov2008graded}, which generalise monads. In this work we do not consider the
latter, dealing only with graded comonadic modalities, which are closely related to
linearity. In general, we will refer to systems which make use of graded modal
types as \emph{graded} type systems. 

Returning to our example in a graded setting, the arguments of the function now
have \emph{grades} (annotations) that, in this context, are natural numbers
describing the exact number of times the parameters must be used (the choice
here was ours):
%
\begin{align*}
  \begin{array}{ll}
    f : A^{2} \rightarrow A^{0} \rightarrow A^{0} \rightarrow A \times A &
    \\
    f\ x\ y\ z =\ ? &
  \end{array}
\end{align*}
The first $A$ is annotated with a grade $2$, which in this context indicates that it
\textit{must} be used twice. Likewise, the types of $y$ and $z$
are graded with $0$, enforcing zero usage, i.e., we are not allowed
to use them in the body of $f$ and must discard them.

The result is that there is only one (normal form) inhabitant for this type: $[[
(x, x) ]]$; the other assumptions will not even be considered in synthesis,
allowing us to effectively prune out branches which use resources in a way which
violates the grades. In this example, these annotations take the form of natural
numbers explaining how many times a value can be used, but we may instead wish
to represent different kinds of program properties, such as sensitivity,
strictness, or security levels for tracking non-interference, all of which are
well-known instances of graded type
systems~\citep{DBLP:journals/pacmpl/OrchardLE19,DBLP:conf/icfp/GaboardiKOBU16,DBLP:journals/pacmpl/AbelB20}.
Note that all of these examples are technically graded presentations of
\emph{coeffects}, tracking how a programs uses its context, in contrast with
graded types for side
\emph{effects}~\citep{DBLP:journals/corr/OrchardPM14,DBLP:conf/popl/Katsumata14},
which we do not consider here.

We can divide these graded coeffect systems into two main classes, which we
refer to as ``linear graded'' (or \emph{linear base}) and ``fully graded'' (or
\emph{graded base}) throughout. The first are those which trace their origins to
linear logic, refining an existing non-graded type system with graded modal
types. The functional programming language Granule is one such system. Granule
is based on an underlying linear type system, with graded modal types introduced
and eliminated explicitly, through a family of indexed modal operators $\Box_r$.
For instance, $\Box_2 A \multimap B$ represents the type of a linear function
where the argument type is a graded modal type, graded by $2$. By ``unboxing''
the modality, we obtain a value which can be used twice inside the body of the
function. This system is the default basis of
Granule~\citep{DBLP:journals/pacmpl/OrchardLE19}. The second approach does away
with this linear basis, embedding graded modalities into the function types à la
Idris 2~\citep{DBLP:journals/corr/abs-2104-00480}, McBride's
QTT~\citep{McBride2016,quantitative-type-theory}, the core of Linear
Haskell~\citep{DBLP:journals/pacmpl/BernardyBNJS18}, and the unified graded
modal calculus of \citep{DBLP:journals/pacmpl/AbelB20}. In these systems, grades
are expressed as annotations function arrows, i.e. $A^2 \rightarrow B$ expresses
a similar idea as before, where $A$ is an argument type graded by $2$. However,
in these systems, we do not need to do any unboxing to be able to use the value
of type $A$ according to its grade in the body of the function.

As graded type systems develop into fully fledged programming languages such as
Idris 2~\citep{DBLP:journals/corr/abs-2104-00480}, and as traditionally non
graded languages, such as Haskell, incorporate features from graded types into
their existing type systems~\citep{DBLP:journals/pacmpl/BernardyBNJS18}, the
proposition of harnessing the information conveyed by these richer types becomes
increasingly attractive. As of this moment, program synthesis in the context of
graded types has remained relatively unexplored despite the growing prevalence
of such systems. Past works have tackled the problem of proof search in Linear
Logic~\citep{HODAS1994327, CERVESATO2000133}, which are analogous, through the
lens of Curry-Howard, to program synthesis solutions for linear types. These
works lay the foundations for our approach here, however, graded types
pose an entirely new set of challenges and integrating them into these existing
approaches is complex.

For this work, we consider Granule~\citep{DBLP:journals/pacmpl/OrchardLE19} to
be an ideal candidate for the target language of a program synthesis tool for
graded type systems. As mentioned, Granule is a functional programming language
which combines a linear type system at its core with indexed data types. On top
of this, graded modalities are integrated both as graded comonads and graded
monads. We forgo the treatment of Granule's indexed types in synthesis, leaving
this as future work. Instead we focus on building a synthesis tool for Granule,
which unlocks the expressivity held by graded comonads for use inside the
synthesis algorithm itself.

Granule also provides a language extension, known as \emph{GradedBase}, which
does away with this linear type system as its foundation. Instead, grades
permeate the type system à la~\citet{petricek2014coeffects} and other graded
base systems. This extension provides us with a language capable of giving a
broad representation of graded type systems. This will factor heavily into the
design of our calculi, and we structure the rest of the thesis with this in
mind.

\section{Contributions}
Linear and graded types provide a rich specification of program behaviour, which
is enforced statically by the type checker. But how do we harness this
information to make    
writing programs automatically easier?  
 
The primary aim of this work is to demonstrate the feasibility and power of
using resourceful types as the basis of a type-directed program synthesis tool,
culminating in the development and implementation of an expressive, efficient,
and feature-rich program synthesis tool for the Granule programming language. 

We show that not only is program synthesis feasible in the context
of a graded type system, but that the information conveyed by the grades can be
used to prune the search space of programs in synthesis. This results in our
synthesis algorithm needing to consider far less potential programs when
constructing a program, as grade-violating candidate programs may be pruned out. 

Specifically, this work makes the following contributions:
\begin{itemize}
  \item We identify an approach which makes type-directed program synthesis in
        a resourceful setting feasible. Drawing inspiration from the work of
        ~\citet{HODAS1994327} on theorem proving, we adapt their
        work to graded types, and propose a dual to their own method, yielding two
        schemes for managing resources in the
        synthesis of a program term. 

  \item We use these schemes to construct two simple synthesis calculi for a
        simplified core of Granule, which demonstrate their effectiveness as
        tools for resourceful program synthesis. We implement both approaches as 
        part of the Granule toolchain and compare their performance.

  \item We showcase an alternative and complementary approach to generating a
        subset of Granule programs, based around useful combinators for working
        in this setting. To do so, we make use of a system inspired by Haskell's
        generic type deriving mechanism~\citep{generic-deriving} adapted to
        graded types.

  \item We then define a synthesis calculus for a fully graded type system. This
        type system is a feature-rich language based on Granule's \emph{graded
        base} language extension, which includes recursion, recursive types, and
        user-defined ADTs. Furthermore, we again implement this calculus as part
        of the Granule toolchain.

  \item We evaluate our tool on a benchmark suite of recursive functional
        programs leveraging standard data types like lists, streams, and trees.
        We compare against non-graded synthesis provided by
        \textsc{Myth}~\citep{oseraMYTH1}. We also compare against our own tool
        \emph{modulo grades}, i.e. we compare the number of synthesis paths
        explored by our tool when taking grades into account versus traditional
        un-graded synthesis.

  \item We prove that each of these systems is sound, i.e., synthesised programs
        are typed by the goal type. A ket property here is that synthesised programs 
        are not well-typed, but also \emph{well-resourced}, meaning that all values  
        inside the program are used according to their resource constraints. We
        show that this property holds for each of our synthesis calculi as part
        of their soundness proofs. 

  \item We demonstrate how our approach to resourceful program synthesis can be
        readily applied to other graded systems. Leveraging our calculus and
        implementation, we provide a prototype tool for synthesising Haskell
        programs written using GHC 9's Linear Types language extension.
\end{itemize}
\section{Structure}

This dissertation is structured into six chapters. In the next chapter,
Chapter~\ref{chapter:background}, the theoretical background of linear and
graded types is laid out. In doing so, we introduce two core calculi with simple
types and grades, which demonstrate the two major lineages of resourceful type
systems; linear base and graded base.

The rest of the dissertation is structured such that synthesis calculi for both
of these systems are defined and presented, minimising any redundancy in the
presentation. Despite the variances between the core calculi
of Chapter~\ref{chapter:background}, there is a substantial degree of overlap between
the two. Thus, we adopt the following structure: 
\begin{enumerate}
        \item Chapter~\ref{chapter:core} introduces the core concepts of
        type-directed program synthesis from resourceful type systems using an
        extension of the typing calculus of Section~\ref{sec:linear-base}.
        Specifically this chapter introduces the \textit{resource management
        problem} as it relates to program synthesis: how do we ensure that a
        synthesised program is not only well-typed but also well-resourced? To
        address this question, we define two calculi of synthesis rules based on
        a linear $\lambda$-calculus extended with graded modal types which
        tackle the problem in different ways. To better illustrate and test the
        practicality of the synthesis calculi, we extend the language with
        multiplicative conjunction (product types $\otimes$ and unit $\mathsf{Unit}$) and
        additive disjunction (sum types $\otimes$). These calculi are then
        implemented targeting default Granule. We produce a comparative
        evaluation of the implemented synthesis tool, contrasting the efficiency
        of the two resource management approaches against each other, before
        selecting the most performant to use going forward.  

        \item We then consider an alternative approach to type-directed synthesis,
        exploring a mechanism for automatically deriving programs from their
        type à la Haskell's generic deriving mechanism~\citep{generic-deriving}.
        Again for this we base the approach on the graded linear
        $\lambda$-calculus, extending it further with data constructors, pattern
        matching, and recursive data types.
        \item Finally, in chapter~\ref{chapter:extended}, we present a synthesis
        calculus for a target language based instead on the core graded
        $\lambda$-calculus of~\ref{sec:graded-base}. This calculus incorporates
        all of the language features that have been introduced in the previous chapters,
        for a rich synthesis tool implementation targeting Granule's
        \emph{graded base} language extension. Furthermore, we outline several other useful 
        extensions to the synthesis tool, such as the inclusion of example-based synthesis,
        and a post-synthesis refactoring process which re-writes synthesised programs in 
        a more idiomatic style.

        We then evaluate the implementation on a set of benchmarks, including
        several non-trivial programs which make use of these new features. In
        this evaluation, we compare synthesis in a setting which uses grades to
        prune the search space, to un-graded purely type-directed synthesis.
        From our evaluation we find that using grades in synthesis outperforms
        purely type-driven program synthesis in terms of number of paths
        explored in the search space, number of input-output examples required
        or number of retries to get the desired program.
        
        Finally, to demonstrate the practicality and versatility of our
        approach, we apply our synthesis algorithm of
        Chapter~\ref{chapter:extended} to synthesising programs in Haskell from
        type signatures using GHC's \emph{linear types} extension (which is
        implemented underneath by a graded type system).

\end{enumerate}

Table~\ref{table:chapter-calculi} summarises the structure of the thesis by chapter, with respect 
to the variants of resourceful calculi we consider.
\begin{table}[H]
\begin{align*}
\setlength{\arraycolsep}{0.2em}
\begin{array}{r|c|c}
      & \text{ Linear} + \text{Graded } & \text{ Graded}\\
      \hline
      \text{Chapter~\ref{chapter:background}} & \multimap,\ \Box_r & \rightarrow \\
      \text{Chapter~\ref{chapter:core}} & \multimap,\ \Box_r,\ \otimes,\ \oplus,\ \mathsf{Unit} &  \\
      \text{Chapter~\ref{chapter:deriving}} & \multimap,\ \Box_r,\ \text{ADTs} & \\
      \text{Chapter~\ref{chapter:extended}} & & \rightarrow,\ \text{ADTs},\ \text{Recursion}
\end{array}
\end{align*}
\caption{Typing calculi per chapter}
\label{table:chapter-calculi}
\end{table}

This approach strikes a balance between maximising coverage of different
approaches to resourceful type systems, and avoiding unnecessary repetition,
whilst gradually increasing the complexity of the target language. By the end, we
will then have two synthesis tool implementations for Granule, targeting both
styles of graded type systems.

\section{Publications}
In this thesis, the content of some chapters is formed from
previously published papers:
\begin{itemize}
    \item \citet{DBLP:conf/lopstr/HughesO20} Resourceful Program Synthesis from
    Graded Linear Types. In Logic-Based Program Synthesis and Transformation -
    30th International Symposium, pp. 151-170. 
    
    This paper builds on the resource management techniques of previous work for
    Linear Logic proof search to graded types. It presents two synthesis
    calculi based on these approaches, and evaluates them against each other 
    using a suite of benchmarks of Granule programs. 
    
    This paper constitutes a large part of Chapter~\ref{chapter:core}, where a
    program synthesis tool for a linear graded type system is introduced. 

    \item \citet{DBLP:journals/corr/abs-2112-14966} Deriving Distributive Laws
    for Graded Linear Types. In 6th edition of the International Workshop on
    Linearity and of the 4th edition of the International Workshop on Trends in
    Linear Logic and its Applications. 

    The majority of Chapter~\ref{chapter:deriving} is derived from this paper,
    which discusses an approach for automatically deriving certain classes of
    graded programs in Granule as an alternative to search-based synthesis. The
    focus is on deriving distributive laws of graded programs in Granule, but
    also considers deriving other useful structural combinators. The core
    contribution of this work is a tool in the vein of Haskell's Generic Type
    Deriving mechanism~\citep{generic-deriving}, which is implemented as part of
    the Granule compiler.

    \item \citet{hughes:lirmm-03271465} Linear Exponentials as Graded Modal
    Types. In In 5th International Workshop on Trends in Linear Logic and
    Applications.  

    In Chapter~\ref{chapter:core} and subsequently, we make reference to
    the disparity between Granule's $\Box$ modality and Linear Logic's !, and
    how ! can be recovered in Granule through the use of a grade-level operator
    defined in this paper.  
\end{itemize}
