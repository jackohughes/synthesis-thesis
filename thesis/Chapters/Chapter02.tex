\chapter{Background}
\label{chapter:background}
Since Girard's original work on Linear Logic in 1987~\cite{girard1987linear}, the development of type
systems which convey additional information about the program's structure has
evolved into a distinct paradigm, culminating in recent years with the notion of
\textit{graded types}. Approaches to graded type systems run the gamut,
incorporating a wide range of effect and coeffect systems, however, they 
can typically be distilled into two categories, with distinct lineages: 
\begin{itemize}
  \item Systems where a graded modal type operator introduces and eliminates
        graded modalities above some existing type system. This is the default
        approach of Granule, where the underlying type system is linear,
        and grade modalities are introduced and eliminated via the $\Box$ modal
        type operator.
  \item Systems where grades permeate the program, and are introduced via annotations 
        on function arrows. This is the approach taken by Linear Haskell~\cite{DBLP:journals/pacmpl/BernardyBNJS18}, where 
        grades (or ``multiplicities'') are specified using the \granin{\%} operator. 
\end{itemize}  
These two different styles to graded types mirror the dual development of 
effect systems and graded monadic systems in the literature. In the latter case, 
the two were eventually found to be equivalent. 

% The same treatment for graded systems is ongoing 

% In the first, modern resourceful type systems trace their roots to Girard's
% Linear Logic~\cite{} which was one of the first treatments of data as a resource
% inside a program. Bounded Linear Logic (BLL) developed this idea further,
% refining the coarse grained view of data as either linear or non linear. Several
% subsequent works generalised BLL, coalescing into the notion of a
% \textit{graded} type system in languages such as Granule~\cite{}. This lineage
% treats these systems essentially as refinements of an underlying linear
% structure.



% \begin{table}[htbp]
% \centering
% \caption{Timeline.}
% \label{tbl:1}
% \begin{tabular}{r l l}
%   Year & \ \ Linear & \ \ Graded \\
%   \toprule
%   1986 & & \textasteriskcentered\ D.K. Gifford, J.M. Lucassen \\
%        & & \ \ \textit{Integrating Functional and} \\
%        & & \ \ \textit{Imperative Programming}  \\

%   1987 & \textasteriskcentered\ J.Y. Girard \\
%        & \ \ \textit{Linear Logic} & \\

%   1990 & & \textasteriskcentered\ E. Moggi \\
%        & & \ \ \textit{Notions of Computation and Monads} \\

%   1991 & \textasteriskcentered\ J.Y. Girard et al. & \\
%        & \ \ \textit{Bounded Linear Logic} & \\

%   1994 & \textasteriskcentered\ J.S. Hodas, D. Miller &  \\
%        & \ \ \textit{Logic Programming in a Fragment} & \\
%        & \ \ \textit{of Intuitionistic Linear Logic} & \\

%   2000 & & \textasteriskcentered\ P. Wadler, P. Thiemann \\
%        & & \ \ \textit{Marriage of effects and monads} \\
%   2011 & \textasteriskcentered\ U.D. Lago, M. Gaboardi &  \\
%        & \ \ \textit{Linear Dependent Types} & \\
%        & \ \ \textit{and Relative Completeness} & \\
%   2013 & & \textasteriskcentered\ T. Petricek et al. \\
%        & & \ \ \textit{Coeffects: Unified Static} \\
%        & & \ \ \textit{Analysis of Context-Dependence} \\
%   2014 & \textasteriskcentered\ D.R. Ghica, A. I. Smith & \textasteriskcentered\
%   T. Petricek et al. \\
%        & \ \ \textit{Bounded Linear Types in} & \ \ \textit{Coeffects: a
%                                                 Calculus of} \\
%        & \ \ \textit{a Resource Semiring}  & \ \ \textit{Context-Dependent Computation} \\
%        & \textasteriskcentered\ A. Brunel et al. & \\
%        & \ \ \textit{A Core Quantative Coeffect Calculus} & \\
%   2016 & \textasteriskcentered\ M. Gaboardi et al. & \textasteriskcentered\ C. McBride \\
%        & \ \ \textit{Combining Effects and} & \ \ \textit{I Got Plenty o' Nuttin'} \\
%        & \ \ \textit{Coeffects via Grading} & \\
%   2017 & & \textasteriskcentered\ J.P. Bernardy et al. \\
%        & & \ \ \textit{Linear Haskell} \\
%   2018 & & \textasteriskcentered\ R. Atkey \\
%        & & \ \ \textit{Syntax and Semantics of} \\
%        & & \ \ \textit{Quantative Type Theory} \\
%   2019 & \textasteriskcentered\ Orchard et al. & \\
%        & \ \ \textit{Quantitative Program Reasoning} & \\
%        & \ \ \textit{with Graded Modal Types} & \\
%   % 2020 & LOPSTR &  \\
%   % 2021 & & \textasteriskcentered\ A. Abel, J.P. Bernardy \\
%   %      & & \ \ \textit{A Unified View of Modalities} \\
%   %      & & \ \ \textit{in Type Systems} \\
%   %      & & \textasteriskcentered\ P. Choudhury et al. \\
%   %      & & \ \ \textit{A Graded Dependent Type System} \\
%   %      & & \ \ \textit{with a Usage-Aware Semantics} \\
%   %      & & \textasteriskcentered\ B. Moon et al. \\
%   %      & & \ \ \textit{Graded Modal Dependent} \\
%   %      & & \ \ \textit{Type Theory} \\
%   %      & & \textasteriskcentered\ E. Brady \\
%   %      & & \ \ \textit{Idris 2: Quantative Type} \\
%   %      & & \ \ \textit{Theory in Practice} \\
%   %
%   \bottomrule
% \end{tabular}
% \end{table}


\section{Terminology}
Before delving into linear and graded systems, we briefly frame the approach we
will take to discussing the relevant background material. Throughout we will
tend towards using a \textit{types-and-programs} terminology rather than
\textit{propositions-and-proofs}. Through the lens of the Curry-Howard
correspondence, one can switch smoothly to viewing our approach to program
synthesis as proof search in logic.

The functional programming languages we discuss are presented as typed
calculi given by sets of \textit{types}, \textit{terms} (programs), and \textit{typing
 rules} that relate a term to its type. The most well-known typed
calculus is the simply-typed $\lambda$-calculus, which corresponds to
intuitionistic logic.

A \textit{judgment} defines the typing relation between a type and a term based on a
\textit{context}. In the simple typed $\lambda$-calculus, judgments have the
form: $\Gamma \vdash t : A$, stating that under some context of
\textit{assumptions} $\Gamma$ the program term $t$ can be assigned the type $A$.
An assumption is a name with an associated type, written $x : A$ and
corresponds to an in-scope variable in a program.

A term can be related to a type if we can derive a valid judgment through the
application of typing rules. The application of these rules forms a tree
structure known as a \textit{typing derivation}.


\section{Linear and substructural logics}
Linear logic~\cite{} was introduced by Girard as a way of being more descriptive
about the properties of a derivation in intuitionistic logic. In type systems
such as the simply typed $\lambda$-calculus, the properties of
\textit{weakening}, \textit{contraction}, and \textit{exchange} are assumed
implicitly. These are typing rules which are \textit{structural} as they
determine how the context may be used rather than being directed by the syntax.
Weakening is a rule which allows terms that are not needed in a typing
derivation to be discarded. Contraction works as a dual to weakening, allowing
an assumption in the context to be used more than once. Finally, exchange allows
assumptions in a context to arbitrarily re-ordered.


\begin{figure}[H]
  \begin{align*}
  \begin{array}{c}
    \inferrule*[right=Weakening]{\Gamma \vdash t : B}{\Gamma, x : A \vdash t : B}
    \;\;\;\;
    \;\;\;\;
    \;\;\;\;
    \inferrule*[right=Contraction]{\Gamma, x : A, y : A \vdash t : B}{\Gamma, x : A \vdash t : B}
    \\\\
    \inferrule*[right=Exchange]{\Gamma_{1}, y : B, x : A, \Gamma_{2} \vdash t : C} {\Gamma_{1}, x : A, y : B, \Gamma_{2} \vdash t : C}
    \end{array}
  \end{align*}
  \caption{Substructural rules for weakening, contraction, and exchange}
\end{figure}

Linear logic is known as a \textit{substructural} logic because it lacks the weakening
and contraction rules, while permitting exchange.

The disallowance of these rules means that in order to construct of a typing
derivation, each assumption must be used exactly once --- arbitrarily copying or
discarding values is disallowed, excluding a vast number of programs from being
typeable in linear logic. Non-restricted usage of a value is recovered through
the modal operator $!$ (also called ``bang'', ``of-course'', or the
\textit{exponential} modality). Affixing $!$ to a type captures the notion that values of
that type may be used freely in a program.

providing a binary view of data as a
resource inside a program: values are either linear or completely unrestricted.

Bounded Linear Logic, took this idea further --- instead of a single modal
operator, $!$ is replaced with a family of modal operators indexed by terms
which provide an upper bound on usage~\cite{}. These terms provide an upper bound on
the usage of a values inside term, e.g. $!_{3}A$ is the type of $A$ values which
may be used up to 3 times.




\section{The Graded Linear $\lambda$-calculus}
\label{sec:linear-base}
We now define a core type system, based on the linear
$\lambda$-calculus extended with a graded modal type. 
This calculus is equivalent to the core calculus of Granule,
\textsc{GrMini}\cite{}. Granule's full type system extends this
graded linear core with polymorphism, algebraic data types (ADTs), indexed types, pattern
matching. We refer to this system as the \textit{graded linear $\lambda$-calculus}, reflecting the underlying
linear structure of the system.

This system forms the basis of the target language for our synthesis tool in
chapter~\ref{chapter:core}. 

The types of the graded linear $\lambda$-calculus are defined as:
\begin{align*}
\hspace{-0.9em}[[ A ]] , [[ B ]] & ::=
       [[ A ]] \multimap [[ B ]]
  \mid [[ [] r A ]]
{\small{\tag{types}}}
\end{align*}
where the type $\Box_{r} A$ is an indexed family of type operators where $r$ is a
\textit{grade} ranging over the elements of a pre-ordered semiring
$({\mathcal{R}}, {*}, {1}, {+}, {0},
{\sqsubseteq})$ parameterising the calculus (where $\ast$ and $+$
are monotonic with respect to the pre-order $\sqsubseteq$). Linear functions 
are specified by the $\multimap$ arrow\footnote{Although Granule uses $\rightarrow$ syntax rather than
$\multimap$ for linear functions for the sake of familiarity with
standard functional languages}.

The syntax of terms provides the elimination and introduction forms:
\begin{align*}
\hspace{-0.8em} [[ t ]] ::= \;
       & [[ x ]]
  \mid [[ \x . t ]]
  \mid [[ t1 t2 ]]
  \mid [[ [t] ]]
  \mid [[ let [ x ] = t1 in t2 ]]
{\small{\tag{terms}}}
\end{align*}
In addition the the terms of the linear $\lambda$-calculus, we also have the
construct $[[ [t] ]]$ which introduces a graded modal type $[[ [] r A
]]$ by `promoting' a term $t$ to the graded modality, and it's dual $[[ let [x] = t1
in t2 ]]$ eliminates a graded modal value $[[ t1 ]]$, binding a graded variable $x$
in scope of $[[ t2 ]]$. The typing rules provide further understanding of the
behaviour of these terms.

Typing judgments are of the form $[[ G |- t : A ]]$, where $[[ G ]]$ ranges over contexts:
\begin{equation*}
  [[ G ]] ::= \emptyset
  \mid [[ G , x : A ]]
  \mid [[ G , x : [ A ] r ]]
\tag{contexts}
\end{equation*}

Thus, a context may be empty $\emptyset$, extended with a
linear assumption $[[ x : A ]]$ or extended with a graded assumption $[[x : [A]
r]]$. For linear assumptions, structural rules of weakening
and contraction are disallowed. Graded assumptions may be used
non-linearly according to the constraints given by their grade, the
semiring element $r$. Throughout, comma denotes disjoint context
concatenation.

Various operations on contexts are used to capture non-linear data flow
via grading. Firstly, \emph{context addition}~\eqref{def:contextAdd} provides an
analogue to contraction, combining contexts that have come from
typing multiple subterms in a rule.
Context addition, written $[[ G1 + G2]]$, is undefined if $[[ G1 ]]$ and $[[ G2 ]]$
overlap in their linear assumptions. Otherwise graded assumptions appearing
in both contexts are combined via the semiring $+$ of their grades.

\begin{definition}[Context addition]\label{def:contextAdd}

\begin{align*}
    \setlength{\arraycolsep}{0.1em}
    \begin{array}{cc}
    \begin{array}{rl}
    [[ (G, x : A) + G' ]] & = [[ (G + G'), x : A ]] \quad \text{iff} \,\; x
    \not\in | [[ G' ]] | \\
    [[ G + (G', x : A) ]] & = [[ (G + G'), x : A ]] \quad \text{iff} \,\; x \not\in | [[ G]] | \\
    [[ (G, x : [ A ] r) + (G', x : [ A ] s) ]] & = [[ (G + G'), x : [ A ] (r + s) ]]
    \end{array}
      \;\;\quad & \quad\;\;
    \begin{array}{rl}
      [[ . + G ]] & = [[ G ]] \\
      [[ G + . ]] & = [[ G ]] \\ \\
    \end{array}
  \end{array}
  \end{align*}

Note that this is a declarative specification of context addition. Graded
assumptions may appear in any position in $\Gamma$ and $\Gamma'$ as witnessed by
the algorithmic specification where for all $[[ G1 ]], [[ G2 ]]$
  \emph{context addition} is defined
as follows by ordered cases matching inductively on the structure of
$[[ G2 ]]$:
\begin{align*}
[[G1 + G2]] = \left\{\begin{matrix}
    \begin{array}{ll}
    [[G1]] &
     [[G2]] = \emptyset
             \\
      (([[G1']], [[G1'']]) + [[G2']]), [[x : [A] (r + s)]] \; &
[[ G2]] = [[G2', x : [A] s]] \wedge [[G1]] = [[ G1',x : [A] r]],[[G1'']]  \\
 ([[G1]] + [[G2']]), [[x : A]] & [[G2]] = [[G2', x : A]]\ \wedge\  [[x
                                  : A]] \notin [[ G1 ]]
    \end{array}
  \end{matrix}\right.
\end{align*}
\end{definition}

\begin{figure}[H]
\hspace{-0.5em}
\begin{align*}
\hspace{-0.5em}
  \begin{array}{c}
  \inferrule*[right = Var]
  {\;}
  {[[ x : A |- x : A ]]}
\;\;
  \inferrule*[right = Abs]
  {[[ G , x : A |- t : B ]]}
  {[[ G |- \x . t : A -> B ]]}
\\[0.75em]
  \inferrule*[right = App]
  {[[ G1 |- t1 : A -> B ]] \;\;\;
   [[ G2 |- t2 : A ]] }
  {[[ G1 + G2 |- t1 t2 : B ]]}
\\[0.75em]
 \inferrule*[right = Weak]
  {[[ G |- t : A ]]}
  {[[ G , {[ D , 0 ]} |- t : A ]]}
\;\;\;
\inferrule*[right = Der]
  {[[ G , x : A |- t : B ]]}
  {[[ G , x : [ A ] 1 |- t : B]]}
\\[0.75em]
\inferrule*[right = Approx]
{[[ {G, x : [A] r}, G' |- t : A ]] \quad r \sqsubseteq s }
{[[ {G, x : [A] s}, G' |- t : A ]]}
\\[0.75em]
\inferrule*[right = Pr]
  {[[ [ G ] |- t : A ]]}
  {[[ r * {[ G ]} |- [t] : [] r A ]]}
\;\;\;
\inferrule*[right = Let$\Box$]
  {[[ G1 |- t1 : [] r A ]] \;\;\;
   [[ G2, x : [ A ] r |- t2 : B ]] }
    {[[ G1 + G2 |- let [x] = t1 in t2 : B ]]}
\end{array}
\end{align*}
\vspace{-1.25em}
  \caption{Typing rules of the graded linear $\lambda$-calculus}
\label{fig:typing}
\vspace{-0.65em}
 \end{figure}


Figure~\ref{fig:typing} defines the typing rules.
Linear variables are typed in a singleton context
(\textsc{Var}). Abstraction (\textsc{Abs}) and application (\textsc{App})
follow the rules of the linear $\lambda$-calculus.
The $\textsc{Weak}$ rule captures
weakening of assumptions graded by $0$ (where $[[ [ D , 0 ] ]]$ denotes a context
containing only graded assumptions graded by $0$). Context addition and
\textsc{Weak} together therefore provide the rules of substructural rules of contraction
and weakening.
Dereliction ($\textsc{Der}$),
allows a linear assumption to be converted to a graded assumption with grade
$1$. Grade approximation is captured by the $\textsc{Approx}$
rule, which allows a grade $s$ to be converted to another grade $r$,
providing that $r$ \textit{approximates} $s$, where the relation
$\sqsubseteq$ is the pre-order provided
with the semiring.
Introduction and elimination of the graded modality is provided by the
$\textsc{Pr}$ and $\textsc{Let}$ rules
respectively. The $\textsc{Pr}$ rule propagates the grade $r$ to the
assumptions through \emph{scalar multiplication} of $[[G]]$ by $r$ where
every assumption in $[[ G ]]$ must already be graded (written $[[ [ G
] ]]$ in the rule), given by definition~\eqref{def:scalar}.
%
%
\begin{definition}[Scalar context multiplication]
  \label{def:scalar}
 A context which consists solely of graded assumptions can be multiplied by a
 semiring grade $r \in \mathcal{R}$
\begin{align*}
   [[ r * . ]] = \emptyset
    \qquad\qquad
    [[ r * (G , x : [ A ] s) ]] = [[ (r * G), x : [ A ] (r * s) ]]
\end{align*}
\end{definition}

The $\textsc{Let}$ rule eliminates a graded modal value $[[ [] r A ]]$
into a graded assumption $[[ x : [ A ] r ]]$ with a matching
grade in the scope of the \textbf{let} body. This is also referred to as
``unboxing''.

We give an example of graded modalities using a graded modality indexed
by the semiring of natural numbers.

%SKI
\begin{example}
\label{ex:s-comb}
  The natural number semiring with discrete ordering
  $(\mathbb{N}, \ast, 1, +, 0, \equiv)$ provides a graded modality
  that counts exactly how many times non-linear values are used. As a
  simple example, the \emph{S} combinator from the SKI system of combinatory logic is typed and defined:
% then2 : forall {a b c : Type} . (a -> (b -> c)) -> (a -> b) -> (a
% [2] -> c)
\begin{align*}
s & : [[ (A -> (B -> C)) -> {(A -> B) -> ({[] 2 A} -> C)} ]] \\
s & = [[ \x . {\y . {\z' . {let [ z ] = z' in {(x z) (y z)}}}} ]]
\end{align*}
The graded modal value $z'$ captures the `capability' for a value
of type $A$ to be used twice. This capability is made available by eliminating
$\Box$ (via \textbf{let}) to the variable $z$, which is
graded $z : [A]_2$ in the scope of the body.
\end{example}



\paragraph{Metatheory}
The admissibility of substitution is a key result that holds
for this language~\cite{DBLP:journals/pacmpl/OrchardLE19}, which is
leveraged in soundness of the synthesis calculi.
%
\begin{restatable}[Admissibility of substitution]{lemma}{linearSubst}
Let $[[ D |- t' : A]]$, then:
\label{lemma:substitution}
\begin{itemize}[leftmargin=1em]
\item (Linear) \hspace{0.04em} If $[[ {G, x : A}
    ,, { G' } |- t : B]]$ then $[[ G + D + G' |-
[ t' / x ] t : B ]]$
\item (Graded) If $[[ {G, x : [A] r} ,, { G' } |- t : B]]$
then $[[ G + (r * D) + G' |- [ t' / x ] t : B ]]$
\end{itemize}
\end{restatable}

\section{The Fully Graded $\lambda$-calculus}
\label{sec:graded-base}
We now define a core calculus for a fully graded type system, in the vein of
systems where grades permeate the entire program, drawing from the coeffect
calculus of \citet{petricek2014coeffects}, Quantitative Type Theory (QTT) by
\citet{McBride2016} and refined further by \citet{quantitative-type-theory}
(although we omit dependent types from our language), the calculus of
\citet{DBLP:journals/pacmpl/AbelB20}, and other graded dependent type
theories~\cite{quantitative-type-theory,DBLP:conf/esop/MoonEO21}. Similar
systems also form the basis of the core of the linear types extension to
Haskell~\cite{DBLP:journals/pacmpl/BernardyBNJS18}. We refer to this system as
the \textit{fully graded $\lambda$-calculus} to differentiate it from its linearly-based
counterpart.

The syntax of graded-base types is given by:
\begin{align*}
\hspace{-0.9em}[[ A ]] , [[ B ]] & ::=
       [[ A ^ r -> B ]]
  \mid [[ [] r A ]]
{\small{\tag{\textit{types}}}}
\end{align*}
where the function arrow $[[ A ^ r -> B ]]$ annotates the input type with a \emph{grade} $[[ r ]]$
which is again drawn from a pre-ordered semiring
$(\mathcal{R}, {\ast}, {1}, {+}, {0}, \sqsubseteq)$ paramterising the
calculus. The graded necessity modality $[[ [] r A ]]$ is similarly annotated by the grade
$[[ r ]]$ being an element of the semiring. 

The syntax of terms is given as:
%
\begin{align*}
\hspace{-0.8em} [[ t ]] ::= \;
       & [[ x ]]
  \mid [[ \x ^ c . t ]]
  \mid [[ t1 t2 ]]
  \mid [[ [t] ]]
  % \mid [[ case t of p1 -> t1 ; * ; pn -> tn  ]]
{\small{\tag{\textit{terms} }}}
\end{align*}
%
Terms comprise the $\lambda$-calculus, extended
with the \textit{promotion} construct [t] as seen in section~\ref{sec:linear-base}.
Typing judgements have the same form as section~\ref{sec:linear-base}, however, variable contexts 
are instead given by: 
\begin{equation*}
  [[ D ]], [[ G ]] ::= \emptyset
  \mid [[ G , x : [ A ] r ]]
\tag{\textit{contexts}}
\end{equation*}
That is, a context may be empty $\emptyset$ or extended with a \textit{graded}
assumption $ [[ x : [A] r ]]$, which must be used in a
way which adheres to the constraints of the grade $[[ r ]]$. As before, structural
exchange is permitted, allowing a context to be arbitrarily reordered. 


\begin{figure}[H]
\hspace{-0.5em}
\begin{align*}
\hspace{-0.5em}
\begin{array}{c}
\GRANULEdruleTyVar{}
\;\;\;
\GRANULEdruleTyAbs{}
\\[0.75em]
\GRANULEdruleTyApp{}
\\[0.75em]
\GRANULEdruleTyPr{}
\;\;\;
\GRANULEdruleTyApprox{}
\end{array}
\end{align*}
\vspace{-0.5em}
\caption{Typing rules for graded-base}
\label{fig:graded-typing}
\vspace{-0.5em}
 \end{figure}

Figure~\ref{fig:graded-typing} gives the full typing rules, which explains the meaning of
the syntax with reference to their static semantics.

Variables (rule \textsc{Var}) are typed in a context where the variable $x$ has
grade $1$ denoting its single usage here. All other variable assumptions are
given the grade of the $0$ semiring element (providing \emph{weakening}), using
\textit{scalar multiplication} of contexts by a grade, re-using
definition~\eqref{def:scalar}.

Abstraction (\textsc{Abs}) captures the assumption's grade $[[ r ]]$ onto the
function arrow in the conclusion, that is, abstraction binds a variable $[[x]]$
which may be used in the body $[[t]]$ according to grade $[[ r ]]$. Application
(\textsc{App}) makes use of context addition to combine the contexts used to
type the two subterms in the premises of the application rule (providing
\emph{contraction}):

\begin{definition}[Graded context addition]\label{def:contextAddGraded}
\begin{align*}
[[G1 + G2]] = \left\{\begin{matrix}
    \begin{array}{ll}
    [[G1]] &
    [[G2]] = \emptyset
             \\
      (([[G1']], [[G1'']]) + [[G2']]), [[x : [A] (r + s)]] \; &
[[ G2]] = [[ G2', x : [A] s]] \wedge [[G1]] = [[ G1',x : [A] r]],[[G1'']] \\
 [[ (G1 + G2'), x : [A] s ]] & [[ G2 ]] = [[ G2' , x : [A] s ]] \wedge [[ x ]] \not\in \mathsf{dom}([[ G1 ]])
\end{array}
  \end{matrix}\right.
\end{align*}
\end{definition}
Note that~\ref{def:contextAddGraded} differs only from
~\ref{def:contextAdd}, in that the former need not consider linear
assumptions.

Explicit introduction of graded modalities is achieved via the rule for
promotion (\textsc{Pr}). The grade $[[ r ]]$ is propagated to the assumptions in
$[[ G ]]$ through the scaling of $[[G]]$ by $[[r]]$. Approximation
(\textsc{Approx}) allows a grade $[[ r ]]$ to be converted to another grade $[[
s ]]$, provided that $[[ r ]]$ \textit{approximates} $[[ s ]]$. Here, the
pre-order relation of the semiring $\sqsubseteq$ provides approximation. This
relation is occasionally lifted pointwise to contexts: we write $[[ G <<= G' ]]$
to mean that $[[ G' ]]$ overapproximates $[[ G ]]$ meaning that for all $[[ (x :
[A] r) ]] \in [[ G ]]$ then $[[ (x : [A] r') ]] \in [[ G' ]]$ and $[[ r <<= r'
]]$.

\paragraph{Metatheory}
Lastly we note that the fully graded $\lambda$-calculus also enjoys admissibility of
substitution~\citep{DBLP:journals/pacmpl/AbelB20}
which is critical in type preservation proofs,
and is needed in our proof of soundness for synthesis:
%
\begin{restatable}[Admissibility of substitution]{lemma}{subst}
\label{lemma:substitution}
Let $[[ D |- t' : A]]$, then:
If $[[ {G, x : [A] r} , G' |- t : B]]$
then $[[ G + (r * D) + G' |- [ t' / x ] t : B ]]$
\end{restatable}

\section{Two Typing Calculi}
Having outlined the two lineages of graded type systems, we are left with the
question: what approach should we use as the basis of a target language for a
program synthesis tool? Both systems embed properties for reasoning about
program structure into the language, however, they differ in how this
information is expressed, as shown by the variance in typing and syntax between
sections ~\ref{sec:linear-base} and ~\ref{sec:graded-base}. 

Rather than focus entirely on one approach, we opt to instead build synthesis
tools which target both languages. As we have seen, systems based on both
approaches are used by many languages today, and both pose their unique
challenges in designing a synthesis tool, which makes favouring a particular
approach difficult to justify. Furthermore, the target programming language
Granule of our implementations includes both approaches~\footnote{As of Granule
v0.9.3.0}. 