\documentclass[12pt]{UoKthesis}

% Note: The UoKextentions package includes the xcolor package with the [usenames]
% options. If you need to add further options, these can be given to UoKextentions
% to be propogated through.
\usepackage{UoKextentions}
\usepackage{lipsum}
\usepackage{comment}
\usepackage{dashrule}
\usepackage{color}
\usepackage{mathpartir}
\usepackage{amsmath}
\usepackage{amsfonts}
\usepackage{amssymb}
\usepackage{amsthm}
\usepackage{tikz}
\usepackage{tikz-cd}
\usepackage{natbib}
\usepackage{caption}
\usepackage{xypic}
\usepackage{enumitem}
\usepackage{hyperref}
\usepackage{graphicx}
\usepackage{float}

\usepackage[tableposition = top]{caption}
\usepackage{booktabs}

\usepackage{multirow}

\newcommand{\jnote}[1]{\textcolor{purple}{\textsc{joh}: #1}}

\usetikzlibrary{automata, positioning, arrows}
\usetikzlibrary{shapes.multipart}
\usepackage{thmtools}
\usepackage{thm-restate}
\declaretheorem[name=Proposition,numberwithin=section]{prop}

\newcommand{\coeff}[1]{\textcolor{coeffectColor}{#1}}

\theoremstyle{plain}% default
\newtheorem{theorem}{Theorem}[section]
\newtheorem{lemma}[theorem]{Lemma}
\newtheorem{proposition}{Proposition}
\newtheorem*{corollary}{Corollary}
\theoremstyle{definition}
\newtheorem{definition}{Definition}[section]
\newtheorem{conjecture}{Conjecture}[section]
\newtheorem{example}{Example}[section]
\newtheorem{remark}{Remark}
\newtheorem{question}{Question}[section]



\definecolor{mypink3}{cmyk}{0, 0.82, 0.98, 0.28}
\newcommand{\maximal}[4]{\forall #2 . #3 \sqsupseteq #4 \implies #1 \sqsupseteq #2}

% Used to configure the algorithm enviroment
\usepackage[ruled,vlined,linesnumbered,commentsnumbered,resetcount]{algorithm2e}

% Used to generate sample image.
\let\oldincludegraphics\includegraphics\renewcommand{\includegraphics}[1]{\IfFileExists{#1}{\oldincludegraphics{#1}}{\fbox{\phantom{\rule{150pt}{100pt}}}}}


% generated by Ott 0.31 from: ../granule.ott
\newcommand{\GRANULEdrule}[4][]{{\displaystyle\frac{\begin{array}{l}#2\end{array}}{#3}\quad\GRANULEdrulename{#4}}}
\newcommand{\GRANULEusedrule}[1]{\[#1\]}
\newcommand{\GRANULEpremise}[1]{ #1 \\}
\newenvironment{GRANULEdefnblock}[3][]{ \framebox{\mbox{#2}} \quad #3 \\[0pt]}{}
\newenvironment{GRANULEfundefnblock}[3][]{ \framebox{\mbox{#2}} \quad #3 \\[0pt]\begin{displaymath}\begin{array}{l}}{\end{array}\end{displaymath}}
\newcommand{\GRANULEfunclause}[2]{ #1 \equiv #2 \\}
\newcommand{\GRANULEnt}[1]{\mathit{#1}}
\newcommand{\GRANULEmv}[1]{\mathit{#1}}
\newcommand{\GRANULEkw}[1]{\mathbf{#1}}
\newcommand{\GRANULEsym}[1]{#1}
\newcommand{\GRANULEcom}[1]{\text{#1}}
\newcommand{\GRANULEdrulename}[1]{\textsc{#1}}
\newcommand{\GRANULEcomplu}[5]{\overline{#1}^{\,#2\in #3 #4 #5}}
\newcommand{\GRANULEcompu}[3]{\overline{#1}^{\,#2<#3}}
\newcommand{\GRANULEcomp}[2]{\overline{#1}^{\,#2}}
\newcommand{\GRANULEgrammartabular}[1]{\begin{supertabular}{llcllllll}#1\end{supertabular}}
\newcommand{\GRANULEmetavartabular}[1]{\begin{supertabular}{ll}#1\end{supertabular}}
\newcommand{\GRANULErulehead}[3]{$#1$ & & $#2$ & & & \multicolumn{2}{l}{#3}}
\newcommand{\GRANULEprodline}[6]{& & $#1$ & $#2$ & $#3 #4$ & $#5$ & $#6$}
\newcommand{\GRANULEfirstprodline}[6]{\GRANULEprodline{#1}{#2}{#3}{#4}{#5}{#6}}
\newcommand{\GRANULElongprodline}[2]{& & $#1$ & \multicolumn{4}{l}{$#2$}}
\newcommand{\GRANULEfirstlongprodline}[2]{\GRANULElongprodline{#1}{#2}}
\newcommand{\GRANULEbindspecprodline}[6]{\GRANULEprodline{#1}{#2}{#3}{#4}{#5}{#6}}
\newcommand{\GRANULEprodnewline}{\\}
\newcommand{\GRANULEinterrule}{\\[5.0mm]}
\newcommand{\GRANULEafterlastrule}{\\}

\IfFileExists{xcolor.sty}{
	\RequirePackage{xcolor}
}{
	\RequirePackage{color}
}


 \usepackage{stmaryrd}
 \usepackage{mathtools}
 \usepackage{halloweenmath}
 %\definecolor{coeffectColor}{HTML}{0750D0}
 \definecolor{coeffectColor}{HTML}{000000}
 \definecolor{effectColor}{HTML}{D64800}

 
\newcommand{\GRANULEmetavars}{
\GRANULEmetavartabular{
 $ \GRANULEmv{termvar} ,\, \alpha ,\, \beta ,\, \gamma ,\, \mathcal{L} ,\, \GRANULEmv{x} ,\, \GRANULEmv{y} ,\, \GRANULEmv{z} ,\, \GRANULEmv{f} ,\, \GRANULEmv{g} ,\, \GRANULEmv{h} ,\, \mathit{adv} $ &  \\
 $ \GRANULEmv{index} ,\, \GRANULEmv{i} ,\, \GRANULEmv{j} ,\, \GRANULEmv{n} ,\, \GRANULEmv{m} $ &  \\
}}

\newcommand{\GRANULEFoc}{
\GRANULErulehead{\GRANULEnt{Foc}}{::=}{\GRANULEcom{Focusing phase}}\GRANULEprodnewline
\GRANULEfirstprodline{|}{ \textsc{A} }{}{}{}{\GRANULEcom{Async}}\GRANULEprodnewline
\GRANULEprodline{|}{ \textsc{S} }{}{}{}{\GRANULEcom{Sync}}}

\newcommand{\GRANULEt}{
\GRANULErulehead{\GRANULEnt{t}  ,\ \GRANULEnt{v}}{::=}{\GRANULEcom{Terms}}\GRANULEprodnewline
\GRANULEfirstprodline{|}{\GRANULEnt{t_{{\mathrm{1}}}} \, \GRANULEnt{t_{{\mathrm{2}}}}}{}{}{}{\GRANULEcom{Application}}\GRANULEprodnewline
\GRANULEprodline{|}{ \textbf{let} \, \langle  \GRANULEnt{p}  \rangle\!\shortleftarrow\!  \GRANULEnt{t_{{\mathrm{1}}}}  \, \textbf{in} \,  \GRANULEnt{t_{{\mathrm{2}}}} }{}{}{}{\GRANULEcom{Effectful Let Binding}}\GRANULEprodnewline
\GRANULEprodline{|}{ \textbf{case} \  \GRANULEnt{t}  \ \textbf{of} \  \GRANULEnt{Cases} }{}{}{}{\GRANULEcom{Case}}\GRANULEprodnewline
\GRANULEprodline{|}{ \lambda  \GRANULEnt{p}  .  \GRANULEnt{t} }{}{}{}{\GRANULEcom{Function}}\GRANULEprodnewline
\GRANULEprodline{|}{ \lambda  \GRANULEnt{p}  .  \GRANULEnt{t} }{}{}{}{\GRANULEcom{Function with graded annotation on its binder}}\GRANULEprodnewline
\GRANULEprodline{|}{\GRANULEsym{[}  \GRANULEnt{t}  \GRANULEsym{]}}{}{}{}{\GRANULEcom{Promote}}\GRANULEprodnewline
\GRANULEprodline{|}{\langle  \GRANULEnt{t}  \rangle}{}{}{}{\GRANULEcom{Pure}}\GRANULEprodnewline
\GRANULEprodline{|}{\GRANULEmv{x}}{}{}{}{\GRANULEcom{Variable}}\GRANULEprodnewline
\GRANULEprodline{|}{C \, \GRANULEnt{t_{{\mathrm{0}}}} \, ... \, \GRANULEnt{t_{\GRANULEmv{n}}}}{}{}{}{\GRANULEcom{Constructor}}\GRANULEprodnewline
\GRANULEprodline{|}{ C }{}{}{}{}\GRANULEprodnewline
\GRANULEprodline{|}{\GRANULEmv{n}}{}{}{}{\GRANULEcom{Integer constructors}}\GRANULEprodnewline
\GRANULEprodline{|}{ \textbf{let} \, [  \GRANULEnt{p}  ] =  \GRANULEnt{t_{{\mathrm{1}}}}  \, \textbf{in} \,  \GRANULEnt{t_{{\mathrm{2}}}} } {\textsf{S}}{}{}{\GRANULEcom{Modal let-binding}}\GRANULEprodnewline
\GRANULEprodline{|}{\GRANULEkw{inl} \, \GRANULEnt{t}}{}{}{}{\GRANULEcom{Inl}}\GRANULEprodnewline
\GRANULEprodline{|}{\GRANULEkw{inr} \, \GRANULEnt{t}}{}{}{}{\GRANULEcom{Inr}}\GRANULEprodnewline
\GRANULEprodline{|}{ \textbf{let} \, ( \GRANULEmv{x_{{\mathrm{1}}}} ,  \GRANULEmv{x_{{\mathrm{2}}}} ) =  \GRANULEnt{t_{{\mathrm{1}}}}  \, \textbf{in} \,  \GRANULEnt{t_{{\mathrm{2}}}} } {\textsf{S}}{}{}{\GRANULEcom{Pair let-binding}}\GRANULEprodnewline
\GRANULEprodline{|}{ ( \GRANULEnt{t_{{\mathrm{1}}}} ,  \GRANULEnt{t_{{\mathrm{2}}}} ) }{}{}{}{\GRANULEcom{A pair of terms}}\GRANULEprodnewline
\GRANULEprodline{|}{ - }{}{}{}{\GRANULEcom{Hole}}\GRANULEprodnewline
\GRANULEprodline{|}{\GRANULEkw{let} \, \GRANULEsym{()}  \GRANULEsym{=}  \GRANULEnt{t_{{\mathrm{1}}}} \, \GRANULEkw{in} \, \GRANULEnt{t_{{\mathrm{2}}}}}{}{}{}{\GRANULEcom{UnitElim}}\GRANULEprodnewline
\GRANULEprodline{|}{\GRANULEsym{()}}{}{}{}{\GRANULEcom{Unit}}}

\newcommand{\GRANULECon}{
\GRANULErulehead{C}{::=}{\GRANULEcom{Constructors}}\GRANULEprodnewline
\GRANULEfirstprodline{|}{\GRANULEsym{(,)}}{}{}{}{\GRANULEcom{Pair constructor}}\GRANULEprodnewline
\GRANULEprodline{|}{\GRANULEkw{inl}}{}{}{}{\GRANULEcom{Left injection}}\GRANULEprodnewline
\GRANULEprodline{|}{\GRANULEkw{inr}}{}{}{}{\GRANULEcom{Right injection}}\GRANULEprodnewline
\GRANULEprodline{|}{ \mathsf{unit} }{}{}{}{\GRANULEcom{Unit}}\GRANULEprodnewline
\GRANULEprodline{|}{ \mathsf{tt} }{}{}{}{}\GRANULEprodnewline
\GRANULEprodline{|}{ \mathsf{ff} }{}{}{}{}\GRANULEprodnewline
\GRANULEprodline{|}{ \mathsf{Just} }{}{}{}{}\GRANULEprodnewline
\GRANULEprodline{|}{ \mathsf{Nothing} }{}{}{}{}}

\newcommand{\GRANULECases}{
\GRANULErulehead{\GRANULEnt{Cases}}{::=}{\GRANULEcom{Value-level cases}}\GRANULEprodnewline
\GRANULEfirstprodline{|}{\GRANULEnt{p}  \rightarrow  \GRANULEnt{t}  \GRANULEsym{;}  \GRANULEnt{Cases}}{}{}{}{\GRANULEcom{Case cons}}\GRANULEprodnewline
\GRANULEprodline{|}{\GRANULEnt{p}  \rightarrow  \GRANULEnt{t}}{}{}{}{\GRANULEcom{One case}}\GRANULEprodnewline
\GRANULEprodline{|}{ \overline{  \GRANULEnt{p}  \mapsto  \GRANULEnt{t} } }{}{}{}{\GRANULEcom{One case overline}}\GRANULEprodnewline
\GRANULEprodline{|}{ \GRANULEnt{p}  \mapsto  \GRANULEnt{t} ; ... ;  \GRANULEnt{p'}  \mapsto  \GRANULEnt{t'} } {\textsf{S}}{}{}{\GRANULEcom{Many cases (syntactic sugar)}}}

\newcommand{\GRANULEp}{
\GRANULErulehead{\GRANULEnt{p}}{::=}{\GRANULEcom{Patterns}}\GRANULEprodnewline
\GRANULEfirstprodline{|}{\GRANULEmv{x}}{}{}{}{\GRANULEcom{Variable}}\GRANULEprodnewline
\GRANULEprodline{|}{ \_ }{}{}{}{\GRANULEcom{Wildcard}}\GRANULEprodnewline
\GRANULEprodline{|}{\GRANULEsym{[}  \GRANULEnt{p}  \GRANULEsym{]}}{}{}{}{\GRANULEcom{Unbox}}\GRANULEprodnewline
\GRANULEprodline{|}{ \lbrack [  \GRANULEnt{p}  ] \rbrack }{}{}{}{\GRANULEcom{Double unboxing}}\GRANULEprodnewline
\GRANULEprodline{|}{C \, \GRANULEnt{p_{{\mathrm{1}}}} \, ... \, \GRANULEnt{p_{\GRANULEmv{n}}}}{}{}{}{\GRANULEcom{Constructor}}\GRANULEprodnewline
\GRANULEprodline{|}{ C  \  \GRANULEnt{p_{{\mathrm{1}}}} ^ \GRANULEnt{Ix_{{\mathrm{3}}}} _ \GRANULEnt{Ix_{{\mathrm{1}}}}  ...  \GRANULEnt{p_{{\mathrm{2}}}} ^ \GRANULEnt{Ix_{{\mathrm{3}}}} _ \GRANULEnt{Ix_{{\mathrm{2}}}} }{}{}{}{\GRANULEcom{ConstructorIndexed}}\GRANULEprodnewline
\GRANULEprodline{|}{ C }{}{}{}{\GRANULEcom{Nullary Constructor}}\GRANULEprodnewline
\GRANULEprodline{|}{\GRANULEmv{n}}{}{}{}{\GRANULEcom{Int constructor}}\GRANULEprodnewline
\GRANULEprodline{|}{\GRANULEsym{(}  \GRANULEnt{p}  \GRANULEsym{)}}{}{}{}{}\GRANULEprodnewline
\GRANULEprodline{|}{ \GRANULEmv{y} ^i_j }{}{}{}{\GRANULEcom{Hack}}\GRANULEprodnewline
\GRANULEprodline{|}{ ( \GRANULEnt{p_{{\mathrm{1}}}} ,  \GRANULEnt{p_{{\mathrm{2}}}} ) }{}{}{}{\GRANULEcom{Pair}}\GRANULEprodnewline
\GRANULEprodline{|}{ \GRANULEnt{p} ^{ \GRANULEnt{Ix} } } {\textsf{S}}{}{}{\GRANULEcom{Pattern at index n+m}}}

\newcommand{\GRANULEIx}{
\GRANULErulehead{\GRANULEnt{Ix}}{::=}{\GRANULEcom{More complex index expressions}}\GRANULEprodnewline
\GRANULEfirstprodline{|}{\GRANULEsym{0}}{}{}{}{}\GRANULEprodnewline
\GRANULEprodline{|}{\GRANULEsym{1}}{}{}{}{}\GRANULEprodnewline
\GRANULEprodline{|}{\GRANULEsym{2}}{}{}{}{}\GRANULEprodnewline
\GRANULEprodline{|}{\GRANULEnt{Ix_{{\mathrm{1}}}}  \GRANULEsym{+}  \GRANULEnt{Ix_{{\mathrm{2}}}}}{}{}{}{}\GRANULEprodnewline
\GRANULEprodline{|}{ m }{}{}{}{}\GRANULEprodnewline
\GRANULEprodline{|}{ 1 }{}{}{}{}\GRANULEprodnewline
\GRANULEprodline{|}{ 2 }{}{}{}{}\GRANULEprodnewline
\GRANULEprodline{|}{ 3 }{}{}{}{}\GRANULEprodnewline
\GRANULEprodline{|}{ 4 }{}{}{}{}\GRANULEprodnewline
\GRANULEprodline{|}{ i }{}{}{}{}\GRANULEprodnewline
\GRANULEprodline{|}{ j }{}{}{}{}\GRANULEprodnewline
\GRANULEprodline{|}{ n }{}{}{}{}\GRANULEprodnewline
\GRANULEprodline{|}{ m }{}{}{}{}}

\newcommand{\GRANULEEqn}{
\GRANULErulehead{\GRANULEnt{Eqn}}{::=}{\GRANULEcom{Equations}}\GRANULEprodnewline
\GRANULEfirstprodline{|}{\GRANULEmv{x} \, \GRANULEnt{p_{{\mathrm{1}}}} \, .. \, \GRANULEnt{p_{\GRANULEmv{n}}}  \GRANULEsym{=}  \GRANULEnt{t}}{}{}{}{\GRANULEcom{Eq}}}

\newcommand{\GRANULEDef}{
\GRANULErulehead{\textit{Def}}{::=}{\GRANULEcom{Definitions}}\GRANULEprodnewline
\GRANULEfirstprodline{|}{\GRANULEmv{x}  \GRANULEsym{:}  C  \GRANULEsym{;}  \GRANULEnt{Eqn_{{\mathrm{1}}}} \, .. \, \GRANULEnt{Eqn_{\GRANULEmv{n}}}}{}{}{}{\GRANULEcom{Multi-eq def}}}

\newcommand{\GRANULEA}{
\GRANULErulehead{\GRANULEnt{A}  ,\ \GRANULEnt{B}  ,\ \GRANULEnt{C}  ,\ \GRANULEnt{E}  ,\ \GRANULEnt{W}  ,\ C}{::=}{\GRANULEcom{Types}}\GRANULEprodnewline
\GRANULEfirstprodline{|}{ \cdot }{}{}{}{\GRANULEcom{Empty}}\GRANULEprodnewline
\GRANULEprodline{|}{\GRANULEnt{A}  \rightarrow  \GRANULEnt{B}}{}{}{}{\GRANULEcom{Function}}\GRANULEprodnewline
\GRANULEprodline{|}{ \GRANULEnt{A} _{ \textsc{Dec} } }{}{}{}{\GRANULEcom{Dec}}\GRANULEprodnewline
\GRANULEprodline{|}{ \GRANULEnt{A} ^ \GRANULEnt{c}  \rightarrow  \GRANULEnt{B} }{}{}{}{\GRANULEcom{Graded Function}}\GRANULEprodnewline
\GRANULEprodline{|}{ \GRANULEnt{A} ^ \GRANULEnt{c}  \multimap  \GRANULEnt{B} }{}{}{}{\GRANULEcom{Graded Linear Function}}\GRANULEprodnewline
\GRANULEprodline{|}{ \GRANULEnt{A}  \multimap  \GRANULEnt{B} } {\textsf{S}}{}{}{\GRANULEcom{Linear Function}}\GRANULEprodnewline
\GRANULEprodline{|}{ K }{}{}{}{\GRANULEcom{Constructor}}\GRANULEprodnewline
\GRANULEprodline{|}{ K  \GRANULEnt{A}  \ldots  \GRANULEnt{B} }{}{}{}{\GRANULEcom{Constructor}}\GRANULEprodnewline
\GRANULEprodline{|}{\alpha}{}{}{}{\GRANULEcom{Variable}}\GRANULEprodnewline
\GRANULEprodline{|}{\GRANULEnt{A} \, \GRANULEnt{B}}{}{}{}{\GRANULEcom{Application}}\GRANULEprodnewline
\GRANULEprodline{|}{ \GRANULEnt{A} ^ \GRANULEnt{Ix_{{\mathrm{1}}}} _ \GRANULEnt{Ix_{{\mathrm{2}}}} }{}{}{}{\GRANULEcom{Var2IndexTy}}\GRANULEprodnewline
\GRANULEprodline{|}{ \Box  \GRANULEnt{A} }{}{}{}{\GRANULEcom{BlankBox}}\GRANULEprodnewline
\GRANULEprodline{|}{ \Box_{  \GRANULEnt{c}  }  \GRANULEnt{A} }{}{}{}{\GRANULEcom{Box}}\GRANULEprodnewline
\GRANULEprodline{|}{ \Box_{\textcolor{coeffectColor}{ \GRANULEnt{c}  :  \GRANULEnt{B} } }  \GRANULEnt{A} }{}{}{}{\GRANULEcom{Box with coeffect type}}\GRANULEprodnewline
\GRANULEprodline{|}{ \mathsf{Int} }{}{}{}{\GRANULEcom{Integers}}\GRANULEprodnewline
\GRANULEprodline{|}{ \mathsf{Char} }{}{}{}{\GRANULEcom{Characters}}\GRANULEprodnewline
\GRANULEprodline{|}{ \mathsf{1} }{}{}{}{\GRANULEcom{unit}}\GRANULEprodnewline
\GRANULEprodline{|}{ \otimes }{}{}{}{\GRANULEcom{Products}}\GRANULEprodnewline
\GRANULEprodline{|}{ \mathbb{B} }{}{}{}{\GRANULEcom{Bool}}\GRANULEprodnewline
\GRANULEprodline{|}{ \mathsf{IO} }{}{}{}{\GRANULEcom{IO}}\GRANULEprodnewline
\GRANULEprodline{|}{\GRANULEnt{R}}{}{}{}{\GRANULEcom{Coeffect types}}\GRANULEprodnewline
\GRANULEprodline{|}{\GRANULEmv{n}}{}{}{}{\GRANULEcom{Type-level integers}}\GRANULEprodnewline
\GRANULEprodline{|}{ \GRANULEnt{A}  \,  \textsf{op} 
  \,  \GRANULEnt{B} }{}{}{}{\GRANULEcom{InfixOp}}\GRANULEprodnewline
\GRANULEprodline{|}{ \GRANULEnt{A}  \, \otimes \,  \GRANULEnt{B} }{}{}{}{\GRANULEcom{Tuple}}\GRANULEprodnewline
\GRANULEprodline{|}{ \GRANULEnt{A}  \, \oplus \,  \GRANULEnt{B} }{}{}{}{\GRANULEcom{Sum}}\GRANULEprodnewline
\GRANULEprodline{|}{ \{  \GRANULEnt{A_{{\mathrm{1}}}} , ...,   \GRANULEnt{A_{\GRANULEmv{n}}}  \} }{}{}{}{}\GRANULEprodnewline
\GRANULEprodline{|}{ {\downarrow\! \kappa  } }{}{}{}{}\GRANULEprodnewline
\GRANULEprodline{|}{ \mathsf{Set} \,  \GRANULEnt{A} }{}{}{}{}\GRANULEprodnewline
\GRANULEprodline{|}{ \GRANULEnt{B_{{\mathrm{1}}}} ^1  \rightarrow \! ... \! \rightarrow   \GRANULEnt{B_{\GRANULEmv{n}}} ^1  \rightarrow   \GRANULEnt{A} } {\textsf{S}}{}{}{\GRANULEcom{n-Ary Function}}\GRANULEprodnewline
\GRANULEprodline{|}{ \GRANULEnt{B_{{\mathrm{1}}}} ^{q_1}  \rightarrow \! ... \! \rightarrow   \GRANULEnt{B_{\GRANULEmv{n}}} ^{q_n}  \rightarrow   \GRANULEnt{A} } {\textsf{S}}{}{}{\GRANULEcom{n-Ary Function w grades}}\GRANULEprodnewline
\GRANULEprodline{|}{ \GRANULEnt{A}  \Uparrow\ }{}{}{}{\GRANULEcom{AsyncTy}}\GRANULEprodnewline
\GRANULEprodline{|}{ \GRANULEnt{A}  \Downarrow\ }{}{}{}{\GRANULEcom{SyncTy}}}

\newcommand{\GRANULEop}{
\GRANULErulehead{\textsf{op}}{::=}{\GRANULEcom{Type operators}}\GRANULEprodnewline
\GRANULEfirstprodline{|}{ + }{}{}{}{}\GRANULEprodnewline
\GRANULEprodline{|}{ * }{}{}{}{}\GRANULEprodnewline
\GRANULEprodline{|}{ - }{}{}{}{}\GRANULEprodnewline
\GRANULEprodline{|}{ \leq }{}{}{}{}\GRANULEprodnewline
\GRANULEprodline{|}{ < }{}{}{}{}\GRANULEprodnewline
\GRANULEprodline{|}{ > }{}{}{}{}\GRANULEprodnewline
\GRANULEprodline{|}{ \geq }{}{}{}{}\GRANULEprodnewline
\GRANULEprodline{|}{ = }{}{}{}{}\GRANULEprodnewline
\GRANULEprodline{|}{ \neq }{}{}{}{}\GRANULEprodnewline
\GRANULEprodline{|}{ \sqcup }{}{}{}{}\GRANULEprodnewline
\GRANULEprodline{|}{ \sqcap }{}{}{}{}}

\newcommand{\GRANULER}{
\GRANULErulehead{\GRANULEnt{R}  ,\ \GRANULEnt{S}}{::=}{\GRANULEcom{Coeffect types}}\GRANULEprodnewline
\GRANULEfirstprodline{|}{ \mathsf{Nat} }{}{}{}{\GRANULEcom{Nat}}\GRANULEprodnewline
\GRANULEprodline{|}{ \mathsf{Level} }{}{}{}{\GRANULEcom{Level}}\GRANULEprodnewline
\GRANULEprodline{|}{ \mathsf{Ext} }{}{}{}{\GRANULEcom{Extending}}\GRANULEprodnewline
\GRANULEprodline{|}{ \mathsf{Interval} }{}{}{}{\GRANULEcom{Interval}}\GRANULEprodnewline
\GRANULEprodline{|}{ \GRANULEnt{A}  \times  \GRANULEnt{B} }{}{}{}{\GRANULEcom{Products}}\GRANULEprodnewline
\GRANULEprodline{|}{\GRANULEnt{R} \, \GRANULEnt{S}}{}{}{}{\GRANULEcom{Application}}\GRANULEprodnewline
\GRANULEprodline{|}{ \alpha }{}{}{}{\GRANULEcom{Variable}}\GRANULEprodnewline
\GRANULEprodline{|}{ \beta }{}{}{}{\GRANULEcom{VariableB}}}

\newcommand{\GRANULEConstraints}{
\GRANULErulehead{\textit{Cons}}{::=}{\GRANULEcom{Constraints/Predicates}}\GRANULEprodnewline
\GRANULEfirstprodline{|}{\GRANULEnt{A}  \GRANULEsym{,}  \textit{Cons}}{}{}{}{}\GRANULEprodnewline
\GRANULEprodline{|}{\GRANULEnt{A}}{}{}{}{}\GRANULEprodnewline
\GRANULEprodline{|}{\GRANULEnt{A_{{\mathrm{1}}}}  \GRANULEsym{,} \, .. \, \GRANULEsym{,}  \GRANULEnt{A_{\GRANULEmv{n}}}}{}{}{}{}\GRANULEprodnewline
\GRANULEprodline{|}{ \overrightarrow{ \GRANULEnt{A} } }{}{}{}{}\GRANULEprodnewline
\GRANULEprodline{|}{\GRANULEsym{(}  \textit{Cons}  \GRANULEsym{)}}{}{}{}{}}

\newcommand{\GRANULEc}{
\GRANULErulehead{\GRANULEnt{c}  ,\ \GRANULEnt{r}  ,\ \GRANULEnt{s}  ,\ \GRANULEnt{q}}{::=}{}\GRANULEprodnewline
\GRANULEfirstprodline{|}{ \GRANULEnt{c} ^ \GRANULEnt{Ix_{{\mathrm{1}}}} _ \GRANULEnt{Ix_{{\mathrm{2}}}} }{}{}{}{\GRANULEcom{Var2Index}}\GRANULEprodnewline
\GRANULEprodline{|}{ \GRANULEnt{c} _ \GRANULEnt{Ix_{{\mathrm{1}}}} }{}{}{}{\GRANULEcom{Var1Index}}\GRANULEprodnewline
\GRANULEprodline{|}{\GRANULEnt{c_{{\mathrm{1}}}}  \GRANULEsym{+}  \GRANULEnt{c_{{\mathrm{2}}}}}{}{}{}{\GRANULEcom{Addition}}\GRANULEprodnewline
\GRANULEprodline{|}{\GRANULEnt{c_{{\mathrm{1}}}}  \textcolor{coeffectColor}{\,\cdot\,}  \GRANULEnt{c_{{\mathrm{2}}}}}{}{}{}{\GRANULEcom{Multiplication}}\GRANULEprodnewline
\GRANULEprodline{|}{ 0 }{}{}{}{\GRANULEcom{Additive Unit}}\GRANULEprodnewline
\GRANULEprodline{|}{ 1 }{}{}{}{\GRANULEcom{Multiplicative Unit}}\GRANULEprodnewline
\GRANULEprodline{|}{ \GRANULEnt{c_{{\mathrm{1}}}}  \sqcup  \GRANULEnt{c_{{\mathrm{2}}}} }{}{}{}{\GRANULEcom{Join}}\GRANULEprodnewline
\GRANULEprodline{|}{ \GRANULEnt{c_{{\mathrm{1}}}}  \sqcap  \GRANULEnt{c_{{\mathrm{2}}}} }{}{}{}{\GRANULEcom{Meet}}\GRANULEprodnewline
\GRANULEprodline{|}{ \GRANULEnt{c_{{\mathrm{1}}}}  \sqcap ... \sqcap  \GRANULEnt{c_{{\mathrm{2}}}} }{}{}{}{\GRANULEcom{MultiMeet}}\GRANULEprodnewline
\GRANULEprodline{|}{ \GRANULEnt{c_{{\mathrm{1}}}}  \sqcup ... \sqcup  \GRANULEnt{c_{{\mathrm{2}}}} }{}{}{}{\GRANULEcom{MultiJoin}}\GRANULEprodnewline
\GRANULEprodline{|}{ \GRANULEnt{c_{{\mathrm{1}}}}  =  \GRANULEnt{c_{{\mathrm{2}}}} }{}{}{}{\GRANULEcom{CoeffEq}}\GRANULEprodnewline
\GRANULEprodline{|}{ \bigsqcup_1^n  \GRANULEnt{c} }{}{}{}{\GRANULEcom{BigJoin}}\GRANULEprodnewline
\GRANULEprodline{|}{ \textsf{flatten}  ( \GRANULEnt{c_{{\mathrm{1}}}} ,  \GRANULEnt{A} ,  \GRANULEnt{c_{{\mathrm{2}}}} ,  \GRANULEnt{B} ) }{}{}{}{\GRANULEcom{Flatten}}\GRANULEprodnewline
\GRANULEprodline{|}{ \mathsf{Hi} }{}{}{}{}\GRANULEprodnewline
\GRANULEprodline{|}{ \mathsf{Lo} }{}{}{}{}\GRANULEprodnewline
\GRANULEprodline{|}{ \textcolor{coeffectColor}{ \GRANULEnt{c_{{\mathrm{1}}}} }..\textcolor{coeffectColor}{ \GRANULEnt{c_{{\mathrm{2}}}} } }{}{}{}{}\GRANULEprodnewline
\GRANULEprodline{|}{ \infty }{}{}{}{}\GRANULEprodnewline
\GRANULEprodline{|}{ ( \GRANULEnt{c_{{\mathrm{1}}}} ,  \GRANULEnt{c_{{\mathrm{2}}}} ) }{}{}{}{}\GRANULEprodnewline
\GRANULEprodline{|}{ \theta   \GRANULEnt{c} }{}{}{}{}\GRANULEprodnewline
\GRANULEprodline{|}{\GRANULEsym{2}} {\textsf{S}}{}{}{}\GRANULEprodnewline
\GRANULEprodline{|}{\GRANULEsym{3}} {\textsf{S}}{}{}{}\GRANULEprodnewline
\GRANULEprodline{|}{\GRANULEsym{4}} {\textsf{S}}{}{}{}\GRANULEprodnewline
\GRANULEprodline{|}{\GRANULEsym{5}} {\textsf{S}}{}{}{}\GRANULEprodnewline
\GRANULEprodline{|}{\GRANULEsym{6}} {\textsf{S}}{}{}{}\GRANULEprodnewline
\GRANULEprodline{|}{\GRANULEsym{10}} {\textsf{S}}{}{}{}\GRANULEprodnewline
\GRANULEprodline{|}{\GRANULEsym{15}} {\textsf{S}}{}{}{}\GRANULEprodnewline
\GRANULEprodline{|}{\GRANULEsym{20}} {\textsf{S}}{}{}{}\GRANULEprodnewline
\GRANULEprodline{|}{ \GRANULEnt{c} _{i - 1} }{}{}{}{}\GRANULEprodnewline
\GRANULEprodline{|}{ \mathsf{Unused} }{}{}{}{}\GRANULEprodnewline
\GRANULEprodline{|}{ [  \GRANULEnt{r}  ...  \GRANULEnt{s}  ] }{}{}{}{}}

\newcommand{\GRANULERel}{
\GRANULErulehead{\GRANULEnt{Rel}}{::=}{\GRANULEcom{Relations on grades}}\GRANULEprodnewline
\GRANULEfirstprodline{|}{ \GRANULEnt{c_{{\mathrm{1}}}}  \sqsubseteq  \GRANULEnt{c_{{\mathrm{2}}}} }{}{}{}{}\GRANULEprodnewline
\GRANULEprodline{|}{ \GRANULEnt{c_{{\mathrm{1}}}}  \sqsubseteq  \GRANULEnt{c_{{\mathrm{2}}}}  \sqsubseteq  \GRANULEnt{c_{{\mathrm{3}}}} }{}{}{}{}\GRANULEprodnewline
\GRANULEprodline{|}{ \GRANULEnt{c_{{\mathrm{1}}}}  \sqsupseteq  \GRANULEnt{c_{{\mathrm{2}}}} }{}{}{}{}\GRANULEprodnewline
\GRANULEprodline{|}{ \GRANULEnt{c_{{\mathrm{1}}}}  \sqsupseteq  \GRANULEnt{c_{{\mathrm{2}}}}  \sqsupseteq  \GRANULEnt{c_{{\mathrm{3}}}} }{}{}{}{}\GRANULEprodnewline
\GRANULEprodline{|}{ \GRANULEnt{Rel_{{\mathrm{1}}}}  \wedge  \GRANULEnt{Rel_{{\mathrm{2}}}} }{}{}{}{}}

\newcommand{\GRANULEk}{
\GRANULErulehead{\kappa}{::=}{}\GRANULEprodnewline
\GRANULEfirstprodline{|}{ \mathsf{Type} }{}{}{}{\GRANULEcom{Type}}\GRANULEprodnewline
\GRANULEprodline{|}{ {\uparrow\! \GRANULEnt{A}  } }{}{}{}{\GRANULEcom{Promote a type to a kind}}\GRANULEprodnewline
\GRANULEprodline{|}{ \mathsf{Effect} }{}{}{}{\GRANULEcom{Effect grades}}\GRANULEprodnewline
\GRANULEprodline{|}{ \mathsf{Coeffect} }{}{}{}{\GRANULEcom{Coeffect grades}}\GRANULEprodnewline
\GRANULEprodline{|}{ \mathsf{Eff} }{}{}{}{}\GRANULEprodnewline
\GRANULEprodline{|}{ \mathsf{Coeff} }{}{}{}{}\GRANULEprodnewline
\GRANULEprodline{|}{ \mathsf{(Co)eff} }{}{}{}{}\GRANULEprodnewline
\GRANULEprodline{|}{ \mathsf{Predicate} }{}{}{}{\GRANULEcom{Predicates}}\GRANULEprodnewline
\GRANULEprodline{|}{\kappa_{{\mathrm{1}}}  \rightarrow  \kappa_{{\mathrm{2}}}}{}{}{}{\GRANULEcom{Kind function}}\GRANULEprodnewline
\GRANULEprodline{|}{ \kappa_{{\mathrm{1}}}  \cup  \kappa_{{\mathrm{2}}} }{}{}{}{}\GRANULEprodnewline
\GRANULEprodline{|}{ \theta \kappa }{}{}{}{\GRANULEcom{Substitutions}}\GRANULEprodnewline
\GRANULEprodline{|}{\GRANULEsym{(}  \kappa  \GRANULEsym{)}}{}{}{}{}}

\newcommand{\GRANULEDefines}{
\GRANULErulehead{D}{::=}{}\GRANULEprodnewline
\GRANULEfirstprodline{|}{ \emptyset }{}{}{}{\GRANULEcom{Empty}}\GRANULEprodnewline
\GRANULEprodline{|}{D_{{\mathrm{1}}}  \GRANULEsym{,}  D_{{\mathrm{2}}}}{}{}{}{}\GRANULEprodnewline
\GRANULEprodline{|}{\GRANULEsym{(}  D  \GRANULEsym{)}}{}{}{}{}}

\newcommand{\GRANULEAss}{
\GRANULErulehead{As}{::=}{}\GRANULEprodnewline
\GRANULEfirstprodline{|}{ \GRANULEmv{x}  :  C }{}{}{}{\GRANULEcom{Singleton context}}\GRANULEprodnewline
\GRANULEprodline{|}{ \GRANULEmv{x}  :_{\textcolor{coeffectColor}{  \GRANULEnt{r}  } }   C }{}{}{}{\GRANULEcom{Singleton context w/ graded assumption}}\GRANULEprodnewline
\GRANULEprodline{|}{ {  \GRANULEmv{x} ^ \GRANULEnt{Ix_{{\mathrm{1}}}} _ \GRANULEnt{Ix_{{\mathrm{2}}}}  }:_{\textcolor{coeffectColor}{  \GRANULEnt{c}  } }   C }{}{}{}{\GRANULEcom{Indexed Variable}}\GRANULEprodnewline
\GRANULEprodline{|}{ {  \GRANULEmv{x} ^ \GRANULEnt{Ix_{{\mathrm{1}}}}  }:_{\textcolor{coeffectColor}{  \GRANULEnt{c}  } }   C }{}{}{}{\GRANULEcom{Indexed Variable 1}}}

\newcommand{\GRANULEG}{
\GRANULErulehead{\Gamma  ,\ \Delta  ,\ \Omega}{::=}{}\GRANULEprodnewline
\GRANULEfirstprodline{|}{ \emptyset }{}{}{}{\GRANULEcom{Empty}}\GRANULEprodnewline
\GRANULEprodline{|}{As}{}{}{}{\GRANULEcom{Single assumption}}}

\newcommand{\GRANULECoeffInfo}{
\GRANULErulehead{\textit{r} :? \textit{R}}{::=}{}\GRANULEprodnewline
\GRANULEfirstprodline{|}{ \GRANULEnt{c} }{}{}{}{}\GRANULEprodnewline
\GRANULEprodline{|}{ - }{}{}{}{}\GRANULEprodnewline
\GRANULEprodline{|}{\GRANULEsym{(}  \textit{r} :? \textit{R}  \GRANULEsym{)}}{}{}{}{}\GRANULEprodnewline
\GRANULEprodline{|}{ \theta \textit{r} :? \textit{R} }{}{}{}{}}

\newcommand{\GRANULEP}{
\GRANULErulehead{P}{::=}{}\GRANULEprodnewline
\GRANULEfirstprodline{|}{ P_{{\mathrm{1}}}  \wedge  P_{{\mathrm{2}}} }{}{}{}{}\GRANULEprodnewline
\GRANULEprodline{|}{ P_{{\mathrm{1}}}  \wedge\!...\!\wedge  P_{\GRANULEmv{n}} }{}{}{}{}\GRANULEprodnewline
\GRANULEprodline{|}{ P_{{\mathrm{1}}}  \vee  P_{{\mathrm{2}}} }{}{}{}{}\GRANULEprodnewline
\GRANULEprodline{|}{ P_{{\mathrm{1}}}  \rightarrow  P_{{\mathrm{2}}} }{}{}{}{}\GRANULEprodnewline
\GRANULEprodline{|}{ \forall  \alpha  .  P }{}{}{}{}\GRANULEprodnewline
\GRANULEprodline{|}{ \neg  P }{}{}{}{}\GRANULEprodnewline
\GRANULEprodline{|}{ \exists  \alpha  .  P }{}{}{}{}\GRANULEprodnewline
\GRANULEprodline{|}{ \mathbf{t}_{{\mathrm{1}}}  \equiv  \mathbf{t}_{{\mathrm{2}}} }{}{}{}{}\GRANULEprodnewline
\GRANULEprodline{|}{ \mathbf{t}_{{\mathrm{1}}}  \sqsubseteq  \mathbf{t}_{{\mathrm{2}}} }{}{}{}{}\GRANULEprodnewline
\GRANULEprodline{|}{ \top }{}{}{}{}\GRANULEprodnewline
\GRANULEprodline{|}{\GRANULEsym{(}  P  \GRANULEsym{)}}{}{}{}{}\GRANULEprodnewline
\GRANULEprodline{|}{ \llbracket{ \theta }\rrbracket }{}{}{}{}\GRANULEprodnewline
\GRANULEprodline{|}{ \llbracket{ \textit{Cons} }\rrbracket }{}{}{}{}}

\newcommand{\GRANULEComp}{
\GRANULErulehead{\mathbf{t}}{::=}{}\GRANULEprodnewline
\GRANULEfirstprodline{|}{ \llbracket{\textcolor{coeffectColor}{ \GRANULEnt{c} } }\rrbracket }{}{}{}{}\GRANULEprodnewline
\GRANULEprodline{|}{ \llbracket{\textcolor{coeffectColor}{ \GRANULEnt{A}  } }\rrbracket }{}{}{}{}}

\newcommand{\GRANULETh}{
\GRANULErulehead{\theta  ,\ \theta_\kappa}{::=}{}\GRANULEprodnewline
\GRANULEfirstprodline{|}{ \emptyset }{}{}{}{\GRANULEcom{Empty}}\GRANULEprodnewline
\GRANULEprodline{|}{ \theta_{{\mathrm{1}}}  \uplus  \theta_{{\mathrm{2}}} }{}{}{}{\GRANULEcom{Union}}\GRANULEprodnewline
\GRANULEprodline{|}{\GRANULEmv{x}  \mapsto  \GRANULEnt{B}}{}{}{}{\GRANULEcom{SingletonTy}}\GRANULEprodnewline
\GRANULEprodline{|}{\GRANULEmv{x}  \mapsto  \kappa}{}{}{}{\GRANULEcom{SingletonKind}}\GRANULEprodnewline
\GRANULEprodline{|}{\GRANULEmv{x}  \mapsto  \GRANULEnt{c}}{}{}{}{\GRANULEcom{Singleton Coeffect}}\GRANULEprodnewline
\GRANULEprodline{|}{ \theta   \theta' }{}{}{}{\GRANULEcom{Substitution over a substitution}}\GRANULEprodnewline
\GRANULEprodline{|}{ \theta_{{\mathrm{1}}}  \uplus \ldots \uplus  \theta_{{\mathrm{2}}} }{}{}{}{}\GRANULEprodnewline
\GRANULEprodline{|}{\theta  \GRANULEsym{,}  \theta'}{}{}{}{\GRANULEcom{Disjoint cat}}\GRANULEprodnewline
\GRANULEprodline{|}{ \theta   \setminus   \GRANULEmv{x} }{}{}{}{\GRANULEcom{Remove a substition for a
    variable}}}

\newcommand{\GRANULECostDict}{
\GRANULErulehead{\mathcal{C}}{::=}{}\GRANULEprodnewline
\GRANULEfirstprodline{|}{\GRANULEmv{x}  \mapsto  \GRANULEnt{c}}{}{}{}{}\GRANULEprodnewline
\GRANULEprodline{|}{\mathcal{C}  \GRANULEsym{+}  \mathcal{C}'}{}{}{}{}\GRANULEprodnewline
\GRANULEprodline{|}{\mathcal{C}  \GRANULEsym{,}  \mathcal{C}'}{}{}{}{}\GRANULEprodnewline
\GRANULEprodline{|}{\GRANULEsym{(}  \mathcal{C}  \GRANULEsym{)}}{}{}{}{}\GRANULEprodnewline
\GRANULEprodline{|}{ \mathcal{C} }{}{}{}{}\GRANULEprodnewline
\GRANULEprodline{|}{ \emptyset }{}{}{}{}\GRANULEprodnewline
\GRANULEprodline{|}{ \textcolor{coeffectColor}{ \GRANULEnt{c} \textcolor{coeffectColor}{\,\cdot\,} } \mathcal{C} }{}{}{}{}}

\newcommand{\GRANULEgrammar}{\GRANULEgrammartabular{
\GRANULEFoc\GRANULEinterrule
\GRANULEt\GRANULEinterrule
\GRANULECon\GRANULEinterrule
\GRANULECases\GRANULEinterrule
\GRANULEp\GRANULEinterrule
\GRANULEIx\GRANULEinterrule
\GRANULEEqn\GRANULEinterrule
\GRANULEDef\GRANULEinterrule
\GRANULEA\GRANULEinterrule
\GRANULEop\GRANULEinterrule
\GRANULER\GRANULEinterrule
\GRANULEConstraints\GRANULEinterrule
\GRANULEc\GRANULEinterrule
\GRANULERel\GRANULEinterrule
\GRANULEk\GRANULEinterrule
\GRANULEDefines\GRANULEinterrule
\GRANULEAss\GRANULEinterrule
\GRANULEG\GRANULEinterrule
\GRANULECoeffInfo\GRANULEinterrule
\GRANULEP\GRANULEinterrule
\GRANULEComp\GRANULEinterrule
\GRANULETh\GRANULEinterrule
\GRANULECostDict\GRANULEafterlastrule
}}

% defnss
% defns Judgements
%% defn checkMiniDecl
\newcommand{\GRANULEdruleTyVarName}[0]{\GRANULEdrulename{TyVar}}
\newcommand{\GRANULEdruleTyVar}[1]{\GRANULEdrule[#1]{%
}{
   \textcolor{coeffectColor}{  0    \textcolor{coeffectColor}{\,\cdot\,} }  \Gamma    \GRANULEsym{,}   \GRANULEmv{x}  :_{\textcolor{coeffectColor}{   1   } }   \GRANULEnt{A}   \vdash  \GRANULEmv{x}  :  \GRANULEnt{A} }{%
{\GRANULEdruleTyVarName}{}%
}}


\newcommand{\GRANULEdruleTyAbsName}[0]{\GRANULEdrulename{TyAbs}}
\newcommand{\GRANULEdruleTyAbs}[1]{\GRANULEdrule[#1]{%
\GRANULEpremise{ \Gamma  \GRANULEsym{,}   \GRANULEmv{x}  :_{\textcolor{coeffectColor}{  \GRANULEnt{r}  } }   \GRANULEnt{A}   \vdash  \GRANULEnt{t}  :  \GRANULEnt{B} }%
}{
 \Gamma  \vdash   \lambda  \GRANULEmv{x}  .  \GRANULEnt{t}   :   \GRANULEnt{A} ^ \GRANULEnt{r}  \rightarrow  \GRANULEnt{B}  }{%
{\GRANULEdruleTyAbsName}{}%
}}


\newcommand{\GRANULEdruleTyAppName}[0]{\GRANULEdrulename{TyApp}}
\newcommand{\GRANULEdruleTyApp}[1]{\GRANULEdrule[#1]{%
\GRANULEpremise{  \Gamma_{{\mathrm{1}}}  \vdash  \GRANULEnt{t_{{\mathrm{1}}}}  :   \GRANULEnt{A} ^ \GRANULEnt{r}  \rightarrow  \GRANULEnt{B}    \quad\;\;   \Gamma_{{\mathrm{2}}}  \vdash  \GRANULEnt{t_{{\mathrm{2}}}}  :  \GRANULEnt{A}  }%
}{
 \Gamma_{{\mathrm{1}}}  \GRANULEsym{+}   \textcolor{coeffectColor}{ \GRANULEnt{r}   \textcolor{coeffectColor}{\,\cdot\,} }  \Gamma_{{\mathrm{2}}}   \vdash  \GRANULEnt{t_{{\mathrm{1}}}} \, \GRANULEnt{t_{{\mathrm{2}}}}  :  \GRANULEnt{B} }{%
{\GRANULEdruleTyAppName}{}%
}}


\newcommand{\GRANULEdruleTyConName}[0]{\GRANULEdrulename{TyCon}}
\newcommand{\GRANULEdruleTyCon}[1]{\GRANULEdrule[#1]{%
\GRANULEpremise{ (  C  :  \GRANULEnt{B_{{\mathrm{1}}}} ^ \GRANULEnt{q_{{\mathrm{1}}}}  \rightarrow ... \rightarrow  \GRANULEnt{B_{\GRANULEmv{n}}} ^ \GRANULEnt{q_{\GRANULEmv{n}}}  \rightarrow     K  \,   \vec{ \GRANULEnt{A} }     ) \in  D }%
}{
  \textcolor{coeffectColor}{  0    \textcolor{coeffectColor}{\,\cdot\,} }  \Gamma   \vdash   C   :    \GRANULEnt{B_{{\mathrm{1}}}} ^{q_1}  \rightarrow \! ... \! \rightarrow   \GRANULEnt{B_{\GRANULEmv{n}}} ^{q_n}  \rightarrow     K  \,   \vec{ \GRANULEnt{A} }      }{%
{\GRANULEdruleTyConName}{}%
}}


\newcommand{\GRANULEdruleTyPrName}[0]{\GRANULEdrulename{TyPr}}
\newcommand{\GRANULEdruleTyPr}[1]{\GRANULEdrule[#1]{%
\GRANULEpremise{ \Gamma  \vdash  \GRANULEnt{t}  :  \GRANULEnt{A} }%
}{
  \textcolor{coeffectColor}{ \GRANULEnt{r}   \textcolor{coeffectColor}{\,\cdot\,} }  \Gamma   \vdash  \GRANULEsym{[}  \GRANULEnt{t}  \GRANULEsym{]}  :   \Box_{  \GRANULEnt{r}  }  \GRANULEnt{A}  }{%
{\GRANULEdruleTyPrName}{}%
}}


\newcommand{\GRANULEdruleTyApproxName}[0]{\GRANULEdrulename{TyApprox}}
\newcommand{\GRANULEdruleTyApprox}[1]{\GRANULEdrule[#1]{%
\GRANULEpremise{  \Gamma  \GRANULEsym{,}   \GRANULEmv{x}  :_{\textcolor{coeffectColor}{  \GRANULEnt{r}  } }   \GRANULEnt{A}   \GRANULEsym{,}  \Gamma'  \vdash  \GRANULEnt{t}  :  \GRANULEnt{B}   \quad\;\;   \GRANULEnt{r}  \, \textcolor{coeffectColor}{\sqsubseteq} \,  \GRANULEnt{s}  }%
}{
 \Gamma  \GRANULEsym{,}   \GRANULEmv{x}  :_{\textcolor{coeffectColor}{  \GRANULEnt{s}  } }   \GRANULEnt{A}   \GRANULEsym{,}  \Gamma'  \vdash  \GRANULEnt{t}  :  \GRANULEnt{B} }{%
{\GRANULEdruleTyApproxName}{}%
}}


\newcommand{\GRANULEdruleTyCaseName}[0]{\GRANULEdrulename{TyCase}}
\newcommand{\GRANULEdruleTyCase}[1]{\GRANULEdrule[#1]{%
\GRANULEpremise{    \Gamma  \vdash  \GRANULEnt{t}  :  \GRANULEnt{A}   \quad\;\;   \GRANULEnt{r}  \vdash \,  \GRANULEnt{p_{\GRANULEmv{i}}}  :  \GRANULEnt{A}  \, \rhd \,  \Delta_{\GRANULEmv{i}}     \quad\;\;   \Gamma'  \GRANULEsym{,}  \Delta_{\GRANULEmv{i}}  \vdash  \GRANULEnt{t_{\GRANULEmv{i}}}  :  \GRANULEnt{B}  }%
}{
   \textcolor{coeffectColor}{ \GRANULEnt{r}   \textcolor{coeffectColor}{\,\cdot\,} }  \Gamma    \GRANULEsym{+}  \Gamma'  \vdash   \textbf{case} \  \GRANULEnt{t}  \ \textbf{of} \   \GRANULEnt{p_{{\mathrm{1}}}}  \mapsto  \GRANULEnt{t_{{\mathrm{1}}}} ; ... ;  \GRANULEnt{p_{\GRANULEmv{n}}}  \mapsto  \GRANULEnt{t_{\GRANULEmv{n}}}    :  \GRANULEnt{B} }{%
{\GRANULEdruleTyCaseName}{}%
}}

\newcommand{\GRANULEdefncheckMiniDecl}[1]{\begin{GRANULEdefnblock}[#1]{$ \Gamma  \vdash  \GRANULEnt{t}  :  \GRANULEnt{A} $}{\GRANULEcom{\section{Typing}}}
\GRANULEusedrule{\GRANULEdruleTyVar{}}
\GRANULEusedrule{\GRANULEdruleTyAbs{}}
\GRANULEusedrule{\GRANULEdruleTyApp{}}
\GRANULEusedrule{\GRANULEdruleTyCon{}}
\GRANULEusedrule{\GRANULEdruleTyPr{}}
\GRANULEusedrule{\GRANULEdruleTyApprox{}}
\GRANULEusedrule{\GRANULEdruleTyCase{}}
\end{GRANULEdefnblock}}

%% defn patternTyping
\newcommand{\GRANULEdrulePatWildName}[0]{\GRANULEdrulename{PatWild}}
\newcommand{\GRANULEdrulePatWild}[1]{\GRANULEdrule[#1]{%
\GRANULEpremise{  0   \, \textcolor{coeffectColor}{\sqsubseteq} \,  \GRANULEnt{r} }%
}{
 \GRANULEnt{r}  \vdash \,   \_   :  \GRANULEnt{A}  \, \rhd \,   \emptyset  }{%
{\GRANULEdrulePatWildName}{}%
}}


\newcommand{\GRANULEdrulePatVarName}[0]{\GRANULEdrulename{PatVar}}
\newcommand{\GRANULEdrulePatVar}[1]{\GRANULEdrule[#1]{%
}{
 \GRANULEnt{r}  \vdash \,  \GRANULEnt{p}  :  \GRANULEnt{A}  \, \rhd \,   \GRANULEmv{x}  :_{\textcolor{coeffectColor}{  \GRANULEnt{r}  } }   \GRANULEnt{A}  }{%
{\GRANULEdrulePatVarName}{}%
}}


\newcommand{\GRANULEdrulePatBoxName}[0]{\GRANULEdrulename{PatBox}}
\newcommand{\GRANULEdrulePatBox}[1]{\GRANULEdrule[#1]{%
\GRANULEpremise{ \GRANULEnt{r}  \textcolor{coeffectColor}{\,\cdot\,}  \GRANULEnt{s}  \vdash \,  \GRANULEnt{p}  :  \GRANULEnt{A}  \, \rhd \,  \Gamma }%
}{
 \GRANULEnt{r}  \vdash \,  \GRANULEsym{[}  \GRANULEnt{p}  \GRANULEsym{]}  :   \Box_{  \GRANULEnt{s}  }  \GRANULEnt{A}   \, \rhd \,  \Gamma }{%
{\GRANULEdrulePatBoxName}{}%
}}


\newcommand{\GRANULEdrulePatConName}[0]{\GRANULEdrulename{PatCon}}
\newcommand{\GRANULEdrulePatCon}[1]{\GRANULEdrule[#1]{%
\GRANULEpremise{ (  C  :  \GRANULEnt{B_{{\mathrm{1}}}} ^ \GRANULEnt{q_{{\mathrm{1}}}}  \rightarrow ... \rightarrow  \GRANULEnt{B_{\GRANULEmv{n}}} ^ \GRANULEnt{q_{\GRANULEmv{n}}}  \rightarrow     K  \,   \vec{ \GRANULEnt{A} }     ) \in  D }%
\GRANULEpremise{  \GRANULEnt{q_{\GRANULEmv{i}}}  \textcolor{coeffectColor}{\,\cdot\,}  \GRANULEnt{r}  \vdash \,  \GRANULEnt{p_{\GRANULEmv{i}}}  :  \GRANULEnt{B_{\GRANULEmv{i}}}  \, \rhd \,  \Gamma_{\GRANULEmv{i}}   \quad\;\;    |    K  \,   \vec{ \GRANULEnt{A} }     | > 1 \Rightarrow 1 \sqsubseteq  \GRANULEnt{r}   }%
}{
 \GRANULEnt{r}  \vdash \,  C \, \GRANULEnt{p_{{\mathrm{1}}}} \, ... \, \GRANULEnt{p_{\GRANULEmv{n}}}  :    K  \,   \vec{ \GRANULEnt{A} }     \, \rhd \,   \overrightarrow{ \Gamma_{\GRANULEmv{i}} }  }{%
{\GRANULEdrulePatConName}{}%
}}

\newcommand{\GRANULEdefnpatternTyping}[1]{\begin{GRANULEdefnblock}[#1]{$ \GRANULEnt{c}  \vdash \,  \GRANULEnt{p}  :  \GRANULEnt{A}  \, \rhd \,  \Gamma $}{\GRANULEcom{Declarative pattern checking for Granule Mini (monomorphic)}}
\GRANULEusedrule{\GRANULEdrulePatWild{}}
\GRANULEusedrule{\GRANULEdrulePatVar{}}
\GRANULEusedrule{\GRANULEdrulePatBox{}}
\GRANULEusedrule{\GRANULEdrulePatCon{}}
\end{GRANULEdefnblock}}

%% defn synthesiseSub

\newcommand{\GRANULEdefnsynthesiseSub}[1]{\begin{GRANULEdefnblock}[#1]{$ \Gamma_{{\mathrm{1}}}  \vdash  \GRANULEnt{A}  \Rightarrow^{-}  \GRANULEnt{t} \ |\  \Gamma_{{\mathrm{2}}} $}{}
\end{GRANULEdefnblock}}

%% defn SynthesiseAdd
\newcommand{\GRANULEdruleVarName}[0]{\GRANULEdrulename{Var}}
\newcommand{\GRANULEdruleVar}[1]{\GRANULEdrule[#1]{%
}{
 \Gamma  \GRANULEsym{,}   \GRANULEmv{x}  :_{\textcolor{coeffectColor}{  \GRANULEnt{r}  } }   \GRANULEnt{A}   \vdash  \GRANULEnt{A}  \Rightarrow  \GRANULEmv{x}  \mid    \textcolor{coeffectColor}{  0    \textcolor{coeffectColor}{\,\cdot\,} }  \Gamma    \GRANULEsym{,}   \GRANULEmv{x}  :_{\textcolor{coeffectColor}{   1   } }   \GRANULEnt{A}  }{%
{\GRANULEdruleVarName}{}%
}}


\newcommand{\GRANULEdruleAbsName}[0]{\GRANULEdrulename{Abs}}
\newcommand{\GRANULEdruleAbs}[1]{\GRANULEdrule[#1]{%
\GRANULEpremise{  \Gamma  \GRANULEsym{,}   \GRANULEmv{x}  :_{\textcolor{coeffectColor}{  \GRANULEnt{q}  } }   \GRANULEnt{A}   \vdash  \GRANULEnt{B}  \Rightarrow  \GRANULEnt{t}  \mid  \Delta  \GRANULEsym{,}   \GRANULEmv{x}  :_{\textcolor{coeffectColor}{  \GRANULEnt{r}  } }   \GRANULEnt{A}    \quad\;\;   \GRANULEnt{r}  \, \textcolor{coeffectColor}{\sqsubseteq} \,  \GRANULEnt{q}  }%
}{
 \Gamma  \vdash    \GRANULEnt{A} ^ \GRANULEnt{q}  \rightarrow  \GRANULEnt{B}    \Rightarrow   \lambda  \GRANULEmv{x}  .  \GRANULEnt{t}   \mid  \Delta }{%
{\GRANULEdruleAbsName}{}%
}}


\newcommand{\GRANULEdruleAppName}[0]{\GRANULEdrulename{App}}
\newcommand{\GRANULEdruleApp}[1]{\GRANULEdrule[#1]{%
\GRANULEpremise{ \Gamma  \GRANULEsym{,}   \GRANULEmv{x_{{\mathrm{1}}}}  :_{\textcolor{coeffectColor}{  \GRANULEnt{r_{{\mathrm{1}}}}  } }    \GRANULEnt{A} ^ \GRANULEnt{q}  \rightarrow  \GRANULEnt{B}    \GRANULEsym{,}   \GRANULEmv{x_{{\mathrm{2}}}}  :_{\textcolor{coeffectColor}{  \GRANULEnt{r_{{\mathrm{1}}}}  } }   \GRANULEnt{B}   \vdash  \GRANULEnt{C}  \Rightarrow  \GRANULEnt{t_{{\mathrm{1}}}}  \mid  \Delta_{{\mathrm{1}}}  \GRANULEsym{,}   \GRANULEmv{x_{{\mathrm{1}}}}  :_{\textcolor{coeffectColor}{  \GRANULEnt{s_{{\mathrm{1}}}}  } }    \GRANULEnt{A} ^ \GRANULEnt{q}  \rightarrow  \GRANULEnt{B}    \GRANULEsym{,}   \GRANULEmv{x_{{\mathrm{2}}}}  :_{\textcolor{coeffectColor}{  \GRANULEnt{s_{{\mathrm{2}}}}  } }   \GRANULEnt{B}  }%
\GRANULEpremise{ \Gamma  \GRANULEsym{,}   \GRANULEmv{x_{{\mathrm{1}}}}  :_{\textcolor{coeffectColor}{  \GRANULEnt{r_{{\mathrm{1}}}}  } }    \GRANULEnt{A} ^ \GRANULEnt{q}  \rightarrow  \GRANULEnt{B}    \vdash  \GRANULEnt{A}  \Rightarrow  \GRANULEnt{t_{{\mathrm{2}}}}  \mid  \Delta_{{\mathrm{2}}}  \GRANULEsym{,}   \GRANULEmv{x_{{\mathrm{1}}}}  :_{\textcolor{coeffectColor}{  \GRANULEnt{s_{{\mathrm{3}}}}  } }    \GRANULEnt{A} ^ \GRANULEnt{q}  \rightarrow  \GRANULEnt{B}   }%
}{
 \Gamma  \GRANULEsym{,}   \GRANULEmv{x_{{\mathrm{1}}}}  :_{\textcolor{coeffectColor}{  \GRANULEnt{r_{{\mathrm{1}}}}  } }    \GRANULEnt{A} ^ \GRANULEnt{q}  \rightarrow  \GRANULEnt{B}    \vdash  \GRANULEnt{C}  \Rightarrow    [  \GRANULEsym{(}  \GRANULEmv{x_{{\mathrm{1}}}} \, \GRANULEnt{t_{{\mathrm{2}}}}  \GRANULEsym{)}  /  \GRANULEmv{x_{{\mathrm{2}}}}  ]  \GRANULEnt{t_{{\mathrm{1}}}}    \mid  \GRANULEsym{(}  \Delta_{{\mathrm{1}}}  \GRANULEsym{+}   \textcolor{coeffectColor}{ \GRANULEnt{s_{{\mathrm{2}}}}   \textcolor{coeffectColor}{\,\cdot\,} }    \textcolor{coeffectColor}{ \GRANULEnt{q}   \textcolor{coeffectColor}{\,\cdot\,} }  \Delta_{{\mathrm{2}}}     \GRANULEsym{)}  \GRANULEsym{,}   \GRANULEmv{x_{{\mathrm{1}}}}  :_{\textcolor{coeffectColor}{   \GRANULEnt{s_{{\mathrm{2}}}}  \GRANULEsym{+}   \GRANULEnt{s_{{\mathrm{1}}}}  \GRANULEsym{+}   \GRANULEsym{(}  \GRANULEnt{s_{{\mathrm{2}}}}  \textcolor{coeffectColor}{\,\cdot\,}   \GRANULEnt{q}  \textcolor{coeffectColor}{\,\cdot\,}  \GRANULEnt{s_{{\mathrm{3}}}}   \GRANULEsym{)}     } }    \GRANULEnt{A} ^ \GRANULEnt{q}  \rightarrow  \GRANULEnt{B}   }{%
{\GRANULEdruleAppName}{}%
}}


\newcommand{\GRANULEdruleBoxName}[0]{\GRANULEdrulename{Box}}
\newcommand{\GRANULEdruleBox}[1]{\GRANULEdrule[#1]{%
\GRANULEpremise{ \Gamma  \vdash  \GRANULEnt{A}  \Rightarrow  \GRANULEnt{t}  \mid  \Delta }%
}{
 \Gamma  \vdash    \Box_{  \GRANULEnt{r}  }  \GRANULEnt{A}    \Rightarrow  \GRANULEsym{[}  \GRANULEnt{t}  \GRANULEsym{]}  \mid   \textcolor{coeffectColor}{ \GRANULEnt{r}   \textcolor{coeffectColor}{\,\cdot\,} }  \Delta  }{%
{\GRANULEdruleBoxName}{}%
}}


\newcommand{\GRANULEdruleConName}[0]{\GRANULEdrulename{Con}}
\newcommand{\GRANULEdruleCon}[1]{\GRANULEdrule[#1]{%
\GRANULEpremise{ (  C  :  \GRANULEnt{B_{{\mathrm{1}}}} ^ \GRANULEnt{q_{{\mathrm{1}}}}  \rightarrow ... \rightarrow  \GRANULEnt{B_{\GRANULEmv{n}}} ^ \GRANULEnt{q_{\GRANULEmv{n}}}  \rightarrow     K  \,   \vec{ \GRANULEnt{A} }     ) \in  D }%
\GRANULEpremise{ \Gamma  \vdash  \GRANULEnt{B_{\GRANULEmv{i}}}  \Rightarrow  \GRANULEnt{t_{\GRANULEmv{i}}}  \mid  \Delta_{\GRANULEmv{i}} }%
}{
 \Gamma  \vdash   K  \,   \vec{ \GRANULEnt{A} }    \Rightarrow  C \, \GRANULEnt{t_{{\mathrm{1}}}} \, ... \, \GRANULEnt{t_{\GRANULEmv{n}}}  \mid    \textcolor{coeffectColor}{  0    \textcolor{coeffectColor}{\,\cdot\,} }  \Gamma    \GRANULEsym{+}   \GRANULEsym{(}   \textcolor{coeffectColor}{ \GRANULEnt{q_{{\mathrm{1}}}}   \textcolor{coeffectColor}{\,\cdot\,} }  \Delta_{{\mathrm{1}}}   \GRANULEsym{)}  \GRANULEsym{+} \, ... \, \GRANULEsym{+}  \GRANULEsym{(}   \textcolor{coeffectColor}{ \GRANULEnt{q_{\GRANULEmv{n}}}   \textcolor{coeffectColor}{\,\cdot\,} }  \Delta_{\GRANULEmv{n}}   \GRANULEsym{)}  }{%
{\GRANULEdruleConName}{}%
}}


\newcommand{\GRANULEdruleConAltName}[0]{\GRANULEdrulename{ConAlt}}
\newcommand{\GRANULEdruleConAlt}[1]{\GRANULEdrule[#1]{%
\GRANULEpremise{ (  C  :  \GRANULEnt{B_{{\mathrm{1}}}} ^ \GRANULEnt{r_{{\mathrm{1}}}}  \rightarrow ... \rightarrow  \GRANULEnt{B_{\GRANULEmv{n}}} ^ \GRANULEnt{r_{\GRANULEmv{n}}}  \rightarrow     K  \, \GRANULEnt{A}   ) \in  D }%
\GRANULEpremise{ \Gamma  \GRANULEsym{,}   \GRANULEmv{x}  :_{\textcolor{coeffectColor}{   1   } }    \GRANULEnt{B_{{\mathrm{1}}}} ^{q_1}  \rightarrow \! ... \! \rightarrow   \GRANULEnt{B_{\GRANULEmv{n}}} ^{q_n}  \rightarrow     K  \, \GRANULEnt{A}     \vdash  \GRANULEnt{A_{{\mathrm{1}}}}  \Rightarrow  \GRANULEnt{t}  \mid  \Delta  \GRANULEsym{,}   \GRANULEmv{x}  :_{\textcolor{coeffectColor}{   1   } }    \GRANULEnt{B_{{\mathrm{1}}}} ^{q_1}  \rightarrow \! ... \! \rightarrow   \GRANULEnt{B_{\GRANULEmv{n}}} ^{q_n}  \rightarrow     K  \, \GRANULEnt{A}    }%
}{
 \Gamma  \vdash  \GRANULEnt{A_{{\mathrm{1}}}}  \Rightarrow  \GRANULEnt{t}  \mid  \Delta }{%
{\GRANULEdruleConAltName}{}%
}}


\newcommand{\GRANULEdruleCaseName}[0]{\GRANULEdrulename{Case}}
\newcommand{\GRANULEdruleCase}[1]{\GRANULEdrule[#1]{%
\GRANULEpremise{ (  C_{\GRANULEmv{i}}  :  \GRANULEnt{B_{{\mathrm{1}}}} ^{q_1^i} \rightarrow ... \rightarrow  \GRANULEnt{B_{\GRANULEmv{n}}} ^{q_n^i} \rightarrow     K  \,   \vec{ \GRANULEnt{A} }     ) \in  D }%
\GRANULEpremise{    \Gamma  \GRANULEsym{,}   \GRANULEmv{x}  :_{\textcolor{coeffectColor}{   \GRANULEnt{r}   } }    K  \, \GRANULEnt{A}    \GRANULEsym{,}   {  \GRANULEmv{y} ^  i  _  1   }:_{\textcolor{coeffectColor}{     \GRANULEnt{q} ^  i  _  1      } }   \GRANULEnt{B_{{\mathrm{1}}}}    , ... ,   {  \GRANULEmv{y} ^  i  _  n   }:_{\textcolor{coeffectColor}{     \GRANULEnt{q} ^  i  _  n      } }   \GRANULEnt{B_{\GRANULEmv{n}}}    \vdash  \GRANULEnt{B}  \Rightarrow^+  \GRANULEnt{t_{\GRANULEmv{i}}}  ;\,     \Delta_{\GRANULEmv{i}}  \GRANULEsym{,}   \GRANULEmv{x}  :_{\textcolor{coeffectColor}{   \GRANULEnt{r_{\GRANULEmv{i}}}   } }    K  \, \GRANULEnt{A}    \GRANULEsym{,}   {  \GRANULEmv{y} ^  i  _  1   }:_{\textcolor{coeffectColor}{    \GRANULEnt{s} ^  i  _  1     } }   \GRANULEnt{B_{{\mathrm{1}}}}    , ... ,   {  \GRANULEmv{y} ^  i  _  n   }:_{\textcolor{coeffectColor}{    \GRANULEnt{s} ^  i  _  n     } }   \GRANULEnt{B_{\GRANULEmv{n}}}   }%
\GRANULEpremise{   \GRANULEnt{s} ^  i  _  j     \, \textcolor{coeffectColor}{\sqsubseteq} \,    \GRANULEnt{q} ^  i  _  j    }%
\GRANULEpremise{   \GRANULEnt{s} _  i     =      \GRANULEnt{s} ^  i  _  1     \sqcup ... \sqcup    \GRANULEnt{s} ^  i  _  n      }%
}{
 \Gamma  \GRANULEsym{,}   \GRANULEmv{x}  :_{\textcolor{coeffectColor}{  \GRANULEnt{r}  } }     K  \, \GRANULEnt{A}    \vdash  \GRANULEnt{B}  \Rightarrow   \textbf{case} \  \GRANULEmv{x}  \ \textbf{of} \   \overline{   C_{\GRANULEmv{i}}  \  \GRANULEmv{y} ^  i  _  1   ...  \GRANULEmv{y} ^  i  _  n    \mapsto  \GRANULEnt{t_{\GRANULEmv{i}}} }    \mid  \GRANULEsym{(}   \Delta_{{\mathrm{1}}}  \sqcup ...  \sqcup  \Delta_{\GRANULEmv{n}}   \GRANULEsym{)}  \GRANULEsym{,}    \GRANULEmv{x}  :_{\textcolor{coeffectColor}{   \GRANULEsym{(}     \GRANULEnt{r} _  1     \sqcup ... \sqcup    \GRANULEnt{r} _  n      \GRANULEsym{)}   \GRANULEsym{+}   \GRANULEsym{(}     \GRANULEnt{s} _  1     \sqcup ... \sqcup    \GRANULEnt{s} _  n      \GRANULEsym{)}   } }     K  \, \GRANULEnt{A}    }{%
{\GRANULEdruleCaseName}{}%
}}


\newcommand{\GRANULEdruleCaseAltName}[0]{\GRANULEdrulename{CaseAlt}}
\newcommand{\GRANULEdruleCaseAlt}[1]{\GRANULEdrule[#1]{%
\GRANULEpremise{ (  C_{\GRANULEmv{i}}  :  \GRANULEnt{B_{{\mathrm{1}}}} ^{q_1^i} \rightarrow ... \rightarrow  \GRANULEnt{B_{\GRANULEmv{n}}} ^{q_n^i} \rightarrow     K  \,   \vec{ \GRANULEnt{A} }     ) \in  D }%
\GRANULEpremise{    \Gamma  \GRANULEsym{,}   \GRANULEmv{x}  :_{\textcolor{coeffectColor}{   \GRANULEnt{r}   } }    K  \,   \vec{ \GRANULEnt{A} }      \GRANULEsym{,}   {  \GRANULEmv{y} ^  i  _  1   }:_{\textcolor{coeffectColor}{   \GRANULEnt{r}  \textcolor{coeffectColor}{\,\cdot\,}    \GRANULEnt{q} ^  i  _  1      } }   \GRANULEnt{B_{{\mathrm{1}}}}    , ... ,   {  \GRANULEmv{y} ^  i  _  n   }:_{\textcolor{coeffectColor}{   \GRANULEnt{r}  \textcolor{coeffectColor}{\,\cdot\,}    \GRANULEnt{q} ^  i  _  1      } }   \GRANULEnt{B_{\GRANULEmv{n}}}    \vdash  \GRANULEnt{B}  \Rightarrow  \GRANULEnt{t_{\GRANULEmv{i}}}  \mid     \Delta_{\GRANULEmv{i}}  \GRANULEsym{,}   \GRANULEmv{x}  :_{\textcolor{coeffectColor}{   \GRANULEnt{r_{\GRANULEmv{i}}}   } }    K  \,   \vec{ \GRANULEnt{A} }      \GRANULEsym{,}   {  \GRANULEmv{y} ^  i  _  1   }:_{\textcolor{coeffectColor}{    \GRANULEnt{s} ^  i  _  1     } }   \GRANULEnt{B_{{\mathrm{1}}}}    , ... ,   {  \GRANULEmv{y} ^  i  _  n   }:_{\textcolor{coeffectColor}{    \GRANULEnt{s} ^  i  _  n     } }   \GRANULEnt{B_{\GRANULEmv{n}}}   }%
\GRANULEpremise{ \exists    \GRANULEnt{s'} ^  i  _  j     .\,     \GRANULEnt{s} ^  i  _  j     \sqsubseteq    \GRANULEnt{s'} ^  i  _  j     \textcolor{coeffectColor}{\,\cdot\,}    \GRANULEnt{q} ^  i  _  j     \sqsubseteq  \GRANULEnt{r}  \textcolor{coeffectColor}{\,\cdot\,}    \GRANULEnt{q} ^  i  _  j     }%
\GRANULEpremise{   \GRANULEnt{s} _  i     =      \GRANULEnt{s'} ^  i  _  1     \sqcup ... \sqcup    \GRANULEnt{s'} ^  i  _  n      }%
\GRANULEpremise{  |    K  \,   \vec{ \GRANULEnt{A} }     | > 1 \Rightarrow 1 \sqsubseteq      \GRANULEnt{s} _  1     \sqcup ... \sqcup    \GRANULEnt{s} _  m       }%
}{
 \Gamma  \GRANULEsym{,}   \GRANULEmv{x}  :_{\textcolor{coeffectColor}{  \GRANULEnt{r}  } }     K  \,   \vec{ \GRANULEnt{A} }      \vdash  \GRANULEnt{B}  \Rightarrow   \textbf{case} \  \GRANULEmv{x}  \ \textbf{of} \   \overline{   C_{\GRANULEmv{i}}  \  \GRANULEmv{y} ^  i  _  1   ...  \GRANULEmv{y} ^  i  _  n    \mapsto  \GRANULEnt{t_{\GRANULEmv{i}}} }    \mid  \GRANULEsym{(}   \Delta_{{\mathrm{1}}}  \sqcup ...  \sqcup  \Delta_{\GRANULEmv{m}}   \GRANULEsym{)}  \GRANULEsym{,}    \GRANULEmv{x}  :_{\textcolor{coeffectColor}{   \GRANULEsym{(}     \GRANULEnt{r} _  1     \sqcup ... \sqcup    \GRANULEnt{r} _  m      \GRANULEsym{)}   \GRANULEsym{+}   \GRANULEsym{(}     \GRANULEnt{s} _  1     \sqcup ... \sqcup    \GRANULEnt{s} _  m      \GRANULEsym{)}   } }     K  \,   \vec{ \GRANULEnt{A} }      }{%
{\GRANULEdruleCaseAltName}{}%
}}


\newcommand{\GRANULEdruleCaseAltAltName}[0]{\GRANULEdrulename{CaseAltAlt}}
\newcommand{\GRANULEdruleCaseAltAlt}[1]{\GRANULEdrule[#1]{%
\GRANULEpremise{ (  C_{\GRANULEmv{i}}  :  \GRANULEnt{B_{{\mathrm{1}}}} ^{q_1^i} \rightarrow ... \rightarrow  \GRANULEnt{B_{\GRANULEmv{n}}} ^{q_n^i} \rightarrow     K  \, \GRANULEnt{A}   ) \in  D }%
\GRANULEpremise{    \Gamma  \GRANULEsym{,}   \GRANULEmv{x}  :_{\textcolor{coeffectColor}{   \GRANULEnt{r}   } }    K  \, \GRANULEnt{A}    \GRANULEsym{,}   {  \GRANULEmv{y} ^  i  _  1   }:_{\textcolor{coeffectColor}{   \GRANULEnt{r}  \textcolor{coeffectColor}{\,\cdot\,}    \GRANULEnt{q} ^  i  _  1      } }   \GRANULEnt{B_{{\mathrm{1}}}}    , ... ,   {  \GRANULEmv{y} ^  i  _  n   }:_{\textcolor{coeffectColor}{   \GRANULEnt{r}  \textcolor{coeffectColor}{\,\cdot\,}    \GRANULEnt{q} ^  i  _  1      } }   \GRANULEnt{B_{\GRANULEmv{n}}}    \vdash  \GRANULEnt{B}  \Rightarrow  \GRANULEnt{t_{\GRANULEmv{i}}}  \mid     \Delta_{\GRANULEmv{i}}  \GRANULEsym{,}   \GRANULEmv{x}  :_{\textcolor{coeffectColor}{   \GRANULEnt{r_{\GRANULEmv{i}}}   } }    K  \, \GRANULEnt{A}    \GRANULEsym{,}   {  \GRANULEmv{y} ^  i  _  1   }:_{\textcolor{coeffectColor}{    \GRANULEnt{s} ^  i  _  1     } }   \GRANULEnt{B_{{\mathrm{1}}}}    , ... ,   {  \GRANULEmv{y} ^  i  _  n   }:_{\textcolor{coeffectColor}{    \GRANULEnt{s} ^  i  _  n     } }   \GRANULEnt{B_{\GRANULEmv{n}}}   }%
\GRANULEpremise{   \GRANULEnt{s} _  i     =      \GRANULEnt{s} ^  i  _  1     \sqcup ... \sqcup    \GRANULEnt{s} ^  i  _  n      }%
\GRANULEpremise{   \GRANULEnt{q} _  i     =      \GRANULEnt{q} ^  i  _  1     \sqcup ... \sqcup    \GRANULEnt{q} ^  i  _  n      }%
\GRANULEpremise{ \exists    \GRANULEnt{s'} _  i     .\,     \GRANULEnt{s} _  i     \sqsubseteq    \GRANULEnt{s'} _  i     \textcolor{coeffectColor}{\,\cdot\,}    \GRANULEnt{q} _  i     \sqsubseteq  \GRANULEnt{r}  \textcolor{coeffectColor}{\,\cdot\,}    \GRANULEnt{q} _  i     }%
}{
 \Gamma  \GRANULEsym{,}   \GRANULEmv{x}  :_{\textcolor{coeffectColor}{  \GRANULEnt{r}  } }     K  \, \GRANULEnt{A}    \vdash  \GRANULEnt{B}  \Rightarrow   \textbf{case} \  \GRANULEmv{x}  \ \textbf{of} \   \overline{   C_{\GRANULEmv{i}}  \  \GRANULEmv{y} ^  i  _  1   ...  \GRANULEmv{y} ^  i  _  n    \mapsto  \GRANULEnt{t_{\GRANULEmv{i}}} }    \mid  \GRANULEsym{(}   \Delta_{{\mathrm{1}}}  \sqcup ...  \sqcup  \Delta_{\GRANULEmv{n}}   \GRANULEsym{)}  \GRANULEsym{,}    \GRANULEmv{x}  :_{\textcolor{coeffectColor}{   \GRANULEsym{(}     \GRANULEnt{r} _  1     \sqcup ... \sqcup    \GRANULEnt{r} _  n      \GRANULEsym{)}   \GRANULEsym{+}   \GRANULEsym{(}     \GRANULEnt{s'} _  1     \sqcup ... \sqcup    \GRANULEnt{s'} _  n      \GRANULEsym{)}   } }     K  \, \GRANULEnt{A}    }{%
{\GRANULEdruleCaseAltAltName}{}%
}}


\newcommand{\GRANULEdruleUnboxName}[0]{\GRANULEdrulename{Unbox}}
\newcommand{\GRANULEdruleUnbox}[1]{\GRANULEdrule[#1]{%
\GRANULEpremise{ \Gamma  \GRANULEsym{,}   \GRANULEmv{y}  :_{\textcolor{coeffectColor}{   \GRANULEnt{r}  \textcolor{coeffectColor}{\,\cdot\,}  \GRANULEnt{q}   } }   \GRANULEnt{A}   \GRANULEsym{,}   \GRANULEmv{x}  :_{\textcolor{coeffectColor}{  \GRANULEnt{r}  } }    \Box_{  \GRANULEnt{q}  }  \GRANULEnt{A}    \vdash  \GRANULEnt{B}  \Rightarrow  \GRANULEnt{t}  \mid  \Delta  \GRANULEsym{,}   \GRANULEmv{y}  :_{\textcolor{coeffectColor}{   \GRANULEnt{s_{{\mathrm{1}}}}   } }   \GRANULEnt{A}   \GRANULEsym{,}   \GRANULEmv{x}  :_{\textcolor{coeffectColor}{  \GRANULEnt{s_{{\mathrm{2}}}}  } }    \Box_{  \GRANULEnt{q}  }  \GRANULEnt{A}   }%
\GRANULEpremise{ \exists  \GRANULEnt{s_{{\mathrm{3}}}}  .\,   \GRANULEnt{s_{{\mathrm{1}}}}  \sqsubseteq   \GRANULEnt{s_{{\mathrm{3}}}}  \textcolor{coeffectColor}{\,\cdot\,}  \GRANULEnt{q}   \sqsubseteq   \GRANULEnt{r}  \textcolor{coeffectColor}{\,\cdot\,}  \GRANULEnt{q}   }%
}{
 \Gamma  \GRANULEsym{,}   \GRANULEmv{x}  :_{\textcolor{coeffectColor}{  \GRANULEnt{r}  } }    \Box_{  \GRANULEnt{q}  }  \GRANULEnt{A}    \vdash  \GRANULEnt{B}  \Rightarrow   \textbf{case} \  \GRANULEmv{x}  \ \textbf{of} \  \GRANULEsym{[}  \GRANULEmv{y}  \GRANULEsym{]}  \rightarrow  \GRANULEnt{t}   \mid   \Delta   \GRANULEsym{,}   \GRANULEmv{x}  :_{\textcolor{coeffectColor}{   \GRANULEnt{s_{{\mathrm{3}}}}  \GRANULEsym{+}  \GRANULEnt{s_{{\mathrm{2}}}}   } }    \Box_{  \GRANULEnt{q}  }  \GRANULEnt{A}   }{%
{\GRANULEdruleUnboxName}{}%
}}


\newcommand{\GRANULEdruleCaseSubName}[0]{\GRANULEdrulename{CaseSub}}
\newcommand{\GRANULEdruleCaseSub}[1]{\GRANULEdrule[#1]{%
\GRANULEpremise{ (  C_{\GRANULEmv{i}}  :  \GRANULEnt{B_{{\mathrm{1}}}} ^ \GRANULEnt{q_{{\mathrm{1}}}}  \rightarrow ... \rightarrow  \GRANULEnt{B_{\GRANULEmv{n}}} ^ \GRANULEnt{q_{\GRANULEmv{n}}}  \rightarrow     K  \, \GRANULEnt{A}   ) \in  D }%
\GRANULEpremise{    \Gamma  \GRANULEsym{,}   \GRANULEmv{x}  :_{\textcolor{coeffectColor}{   \GRANULEnt{r}   } }     K  \, \GRANULEnt{A}     \GRANULEsym{,}   {  \GRANULEmv{y} ^  i  _  1   }:_{\textcolor{coeffectColor}{   \GRANULEnt{r}  \textcolor{coeffectColor}{\,\cdot\,}  \GRANULEnt{q_{{\mathrm{1}}}}   } }   \GRANULEnt{B_{{\mathrm{1}}}}    , ... ,   {  \GRANULEmv{y} ^  i  _  n   }:_{\textcolor{coeffectColor}{   \GRANULEnt{r}  \textcolor{coeffectColor}{\,\cdot\,}  \GRANULEnt{q_{\GRANULEmv{n}}}   } }   \GRANULEnt{B_{\GRANULEmv{n}}}    \vdash  \GRANULEnt{B}  \Rightarrow  \GRANULEnt{t_{\GRANULEmv{i}}}  \mid     \Delta_{\GRANULEmv{i}}  \GRANULEsym{,}   \GRANULEmv{x}  :_{\textcolor{coeffectColor}{    \GRANULEnt{r} ^  i  _  1     } }    K  \, \GRANULEnt{A}    \GRANULEsym{,}   {  \GRANULEmv{y} ^  i  _  1   }:_{\textcolor{coeffectColor}{    \GRANULEnt{s} ^  i  _  1     } }   \GRANULEnt{B_{{\mathrm{1}}}}    , ... ,   {  \GRANULEmv{y} ^  i  _  n   }:_{\textcolor{coeffectColor}{    \GRANULEnt{s} ^  i  _  n     } }   \GRANULEnt{B_{\GRANULEmv{n}}}   }%
\GRANULEpremise{ \exists    \GRANULEnt{r} ^  i  _  2     .\,   \GRANULEnt{r}  \sqsupseteq    \GRANULEnt{r} ^  i  _  2     \GRANULEsym{+}     \GRANULEnt{r} ^  i  _  1     \textcolor{coeffectColor}{\,\cdot\,}  \GRANULEsym{(}     \GRANULEnt{s} ^  i  _  1     \sqcap ... \sqcap    \GRANULEnt{s} ^  n  _  1      \GRANULEsym{)}   }%
}{
 \Gamma  \GRANULEsym{,}   \GRANULEmv{x}  :_{\textcolor{coeffectColor}{  \GRANULEnt{r_{{\mathrm{1}}}}  } }     K  \, \GRANULEnt{A}    \vdash  \GRANULEnt{B}  \Rightarrow   \textbf{case} \  \GRANULEmv{x}  \ \textbf{of} \   \overline{   C_{\GRANULEmv{i}}  \  \GRANULEmv{y} ^  i  _  1   ...  \GRANULEmv{y} ^  i  _  n    \mapsto  \GRANULEnt{t_{\GRANULEmv{i}}} }    \mid  \GRANULEsym{(}   \Delta_{{\mathrm{1}}}  \sqcap ... \sqcap  \Delta_{\GRANULEmv{n}}   \GRANULEsym{)}  \GRANULEsym{,}   \GRANULEmv{x}  :_{\textcolor{coeffectColor}{      \GRANULEnt{r} ^  1  _  2     \sqcap ... \sqcap    \GRANULEnt{r} ^  n  _  2       } }     K  \, \GRANULEnt{A}   }{%
{\GRANULEdruleCaseSubName}{}%
}}

\newcommand{\GRANULEdefnSynthesiseAdd}[1]{\begin{GRANULEdefnblock}[#1]{$ \Gamma  \vdash  \GRANULEnt{A}  \Rightarrow  \GRANULEnt{t}  \mid  \Delta $}{}
\GRANULEusedrule{\GRANULEdruleVar{}}
\GRANULEusedrule{\GRANULEdruleAbs{}}
\GRANULEusedrule{\GRANULEdruleApp{}}
\GRANULEusedrule{\GRANULEdruleBox{}}
\GRANULEusedrule{\GRANULEdruleCon{}}
\GRANULEusedrule{\GRANULEdruleConAlt{}}
\GRANULEusedrule{\GRANULEdruleCase{}}
\GRANULEusedrule{\GRANULEdruleCaseAlt{}}
\GRANULEusedrule{\GRANULEdruleCaseAltAlt{}}
\GRANULEusedrule{\GRANULEdruleUnbox{}}
\GRANULEusedrule{\GRANULEdruleCaseSub{}}
\end{GRANULEdefnblock}}

%% defn RASynFocus
\newcommand{\GRANULEdruleAbsFName}[0]{\GRANULEdrulename{AbsF}}
\newcommand{\GRANULEdruleAbsF}[1]{\GRANULEdrule[#1]{%
\GRANULEpremise{  \Gamma  ;  \Omega  \GRANULEsym{,}   \GRANULEmv{x}  :_{\textcolor{coeffectColor}{  \GRANULEnt{q}  } }   \GRANULEnt{A}   \vdash  \GRANULEnt{B}  \Uparrow \Rightarrow  \GRANULEnt{t}  \mid  \Delta  \GRANULEsym{,}   \GRANULEmv{x}  :_{\textcolor{coeffectColor}{  \GRANULEnt{r}  } }   \GRANULEnt{A}    \quad\;\;   \GRANULEnt{r}  \, \textcolor{coeffectColor}{\sqsubseteq} \,  \GRANULEnt{q}  }%
}{
 \Gamma  ;  \Omega  \vdash    \GRANULEnt{A} ^ \GRANULEnt{q}  \rightarrow  \GRANULEnt{B}    \Uparrow \Rightarrow   \lambda  \GRANULEmv{x}  .  \GRANULEnt{t}   \mid  \Delta }{%
{\GRANULEdruleAbsFName}{}%
}}

\newcommand{\GRANULEdefnRASynFocus}[1]{\begin{GRANULEdefnblock}[#1]{$ \Gamma  ;  \Omega  \vdash  \GRANULEnt{A}  \Uparrow \Rightarrow  \GRANULEnt{t}  \mid  \Delta $}{}
\GRANULEusedrule{\GRANULEdruleAbsF{}}
\end{GRANULEdefnblock}}

%% defn LASynFocus
\newcommand{\GRANULEdruleCaseFName}[0]{\GRANULEdrulename{CaseF}}
\newcommand{\GRANULEdruleCaseF}[1]{\GRANULEdrule[#1]{%
\GRANULEpremise{ (  C_{\GRANULEmv{i}}  :  \GRANULEnt{B_{{\mathrm{1}}}} ^{q_1^i} \rightarrow ... \rightarrow  \GRANULEnt{B_{\GRANULEmv{n}}} ^{q_n^i} \rightarrow     K  \,   \vec{ \GRANULEnt{A} }     ) \in  D }%
\GRANULEpremise{ \Gamma  ;  \Omega  \GRANULEsym{,}       \GRANULEmv{x}  :_{\textcolor{coeffectColor}{   \GRANULEnt{r}   } }    K  \,   \vec{ \GRANULEnt{A} }      \GRANULEsym{,}   {  \GRANULEmv{y} ^  i  _  1   }:_{\textcolor{coeffectColor}{   \GRANULEnt{r}  \textcolor{coeffectColor}{\,\cdot\,}    \GRANULEnt{q} ^  i  _  1      } }   \GRANULEnt{B_{{\mathrm{1}}}}    , ... ,   {  \GRANULEmv{y} ^  i  _  n   }:_{\textcolor{coeffectColor}{   \GRANULEnt{r}  \textcolor{coeffectColor}{\,\cdot\,}    \GRANULEnt{q} ^  i  _  1      } }   \GRANULEnt{B_{\GRANULEmv{n}}}     \Uparrow \vdash  \GRANULEnt{B}  \Rightarrow  \GRANULEnt{t_{\GRANULEmv{i}}}  \mid     \Delta_{\GRANULEmv{i}}  \GRANULEsym{,}   \GRANULEmv{x}  :_{\textcolor{coeffectColor}{   \GRANULEnt{r_{\GRANULEmv{i}}}   } }    K  \,   \vec{ \GRANULEnt{A} }      \GRANULEsym{,}   {  \GRANULEmv{y} ^  i  _  1   }:_{\textcolor{coeffectColor}{    \GRANULEnt{s} ^  i  _  1     } }   \GRANULEnt{B_{{\mathrm{1}}}}    , ... ,   {  \GRANULEmv{y} ^  i  _  n   }:_{\textcolor{coeffectColor}{    \GRANULEnt{s} ^  i  _  n     } }   \GRANULEnt{B_{\GRANULEmv{n}}}   }%
\GRANULEpremise{ \exists    \GRANULEnt{s'} ^  i  _  j     .\,     \GRANULEnt{s} ^  i  _  j     \sqsubseteq    \GRANULEnt{s'} ^  i  _  j     \textcolor{coeffectColor}{\,\cdot\,}    \GRANULEnt{q} ^  i  _  j     \sqsubseteq  \GRANULEnt{r}  \textcolor{coeffectColor}{\,\cdot\,}    \GRANULEnt{q} ^  i  _  j     }%
\GRANULEpremise{   \GRANULEnt{s} _  i     =      \GRANULEnt{s'} ^  i  _  1     \sqcup ... \sqcup    \GRANULEnt{s'} ^  i  _  n      }%
\GRANULEpremise{  |    K  \,   \vec{ \GRANULEnt{A} }     | > 1 \Rightarrow 1 \sqsubseteq      \GRANULEnt{s} _  1     \sqcup ... \sqcup    \GRANULEnt{s} _  m       }%
}{
 \Gamma  ;   \Omega  \GRANULEsym{,}   \GRANULEmv{x}  :_{\textcolor{coeffectColor}{  \GRANULEnt{r}  } }     K  \,   \vec{ \GRANULEnt{A} }       \Uparrow \vdash  \GRANULEnt{B}  \Rightarrow   \textbf{case} \  \GRANULEmv{x}  \ \textbf{of} \   \overline{   C_{\GRANULEmv{i}}  \  \GRANULEmv{y} ^  i  _  1   ...  \GRANULEmv{y} ^  i  _  n    \mapsto  \GRANULEnt{t_{\GRANULEmv{i}}} }    \mid  \GRANULEsym{(}   \Delta_{{\mathrm{1}}}  \sqcup ...  \sqcup  \Delta_{\GRANULEmv{m}}   \GRANULEsym{)}  \GRANULEsym{,}    \GRANULEmv{x}  :_{\textcolor{coeffectColor}{   \GRANULEsym{(}     \GRANULEnt{r} _  1     \sqcup ... \sqcup    \GRANULEnt{r} _  m      \GRANULEsym{)}   \GRANULEsym{+}   \GRANULEsym{(}     \GRANULEnt{s} _  1     \sqcup ... \sqcup    \GRANULEnt{s} _  m      \GRANULEsym{)}   } }     K  \,   \vec{ \GRANULEnt{A} }      }{%
{\GRANULEdruleCaseFName}{}%
}}


\newcommand{\GRANULEdruleUnboxFName}[0]{\GRANULEdrulename{UnboxF}}
\newcommand{\GRANULEdruleUnboxF}[1]{\GRANULEdrule[#1]{%
\GRANULEpremise{ \Gamma  ;   \Omega  \GRANULEsym{,}   \GRANULEmv{y}  :_{\textcolor{coeffectColor}{   \GRANULEnt{r}  \textcolor{coeffectColor}{\,\cdot\,}  \GRANULEnt{q}   } }   \GRANULEnt{A}   \GRANULEsym{,}   \GRANULEmv{x}  :_{\textcolor{coeffectColor}{  \GRANULEnt{r}  } }    \Box_{  \GRANULEnt{q}  }  \GRANULEnt{A}     \Uparrow \vdash  \GRANULEnt{B}  \Rightarrow  \GRANULEnt{t}  \mid  \Delta  \GRANULEsym{,}   \GRANULEmv{y}  :_{\textcolor{coeffectColor}{   \GRANULEnt{s_{{\mathrm{1}}}}   } }   \GRANULEnt{A}   \GRANULEsym{,}   \GRANULEmv{x}  :_{\textcolor{coeffectColor}{  \GRANULEnt{s_{{\mathrm{2}}}}  } }    \Box_{  \GRANULEnt{q}  }  \GRANULEnt{A}   }%
\GRANULEpremise{ \exists  \GRANULEnt{s_{{\mathrm{3}}}}  .\,   \GRANULEnt{s_{{\mathrm{3}}}}  \textcolor{coeffectColor}{\,\cdot\,}  \GRANULEnt{q}  \sqsubseteq  \GRANULEnt{s_{{\mathrm{1}}}}  }%
}{
 \Gamma  ;   \Omega  \GRANULEsym{,}   \GRANULEmv{x}  :_{\textcolor{coeffectColor}{  \GRANULEnt{r}  } }    \Box_{  \GRANULEnt{q}  }  \GRANULEnt{A}     \Uparrow \vdash  \GRANULEnt{B}  \Rightarrow   \textbf{case} \  \GRANULEmv{x}  \ \textbf{of} \  \GRANULEsym{[}  \GRANULEmv{y}  \GRANULEsym{]}  \rightarrow  \GRANULEnt{t}   \mid   \Delta   \GRANULEsym{,}   \GRANULEmv{x}  :_{\textcolor{coeffectColor}{   \GRANULEnt{s_{{\mathrm{3}}}}  \GRANULEsym{+}  \GRANULEnt{s_{{\mathrm{2}}}}   } }    \Box_{  \GRANULEnt{q}  }  \GRANULEnt{A}   }{%
{\GRANULEdruleUnboxFName}{}%
}}

\newcommand{\GRANULEdefnLASynFocus}[1]{\begin{GRANULEdefnblock}[#1]{$ \Gamma  ;  \Omega  \Uparrow \vdash  \GRANULEnt{A}  \Rightarrow  \GRANULEnt{t}  \mid  \Delta $}{}
\GRANULEusedrule{\GRANULEdruleCaseF{}}
\GRANULEusedrule{\GRANULEdruleUnboxF{}}
\end{GRANULEdefnblock}}

%% defn RSSynFocus
\newcommand{\GRANULEdruleBoxFName}[0]{\GRANULEdrulename{BoxF}}
\newcommand{\GRANULEdruleBoxF}[1]{\GRANULEdrule[#1]{%
\GRANULEpremise{\texttt{\textcolor{red}{<<no parses (char 19): G ; . \mbox{$\mid$}- A sync => ***t ; D >>}}}%
}{
\texttt{\textcolor{red}{<<no parses (char 26): G ; . \mbox{$\mid$}- \{[] r A\} sync => ***[ t ] ; r * D >>}}}{%
{\GRANULEdruleBoxFName}{}%
}}


\newcommand{\GRANULEdruleConFName}[0]{\GRANULEdrulename{ConF}}
\newcommand{\GRANULEdruleConF}[1]{\GRANULEdrule[#1]{%
\GRANULEpremise{ (  C  :  \GRANULEnt{B_{{\mathrm{1}}}} ^1 \rightarrow ... \rightarrow  \GRANULEnt{B_{\GRANULEmv{n}}} ^1 \rightarrow     K  \,   \vec{ \GRANULEnt{A} }     ) \in  D }%
\GRANULEpremise{\texttt{\textcolor{red}{<<no parses (char 22): G ; . \mbox{$\mid$}- \{Bi sync\} => ***ti ; Di >>}}}%
}{
\texttt{\textcolor{red}{<<no parses (char 23): G ; . \mbox{$\mid$}- \{K A\} sync => ***Con t1 ... tn ; \{D1 + ... + Dn\} >>}}}{%
{\GRANULEdruleConFName}{}%
}}

\newcommand{\GRANULEdefnRSSynFocus}[1]{\begin{GRANULEdefnblock}[#1]{$ \Gamma  \vdash  \GRANULEnt{A}  \Downarrow  \GRANULEnt{t}  \mid  \Delta $}{}
\GRANULEusedrule{\GRANULEdruleBoxF{}}
\GRANULEusedrule{\GRANULEdruleConF{}}
\end{GRANULEdefnblock}}

%% defn LSSynFocus
\newcommand{\GRANULEdruleVarFName}[0]{\GRANULEdrulename{VarF}}
\newcommand{\GRANULEdruleVarF}[1]{\GRANULEdrule[#1]{%
}{
 \Gamma  ;   \GRANULEmv{x}  :_{\textcolor{coeffectColor}{  \GRANULEnt{r}  } }   \GRANULEnt{A}   \Downarrow \vdash  \GRANULEnt{A}  \Rightarrow  \GRANULEmv{x}  \mid    \textcolor{coeffectColor}{  0    \textcolor{coeffectColor}{\,\cdot\,} }  \Gamma    \GRANULEsym{,}   \GRANULEmv{x}  :_{\textcolor{coeffectColor}{   1   } }   \GRANULEnt{A}  }{%
{\GRANULEdruleVarFName}{}%
}}


\newcommand{\GRANULEdruleAppFName}[0]{\GRANULEdrulename{AppF}}
\newcommand{\GRANULEdruleAppF}[1]{\GRANULEdrule[#1]{%
\GRANULEpremise{ \Gamma  ;    \GRANULEmv{x_{{\mathrm{1}}}}  :_{\textcolor{coeffectColor}{  \GRANULEnt{r_{{\mathrm{1}}}}  } }    \GRANULEnt{A} ^ \GRANULEnt{q}  \rightarrow  \GRANULEnt{B}    \GRANULEsym{,}   \GRANULEmv{x_{{\mathrm{2}}}}  :_{\textcolor{coeffectColor}{  \GRANULEnt{r_{{\mathrm{1}}}}  } }   \GRANULEnt{B}    \Downarrow \vdash  \GRANULEnt{C}  \Rightarrow  \GRANULEnt{t_{{\mathrm{1}}}}  \mid  \Delta_{{\mathrm{1}}}  \GRANULEsym{,}   \GRANULEmv{x_{{\mathrm{1}}}}  :_{\textcolor{coeffectColor}{  \GRANULEnt{s_{{\mathrm{1}}}}  } }    \GRANULEnt{A} ^ \GRANULEnt{q}  \rightarrow  \GRANULEnt{B}    \GRANULEsym{,}   \GRANULEmv{x_{{\mathrm{2}}}}  :_{\textcolor{coeffectColor}{  \GRANULEnt{s_{{\mathrm{2}}}}  } }   \GRANULEnt{B}  }%
\GRANULEpremise{ \Gamma  ;   \GRANULEmv{x_{{\mathrm{1}}}}  :_{\textcolor{coeffectColor}{  \GRANULEnt{r_{{\mathrm{1}}}}  } }    \GRANULEnt{A} ^ \GRANULEnt{q}  \rightarrow  \GRANULEnt{B}    \Downarrow \vdash  \GRANULEnt{A}  \Rightarrow  \GRANULEnt{t_{{\mathrm{2}}}}  \mid  \Delta_{{\mathrm{2}}}  \GRANULEsym{,}   \GRANULEmv{x_{{\mathrm{1}}}}  :_{\textcolor{coeffectColor}{  \GRANULEnt{s_{{\mathrm{3}}}}  } }    \GRANULEnt{A} ^ \GRANULEnt{q}  \rightarrow  \GRANULEnt{B}   }%
}{
 \Gamma  ;   \GRANULEmv{x_{{\mathrm{1}}}}  :_{\textcolor{coeffectColor}{  \GRANULEnt{r_{{\mathrm{1}}}}  } }    \GRANULEnt{A} ^ \GRANULEnt{q}  \rightarrow  \GRANULEnt{B}    \Downarrow \vdash  \GRANULEnt{C}  \Rightarrow    [  \GRANULEsym{(}  \GRANULEmv{x_{{\mathrm{1}}}} \, \GRANULEnt{t_{{\mathrm{2}}}}  \GRANULEsym{)}  /  \GRANULEmv{x_{{\mathrm{2}}}}  ]  \GRANULEnt{t_{{\mathrm{1}}}}    \mid  \GRANULEsym{(}  \Delta_{{\mathrm{1}}}  \GRANULEsym{+}   \textcolor{coeffectColor}{ \GRANULEnt{s_{{\mathrm{2}}}}   \textcolor{coeffectColor}{\,\cdot\,} }    \textcolor{coeffectColor}{ \GRANULEnt{q}   \textcolor{coeffectColor}{\,\cdot\,} }  \Delta_{{\mathrm{2}}}     \GRANULEsym{)}  \GRANULEsym{,}   \GRANULEmv{x_{{\mathrm{1}}}}  :_{\textcolor{coeffectColor}{   \GRANULEnt{s_{{\mathrm{2}}}}  \GRANULEsym{+}   \GRANULEnt{s_{{\mathrm{1}}}}  \GRANULEsym{+}   \GRANULEsym{(}  \GRANULEnt{s_{{\mathrm{2}}}}  \textcolor{coeffectColor}{\,\cdot\,}   \GRANULEnt{q}  \textcolor{coeffectColor}{\,\cdot\,}  \GRANULEnt{s_{{\mathrm{3}}}}   \GRANULEsym{)}     } }    \GRANULEnt{A} ^ \GRANULEnt{q}  \multimap  \GRANULEnt{B}   }{%
{\GRANULEdruleAppFName}{}%
}}

\newcommand{\GRANULEdefnLSSynFocus}[1]{\begin{GRANULEdefnblock}[#1]{$ \Gamma  ;  \Omega  \Downarrow \vdash  \GRANULEnt{A}  \Rightarrow  \GRANULEnt{t}  \mid  \Delta $}{}
\GRANULEusedrule{\GRANULEdruleVarF{}}
\GRANULEusedrule{\GRANULEdruleAppF{}}
\end{GRANULEdefnblock}}

%% defn SynthesisSub
\newcommand{\GRANULEdruleSynXXsubXXlinXXvarName}[0]{\GRANULEdrulename{Syn\_sub\_lin\_var}}
\newcommand{\GRANULEdruleSynXXsubXXlinXXvar}[1]{\GRANULEdrule[#1]{%
}{
 \Gamma  \GRANULEsym{,}   \GRANULEmv{x}  :  \GRANULEnt{A}   \vdash  \GRANULEnt{A}  \Rightarrow^-  \GRANULEmv{x} \ |\  \Gamma }{%
{\GRANULEdruleSynXXsubXXlinXXvarName}{}%
}}


\newcommand{\GRANULEdruleSynXXsubXXgrXXvarName}[0]{\GRANULEdrulename{Syn\_sub\_gr\_var}}
\newcommand{\GRANULEdruleSynXXsubXXgrXXvar}[1]{\GRANULEdrule[#1]{%
\GRANULEpremise{\texttt{\textcolor{red}{<<no parses (char 8): exists e***x >>}}}%
}{
 \Gamma  \GRANULEsym{,}   \GRANULEmv{x}  :_{\textcolor{coeffectColor}{  \GRANULEnt{r}  } }   \GRANULEnt{A}   \vdash  \GRANULEnt{A}  \Rightarrow^-  \GRANULEmv{x} \ |\  \Gamma  \GRANULEsym{,}   \GRANULEmv{x}  :_{\textcolor{coeffectColor}{  \GRANULEnt{s}  } }   \GRANULEnt{A}  }{%
{\GRANULEdruleSynXXsubXXgrXXvarName}{}%
}}


\newcommand{\GRANULEdruleSynXXsubXXderName}[0]{\GRANULEdrulename{Syn\_sub\_der}}
\newcommand{\GRANULEdruleSynXXsubXXder}[1]{\GRANULEdrule[#1]{%
\GRANULEpremise{ \Gamma  \GRANULEsym{,}   \GRANULEmv{x}  :_{\textcolor{coeffectColor}{  \GRANULEnt{r}  } }   \GRANULEnt{A}   \GRANULEsym{,}   \GRANULEmv{y}  :  \GRANULEnt{A}   \vdash  \GRANULEnt{B}  \Rightarrow^-  \GRANULEnt{t} \ |\  \Delta  \GRANULEsym{,}   \GRANULEmv{x}  :_{\textcolor{coeffectColor}{  \GRANULEnt{s}  } }   \GRANULEnt{A}  }%
\GRANULEpremise{ \GRANULEmv{y}  \not\in | \Delta | }%
\GRANULEpremise{\texttt{\textcolor{red}{<<no parses (char 22): exists s' . s = s' + 1*** >>}}}%
}{
 \Gamma  \GRANULEsym{,}   \GRANULEmv{x}  :_{\textcolor{coeffectColor}{  \GRANULEnt{r}  } }   \GRANULEnt{A}   \vdash  \GRANULEnt{B}  \Rightarrow^-   [  \GRANULEmv{y}  /  \GRANULEmv{x}  ]  \GRANULEnt{t}  \ |\  \Delta  \GRANULEsym{,}   \GRANULEmv{x}  :_{\textcolor{coeffectColor}{  \GRANULEnt{s'}  } }   \GRANULEnt{A}  }{%
{\GRANULEdruleSynXXsubXXderName}{}%
}}


\newcommand{\GRANULEdruleSynXXsubXXabsName}[0]{\GRANULEdrulename{Syn\_sub\_abs}}
\newcommand{\GRANULEdruleSynXXsubXXabs}[1]{\GRANULEdrule[#1]{%
\GRANULEpremise{ \GRANULEmv{x}  \not\in | \Delta | }%
\GRANULEpremise{ \Gamma  \GRANULEsym{,}   \GRANULEmv{x}  :  \GRANULEnt{A}   \vdash  \GRANULEnt{B}  \Rightarrow^-  \GRANULEnt{t} \ |\  \Delta }%
}{
 \Gamma  \vdash  \GRANULEnt{A}  \rightarrow  \GRANULEnt{B}  \Rightarrow^-   \lambda  \GRANULEmv{x}  .  \GRANULEnt{t}  \ |\  \Delta }{%
{\GRANULEdruleSynXXsubXXabsName}{}%
}}


\newcommand{\GRANULEdruleSynXXsubXXappName}[0]{\GRANULEdrulename{Syn\_sub\_app}}
\newcommand{\GRANULEdruleSynXXsubXXapp}[1]{\GRANULEdrule[#1]{%
\GRANULEpremise{ \GRANULEmv{x_{{\mathrm{2}}}}  \not\in | \Delta_{{\mathrm{1}}} | }%
\GRANULEpremise{ \Gamma  \GRANULEsym{,}   \GRANULEmv{x_{{\mathrm{2}}}}  :  \GRANULEnt{B}   \vdash  \GRANULEnt{C}  \Rightarrow^-  \GRANULEnt{t_{{\mathrm{1}}}} \ |\  \Delta_{{\mathrm{1}}} }%
\GRANULEpremise{ \Delta_{{\mathrm{1}}}  \vdash  \GRANULEnt{A}  \Rightarrow^-  \GRANULEnt{t_{{\mathrm{2}}}} \ |\  \Delta_{{\mathrm{2}}} }%
}{
 \Gamma  \GRANULEsym{,}   \GRANULEmv{x_{{\mathrm{1}}}}  :  \GRANULEnt{A}  \rightarrow  \GRANULEnt{B}   \vdash  \GRANULEnt{C}  \Rightarrow^-   [  \GRANULEsym{(}  \GRANULEmv{x_{{\mathrm{1}}}} \, \GRANULEnt{t_{{\mathrm{2}}}}  \GRANULEsym{)}  /  \GRANULEmv{x_{{\mathrm{2}}}}  ]  \GRANULEnt{t_{{\mathrm{1}}}}  \ |\  \Delta_{{\mathrm{2}}} }{%
{\GRANULEdruleSynXXsubXXappName}{}%
}}


\newcommand{\GRANULEdruleSynXXsubXXboxName}[0]{\GRANULEdrulename{Syn\_sub\_box}}
\newcommand{\GRANULEdruleSynXXsubXXbox}[1]{\GRANULEdrule[#1]{%
\GRANULEpremise{ \Gamma  \vdash  \GRANULEnt{A}  \Rightarrow^-  \GRANULEnt{t} \ |\  \Delta }%
}{
 \Gamma  \vdash   \Box_{  \GRANULEnt{r}  }  \GRANULEnt{A}   \Rightarrow^-  \GRANULEnt{t} \ |\  \Gamma  \GRANULEsym{-}   \textcolor{coeffectColor}{ \GRANULEnt{r}   \textcolor{coeffectColor}{\,\cdot\,} }  \GRANULEsym{(}  \Gamma  \GRANULEsym{-}  \Delta  \GRANULEsym{)}  }{%
{\GRANULEdruleSynXXsubXXboxName}{}%
}}


\newcommand{\GRANULEdruleSynXXsubXXunboxName}[0]{\GRANULEdrulename{Syn\_sub\_unbox}}
\newcommand{\GRANULEdruleSynXXsubXXunbox}[1]{\GRANULEdrule[#1]{%
\GRANULEpremise{  \Gamma  \GRANULEsym{,}   \GRANULEmv{x_{{\mathrm{2}}}}  :_{\textcolor{coeffectColor}{  \GRANULEnt{r}  } }   \GRANULEnt{A}   \vdash  \GRANULEnt{B}  \Rightarrow^-  \GRANULEnt{t} \ |\  \Delta  \GRANULEsym{,}   \GRANULEmv{x_{{\mathrm{2}}}}  :_{\textcolor{coeffectColor}{  \GRANULEnt{s}  } }   \GRANULEnt{A}    \quad\;\;    0   \, {\leq} \,  \GRANULEnt{s}  }%
}{
 \Gamma  \GRANULEsym{,}   \GRANULEmv{x_{{\mathrm{1}}}}  :   \Box_{  \GRANULEnt{r}  }  \GRANULEnt{A}    \vdash  \GRANULEnt{B}  \Rightarrow^-   \textbf{let} \, [  \GRANULEmv{x_{{\mathrm{2}}}}  ] =  \GRANULEmv{x_{{\mathrm{1}}}}  \, \textbf{in} \,  \GRANULEnt{t}  \ |\  \Delta }{%
{\GRANULEdruleSynXXsubXXunboxName}{}%
}}


\newcommand{\GRANULEdruleSynXXsubXXpairXXintroName}[0]{\GRANULEdrulename{Syn\_sub\_pair\_intro}}
\newcommand{\GRANULEdruleSynXXsubXXpairXXintro}[1]{\GRANULEdrule[#1]{%
\GRANULEpremise{ \Gamma  \vdash  \GRANULEnt{A}  \Rightarrow^-  \GRANULEnt{t_{{\mathrm{1}}}} \ |\  \Delta_{{\mathrm{1}}} }%
\GRANULEpremise{ \Delta_{{\mathrm{1}}}  \vdash  \GRANULEnt{B}  \Rightarrow^-  \GRANULEnt{t_{{\mathrm{2}}}} \ |\  \Delta_{{\mathrm{2}}} }%
}{
\texttt{\textcolor{red}{<<no parses (char 8): G \mbox{$\mid$}- A **** B =>- t1, t2 ; D2 >>}}}{%
{\GRANULEdruleSynXXsubXXpairXXintroName}{}%
}}


\newcommand{\GRANULEdruleSynXXsubXXpairXXelimName}[0]{\GRANULEdrulename{Syn\_sub\_pair\_elim}}
\newcommand{\GRANULEdruleSynXXsubXXpairXXelim}[1]{\GRANULEdrule[#1]{%
\GRANULEpremise{ \GRANULEmv{x_{{\mathrm{1}}}}  \not\in | \Delta | }%
\GRANULEpremise{ \GRANULEmv{x_{{\mathrm{2}}}}  \not\in | \Delta | }%
\GRANULEpremise{ \Gamma  \GRANULEsym{,}   \GRANULEmv{x_{{\mathrm{1}}}}  :  \GRANULEnt{A}   \GRANULEsym{,}   \GRANULEmv{x_{{\mathrm{2}}}}  :  \GRANULEnt{B}   \vdash  \GRANULEnt{C}  \Rightarrow^-  \GRANULEnt{t_{{\mathrm{2}}}} \ |\  \Delta }%
}{
\texttt{\textcolor{red}{<<no parses (char 11): G, x3 : A **** B \mbox{$\mid$}- C =>- let x1, x2 = x3 in t2 ; D >>}}}{%
{\GRANULEdruleSynXXsubXXpairXXelimName}{}%
}}


\newcommand{\GRANULEdruleSynXXsubXXsumXXintroXXleftName}[0]{\GRANULEdrulename{Syn\_sub\_sum\_intro\_left}}
\newcommand{\GRANULEdruleSynXXsubXXsumXXintroXXleft}[1]{\GRANULEdrule[#1]{%
\GRANULEpremise{ \Gamma  \vdash  \GRANULEnt{A}  \Rightarrow^-  \GRANULEnt{t} \ |\  \Delta }%
}{
\texttt{\textcolor{red}{<<no parses (char 8): G \mbox{$\mid$}- A +*** B =>- inl t ; D >>}}}{%
{\GRANULEdruleSynXXsubXXsumXXintroXXleftName}{}%
}}


\newcommand{\GRANULEdruleSynXXsubXXsumXXintroXXrightName}[0]{\GRANULEdrulename{Syn\_sub\_sum\_intro\_right}}
\newcommand{\GRANULEdruleSynXXsubXXsumXXintroXXright}[1]{\GRANULEdrule[#1]{%
\GRANULEpremise{ \Gamma  \vdash  \GRANULEnt{B}  \Rightarrow^-  \GRANULEnt{t} \ |\  \Delta }%
}{
\texttt{\textcolor{red}{<<no parses (char 8): G \mbox{$\mid$}- A +*** B =>- inr t ; D >>}}}{%
{\GRANULEdruleSynXXsubXXsumXXintroXXrightName}{}%
}}


\newcommand{\GRANULEdruleSynXXsubXXsumXXelimName}[0]{\GRANULEdrulename{Syn\_sub\_sum\_elim}}
\newcommand{\GRANULEdruleSynXXsubXXsumXXelim}[1]{\GRANULEdrule[#1]{%
\GRANULEpremise{ \GRANULEmv{x_{{\mathrm{2}}}}  \not\in | \Delta_{{\mathrm{1}}} | }%
\GRANULEpremise{ \GRANULEmv{x_{{\mathrm{3}}}}  \not\in | \Delta_{{\mathrm{2}}} | }%
\GRANULEpremise{ \Gamma  \GRANULEsym{,}   \GRANULEmv{x_{{\mathrm{2}}}}  :  \GRANULEnt{A}   \vdash  \GRANULEnt{C}  \Rightarrow^-  \GRANULEnt{t_{{\mathrm{1}}}} \ |\  \Delta_{{\mathrm{1}}} }%
\GRANULEpremise{ \Gamma  \GRANULEsym{,}   \GRANULEmv{x_{{\mathrm{3}}}}  :  \GRANULEnt{B}   \vdash  \GRANULEnt{C}  \Rightarrow^-  \GRANULEnt{t_{{\mathrm{2}}}} \ |\  \Delta_{{\mathrm{2}}} }%
}{
\texttt{\textcolor{red}{<<no parses (char 14): G, x1 : A + B ***\mbox{$\mid$}- C =>- case x1 of inl x2 -> t1 \mbox{$\mid$} inr x3 -> t2 ; D1 ++- D2 >>}}}{%
{\GRANULEdruleSynXXsubXXsumXXelimName}{}%
}}


\newcommand{\GRANULEdruleSynXXsubXXunitXXintroName}[0]{\GRANULEdrulename{Syn\_sub\_unit\_intro}}
\newcommand{\GRANULEdruleSynXXsubXXunitXXintro}[1]{\GRANULEdrule[#1]{%
}{
 \Gamma  \vdash   \mathsf{1}   \Rightarrow^-  \GRANULEsym{()} \ |\  \Gamma }{%
{\GRANULEdruleSynXXsubXXunitXXintroName}{}%
}}


\newcommand{\GRANULEdruleSynXXsubXXunitXXelimName}[0]{\GRANULEdrulename{Syn\_sub\_unit\_elim}}
\newcommand{\GRANULEdruleSynXXsubXXunitXXelim}[1]{\GRANULEdrule[#1]{%
\GRANULEpremise{ \Gamma  \vdash  \GRANULEnt{C}  \Rightarrow^-  \GRANULEnt{t} \ |\  \Delta }%
}{
 \Gamma  \GRANULEsym{,}   \GRANULEmv{x}  :   \mathsf{1}    \vdash  \GRANULEnt{C}  \Rightarrow^-  \GRANULEkw{let} \, \GRANULEsym{()}  \GRANULEsym{=}  \GRANULEmv{x} \, \GRANULEkw{in} \, \GRANULEnt{t} \ |\  \Delta }{%
{\GRANULEdruleSynXXsubXXunitXXelimName}{}%
}}

\newcommand{\GRANULEdefnSynthesisSub}[1]{\begin{GRANULEdefnblock}[#1]{$ \Gamma  \vdash  \GRANULEnt{A}  \Rightarrow^-  \GRANULEnt{t} \ |\  \Delta $}{}
\GRANULEusedrule{\GRANULEdruleSynXXsubXXlinXXvar{}}
\GRANULEusedrule{\GRANULEdruleSynXXsubXXgrXXvar{}}
\GRANULEusedrule{\GRANULEdruleSynXXsubXXder{}}
\GRANULEusedrule{\GRANULEdruleSynXXsubXXabs{}}
\GRANULEusedrule{\GRANULEdruleSynXXsubXXapp{}}
\GRANULEusedrule{\GRANULEdruleSynXXsubXXbox{}}
\GRANULEusedrule{\GRANULEdruleSynXXsubXXunbox{}}
\GRANULEusedrule{\GRANULEdruleSynXXsubXXpairXXintro{}}
\GRANULEusedrule{\GRANULEdruleSynXXsubXXpairXXelim{}}
\GRANULEusedrule{\GRANULEdruleSynXXsubXXsumXXintroXXleft{}}
\GRANULEusedrule{\GRANULEdruleSynXXsubXXsumXXintroXXright{}}
\GRANULEusedrule{\GRANULEdruleSynXXsubXXsumXXelim{}}
\GRANULEusedrule{\GRANULEdruleSynXXsubXXunitXXintro{}}
\GRANULEusedrule{\GRANULEdruleSynXXsubXXunitXXelim{}}
\end{GRANULEdefnblock}}

%% defn SynthesisAdd
\newcommand{\GRANULEdruleSynXXaddXXlinXXvarName}[0]{\GRANULEdrulename{Syn\_add\_lin\_var}}
\newcommand{\GRANULEdruleSynXXaddXXlinXXvar}[1]{\GRANULEdrule[#1]{%
}{
 \Gamma  \GRANULEsym{,}   \GRANULEmv{x}  :  \GRANULEnt{A}   \vdash  \GRANULEnt{A}  \Rightarrow^+  \GRANULEmv{x}  ;\,   \GRANULEmv{x}  :  \GRANULEnt{A}  }{%
{\GRANULEdruleSynXXaddXXlinXXvarName}{}%
}}


\newcommand{\GRANULEdruleSynXXaddXXgrXXvarName}[0]{\GRANULEdrulename{Syn\_add\_gr\_var}}
\newcommand{\GRANULEdruleSynXXaddXXgrXXvar}[1]{\GRANULEdrule[#1]{%
}{
 \Gamma  \GRANULEsym{,}   \GRANULEmv{x}  :_{\textcolor{coeffectColor}{  \GRANULEnt{r}  } }   \GRANULEnt{A}   \vdash  \GRANULEnt{A}  \Rightarrow^+  \GRANULEmv{x}  ;\,   \GRANULEmv{x}  :_{\textcolor{coeffectColor}{   1   } }   \GRANULEnt{A}  }{%
{\GRANULEdruleSynXXaddXXgrXXvarName}{}%
}}


\newcommand{\GRANULEdruleSynXXaddXXderName}[0]{\GRANULEdrulename{Syn\_add\_der}}
\newcommand{\GRANULEdruleSynXXaddXXder}[1]{\GRANULEdrule[#1]{%
\GRANULEpremise{ \Gamma  \GRANULEsym{,}   \GRANULEmv{x}  :_{\textcolor{coeffectColor}{  \GRANULEnt{s}  } }   \GRANULEnt{A}   \GRANULEsym{,}   \GRANULEmv{y}  :  \GRANULEnt{A}   \vdash  \GRANULEnt{B}  \Rightarrow^+  \GRANULEnt{t}  ;\,  \Delta  \GRANULEsym{,}   \GRANULEmv{y}  :  \GRANULEnt{A}  }%
}{
 \Gamma  \GRANULEsym{,}   \GRANULEmv{x}  :_{\textcolor{coeffectColor}{  \GRANULEnt{s}  } }   \GRANULEnt{A}   \vdash  \GRANULEnt{B}  \Rightarrow^+   [  \GRANULEmv{y}  /  \GRANULEmv{x}  ]  \GRANULEnt{t}   ;\,  \Delta  \GRANULEsym{+}   \GRANULEmv{x}  :_{\textcolor{coeffectColor}{   1   } }   \GRANULEnt{A}  }{%
{\GRANULEdruleSynXXaddXXderName}{}%
}}


\newcommand{\GRANULEdruleSynXXaddXXabsName}[0]{\GRANULEdrulename{Syn\_add\_abs}}
\newcommand{\GRANULEdruleSynXXaddXXabs}[1]{\GRANULEdrule[#1]{%
\GRANULEpremise{ \Gamma  \GRANULEsym{,}   \GRANULEmv{x}  :  \GRANULEnt{A}   \vdash  \GRANULEnt{B}  \Rightarrow^+  \GRANULEnt{t}  ;\,  \Delta  \GRANULEsym{,}   \GRANULEmv{x}  :  \GRANULEnt{A}  }%
}{
 \Gamma  \vdash  \GRANULEnt{A}  \rightarrow  \GRANULEnt{B}  \Rightarrow^+   \lambda  \GRANULEmv{x}  .  \GRANULEnt{t}   ;\,  \Delta }{%
{\GRANULEdruleSynXXaddXXabsName}{}%
}}


\newcommand{\GRANULEdruleSynXXaddXXappName}[0]{\GRANULEdrulename{Syn\_add\_app}}
\newcommand{\GRANULEdruleSynXXaddXXapp}[1]{\GRANULEdrule[#1]{%
\GRANULEpremise{ \Gamma  \GRANULEsym{,}   \GRANULEmv{x_{{\mathrm{2}}}}  :  \GRANULEnt{B}   \vdash  \GRANULEnt{C}  \Rightarrow^+  \GRANULEnt{t_{{\mathrm{1}}}}  ;\,  \Delta_{{\mathrm{1}}}  \GRANULEsym{,}   \GRANULEmv{x_{{\mathrm{2}}}}  :  \GRANULEnt{B}  }%
\GRANULEpremise{ \Gamma  \GRANULEsym{-}  \GRANULEsym{(}  \Delta_{{\mathrm{1}}}  \GRANULEsym{,}   \GRANULEmv{x_{{\mathrm{2}}}}  :  \GRANULEnt{B}   \GRANULEsym{)}  \vdash  \GRANULEnt{A}  \Rightarrow^+  \GRANULEnt{t_{{\mathrm{2}}}}  ;\,  \Delta_{{\mathrm{2}}} }%
}{
 \Gamma  \GRANULEsym{,}   \GRANULEmv{x_{{\mathrm{1}}}}  :  \GRANULEnt{A}  \rightarrow  \GRANULEnt{B}   \vdash  \GRANULEnt{C}  \Rightarrow^+   [  \GRANULEsym{(}  \GRANULEmv{x_{{\mathrm{1}}}} \, \GRANULEnt{t_{{\mathrm{2}}}}  \GRANULEsym{)}  /  \GRANULEmv{x_{{\mathrm{2}}}}  ]  \GRANULEnt{t_{{\mathrm{1}}}}   ;\,  \GRANULEsym{(}  \Delta_{{\mathrm{1}}}  \GRANULEsym{+}  \Delta_{{\mathrm{2}}}  \GRANULEsym{)}  \GRANULEsym{,}   \GRANULEmv{x_{{\mathrm{1}}}}  :  \GRANULEnt{A}  \rightarrow  \GRANULEnt{B}  }{%
{\GRANULEdruleSynXXaddXXappName}{}%
}}


\newcommand{\GRANULEdruleSynXXaddXXboxName}[0]{\GRANULEdrulename{Syn\_add\_box}}
\newcommand{\GRANULEdruleSynXXaddXXbox}[1]{\GRANULEdrule[#1]{%
\GRANULEpremise{ \Gamma  \vdash  \GRANULEnt{A}  \Rightarrow^+  \GRANULEnt{t}  ;\,  \Delta }%
}{
 \Gamma  \vdash   \Box_{  \GRANULEnt{r}  }  \GRANULEnt{A}   \Rightarrow^+  \GRANULEsym{[}  \GRANULEnt{t}  \GRANULEsym{]}  ;\,   \textcolor{coeffectColor}{ \GRANULEnt{r}   \textcolor{coeffectColor}{\,\cdot\,} }  \Delta  }{%
{\GRANULEdruleSynXXaddXXboxName}{}%
}}


\newcommand{\GRANULEdruleSynXXaddXXunboxName}[0]{\GRANULEdrulename{Syn\_add\_unbox}}
\newcommand{\GRANULEdruleSynXXaddXXunbox}[1]{\GRANULEdrule[#1]{%
\GRANULEpremise{  \Gamma  \GRANULEsym{,}   \GRANULEmv{x_{{\mathrm{2}}}}  :_{\textcolor{coeffectColor}{  \GRANULEnt{r}  } }   \GRANULEnt{A}   \vdash  \GRANULEnt{B}  \Rightarrow^+  \GRANULEnt{t}  ;\,  \Delta  \GRANULEsym{,}   \GRANULEmv{x_{{\mathrm{2}}}}  :_{\textcolor{coeffectColor}{  \GRANULEnt{s}  } }   \GRANULEnt{A}    \quad\;\;   \GRANULEnt{s}  \, {\leq} \,  \GRANULEnt{r}  }%
}{
 \Gamma  \GRANULEsym{,}   \GRANULEmv{x_{{\mathrm{1}}}}  :   \Box_{  \GRANULEnt{r}  }  \GRANULEnt{A}    \vdash  \GRANULEnt{B}  \Rightarrow^+   \textbf{let} \, [  \GRANULEmv{x_{{\mathrm{2}}}}  ] =  \GRANULEmv{x_{{\mathrm{1}}}}  \, \textbf{in} \,  \GRANULEnt{t}   ;\,  \Delta  \GRANULEsym{,}   \GRANULEmv{x_{{\mathrm{1}}}}  :   \Box_{  \GRANULEnt{r}  }  \GRANULEnt{A}   }{%
{\GRANULEdruleSynXXaddXXunboxName}{}%
}}


\newcommand{\GRANULEdruleSynXXaddXXpairXXintroName}[0]{\GRANULEdrulename{Syn\_add\_pair\_intro}}
\newcommand{\GRANULEdruleSynXXaddXXpairXXintro}[1]{\GRANULEdrule[#1]{%
\GRANULEpremise{ \Gamma  \vdash  \GRANULEnt{A}  \Rightarrow^+  \GRANULEnt{t_{{\mathrm{1}}}}  ;\,  \Delta_{{\mathrm{1}}} }%
\GRANULEpremise{ \Gamma  \GRANULEsym{-}  \Delta_{{\mathrm{1}}}  \vdash  \GRANULEnt{B}  \Rightarrow^+  \GRANULEnt{t_{{\mathrm{2}}}}  ;\,  \Delta_{{\mathrm{2}}} }%
}{
\texttt{\textcolor{red}{<<no parses (char 8): G \mbox{$\mid$}- A **** B =>+ t1, t2 ; D1 + D2 >>}}}{%
{\GRANULEdruleSynXXaddXXpairXXintroName}{}%
}}


\newcommand{\GRANULEdruleSynXXaddXXpairXXelimName}[0]{\GRANULEdrulename{Syn\_add\_pair\_elim}}
\newcommand{\GRANULEdruleSynXXaddXXpairXXelim}[1]{\GRANULEdrule[#1]{%
\GRANULEpremise{ \Gamma  \GRANULEsym{,}   \GRANULEmv{x_{{\mathrm{1}}}}  :  \GRANULEnt{A}   \GRANULEsym{,}   \GRANULEmv{x_{{\mathrm{2}}}}  :  \GRANULEnt{B}   \vdash  \GRANULEnt{C}  \Rightarrow^+  \GRANULEnt{t_{{\mathrm{2}}}}  ;\,  \Delta  \GRANULEsym{,}   \GRANULEmv{x_{{\mathrm{1}}}}  :  \GRANULEnt{A}   \GRANULEsym{,}   \GRANULEmv{x_{{\mathrm{2}}}}  :  \GRANULEnt{B}  }%
}{
\texttt{\textcolor{red}{<<no parses (char 11): G, x3 : A **** B \mbox{$\mid$}- C =>+ let x1, x2 = x3 in t2 ; D, x3 : A * B >>}}}{%
{\GRANULEdruleSynXXaddXXpairXXelimName}{}%
}}


\newcommand{\GRANULEdruleSynXXaddXXsumXXintroXXleftName}[0]{\GRANULEdrulename{Syn\_add\_sum\_intro\_left}}
\newcommand{\GRANULEdruleSynXXaddXXsumXXintroXXleft}[1]{\GRANULEdrule[#1]{%
\GRANULEpremise{ \Gamma  \vdash  \GRANULEnt{A}  \Rightarrow^+  \GRANULEnt{t}  ;\,  \Delta }%
}{
\texttt{\textcolor{red}{<<no parses (char 8): G \mbox{$\mid$}- A +*** B =>+ inl t ; D >>}}}{%
{\GRANULEdruleSynXXaddXXsumXXintroXXleftName}{}%
}}


\newcommand{\GRANULEdruleSynXXaddXXsumXXintroXXrightName}[0]{\GRANULEdrulename{Syn\_add\_sum\_intro\_right}}
\newcommand{\GRANULEdruleSynXXaddXXsumXXintroXXright}[1]{\GRANULEdrule[#1]{%
\GRANULEpremise{ \Gamma  \vdash  \GRANULEnt{B}  \Rightarrow^+  \GRANULEnt{t}  ;\,  \Delta }%
}{
\texttt{\textcolor{red}{<<no parses (char 8): G \mbox{$\mid$}- A +*** B =>+ inr t ; D >>}}}{%
{\GRANULEdruleSynXXaddXXsumXXintroXXrightName}{}%
}}


\newcommand{\GRANULEdruleSynXXaddXXunitXXintroName}[0]{\GRANULEdrulename{Syn\_add\_unit\_intro}}
\newcommand{\GRANULEdruleSynXXaddXXunitXXintro}[1]{\GRANULEdrule[#1]{%
}{
 \Gamma  \vdash   \mathsf{1}   \Rightarrow^+  \GRANULEsym{()}  ;\,   \emptyset  }{%
{\GRANULEdruleSynXXaddXXunitXXintroName}{}%
}}


\newcommand{\GRANULEdruleSynXXaddXXunitXXelimName}[0]{\GRANULEdrulename{Syn\_add\_unit\_elim}}
\newcommand{\GRANULEdruleSynXXaddXXunitXXelim}[1]{\GRANULEdrule[#1]{%
\GRANULEpremise{ \Gamma  \vdash  \GRANULEnt{C}  \Rightarrow  \GRANULEnt{t}  \mid  \Delta }%
}{
\texttt{\textcolor{red}{<<no parses (char 49): G, x : Unit \mbox{$\mid$}- C => let () = x in t ; D, x : Unit*** >>}}}{%
{\GRANULEdruleSynXXaddXXunitXXelimName}{}%
}}


\newcommand{\GRANULEdruleSynXXaddXXsumXXelimName}[0]{\GRANULEdrulename{Syn\_add\_sum\_elim}}
\newcommand{\GRANULEdruleSynXXaddXXsumXXelim}[1]{\GRANULEdrule[#1]{%
\GRANULEpremise{ \Gamma  \GRANULEsym{,}   \GRANULEmv{x_{{\mathrm{2}}}}  :  \GRANULEnt{A}   \vdash  \GRANULEnt{C}  \Rightarrow^+  \GRANULEnt{t_{{\mathrm{1}}}}  ;\,  \Delta_{{\mathrm{1}}}  \GRANULEsym{,}   \GRANULEmv{x_{{\mathrm{2}}}}  :  \GRANULEnt{A}  }%
\GRANULEpremise{ \Gamma  \GRANULEsym{,}   \GRANULEmv{x_{{\mathrm{3}}}}  :  \GRANULEnt{B}   \vdash  \GRANULEnt{C}  \Rightarrow^+  \GRANULEnt{t_{{\mathrm{2}}}}  ;\,  \Delta_{{\mathrm{2}}}  \GRANULEsym{,}   \GRANULEmv{x_{{\mathrm{3}}}}  :  \GRANULEnt{B}  }%
}{
\texttt{\textcolor{red}{<<no parses (char 14): G, x1 : A + B ***\mbox{$\mid$}- C =>+ case x1 of inl x2 -> t1 \mbox{$\mid$} inr x3 -> t2 ; (D1 +++ D2), x1 : A + B >>}}}{%
{\GRANULEdruleSynXXaddXXsumXXelimName}{}%
}}

\newcommand{\GRANULEdefnSynthesisAdd}[1]{\begin{GRANULEdefnblock}[#1]{$ \Gamma  \vdash  \GRANULEnt{A}  \Rightarrow^+  \GRANULEnt{t}  ;\,  \Delta $}{}
\GRANULEusedrule{\GRANULEdruleSynXXaddXXlinXXvar{}}
\GRANULEusedrule{\GRANULEdruleSynXXaddXXgrXXvar{}}
\GRANULEusedrule{\GRANULEdruleSynXXaddXXder{}}
\GRANULEusedrule{\GRANULEdruleSynXXaddXXabs{}}
\GRANULEusedrule{\GRANULEdruleSynXXaddXXapp{}}
\GRANULEusedrule{\GRANULEdruleSynXXaddXXbox{}}
\GRANULEusedrule{\GRANULEdruleSynXXaddXXunbox{}}
\GRANULEusedrule{\GRANULEdruleSynXXaddXXpairXXintro{}}
\GRANULEusedrule{\GRANULEdruleSynXXaddXXpairXXelim{}}
\GRANULEusedrule{\GRANULEdruleSynXXaddXXsumXXintroXXleft{}}
\GRANULEusedrule{\GRANULEdruleSynXXaddXXsumXXintroXXright{}}
\GRANULEusedrule{\GRANULEdruleSynXXaddXXunitXXintro{}}
\GRANULEusedrule{\GRANULEdruleSynXXaddXXunitXXelim{}}
\GRANULEusedrule{\GRANULEdruleSynXXaddXXsumXXelim{}}
\end{GRANULEdefnblock}}

%% defn FocusSynthesisSub
\newcommand{\GRANULEdrulefocusXXsynXXsubXXlinXXvarName}[0]{\GRANULEdrulename{focus\_syn\_sub\_lin\_var}}
\newcommand{\GRANULEdrulefocusXXsynXXsubXXlinXXvar}[1]{\GRANULEdrule[#1]{%
}{
\texttt{\textcolor{red}{<<no parses (char 18): G ; x : A \mbox{$\mid$}- A => ***x ; G >>}}}{%
{\GRANULEdrulefocusXXsynXXsubXXlinXXvarName}{}%
}}


\newcommand{\GRANULEdrulefocusXXsynXXsubXXgrXXvarName}[0]{\GRANULEdrulename{focus\_syn\_sub\_gr\_var}}
\newcommand{\GRANULEdrulefocusXXsynXXsubXXgrXXvar}[1]{\GRANULEdrule[#1]{%
\GRANULEpremise{  1   \, {\leq} \,  \GRANULEnt{r} }%
\GRANULEpremise{\texttt{\textcolor{red}{<<no parses (char 8): exists e***x >>}}}%
}{
\texttt{\textcolor{red}{<<no parses (char 22): G ; x : [A] r \mbox{$\mid$}- A => ***x ; G, x : [A] s >>}}}{%
{\GRANULEdrulefocusXXsynXXsubXXgrXXvarName}{}%
}}


\newcommand{\GRANULEdrulefocusXXsynXXsubXXderName}[0]{\GRANULEdrulename{focus\_syn\_sub\_der}}
\newcommand{\GRANULEdrulefocusXXsynXXsubXXder}[1]{\GRANULEdrule[#1]{%
\GRANULEpremise{\texttt{\textcolor{red}{<<no parses (char 29): G ; x : [A] r, y : A \mbox{$\mid$}- B => ***t; D, x : [A] s >>}}}%
\GRANULEpremise{ \GRANULEmv{y}  \not\in | \Delta | }%
\GRANULEpremise{ \exists  \GRANULEnt{s}  .\,   \GRANULEnt{r}  \sqsupseteq  \GRANULEnt{s}  \GRANULEsym{+}   1   }%
}{
\texttt{\textcolor{red}{<<no parses (char 22): G ; x : [A] r \mbox{$\mid$}- B => ***[y/x] t; D, x : [A] s' >>}}}{%
{\GRANULEdrulefocusXXsynXXsubXXderName}{}%
}}


\newcommand{\GRANULEdrulefocusXXsynXXsubXXabsName}[0]{\GRANULEdrulename{focus\_syn\_sub\_abs}}
\newcommand{\GRANULEdrulefocusXXsynXXsubXXabs}[1]{\GRANULEdrule[#1]{%
\GRANULEpremise{ \GRANULEmv{x}  \not\in | \Delta | }%
\GRANULEpremise{\texttt{\textcolor{red}{<<no parses (char 21): G ; O, x : A \mbox{$\mid$}- B => ***t ; D >>}}}%
}{
\texttt{\textcolor{red}{<<no parses (char 19): G ; O \mbox{$\mid$}- A -> B => ***\mbox{$\backslash{}$}x . t ; D >>}}}{%
{\GRANULEdrulefocusXXsynXXsubXXabsName}{}%
}}


\newcommand{\GRANULEdrulefocusXXsynXXsubXXappName}[0]{\GRANULEdrulename{focus\_syn\_sub\_app}}
\newcommand{\GRANULEdrulefocusXXsynXXsubXXapp}[1]{\GRANULEdrule[#1]{%
\GRANULEpremise{ \GRANULEmv{x_{{\mathrm{2}}}}  \not\in | \Delta_{{\mathrm{1}}} | }%
\GRANULEpremise{\texttt{\textcolor{red}{<<no parses (char 19): G ; x2 : B \mbox{$\mid$}- C => ***t1 ; D1 >>}}}%
\GRANULEpremise{\texttt{\textcolor{red}{<<no parses (char 15): D1 ; . \mbox{$\mid$}- A => ***t2 ; D2 >>}}}%
}{
\texttt{\textcolor{red}{<<no parses (char 24): G ; x1 : A -> B \mbox{$\mid$}- C => ***[(x1 t2) / x2] t1 ; D2 >>}}}{%
{\GRANULEdrulefocusXXsynXXsubXXappName}{}%
}}


\newcommand{\GRANULEdrulefocusXXsynXXsubXXboxName}[0]{\GRANULEdrulename{focus\_syn\_sub\_box}}
\newcommand{\GRANULEdrulefocusXXsynXXsubXXbox}[1]{\GRANULEdrule[#1]{%
\GRANULEpremise{\texttt{\textcolor{red}{<<no parses (char 14): G ; . \mbox{$\mid$}- A => ***t ; D >>}}}%
}{
\texttt{\textcolor{red}{<<no parses (char 19): G ; . \mbox{$\mid$}- [] r A => ***t ; G - r * (G - D) >>}}}{%
{\GRANULEdrulefocusXXsynXXsubXXboxName}{}%
}}


\newcommand{\GRANULEdrulefocusXXsynXXsubXXunboxName}[0]{\GRANULEdrulename{focus\_syn\_sub\_unbox}}
\newcommand{\GRANULEdrulefocusXXsynXXsubXXunbox}[1]{\GRANULEdrule[#1]{%
\GRANULEpremise{\texttt{\textcolor{red}{<<no parses (char 26): G ; O, x2 : [A] r \mbox{$\mid$}- B => ***t ; D, x2 : [A] s \&\& 0 <= s >>}}}%
}{
\texttt{\textcolor{red}{<<no parses (char 27): G ; O, x1 : [] r A \mbox{$\mid$}- B => ***let [x2] = x1 in t ; D >>}}}{%
{\GRANULEdrulefocusXXsynXXsubXXunboxName}{}%
}}


\newcommand{\GRANULEdrulefocusXXsynXXsubXXpairXXintroName}[0]{\GRANULEdrulename{focus\_syn\_sub\_pair\_intro}}
\newcommand{\GRANULEdrulefocusXXsynXXsubXXpairXXintro}[1]{\GRANULEdrule[#1]{%
\GRANULEpremise{\texttt{\textcolor{red}{<<no parses (char 14): G ; . \mbox{$\mid$}- A => ***t1 ; D1 >>}}}%
\GRANULEpremise{\texttt{\textcolor{red}{<<no parses (char 15): D1 ; . \mbox{$\mid$}- B => ***t2 ; D2 >>}}}%
}{
\texttt{\textcolor{red}{<<no parses (char 12): G ; . \mbox{$\mid$}- A **** B => t1, t2 ; D2 >>}}}{%
{\GRANULEdrulefocusXXsynXXsubXXpairXXintroName}{}%
}}


\newcommand{\GRANULEdrulefocusXXsynXXsubXXpairXXelimName}[0]{\GRANULEdrulename{focus\_syn\_sub\_pair\_elim}}
\newcommand{\GRANULEdrulefocusXXsynXXsubXXpairXXelim}[1]{\GRANULEdrule[#1]{%
\GRANULEpremise{ \GRANULEmv{x_{{\mathrm{1}}}}  \not\in | \Delta | }%
\GRANULEpremise{ \GRANULEmv{x_{{\mathrm{2}}}}  \not\in | \Delta | }%
\GRANULEpremise{\texttt{\textcolor{red}{<<no parses (char 30): G ; O, x1 : A, x2 : B \mbox{$\mid$}- C => ***t2 ; D >>}}}%
}{
\texttt{\textcolor{red}{<<no parses (char 15): G ; O, x3 : A **** B \mbox{$\mid$}- C => let x1, x2 = x3 in t2 ; D >>}}}{%
{\GRANULEdrulefocusXXsynXXsubXXpairXXelimName}{}%
}}


\newcommand{\GRANULEdrulefocusXXsynXXsubXXsumXXintroXXleftName}[0]{\GRANULEdrulename{focus\_syn\_sub\_sum\_intro\_left}}
\newcommand{\GRANULEdrulefocusXXsynXXsubXXsumXXintroXXleft}[1]{\GRANULEdrule[#1]{%
\GRANULEpremise{\texttt{\textcolor{red}{<<no parses (char 14): G ; . \mbox{$\mid$}- A => ***t ; D >>}}}%
}{
\texttt{\textcolor{red}{<<no parses (char 12): G ; . \mbox{$\mid$}- A +*** B => inl t ; D >>}}}{%
{\GRANULEdrulefocusXXsynXXsubXXsumXXintroXXleftName}{}%
}}


\newcommand{\GRANULEdrulefocusXXsynXXsubXXsumXXintroXXrightName}[0]{\GRANULEdrulename{focus\_syn\_sub\_sum\_intro\_right}}
\newcommand{\GRANULEdrulefocusXXsynXXsubXXsumXXintroXXright}[1]{\GRANULEdrule[#1]{%
\GRANULEpremise{\texttt{\textcolor{red}{<<no parses (char 14): G ; . \mbox{$\mid$}- B => ***t ; D >>}}}%
}{
\texttt{\textcolor{red}{<<no parses (char 12): G ; . \mbox{$\mid$}- A +*** B => inr t ; D >>}}}{%
{\GRANULEdrulefocusXXsynXXsubXXsumXXintroXXrightName}{}%
}}


\newcommand{\GRANULEdrulefocusXXsynXXsubXXsumXXelimName}[0]{\GRANULEdrulename{focus\_syn\_sub\_sum\_elim}}
\newcommand{\GRANULEdrulefocusXXsynXXsubXXsumXXelim}[1]{\GRANULEdrule[#1]{%
\GRANULEpremise{ \GRANULEmv{x_{{\mathrm{2}}}}  \not\in | \Delta_{{\mathrm{1}}} | }%
\GRANULEpremise{ \GRANULEmv{x_{{\mathrm{3}}}}  \not\in | \Delta_{{\mathrm{2}}} | }%
\GRANULEpremise{\texttt{\textcolor{red}{<<no parses (char 22): G ; O, x2 : A \mbox{$\mid$}- C => ***t1 ; D1 >>}}}%
\GRANULEpremise{\texttt{\textcolor{red}{<<no parses (char 22): G ; O, x3 : B \mbox{$\mid$}- C => ***t2 ; D2 >>}}}%
}{
\texttt{\textcolor{red}{<<no parses (char 18): G ; O, x1 : A + B ***\mbox{$\mid$}- C => case x1 of inl x2 -> t1 \mbox{$\mid$} inr x3 -> t2 ; D1 ++- D2 >>}}}{%
{\GRANULEdrulefocusXXsynXXsubXXsumXXelimName}{}%
}}

\newcommand{\GRANULEdefnFocusSynthesisSub}[1]{\begin{GRANULEdefnblock}[#1]{$ \Gamma  ;  \Omega  \vdash  \GRANULEnt{A}  \Rightarrow^{-}  \GRANULEnt{t}  \ |\  \Delta $}{}
\GRANULEusedrule{\GRANULEdrulefocusXXsynXXsubXXlinXXvar{}}
\GRANULEusedrule{\GRANULEdrulefocusXXsynXXsubXXgrXXvar{}}
\GRANULEusedrule{\GRANULEdrulefocusXXsynXXsubXXder{}}
\GRANULEusedrule{\GRANULEdrulefocusXXsynXXsubXXabs{}}
\GRANULEusedrule{\GRANULEdrulefocusXXsynXXsubXXapp{}}
\GRANULEusedrule{\GRANULEdrulefocusXXsynXXsubXXbox{}}
\GRANULEusedrule{\GRANULEdrulefocusXXsynXXsubXXunbox{}}
\GRANULEusedrule{\GRANULEdrulefocusXXsynXXsubXXpairXXintro{}}
\GRANULEusedrule{\GRANULEdrulefocusXXsynXXsubXXpairXXelim{}}
\GRANULEusedrule{\GRANULEdrulefocusXXsynXXsubXXsumXXintroXXleft{}}
\GRANULEusedrule{\GRANULEdrulefocusXXsynXXsubXXsumXXintroXXright{}}
\GRANULEusedrule{\GRANULEdrulefocusXXsynXXsubXXsumXXelim{}}
\end{GRANULEdefnblock}}

%% defn FocusSynthesisAdd
\newcommand{\GRANULEdrulefocusXXsynXXaddXXlinXXvarName}[0]{\GRANULEdrulename{focus\_syn\_add\_lin\_var}}
\newcommand{\GRANULEdrulefocusXXsynXXaddXXlinXXvar}[1]{\GRANULEdrule[#1]{%
}{
 \Gamma  ;   \GRANULEmv{x}  :  \GRANULEnt{A}   \vdash  \GRANULEnt{A}  \Rightarrow^{+}  \GRANULEmv{x}  \ | \   \GRANULEmv{x}  :  \GRANULEnt{A}  }{%
{\GRANULEdrulefocusXXsynXXaddXXlinXXvarName}{}%
}}


\newcommand{\GRANULEdrulefocusXXsynXXaddXXgrXXvarName}[0]{\GRANULEdrulename{focus\_syn\_add\_gr\_var}}
\newcommand{\GRANULEdrulefocusXXsynXXaddXXgrXXvar}[1]{\GRANULEdrule[#1]{%
\GRANULEpremise{  1   \, {\leq} \,  \GRANULEnt{r} }%
\GRANULEpremise{\texttt{\textcolor{red}{<<no parses (char 8): exists e***x >>}}}%
}{
 \Gamma  ;   \GRANULEmv{x}  :_{\textcolor{coeffectColor}{  \GRANULEnt{r}  } }   \GRANULEnt{A}   \vdash  \GRANULEnt{A}  \Rightarrow^{+}  \GRANULEmv{x}  \ | \   \GRANULEmv{x}  :_{\textcolor{coeffectColor}{   1   } }   \GRANULEnt{A}  }{%
{\GRANULEdrulefocusXXsynXXaddXXgrXXvarName}{}%
}}


\newcommand{\GRANULEdrulefocusXXsynXXaddXXderName}[0]{\GRANULEdrulename{focus\_syn\_add\_der}}
\newcommand{\GRANULEdrulefocusXXsynXXaddXXder}[1]{\GRANULEdrule[#1]{%
\GRANULEpremise{ \Gamma  ;   \GRANULEmv{x}  :_{\textcolor{coeffectColor}{  \GRANULEnt{s}  } }   \GRANULEnt{A}   \GRANULEsym{,}   \GRANULEmv{y}  :  \GRANULEnt{A}   \vdash  \GRANULEnt{B}  \Rightarrow^{+}  \GRANULEnt{t}  \ | \  \Delta  \GRANULEsym{,}   \GRANULEmv{y}  :  \GRANULEnt{A}  }%
}{
 \Gamma  ;   \GRANULEmv{x}  :_{\textcolor{coeffectColor}{  \GRANULEnt{s}  } }   \GRANULEnt{A}   \vdash  \GRANULEnt{B}  \Rightarrow^{+}   [  \GRANULEmv{y}  /  \GRANULEmv{x}  ]  \GRANULEnt{t}   \ | \  \Delta  \GRANULEsym{+}   \GRANULEmv{x}  :_{\textcolor{coeffectColor}{   1   } }   \GRANULEnt{A}  }{%
{\GRANULEdrulefocusXXsynXXaddXXderName}{}%
}}


\newcommand{\GRANULEdrulefocusXXsynXXaddXXabsName}[0]{\GRANULEdrulename{focus\_syn\_add\_abs}}
\newcommand{\GRANULEdrulefocusXXsynXXaddXXabs}[1]{\GRANULEdrule[#1]{%
\GRANULEpremise{ \Gamma  ;  \Omega  \GRANULEsym{,}   \GRANULEmv{x}  :  \GRANULEnt{A}   \vdash  \GRANULEnt{B}  \Rightarrow^{+}  \GRANULEnt{t}  \ | \  \Delta  \GRANULEsym{,}   \GRANULEmv{x}  :  \GRANULEnt{A}  }%
}{
 \Gamma  ;  \Omega  \vdash  \GRANULEnt{A}  \rightarrow  \GRANULEnt{B}  \Rightarrow^{+}   \lambda  \GRANULEmv{x}  .  \GRANULEnt{t}   \ | \  \Delta }{%
{\GRANULEdrulefocusXXsynXXaddXXabsName}{}%
}}


\newcommand{\GRANULEdrulefocusXXsynXXaddXXappName}[0]{\GRANULEdrulename{focus\_syn\_add\_app}}
\newcommand{\GRANULEdrulefocusXXsynXXaddXXapp}[1]{\GRANULEdrule[#1]{%
\GRANULEpremise{ \Gamma  ;   \GRANULEmv{x_{{\mathrm{2}}}}  :  \GRANULEnt{B}   \vdash  \GRANULEnt{C}  \Rightarrow^{+}  \GRANULEnt{t_{{\mathrm{1}}}}  \ | \  \Delta_{{\mathrm{1}}}  \GRANULEsym{,}   \GRANULEmv{x_{{\mathrm{2}}}}  :  \GRANULEnt{B}  }%
\GRANULEpremise{ \Gamma  \GRANULEsym{-}  \GRANULEsym{(}  \Delta_{{\mathrm{1}}}  \GRANULEsym{,}   \GRANULEmv{x_{{\mathrm{2}}}}  :  \GRANULEnt{B}   \GRANULEsym{)}  ;   \emptyset   \vdash  \GRANULEnt{A}  \Rightarrow^{+}  \GRANULEnt{t_{{\mathrm{2}}}}  \ | \  \Delta_{{\mathrm{2}}} }%
}{
 \Gamma  ;   \GRANULEmv{x_{{\mathrm{1}}}}  :  \GRANULEnt{A}  \rightarrow  \GRANULEnt{B}   \vdash  \GRANULEnt{C}  \Rightarrow^{+}   [  \GRANULEsym{(}  \GRANULEmv{x_{{\mathrm{1}}}} \, \GRANULEnt{t_{{\mathrm{2}}}}  \GRANULEsym{)}  /  \GRANULEmv{x_{{\mathrm{2}}}}  ]  \GRANULEnt{t_{{\mathrm{1}}}}   \ | \  \GRANULEsym{(}  \Delta_{{\mathrm{1}}}  \GRANULEsym{+}  \Delta_{{\mathrm{2}}}  \GRANULEsym{)}  \GRANULEsym{,}   \GRANULEmv{x_{{\mathrm{1}}}}  :  \GRANULEnt{A}  \rightarrow  \GRANULEnt{B}  }{%
{\GRANULEdrulefocusXXsynXXaddXXappName}{}%
}}


\newcommand{\GRANULEdrulefocusXXsynXXaddXXboxName}[0]{\GRANULEdrulename{focus\_syn\_add\_box}}
\newcommand{\GRANULEdrulefocusXXsynXXaddXXbox}[1]{\GRANULEdrule[#1]{%
\GRANULEpremise{ \Gamma  ;   \emptyset   \vdash  \GRANULEnt{A}  \Rightarrow^{+}  \GRANULEnt{t}  \ | \  \Delta }%
}{
 \Gamma  ;   \emptyset   \vdash   \Box_{  \GRANULEnt{r}  }  \GRANULEnt{A}   \Rightarrow^{+}  \GRANULEsym{[}  \GRANULEnt{t}  \GRANULEsym{]}  \ | \   \textcolor{coeffectColor}{ \GRANULEnt{r}   \textcolor{coeffectColor}{\,\cdot\,} }  \Delta  }{%
{\GRANULEdrulefocusXXsynXXaddXXboxName}{}%
}}


\newcommand{\GRANULEdrulefocusXXsynXXaddXXunboxName}[0]{\GRANULEdrulename{focus\_syn\_add\_unbox}}
\newcommand{\GRANULEdrulefocusXXsynXXaddXXunbox}[1]{\GRANULEdrule[#1]{%
\GRANULEpremise{  \Gamma  ;  \Omega  \GRANULEsym{,}   \GRANULEmv{x_{{\mathrm{2}}}}  :_{\textcolor{coeffectColor}{  \GRANULEnt{r}  } }   \GRANULEnt{A}   \vdash  \GRANULEnt{B}  \Rightarrow^{+}  \GRANULEnt{t}  \ | \  \Delta  \GRANULEsym{,}   \GRANULEmv{x_{{\mathrm{2}}}}  :_{\textcolor{coeffectColor}{  \GRANULEnt{s}  } }   \GRANULEnt{A}    \quad\;\;   \GRANULEnt{s}  \, {\leq} \,  \GRANULEnt{r}  }%
}{
 \Gamma  ;  \Omega  \GRANULEsym{,}   \GRANULEmv{x_{{\mathrm{1}}}}  :   \Box_{  \GRANULEnt{r}  }  \GRANULEnt{A}    \vdash  \GRANULEnt{B}  \Rightarrow^{+}   \textbf{let} \, [  \GRANULEmv{x_{{\mathrm{2}}}}  ] =  \GRANULEmv{x_{{\mathrm{1}}}}  \, \textbf{in} \,  \GRANULEnt{t}   \ | \  \Delta  \GRANULEsym{,}   \GRANULEmv{x_{{\mathrm{1}}}}  :   \Box_{  \GRANULEnt{r}  }  \GRANULEnt{A}   }{%
{\GRANULEdrulefocusXXsynXXaddXXunboxName}{}%
}}


\newcommand{\GRANULEdrulefocusXXsynXXaddXXpairXXintroName}[0]{\GRANULEdrulename{focus\_syn\_add\_pair\_intro}}
\newcommand{\GRANULEdrulefocusXXsynXXaddXXpairXXintro}[1]{\GRANULEdrule[#1]{%
\GRANULEpremise{ \Gamma  ;   \emptyset   \vdash  \GRANULEnt{A}  \Rightarrow^{+}  \GRANULEnt{t_{{\mathrm{1}}}}  \ | \  \Delta_{{\mathrm{1}}} }%
\GRANULEpremise{ \Gamma  \GRANULEsym{-}  \Delta_{{\mathrm{1}}}  ;   \emptyset   \vdash  \GRANULEnt{B}  \Rightarrow^{+}  \GRANULEnt{t_{{\mathrm{2}}}}  \ | \  \Delta_{{\mathrm{2}}} }%
}{
\texttt{\textcolor{red}{<<no parses (char 12): G ; . \mbox{$\mid$}- A **** B =>+ t1, t2 ; D1 + D2 >>}}}{%
{\GRANULEdrulefocusXXsynXXaddXXpairXXintroName}{}%
}}


\newcommand{\GRANULEdrulefocusXXsynXXaddXXpairXXelimName}[0]{\GRANULEdrulename{focus\_syn\_add\_pair\_elim}}
\newcommand{\GRANULEdrulefocusXXsynXXaddXXpairXXelim}[1]{\GRANULEdrule[#1]{%
\GRANULEpremise{ \Gamma  ;  \Omega  \GRANULEsym{,}   \GRANULEmv{x_{{\mathrm{1}}}}  :  \GRANULEnt{A}   \GRANULEsym{,}   \GRANULEmv{x_{{\mathrm{2}}}}  :  \GRANULEnt{B}   \vdash  \GRANULEnt{C}  \Rightarrow^{+}  \GRANULEnt{t_{{\mathrm{2}}}}  \ | \  \Delta  \GRANULEsym{,}   \GRANULEmv{x_{{\mathrm{1}}}}  :  \GRANULEnt{A}   \GRANULEsym{,}   \GRANULEmv{x_{{\mathrm{2}}}}  :  \GRANULEnt{B}  }%
}{
\texttt{\textcolor{red}{<<no parses (char 15): G ; O, x3 : A **** B \mbox{$\mid$}- C =>+ let x1, x2 = x3 in t2 ; D, x3 : A * B >>}}}{%
{\GRANULEdrulefocusXXsynXXaddXXpairXXelimName}{}%
}}


\newcommand{\GRANULEdrulefocusXXsynXXaddXXsumXXintroXXleftName}[0]{\GRANULEdrulename{focus\_syn\_add\_sum\_intro\_left}}
\newcommand{\GRANULEdrulefocusXXsynXXaddXXsumXXintroXXleft}[1]{\GRANULEdrule[#1]{%
\GRANULEpremise{ \Gamma  ;   \emptyset   \vdash  \GRANULEnt{A}  \Rightarrow^{+}  \GRANULEnt{t}  \ | \  \Delta }%
}{
\texttt{\textcolor{red}{<<no parses (char 12): G ; . \mbox{$\mid$}- A +*** B =>+ inl t ; D >>}}}{%
{\GRANULEdrulefocusXXsynXXaddXXsumXXintroXXleftName}{}%
}}


\newcommand{\GRANULEdrulefocusXXsynXXaddXXsumXXintroXXrightName}[0]{\GRANULEdrulename{focus\_syn\_add\_sum\_intro\_right}}
\newcommand{\GRANULEdrulefocusXXsynXXaddXXsumXXintroXXright}[1]{\GRANULEdrule[#1]{%
\GRANULEpremise{ \Gamma  ;   \emptyset   \vdash  \GRANULEnt{B}  \Rightarrow^{+}  \GRANULEnt{t}  \ | \  \Delta }%
}{
\texttt{\textcolor{red}{<<no parses (char 12): G ; . \mbox{$\mid$}- A +*** B =>+ inr t ; D >>}}}{%
{\GRANULEdrulefocusXXsynXXaddXXsumXXintroXXrightName}{}%
}}


\newcommand{\GRANULEdrulefocusXXsynXXaddXXsumXXelimName}[0]{\GRANULEdrulename{focus\_syn\_add\_sum\_elim}}
\newcommand{\GRANULEdrulefocusXXsynXXaddXXsumXXelim}[1]{\GRANULEdrule[#1]{%
\GRANULEpremise{ \Gamma  ;  \Omega  \GRANULEsym{,}   \GRANULEmv{x_{{\mathrm{2}}}}  :  \GRANULEnt{A}   \vdash  \GRANULEnt{C}  \Rightarrow^{+}  \GRANULEnt{t_{{\mathrm{1}}}}  \ | \  \Delta_{{\mathrm{1}}}  \GRANULEsym{,}   \GRANULEmv{x_{{\mathrm{2}}}}  :  \GRANULEnt{A}  }%
\GRANULEpremise{ \Gamma  ;  \Omega  \GRANULEsym{,}   \GRANULEmv{x_{{\mathrm{3}}}}  :  \GRANULEnt{B}   \vdash  \GRANULEnt{C}  \Rightarrow^{+}  \GRANULEnt{t_{{\mathrm{2}}}}  \ | \  \Delta_{{\mathrm{2}}}  \GRANULEsym{,}   \GRANULEmv{x_{{\mathrm{3}}}}  :  \GRANULEnt{B}  }%
}{
\texttt{\textcolor{red}{<<no parses (char 18): G ; O, x1 : A + B ***\mbox{$\mid$}- C =>+ case x1 of inl x2 -> t1 \mbox{$\mid$} inr x3 -> t2 ; (D1 +++ D2), x1 : A + B >>}}}{%
{\GRANULEdrulefocusXXsynXXaddXXsumXXelimName}{}%
}}

\newcommand{\GRANULEdefnFocusSynthesisAdd}[1]{\begin{GRANULEdefnblock}[#1]{$ \Gamma  ;  \Omega  \vdash  \GRANULEnt{A}  \Rightarrow^{+}  \GRANULEnt{t}  \ | \  \Delta $}{}
\GRANULEusedrule{\GRANULEdrulefocusXXsynXXaddXXlinXXvar{}}
\GRANULEusedrule{\GRANULEdrulefocusXXsynXXaddXXgrXXvar{}}
\GRANULEusedrule{\GRANULEdrulefocusXXsynXXaddXXder{}}
\GRANULEusedrule{\GRANULEdrulefocusXXsynXXaddXXabs{}}
\GRANULEusedrule{\GRANULEdrulefocusXXsynXXaddXXapp{}}
\GRANULEusedrule{\GRANULEdrulefocusXXsynXXaddXXbox{}}
\GRANULEusedrule{\GRANULEdrulefocusXXsynXXaddXXunbox{}}
\GRANULEusedrule{\GRANULEdrulefocusXXsynXXaddXXpairXXintro{}}
\GRANULEusedrule{\GRANULEdrulefocusXXsynXXaddXXpairXXelim{}}
\GRANULEusedrule{\GRANULEdrulefocusXXsynXXaddXXsumXXintroXXleft{}}
\GRANULEusedrule{\GRANULEdrulefocusXXsynXXaddXXsumXXintroXXright{}}
\GRANULEusedrule{\GRANULEdrulefocusXXsynXXaddXXsumXXelim{}}
\end{GRANULEdefnblock}}


\newcommand{\GRANULEdefnsJudgements}{
\GRANULEdefncheckMiniDecl{}\GRANULEdefnpatternTyping{}\GRANULEdefnsynthesiseSub{}\GRANULEdefnSynthesiseAdd{}\GRANULEdefnRASynFocus{}\GRANULEdefnLASynFocus{}\GRANULEdefnRSSynFocus{}\GRANULEdefnLSSynFocus{}\GRANULEdefnSynthesisSub{}\GRANULEdefnSynthesisAdd{}\GRANULEdefnFocusSynthesisSub{}\GRANULEdefnFocusSynthesisAdd{}}

\newcommand{\GRANULEdefnss}{
\GRANULEdefnsJudgements
}

\newcommand{\GRANULEall}{\GRANULEmetavars\\[0pt]
\GRANULEgrammar\\[5.0mm]
\GRANULEdefnss}


%%% LINEAR BASE NAMES AND RULES %%%
% Subtractive Rule Names
\newcommand{\subLinVarName}{\textsc{LinVar$^{-}$}}
\newcommand{\subGrVarName}{\textsc{GrVar$^{-}$}}
\newcommand{\subAbsName}{$\multimap^{-}_{R}$}
\newcommand{\subAppName}{$\multimap^{-}_{L}$}
\newcommand{\subBoxName}{$\square^{-}_{R}$}
\newcommand{\subUnboxName}{$\square^{-}_{L}$}
\newcommand{\subPairIntroName}{$\otimes^{-}_{R}$}
\newcommand{\subPairElimName}{$\otimes^{-}_{L}$}
\newcommand{\subSumIntroLname}{$\oplus1^{-}_{R}$}
\newcommand{\subSumIntroRname}{$\oplus2^{-}_{R}$}
\newcommand{\subSumElimName}{$\oplus^{-}_{L}$}
\newcommand{\subUnitIntroName}{1$^{-}_{R}$}
\newcommand{\subUnitElimName}{1$^{-}_{L}$}
\newcommand{\subDerName}{\textsc{der$^{-}$}}

% Additive Rule Names
\newcommand{\addLinVarName}{\textsc{LinVar$^{+}$}}
\newcommand{\addGrVarName}{\textsc{GrVar$^{+}$}}
\newcommand{\addDerName}{\textsc{der$^{+}$}}
\newcommand{\addAbsName}{$\multimap^{+}_{R}$}
\newcommand{\addAppName}{$\multimap^{+}_{L}$}
\newcommand{\addPruningAppName}{$\multimap'^{+}_{L}$}
\newcommand{\addBoxName}{$\square^{+}_{R}$}
\newcommand{\addUnboxName}{$\square^{+}_{L}$}
\newcommand{\addPairIntroName}{$\otimes^{+}_{R}$}
\newcommand{\addPruningPairIntroName}{$\otimes'^{+}_{R}$}
\newcommand{\addPairElimName}{$\otimes^{+}_{L}$}
\newcommand{\addSumIntroLName}{$\oplus1_{R}^{+}$}
\newcommand{\addSumIntroRName}{$\oplus2_{R}^{+}$}
\newcommand{\addSumElimName}{$\oplus^{+}_{L}$}
\newcommand{\addUnitIntroName}{1$^{+}_{R}$}
\newcommand{\addUnitElimName}{1$^{+}_{L}$}

% Focusing Subtractive Rule Names
\newcommand*{\fSubAbsName}{$\multimap^{-}_{R}$}
\newcommand*{\fSubRAsyncTransitionName}{$\Uparrow^{-}_{R}$}
\newcommand*{\fSubPairElimName}{$\otimes^{-}_{L}$}
\newcommand*{\fSubSumElimName}{$\oplus^{-}_{L}$}
\newcommand*{\fSubUnboxName}{$\square^{-}_{L}$}
\newcommand*{\fSubUnitElimName}{1$^{-}_{L}$}
\newcommand*{\fSubDerName}{der$^{-}$}
\newcommand*{\fSubLAsyncTransitionName}{$\Uparrow^{-}_{L}$}
\newcommand*{\fSubFocusRName}{focus$^{-}_{R}$}
\newcommand*{\fSubFocusLName}{focus$^{-}_{L}$}
\newcommand*{\fSubPairIntroName}{$\otimes^{-}_{R}$}
\newcommand*{\fSubSumIntroNameL}{$\oplus1_{R}^{-}$}
\newcommand*{\fSubSumIntroNameR}{$\oplus2_{R}^{-}$}
\newcommand*{\fSubBoxName}{$\square^{-}_{R}$}
\newcommand*{\fSubUnitIntroName}{1$^{-}_{R}$}
\newcommand*{\fSubRSyncTransitionName}{$\Downarrow^{-}_{R}$}
\newcommand*{\fSubAppName}{$\multimap^{-}_{L}$}
\newcommand*{\fSubLinVarName}{LinVar$^{-}$}
\newcommand*{\fSubGrVarName}{GrVar$^{-}$}
\newcommand*{\fSubLSyncTransitionName}{$\Downarrow^{-}_{L}$}

% Focusing Additive Rule Names
\newcommand*{\fAddAbsName}{$\multimap^{+}_{R}$}
\newcommand*{\fAddRAsyncTransitionName}{$\Uparrow^{+}_{R}$}
\newcommand*{\fAddPairElimName}{$\otimes^{+}_{L}$}
\newcommand*{\fAddSumElimName}{$\oplus^{+}_{L}$}
\newcommand*{\fAddUnboxName}{$\square^{+}_{L}$}
\newcommand*{\fAddUnitElimName}{1$^{+}_{L}$}
\newcommand*{\fAddDerName}{der$^{+}$}
\newcommand*{\fAddLAsyncTransitionName}{$\Uparrow^{+}_{L}$}
\newcommand*{\fAddFocusRName}{focus$^{+}_{R}$}
\newcommand*{\fAddFocusLName}{focus$^{+}_{L}$}
\newcommand*{\fAddPairIntroName}{$\otimes^{+}_{R}$}
\newcommand*{\fAddSumIntroNameL}{$\oplus1_{L}^{+}$}
\newcommand*{\fAddSumIntroNameR}{$\oplus2_{L}^{+}$}
\newcommand*{\fAddBoxName}{$\square^{+}_{R}$}
\newcommand*{\fAddUnitIntroName}{1$^{+}_{R}$}
\newcommand*{\fAddRSyncTransitionName}{$\Downarrow^{+}_{R}$}
\newcommand*{\fAddAppName}{$\multimap^{+}_{L}$}
\newcommand*{\fAddLinVarName}{LinVar$^{+}$}
\newcommand*{\fAddGrVarName}{GrVar$^{+}$}
\newcommand*{\fAddLSyncTransitionName}{$\Downarrow^{+}_{L}$}

\newcommand*{\fAddAltAppName}{$\multimap'^{+}_{L}$}
\newcommand*{\fAddAltPairIntroName}{$\otimes'^{+}_{R}$}


%%% GRADED BASE NAMES %%%

% Typing renames
\renewcommand{\GRANULEdruleTyVarName}{$\textsc{Var}$}
\renewcommand{\GRANULEdruleTyAbsName}{$\textsc{Abs}$}
\renewcommand{\GRANULEdruleTyAppName}{$\textsc{App}$}
\renewcommand{\GRANULEdruleTyConName}{$\textsc{Con}$}
\renewcommand{\GRANULEdruleTyCaseName}{$\textsc{Case}$}
\renewcommand{\GRANULEdruleTyPrName}{$\textsc{Pr}$}
\renewcommand{\GRANULEdruleTyApproxName}{$\textsc{Approx}$}

% Pattern typing renames
\renewcommand{\GRANULEdrulePatWildName}{$\textsc{PWild}$}
\renewcommand{\GRANULEdrulePatVarName}{$\textsc{PVar}$}
\renewcommand{\GRANULEdrulePatBoxName}{$\textsc{PBox}$}
\renewcommand{\GRANULEdrulePatConName}{$\textsc{PCon}$}

% Synthesis rule names

% Focusing synhtesis rule names
\newcommand*{\fsynRAsyncTransName}{$\Uparrow_{\textsc{R}}$}
\newcommand*{\fsynLAsyncTransName}{$\Uparrow_{\textsc{L}}$}
\newcommand*{\fsynFocusRName}{$\textsc{Foc}_{\textsc{R}}$}
\newcommand*{\fsynFocusLName}{$\textsc{Foc}_{\textsc{L}}$}
\newcommand*{\fsynRSyncTransName}{$\Downarrow_{\textsc{R}}$}
\newcommand*{\fsynLSyncTransName}{$\Downarrow_{\textsc{L}}$}

% GRADED BASE TYPING RULES
\newcommand*{\tyVar}{\GRANULEdruleTyVar{}}
\newcommand*{\tyAbs}{\GRANULEdruleTyAbs{}}
\newcommand*{\tyApp}{\GRANULEdruleTyApp{}}
\newcommand*{\tyApprox}{\GRANULEdruleTyApprox{}}
\newcommand*{\tyPr}{\GRANULEdruleTyPr{}}
\newcommand*{\tyCon}{\GRANULEdruleTyCon{}}
\newcommand*{\tyCase}{\GRANULEdruleTyCase{}}


% DERIVING NAMES 
% \renewcommand{\GRANULEdruleEquivXXbetaName}{$\beta$}
% \renewcommand{\GRANULEdruleEquivXXetaName}{$\eta$}
% \renewcommand{\GRANULEdruleEquivXXletrecBetaName}{$\beta_{letrec}$}
% \renewcommand{\GRANULEdruleEquivXXletrecDistName}{\small{\textsc{LetRecDistrib}}}
% \renewcommand{\GRANULEdruleEquivXXcaseBetaName}{$\beta_{case}$}
% \renewcommand{\GRANULEdruleEquivXXcaseEtaName}{$\eta_{case}$}
% \renewcommand{\GRANULEdruleEquivXXcaseGenName}{\small{\textsc{CaseGen}}}
% \renewcommand{\GRANULEdruleEquivXXcaseDistName}{\small{\textsc{CaseDistrib}}}
% \renewcommand{\GRANULEdruleEquivXXcasePushDerivedName}{\small{\textsc{CasePush}}}
% \renewcommand{\GRANULEdruleEquivXXcaseBoxAssocName}{\small{[\textsc{CaseAssoc}]}}
% \renewcommand{\GRANULEdruleEquivXXcaseAssocName}{\small{\textsc{CaseAssoc}}}

%%% LINEAR BASE NAMES AND RULES %%%
% Subtractive Rule Names
\newcommand{\subLinVarName}{\textsc{LinVar$^{-}$}}
\newcommand{\subGrVarName}{\textsc{GrVar$^{-}$}}
\newcommand{\subAbsName}{$\multimap^{-}_{R}$}
\newcommand{\subAppName}{$\multimap^{-}_{L}$}
\newcommand{\subBoxName}{$\square^{-}_{R}$}
\newcommand{\subUnboxName}{$\square^{-}_{L}$}
\newcommand{\subPairIntroName}{$\otimes^{-}_{R}$}
\newcommand{\subPairElimName}{$\otimes^{-}_{L}$}
\newcommand{\subSumIntroLname}{$\oplus1^{-}_{R}$}
\newcommand{\subSumIntroRname}{$\oplus2^{-}_{R}$}
\newcommand{\subSumElimName}{$\oplus^{-}_{L}$}
\newcommand{\subUnitIntroName}{1$^{-}_{R}$}
\newcommand{\subUnitElimName}{1$^{-}_{L}$}
\newcommand{\subDerName}{\textsc{der$^{-}$}}

% Additive Rule Names
\newcommand{\addLinVarName}{\textsc{LinVar$^{+}$}}
\newcommand{\addGrVarName}{\textsc{GrVar$^{+}$}}
\newcommand{\addDerName}{\textsc{der$^{+}$}}
\newcommand{\addAbsName}{$\multimap^{+}_{R}$}
\newcommand{\addAppName}{$\multimap^{+}_{L}$}
\newcommand{\addPruningAppName}{$\multimap'^{+}_{L}$}
\newcommand{\addBoxName}{$\square^{+}_{R}$}
\newcommand{\addUnboxName}{$\square^{+}_{L}$}
\newcommand{\addPairIntroName}{$\otimes^{+}_{R}$}
\newcommand{\addPruningPairIntroName}{$\otimes'^{+}_{R}$}
\newcommand{\addPairElimName}{$\otimes^{+}_{L}$}
\newcommand{\addSumIntroLName}{$\oplus1_{R}^{+}$}
\newcommand{\addSumIntroRName}{$\oplus2_{R}^{+}$}
\newcommand{\addSumElimName}{$\oplus^{+}_{L}$}
\newcommand{\addUnitIntroName}{1$^{+}_{R}$}
\newcommand{\addUnitElimName}{1$^{+}_{L}$}

% Focusing Subtractive Rule Names
\newcommand*{\fSubAbsName}{$\multimap^{-}_{R}$}
\newcommand*{\fSubRAsyncTransitionName}{$\Uparrow^{-}_{R}$}
\newcommand*{\fSubPairElimName}{$\otimes^{-}_{L}$}
\newcommand*{\fSubSumElimName}{$\oplus^{-}_{L}$}
\newcommand*{\fSubUnboxName}{$\square^{-}_{L}$}
\newcommand*{\fSubUnitElimName}{1$^{-}_{L}$}
\newcommand*{\fSubDerName}{der$^{-}$}
\newcommand*{\fSubLAsyncTransitionName}{$\Uparrow^{-}_{L}$}
\newcommand*{\fSubFocusRName}{focus$^{-}_{R}$}
\newcommand*{\fSubFocusLName}{focus$^{-}_{L}$}
\newcommand*{\fSubPairIntroName}{$\otimes^{-}_{R}$}
\newcommand*{\fSubSumIntroNameL}{$\oplus1_{R}^{-}$}
\newcommand*{\fSubSumIntroNameR}{$\oplus2_{R}^{-}$}
\newcommand*{\fSubBoxName}{$\square^{-}_{R}$}
\newcommand*{\fSubUnitIntroName}{1$^{-}_{R}$}
\newcommand*{\fSubRSyncTransitionName}{$\Downarrow^{-}_{R}$}
\newcommand*{\fSubAppName}{$\multimap^{-}_{L}$}
\newcommand*{\fSubLinVarName}{LinVar$^{-}$}
\newcommand*{\fSubGrVarName}{GrVar$^{-}$}
\newcommand*{\fSubLSyncTransitionName}{$\Downarrow^{-}_{L}$}

% Focusing Additive Rule Names
\newcommand*{\fAddAbsName}{$\multimap^{+}_{R}$}
\newcommand*{\fAddRAsyncTransitionName}{$\Uparrow^{+}_{R}$}
\newcommand*{\fAddPairElimName}{$\otimes^{+}_{L}$}
\newcommand*{\fAddSumElimName}{$\oplus^{+}_{L}$}
\newcommand*{\fAddUnboxName}{$\square^{+}_{L}$}
\newcommand*{\fAddUnitElimName}{1$^{+}_{L}$}
\newcommand*{\fAddDerName}{der$^{+}$}
\newcommand*{\fAddLAsyncTransitionName}{$\Uparrow^{+}_{L}$}
\newcommand*{\fAddFocusRName}{focus$^{+}_{R}$}
\newcommand*{\fAddFocusLName}{focus$^{+}_{L}$}
\newcommand*{\fAddPairIntroName}{$\otimes^{+}_{R}$}
\newcommand*{\fAddSumIntroNameL}{$\oplus1_{L}^{+}$}
\newcommand*{\fAddSumIntroNameR}{$\oplus2_{L}^{+}$}
\newcommand*{\fAddBoxName}{$\square^{+}_{R}$}
\newcommand*{\fAddUnitIntroName}{1$^{+}_{R}$}
\newcommand*{\fAddRSyncTransitionName}{$\Downarrow^{+}_{R}$}
\newcommand*{\fAddAppName}{$\multimap^{+}_{L}$}
\newcommand*{\fAddLinVarName}{LinVar$^{+}$}
\newcommand*{\fAddGrVarName}{GrVar$^{+}$}
\newcommand*{\fAddLSyncTransitionName}{$\Downarrow^{+}_{L}$}

% SUBTRACTIVE RULES
\newcommand*{\subLinVar}{
                             \inferrule*[right=\subLinVarName]
                             {\quad}{\Gamma  \GRANULEsym{,}   \GRANULEmv{x}  :  \GRANULEnt{A}   \vdash  \GRANULEnt{A}  \Rightarrow^-  \GRANULEmv{x} \ |\  \Gamma}
  }

  \newcommand*{\subGrVar}{
      \inferrule*[right=\subGrVarName]
  {\exists  \GRANULEnt{s}  .\,   \GRANULEnt{r}  \sqsubseteq   \GRANULEnt{s}  \GRANULEsym{+}   1}{\Gamma  \GRANULEsym{,}   \GRANULEmv{x}  :_{\textcolor{coeffectColor}{  \GRANULEnt{r}  } }   \GRANULEnt{A}   \vdash  \GRANULEnt{A}  \Rightarrow^-  \GRANULEmv{x} \ |\  \Gamma  \GRANULEsym{,}   \GRANULEmv{x}  :_{\textcolor{coeffectColor}{  \GRANULEnt{s}  } }   \GRANULEnt{A}}
    }

  \newcommand*{\subAbs}{
      \inferrule*[right=\subAbsName]
  {\Gamma  \GRANULEsym{,}   \GRANULEmv{x}  :  \GRANULEnt{A}   \vdash  \GRANULEnt{B}  \Rightarrow^-  \GRANULEnt{t} \ |\  \Delta \quad\; \GRANULEmv{x}  \not\in | \Delta |}{\Gamma  \vdash   \GRANULEnt{A}  \multimap  \GRANULEnt{B}   \Rightarrow^-   \lambda  \GRANULEmv{x}  .  \GRANULEnt{t}  \ |\  \Delta}
    }
  \newcommand*{\subApp}{
  \inferrule*[right=\subAppName]
  {\Gamma  \GRANULEsym{,}   \GRANULEmv{x_{{\mathrm{2}}}}  :  \GRANULEnt{B}   \vdash  \GRANULEnt{C}  \Rightarrow^-  \GRANULEnt{t_{{\mathrm{1}}}} \ |\  \Delta_{{\mathrm{1}}} \qquad \GRANULEmv{x_{{\mathrm{2}}}}  \not\in | \Delta_{{\mathrm{1}}} | \qquad \Delta_{{\mathrm{1}}}  \vdash  \GRANULEnt{A}  \Rightarrow^-  \GRANULEnt{t_{{\mathrm{2}}}} \ |\  \Delta_{{\mathrm{2}}}}{\Gamma  \GRANULEsym{,}   \GRANULEmv{x_{{\mathrm{1}}}}  :   \GRANULEnt{A}  \multimap  \GRANULEnt{B}    \vdash  \GRANULEnt{C}  \Rightarrow^-   [  \GRANULEsym{(}  \GRANULEmv{x_{{\mathrm{1}}}} \, \GRANULEnt{t_{{\mathrm{2}}}}  \GRANULEsym{)}  /  \GRANULEmv{x_{{\mathrm{2}}}}  ]  \GRANULEnt{t_{{\mathrm{1}}}}  \ |\  \Delta_{{\mathrm{2}}}}
    }
  \newcommand*{\subDer}{
      \inferrule*[right=\subDerName]
{\Gamma  \GRANULEsym{,}   \GRANULEmv{x}  :_{\textcolor{coeffectColor}{  \GRANULEnt{s}  } }   \GRANULEnt{A}   \GRANULEsym{,}   \GRANULEmv{y}  :  \GRANULEnt{A}   \vdash  \GRANULEnt{B}  \Rightarrow^-  \GRANULEnt{t} \ |\  \Delta  \GRANULEsym{,}   \GRANULEmv{x}  :_{\textcolor{coeffectColor}{  \GRANULEnt{s'}  } }   \GRANULEnt{A} \\
\GRANULEmv{y}  \not\in | \Delta | \\
\exists  \GRANULEnt{s}  .\,   \GRANULEnt{r}  \sqsupseteq  \GRANULEnt{s}  \GRANULEsym{+}   1
}
{\Gamma  \GRANULEsym{,}   \GRANULEmv{x}  :_{\textcolor{coeffectColor}{  \GRANULEnt{r}  } }   \GRANULEnt{A}   \vdash  \GRANULEnt{B}  \Rightarrow^-   [  \GRANULEmv{x}  /  \GRANULEmv{y}  ]  \GRANULEnt{t}  \ |\  \Delta  \GRANULEsym{,}   \GRANULEmv{x}  :_{\textcolor{coeffectColor}{  \GRANULEnt{s'}  } }   \GRANULEnt{A}}
    }
  \newcommand*{\subBox}{
  \inferrule*[right=\subBoxName]
  {\Gamma  \vdash  \GRANULEnt{A}  \Rightarrow^-  \GRANULEnt{t} \ |\  \Delta}{\Gamma  \vdash   \Box_{  \GRANULEnt{r}  }  \GRANULEnt{A}   \Rightarrow^-  \GRANULEsym{[}  \GRANULEnt{t}  \GRANULEsym{]} \ |\  \Gamma  \GRANULEsym{-}   \textcolor{coeffectColor}{ \GRANULEnt{r}   \textcolor{coeffectColor}{\,\cdot\,} }  \GRANULEsym{(}  \Gamma  \GRANULEsym{-}  \Delta  \GRANULEsym{)}}
    }
  \newcommand*{\subBoxAlt}{
  \inferrule*[right=R${\square'^{-}}$]
  {\texttt{\textcolor{red}{<<no parses (char 3): G /*** r \mbox{$\mid$}- A =>- t ; D >>}}}{\Gamma  \vdash   \Box_{  \GRANULEnt{r}  }  \GRANULEnt{A}   \Rightarrow^-  \GRANULEsym{[}  \GRANULEnt{t}  \GRANULEsym{]} \ |\   \textcolor{coeffectColor}{ \GRANULEnt{r}   \textcolor{coeffectColor}{\,\cdot\,} }  \Delta}
    }
  \newcommand*{\subUnbox}{
  \inferrule*[right=\subUnboxName]
    {\Gamma  \GRANULEsym{,}   \GRANULEmv{x_{{\mathrm{2}}}}  :_{\textcolor{coeffectColor}{  \GRANULEnt{r}  } }   \GRANULEnt{A}   \vdash  \GRANULEnt{B}  \Rightarrow^-  \GRANULEnt{t} \ |\  \Delta  \GRANULEsym{,}   \GRANULEmv{x_{{\mathrm{2}}}}  :_{\textcolor{coeffectColor}{  \GRANULEnt{s}  } }   \GRANULEnt{A} \\ 0 \sqsubseteq \GRANULEnt{s}}{\Gamma  \GRANULEsym{,}   \GRANULEmv{x_{{\mathrm{1}}}}  :   \Box_{  \GRANULEnt{r}  }  \GRANULEnt{A}    \vdash  \GRANULEnt{B}  \Rightarrow^-   \textbf{let} \, [  \GRANULEmv{x_{{\mathrm{2}}}}  ] =  \GRANULEmv{x_{{\mathrm{1}}}}  \, \textbf{in} \,  \GRANULEnt{t}  \ |\  \Delta}
    }
  \newcommand*{\subPairIntro}{
    \inferrule*[right=\subPairIntroName]
    {\Gamma  \vdash  \GRANULEnt{A}  \Rightarrow^-  \GRANULEnt{t_{{\mathrm{1}}}} \ |\  \Delta_{{\mathrm{1}}} \\ \Delta_{{\mathrm{1}}}  \vdash  \GRANULEnt{B}  \Rightarrow^-  \GRANULEnt{t_{{\mathrm{2}}}} \ |\  \Delta_{{\mathrm{2}}}}{\Gamma  \vdash   \GRANULEnt{A}  \, \otimes \,  \GRANULEnt{B}   \Rightarrow^-   ( \GRANULEnt{t_{{\mathrm{1}}}} ,  \GRANULEnt{t_{{\mathrm{2}}}} )  \ |\  \Delta_{{\mathrm{2}}}}
    }
  \newcommand*{\subPairElim}{
    \inferrule*[right=\subPairElimName]
    {\Gamma  \GRANULEsym{,}   \GRANULEmv{x_{{\mathrm{1}}}}  :  \GRANULEnt{A}   \GRANULEsym{,}   \GRANULEmv{x_{{\mathrm{2}}}}  :  \GRANULEnt{B}   \vdash  \GRANULEnt{C}  \Rightarrow^-  \GRANULEnt{t_{{\mathrm{2}}}} \ |\  \Delta \\ \GRANULEmv{x_{{\mathrm{1}}}}  \not\in | \Delta | \\ \GRANULEmv{x_{{\mathrm{2}}}}  \not\in | \Delta |}
    {\Gamma  \GRANULEsym{,}   \GRANULEmv{x_{{\mathrm{3}}}}  :   \GRANULEnt{A}  \, \otimes \,  \GRANULEnt{B}    \vdash  \GRANULEnt{C}  \Rightarrow^-   \textbf{let} \, ( \GRANULEmv{x_{{\mathrm{1}}}} ,  \GRANULEmv{x_{{\mathrm{2}}}} ) =  \GRANULEmv{x_{{\mathrm{3}}}}  \, \textbf{in} \,  \GRANULEnt{t_{{\mathrm{2}}}}  \ |\  \Delta}
    }
  \newcommand*{\subSumIntroL}{
    \inferrule*[right=\subSumIntroLname]
    {\Gamma  \vdash  \GRANULEnt{A}  \Rightarrow^-  \GRANULEnt{t} \ |\  \Delta}
    {\Gamma  \vdash   \GRANULEnt{A}  \, \oplus \,  \GRANULEnt{B}   \Rightarrow^-  \GRANULEkw{inl} \, \GRANULEnt{t} \ |\  \Delta}
    }
  \newcommand*{\subSumIntroR}{
    \inferrule*[right=\subSumIntroRname]
    {\Gamma  \vdash  \GRANULEnt{B}  \Rightarrow^-  \GRANULEnt{t} \ |\  \Delta}
    {\Gamma  \vdash   \GRANULEnt{A}  \, \oplus \,  \GRANULEnt{B}   \Rightarrow^-  \GRANULEkw{inr} \, \GRANULEnt{t} \ |\  \Delta}
    }
  \newcommand*{\subSumElim}{
    \inferrule*[right=\subSumElimName]
      {\Gamma  \GRANULEsym{,}   \GRANULEmv{x_{{\mathrm{2}}}}  :  \GRANULEnt{A}   \vdash  \GRANULEnt{C}  \Rightarrow^-  \GRANULEnt{t_{{\mathrm{1}}}} \ |\  \Delta_{{\mathrm{1}}} \quad\,
       \Gamma  \GRANULEsym{,}   \GRANULEmv{x_{{\mathrm{3}}}}  :  \GRANULEnt{B}   \vdash  \GRANULEnt{C}  \Rightarrow^-  \GRANULEnt{t_{{\mathrm{2}}}} \ |\  \Delta_{{\mathrm{2}}} \quad\, \GRANULEmv{x_{{\mathrm{2}}}}  \not\in | \Delta_{{\mathrm{1}}} | \quad \GRANULEmv{x_{{\mathrm{3}}}}  \not\in | \Delta_{{\mathrm{2}}} |}
{\Gamma, x_1 : \GRANULEnt{A}  \, \oplus \,  \GRANULEnt{B} \vdash C \Rightarrow^- \textbf{case} \ x_{1}\ \textbf{of}\ \textbf{inl}\ x_{2} \rightarrow t_{1};\ \textbf{inr}\ x_{3} \rightarrow t_{2} \ | \ \Delta_1 \sqcap \Delta_2}
    }
  \newcommand*{\subUnitIntro}{
    \inferrule*[right=\subUnitIntroName]
    {\quad}
    {\Gamma  \vdash   \mathsf{1}   \Rightarrow^-  \GRANULEsym{()} \ |\  \Gamma}
    }
  \newcommand*{\subUnitElim}{
    \inferrule*[right=\subUnitElimName]
    {\Gamma  \vdash  \GRANULEnt{C}  \Rightarrow^-  \GRANULEnt{t} \ |\  \Delta}
    {\Gamma  \GRANULEsym{,}   \GRANULEmv{x}  :   \mathsf{1}    \vdash  \GRANULEnt{C}  \Rightarrow^-  \GRANULEkw{let} \, \GRANULEsym{()}  \GRANULEsym{=}  \GRANULEmv{x} \, \GRANULEkw{in} \, \GRANULEnt{t} \ |\  \Delta}
    }

% ADDITIVE RULES
\newcommand*{\addLinVar}{
    \inferrule*[right=LinVar$^{+}$]
    {\quad}
    {\Gamma  \GRANULEsym{,}   \GRANULEmv{x}  :  \GRANULEnt{A}   \vdash  \GRANULEnt{A}  \Rightarrow^+  \GRANULEmv{x}  ;\,   \GRANULEmv{x}  :  \GRANULEnt{A}}
  }

  \newcommand*{\addGrVar}{
    \inferrule*[right=GrVar$^{+}$]
    {\quad}
      {\Gamma  \GRANULEsym{,}   \GRANULEmv{x}  :_{\textcolor{coeffectColor}{  \GRANULEnt{r}  } }   \GRANULEnt{A}   \vdash  \GRANULEnt{A}  \Rightarrow^+  \GRANULEmv{x}  ;\,   \GRANULEmv{x}  :_{\textcolor{coeffectColor}{   1   } }   \GRANULEnt{A} }
    }

  \newcommand*{\addAbs}{
    \inferrule*[right=\addAbsName]
    {\Gamma  \GRANULEsym{,}   \GRANULEmv{x}  :  \GRANULEnt{A}   \vdash  \GRANULEnt{B}  \Rightarrow^+  \GRANULEnt{t}  ;\,  \Delta  \GRANULEsym{,}   \GRANULEmv{x}  :  \GRANULEnt{A}}{\Gamma  \vdash   \GRANULEnt{A}  \multimap  \GRANULEnt{B}   \Rightarrow^+   \lambda  \GRANULEmv{x}  .  \GRANULEnt{t}   ;\,  \Delta}
    }
  \newcommand*{\addApp}{
    \inferrule*[right=\addAppName]
    {\Gamma  \GRANULEsym{,}   \GRANULEmv{x_{{\mathrm{2}}}}  :  \GRANULEnt{B}   \vdash  \GRANULEnt{C}  \Rightarrow^+  \GRANULEnt{t_{{\mathrm{1}}}}  ;\,  \Delta_{{\mathrm{1}}}  \GRANULEsym{,}   \GRANULEmv{x_{{\mathrm{2}}}}  :  \GRANULEnt{B} \\ \Gamma  \vdash  \GRANULEnt{A}  \Rightarrow^+  \GRANULEnt{t_{{\mathrm{2}}}}  ;\,  \Delta_{{\mathrm{2}}}}{\Gamma  \GRANULEsym{,}   \GRANULEmv{x_{{\mathrm{1}}}}  :   \GRANULEnt{A}  \multimap  \GRANULEnt{B}    \vdash  \GRANULEnt{C}  \Rightarrow^+   [  \GRANULEsym{(}  \GRANULEmv{x_{{\mathrm{1}}}} \, \GRANULEnt{t_{{\mathrm{2}}}}  \GRANULEsym{)}  /  \GRANULEmv{x_{{\mathrm{2}}}}  ]  \GRANULEnt{t_{{\mathrm{1}}}}   ;\,  \GRANULEsym{(}  \Delta_{{\mathrm{1}}}  \GRANULEsym{+}  \Delta_{{\mathrm{2}}}  \GRANULEsym{)}  \GRANULEsym{,}   \GRANULEmv{x_{{\mathrm{1}}}}  :   \GRANULEnt{A}  \multimap  \GRANULEnt{B} }
    }
\newcommand*{\addPruneApp}{
\inferrule*[right=\addPruningAppName]
    {\Gamma  \GRANULEsym{,}   \GRANULEmv{x_{{\mathrm{2}}}}  :  \GRANULEnt{B}   \vdash  \GRANULEnt{C}  \Rightarrow^+  \GRANULEnt{t_{{\mathrm{1}}}}  ;\,  \Delta_{{\mathrm{1}}}  \GRANULEsym{,}   \GRANULEmv{x_{{\mathrm{2}}}}  :  \GRANULEnt{B} \\ \Gamma  \GRANULEsym{-}  \Delta_{{\mathrm{1}}}  \vdash  \GRANULEnt{A}  \Rightarrow^+  \GRANULEnt{t_{{\mathrm{2}}}}  ;\,  \Delta_{{\mathrm{2}}}}{\Gamma  \GRANULEsym{,}   \GRANULEmv{x_{{\mathrm{1}}}}  :   \GRANULEnt{A}  \multimap  \GRANULEnt{B}    \vdash  \GRANULEnt{C}  \Rightarrow^+   [  \GRANULEsym{(}  \GRANULEmv{x_{{\mathrm{1}}}} \, \GRANULEnt{t_{{\mathrm{2}}}}  \GRANULEsym{)}  /  \GRANULEmv{x_{{\mathrm{2}}}}  ]  \GRANULEnt{t_{{\mathrm{1}}}}   ;\,  \GRANULEsym{(}  \Delta_{{\mathrm{1}}}  \GRANULEsym{+}  \Delta_{{\mathrm{2}}}  \GRANULEsym{)}  \GRANULEsym{,}   \GRANULEmv{x_{{\mathrm{1}}}}  :   \GRANULEnt{A}  \multimap  \GRANULEnt{B} }
}
  \newcommand*{\addDer}{
\inferrule*[right=\addDerName]
{ \Gamma  \GRANULEsym{,}   \GRANULEmv{x}  :_{\textcolor{coeffectColor}{  \GRANULEnt{s}  } }   \GRANULEnt{A}   \GRANULEsym{,}   \GRANULEmv{y}  :  \GRANULEnt{A}   \vdash  \GRANULEnt{B}  \Rightarrow^+  \GRANULEnt{t}  ;\,  \Delta  \GRANULEsym{,}   \GRANULEmv{y}  :  \GRANULEnt{A} }
{ \Gamma  \GRANULEsym{,}   \GRANULEmv{x}  :_{\textcolor{coeffectColor}{  \GRANULEnt{s}  } }   \GRANULEnt{A}   \vdash  \GRANULEnt{B}  \Rightarrow^+   [  \GRANULEmv{x}  /  \GRANULEmv{y}  ]  \GRANULEnt{t}   ;\,  \Delta  \GRANULEsym{+}   \GRANULEmv{x}  :_{\textcolor{coeffectColor}{   1   } }   \GRANULEnt{A} }
    }
  \newcommand*{\addBox}{
    \inferrule*[right=\addBoxName]
    {\Gamma  \vdash  \GRANULEnt{A}  \Rightarrow^+  \GRANULEnt{t}  ;\,  \Delta}{\Gamma  \vdash   \Box_{  \GRANULEnt{r}  }  \GRANULEnt{A}   \Rightarrow^+  \GRANULEsym{[}  \GRANULEnt{t}  \GRANULEsym{]}  ;\,   \textcolor{coeffectColor}{ \GRANULEnt{r}   \textcolor{coeffectColor}{\,\cdot\,} }  \Delta}
    }
  \newcommand*{\addUnbox}{
    \inferrule*[right=\addUnboxName]
    {\Gamma  \GRANULEsym{,}   \GRANULEmv{x_{{\mathrm{2}}}}  :_{\textcolor{coeffectColor}{  \GRANULEnt{r}  } }   \GRANULEnt{A}   \vdash  \GRANULEnt{B}  \Rightarrow^+  \GRANULEnt{t}  ;\,  \Delta \\ \textit{if}\ \GRANULEmv{x_{{\mathrm{2}}}}  :_{\textcolor{coeffectColor}{  \GRANULEnt{s}  } }   \GRANULEnt{A} \in
      \Delta\ \textit{then}\ s \sqsubseteq r \ \textit{else}\  0 \sqsubseteq r}{\Gamma  \GRANULEsym{,}   \GRANULEmv{x_{{\mathrm{1}}}}  :   \Box_{  \GRANULEnt{r}  }  \GRANULEnt{A}    \vdash  \GRANULEnt{B}  \Rightarrow^+   \textbf{let} \, [  \GRANULEmv{x_{{\mathrm{2}}}}  ] =  \GRANULEmv{x_{{\mathrm{1}}}}  \, \textbf{in} \,  \GRANULEnt{t}   ;\,  \GRANULEsym{(}   \Delta \!\setminus\!  \GRANULEmv{x_{{\mathrm{2}}}}   \GRANULEsym{)}  \GRANULEsym{,}   \GRANULEmv{x_{{\mathrm{1}}}}  :   \Box_{  \GRANULEnt{r}  }  \GRANULEnt{A}}
    }
  \newcommand*{\addPairIntro}{
    \inferrule*[right=\addPairIntroName]
    {\Gamma  \vdash  \GRANULEnt{A}  \Rightarrow^+  \GRANULEnt{t_{{\mathrm{1}}}}  ;\,  \Delta_{{\mathrm{1}}} \\ \Gamma  \vdash  \GRANULEnt{B}  \Rightarrow^+  \GRANULEnt{t_{{\mathrm{2}}}}  ;\,  \Delta_{{\mathrm{2}}}}
    {\Gamma  \vdash   \GRANULEnt{A}  \, \otimes \,  \GRANULEnt{B}   \Rightarrow^+   ( \GRANULEnt{t_{{\mathrm{1}}}} ,  \GRANULEnt{t_{{\mathrm{2}}}} )   ;\,  \Delta_{{\mathrm{1}}}  \GRANULEsym{+}  \Delta_{{\mathrm{2}}}}
    }
  \newcommand*{\addPrunePairIntro}{
    \inferrule*[right=\addPruningPairIntroName]
    {\Gamma  \vdash  \GRANULEnt{A}  \Rightarrow^+  \GRANULEnt{t_{{\mathrm{1}}}}  ;\,  \Delta_{{\mathrm{1}}} \\ \Gamma  \GRANULEsym{-}  \Delta_{{\mathrm{1}}}  \vdash  \GRANULEnt{B}  \Rightarrow^+  \GRANULEnt{t_{{\mathrm{2}}}}  ;\,  \Delta_{{\mathrm{2}}}}
    {\Gamma  \vdash   \GRANULEnt{A}  \, \otimes \,  \GRANULEnt{B}   \Rightarrow^+   ( \GRANULEnt{t_{{\mathrm{1}}}} ,  \GRANULEnt{t_{{\mathrm{2}}}} )   ;\,  \Delta_{{\mathrm{1}}}  \GRANULEsym{+}  \Delta_{{\mathrm{2}}}}
    }
  \newcommand*{\addPairElim}{
    \inferrule*[right=\addPairElimName]
    {\Gamma  \GRANULEsym{,}   \GRANULEmv{x_{{\mathrm{1}}}}  :  \GRANULEnt{A}   \GRANULEsym{,}   \GRANULEmv{x_{{\mathrm{2}}}}  :  \GRANULEnt{B}   \vdash  \GRANULEnt{C}  \Rightarrow^+  \GRANULEnt{t_{{\mathrm{2}}}}  ;\,  \Delta  \GRANULEsym{,}   \GRANULEmv{x_{{\mathrm{1}}}}  :  \GRANULEnt{A}   \GRANULEsym{,}   \GRANULEmv{x_{{\mathrm{2}}}}  :  \GRANULEnt{B}}
    {\Gamma  \GRANULEsym{,}   \GRANULEmv{x_{{\mathrm{3}}}}  :   \GRANULEnt{A}  \, \otimes \,  \GRANULEnt{B}    \vdash  \GRANULEnt{C}  \Rightarrow^+   \textbf{let} \, ( \GRANULEmv{x_{{\mathrm{1}}}} ,  \GRANULEmv{x_{{\mathrm{2}}}} ) =  \GRANULEmv{x_{{\mathrm{3}}}}  \, \textbf{in} \,  \GRANULEnt{t_{{\mathrm{2}}}}   ;\,  \Delta  \GRANULEsym{,}   \GRANULEmv{x_{{\mathrm{3}}}}  :   \GRANULEnt{A}  \, \otimes \,  \GRANULEnt{B}}
    }
  \newcommand*{\addSumIntroL}{
    \inferrule*[right=\addSumIntroLName]
    {\Gamma  \vdash  \GRANULEnt{A}  \Rightarrow^+  \GRANULEnt{t}  ;\,  \Delta}
    {\Gamma  \vdash   \GRANULEnt{A}  \, \oplus \,  \GRANULEnt{B}   \Rightarrow^+  \GRANULEkw{inl} \, \GRANULEnt{t}  ;\,  \Delta}
    }
  \newcommand*{\addSumIntroR}{
    \inferrule*[right=\addSumIntroRName]
    {\Gamma  \vdash  \GRANULEnt{B}  \Rightarrow^+  \GRANULEnt{t}  ;\,  \Delta}
    {\Gamma  \vdash   \GRANULEnt{A}  \, \oplus \,  \GRANULEnt{B}   \Rightarrow^+  \GRANULEkw{inr} \, \GRANULEnt{t}  ;\,  \Delta}
    }
  \newcommand*{\addSumElim}{
    \inferrule*[right=\addSumElimName]
    {\Gamma  \GRANULEsym{,}   \GRANULEmv{x_{{\mathrm{2}}}}  :  \GRANULEnt{A}   \vdash  \GRANULEnt{C}  \Rightarrow^+  \GRANULEnt{t_{{\mathrm{1}}}}  ;\,  \Delta_{{\mathrm{1}}}  \GRANULEsym{,}   \GRANULEmv{x_{{\mathrm{2}}}}  :  \GRANULEnt{A} \\ \Gamma  \GRANULEsym{,}   \GRANULEmv{x_{{\mathrm{3}}}}  :  \GRANULEnt{B}   \vdash  \GRANULEnt{C}  \Rightarrow^+  \GRANULEnt{t_{{\mathrm{2}}}}  ;\,  \Delta_{{\mathrm{2}}}  \GRANULEsym{,}   \GRANULEmv{x_{{\mathrm{3}}}}  :  \GRANULEnt{B}}
{\Gamma, x_1 : \GRANULEnt{A}  \, \oplus \,  \GRANULEnt{B} \vdash C \Rightarrow^- \textbf{case} \ x_{1}\ \textbf{of}\ \textbf{inl}\ x_{2} \rightarrow t_{1};\ \textbf{inr}\ x_{3} \rightarrow t_{2} \ |\ \Delta_1 \sqcup \Delta_2, \GRANULEmv{x_{{\mathrm{1}}}}  :   \GRANULEnt{A}  \, \oplus \,  \GRANULEnt{B}}
    }
  \newcommand*{\addUnitIntro}{
    \inferrule*[right=\addUnitIntroName]
    {\quad}
    {\Gamma  \vdash   \mathsf{1}   \Rightarrow^+  \GRANULEsym{()}  ;\,   \emptyset}
    }
  \newcommand*{\addUnitElim}{
    \inferrule*[right=\addUnitElimName]
    {\Gamma  \vdash  \GRANULEnt{C}  \Rightarrow^+  \GRANULEnt{t}  ;\,  \Delta}
    {\Gamma  \GRANULEsym{,}   \GRANULEmv{x}  :   \mathsf{1}    \vdash  \GRANULEnt{C}  \Rightarrow^+  \GRANULEkw{let} \, \GRANULEsym{()}  \GRANULEsym{=}  \GRANULEmv{x} \, \GRANULEkw{in} \, \GRANULEnt{t}  ;\,  \Delta  \GRANULEsym{,}   \GRANULEmv{x}  :   \mathsf{1}}
    }


% FOCUSED SUBTRACTIVE RULES
\newcommand*{\fSubAbsRule}{
    \inferrule*[right=\fSubAbsName,Lab=RightAsync]
    {\Gamma  ;  \Omega  \GRANULEsym{,}   \GRANULEmv{x}  :  \GRANULEnt{A}   \vdash   \GRANULEnt{B}  \Uparrow\   \Rightarrow^{-}  \GRANULEnt{t}  \ |\  \Delta \quad\; \GRANULEmv{x}  \not\in | \Delta |}{\Gamma  ;  \Omega  \vdash     \GRANULEnt{A}  \multimap  \GRANULEnt{B}    \Uparrow\   \Rightarrow^{-}   \lambda  \GRANULEmv{x}  .  \GRANULEnt{t}   \ |\  \Delta}
}
\newcommand*{\fSubAbsRuleNoLabel}{
    \inferrule*[right=\fSubAbsName]
    {\Gamma  ;  \Omega  \GRANULEsym{,}   \GRANULEmv{x}  :  \GRANULEnt{A}   \vdash   \GRANULEnt{B}  \Uparrow\   \Rightarrow^{-}  \GRANULEnt{t}  \ |\  \Delta \quad\; \GRANULEmv{x}  \not\in | \Delta |}{\Gamma  ;  \Omega  \vdash     \GRANULEnt{A}  \multimap  \GRANULEnt{B}    \Uparrow\   \Rightarrow^{-}   \lambda  \GRANULEmv{x}  .  \GRANULEnt{t}   \ |\  \Delta}
}
\newcommand*{\fSubRAsyncTransitionRule}{
  \inferrule*[right=\fSubRAsyncTransitionName]
  {\Gamma  ;   \Omega  \Uparrow\   \vdash  \GRANULEnt{C}  \Rightarrow^{-}  \GRANULEnt{t}  \ |\  \Delta \\ \GRANULEnt{C} \text{ not right async}}{\Gamma  ;  \Omega  \vdash   \GRANULEnt{C}  \Uparrow\   \Rightarrow^{-}  \GRANULEnt{t}  \ |\  \Delta}
}
\newcommand*{\fSubPairElimRule}{
    \inferrule*[right=\fSubPairElimName,lab=LeftAsync]
    {\Gamma  ;    \Omega  \GRANULEsym{,}   \GRANULEmv{x_{{\mathrm{1}}}}  :  \GRANULEnt{A}   \GRANULEsym{,}   \GRANULEmv{x_{{\mathrm{2}}}}  :  \GRANULEnt{B}    \Uparrow\   \vdash  \GRANULEnt{C}  \Rightarrow^{-}  \GRANULEnt{t_{{\mathrm{2}}}}  \ |\  \Delta \\ \GRANULEmv{x_{{\mathrm{1}}}}  \not\in | \Delta | \\ \GRANULEmv{x_{{\mathrm{2}}}}  \not\in | \Delta |}{\Gamma  ;    \Omega  \GRANULEsym{,}   \GRANULEmv{x_{{\mathrm{3}}}}  :   \GRANULEnt{A}  \, \otimes \,  \GRANULEnt{B}     \Uparrow\   \vdash  \GRANULEnt{C}  \Rightarrow^{-}   \textbf{let} \, ( \GRANULEmv{x_{{\mathrm{1}}}} ,  \GRANULEmv{x_{{\mathrm{2}}}} ) =  \GRANULEmv{x_{{\mathrm{3}}}}  \, \textbf{in} \,  \GRANULEnt{t_{{\mathrm{2}}}}   \ |\  \Delta}
}
\newcommand*{\fSubPairElimRuleNoLabel}{
    \inferrule*[right=\fSubPairElimName]
    {\Gamma  ;    \Omega  \GRANULEsym{,}   \GRANULEmv{x_{{\mathrm{1}}}}  :  \GRANULEnt{A}   \GRANULEsym{,}   \GRANULEmv{x_{{\mathrm{2}}}}  :  \GRANULEnt{B}    \Uparrow\   \vdash  \GRANULEnt{C}  \Rightarrow^{-}  \GRANULEnt{t_{{\mathrm{2}}}}  \ |\  \Delta \\ \GRANULEmv{x_{{\mathrm{1}}}}  \not\in | \Delta | \\ \GRANULEmv{x_{{\mathrm{2}}}}  \not\in | \Delta |}{\Gamma  ;    \Omega  \GRANULEsym{,}   \GRANULEmv{x_{{\mathrm{3}}}}  :   \GRANULEnt{A}  \, \otimes \,  \GRANULEnt{B}     \Uparrow\   \vdash  \GRANULEnt{C}  \Rightarrow^{-}   \textbf{let} \, ( \GRANULEmv{x_{{\mathrm{1}}}} ,  \GRANULEmv{x_{{\mathrm{2}}}} ) =  \GRANULEmv{x_{{\mathrm{3}}}}  \, \textbf{in} \,  \GRANULEnt{t_{{\mathrm{2}}}}   \ |\  \Delta}
}
\newcommand*{\fSubSumElimRule}{
    \inferrule*[right=\fSubSumElimName]
    {\Gamma  ;    \Omega  \GRANULEsym{,}   \GRANULEmv{x_{{\mathrm{2}}}}  :  \GRANULEnt{A}    \Uparrow\   \vdash  \GRANULEnt{C}  \Rightarrow^{-}  \GRANULEnt{t_{{\mathrm{1}}}}  \ |\  \Delta_{{\mathrm{1}}} \quad\, \Gamma  ;    \Omega  \GRANULEsym{,}   \GRANULEmv{x_{{\mathrm{3}}}}  :  \GRANULEnt{B}    \Uparrow\   \vdash  \GRANULEnt{C}  \Rightarrow^{-}  \GRANULEnt{t_{{\mathrm{2}}}}  \ |\  \Delta_{{\mathrm{2}}} \quad\, \GRANULEmv{x_{{\mathrm{2}}}}  \not\in | \Delta_{{\mathrm{1}}} | \quad \GRANULEmv{x_{{\mathrm{3}}}}  \not\in | \Delta_{{\mathrm{2}}} |}
{\Gamma ; \Omega  \GRANULEsym{,}   \GRANULEmv{x_{{\mathrm{1}}}}  :   \GRANULEnt{A}  \, \oplus \,  \GRANULEnt{B}     \Uparrow\ \vdash C \Rightarrow^- \textbf{case} \ x_{1}\ \textbf{of}\ \textbf{inl}\ x_{2} \rightarrow t_{1};\ \textbf{inr}\ x_{3} \rightarrow t_{2} \ | \ \Delta_1 \sqcap \Delta_2}
    % {\texttt{\textcolor{red}{<<no parses (char 19): G ; \{O, x1 : A + B\}*** async \mbox{$\mid$}- C =>- case x2 of inl x2 -> t1 \mbox{$\mid$} inr x3 -> t2 ; D1 ++- D2 >>}}}
}
\newcommand*{\fSubUnboxRule}{
    \inferrule*[right=\fSubUnboxName]
    {\Gamma  ;    \Omega  \GRANULEsym{,}   \GRANULEmv{x_{{\mathrm{2}}}}  :_{\textcolor{coeffectColor}{  \GRANULEnt{r}  } }   \GRANULEnt{A}    \Uparrow\   \vdash  \GRANULEnt{B}  \Rightarrow^{-}  \GRANULEnt{t}  \ |\  \Delta  \GRANULEsym{,}   \GRANULEmv{x_{{\mathrm{2}}}}  :_{\textcolor{coeffectColor}{  \GRANULEnt{s}  } }   \GRANULEnt{A} \\ 0 \sqsubseteq \GRANULEnt{s}}{\Gamma  ;    \Omega  \GRANULEsym{,}   \GRANULEmv{x_{{\mathrm{1}}}}  :   \Box_{  \GRANULEnt{r}  }  \GRANULEnt{A}     \Uparrow\   \vdash  \GRANULEnt{B}  \Rightarrow^{-}   \textbf{let} \, [  \GRANULEmv{x_{{\mathrm{2}}}}  ] =  \GRANULEmv{x_{{\mathrm{1}}}}  \, \textbf{in} \,  \GRANULEnt{t}   \ |\  \Delta}
}
\newcommand*{\fSubUnitElimRule}{
    \inferrule*[right=\fSubUnitElimName]
    {\Gamma  ;   \emptyset   \vdash  \GRANULEnt{C}  \Rightarrow^{-}  \GRANULEnt{t}  \ |\  \Delta}
    {\Gamma  ;   \GRANULEmv{x}  :   \mathsf{1}    \vdash  \GRANULEnt{C}  \Rightarrow^{-}  \GRANULEkw{let} \, \GRANULEsym{()}  \GRANULEsym{=}  \GRANULEmv{x} \, \GRANULEkw{in} \, \GRANULEnt{t}  \ |\  \Delta}
}
\newcommand*{\fSubDerRule}{
\inferrule*[right=\fSubDerName]
{\Gamma  ;     \GRANULEmv{x}  :_{\textcolor{coeffectColor}{  \GRANULEnt{s}  } }   \GRANULEnt{A}   \GRANULEsym{,}   \GRANULEmv{y}  :  \GRANULEnt{A}    \Uparrow\   \vdash  \GRANULEnt{B}  \Rightarrow^{-}  \GRANULEnt{t}  \ |\  \Delta  \GRANULEsym{,}   \GRANULEmv{x}  :_{\textcolor{coeffectColor}{  \GRANULEnt{s'}  } }   \GRANULEnt{A} \\
\GRANULEmv{y}  \not\in | \Delta | \\
\exists  \GRANULEnt{s}  .\,   \GRANULEnt{r}  \sqsupseteq  \GRANULEnt{s}  \GRANULEsym{+}   1
}
{\Gamma  ;     \GRANULEmv{x}  :_{\textcolor{coeffectColor}{  \GRANULEnt{r}  } }   \GRANULEnt{A}    \Uparrow\   \vdash  \GRANULEnt{B}  \Rightarrow^{-}   [  \GRANULEmv{x}  /  \GRANULEmv{y}  ]  \GRANULEnt{t}   \ |\  \Delta  \GRANULEsym{,}   \GRANULEmv{x}  :_{\textcolor{coeffectColor}{  \GRANULEnt{s'}  } }   \GRANULEnt{A}}
}
\newcommand*{\fSubLAsyncTransitionRule}{
  \inferrule*[right=\fSubLAsyncTransitionName]
  {\Gamma  \GRANULEsym{,}   \GRANULEmv{x}  :  \GRANULEnt{A}   ;   \Omega  \Uparrow\   \vdash  \GRANULEnt{C}  \Rightarrow^{-}  \GRANULEnt{t}  \ |\  \Delta \\ \text{ A not left async}}{\Gamma  ;    \Omega  \GRANULEsym{,}   \GRANULEmv{x}  :  \GRANULEnt{A}    \Uparrow\   \vdash  \GRANULEnt{C}  \Rightarrow^{-}  \GRANULEnt{t}  \ |\  \Delta}
}
\newcommand*{\fSubFocusRRule}{
\inferrule*[right=\fSubFocusRName,lab=Focus]
  {\Gamma  ;   \emptyset   \vdash   \GRANULEnt{C}  \Downarrow\   \Rightarrow^{-}  \GRANULEnt{t}  \ |\  \Delta \\ \text{ C not atomic}}{\Gamma  ;    \emptyset   \Uparrow\   \vdash  \GRANULEnt{C}  \Rightarrow^{-}  \GRANULEnt{t}  \ |\  \Delta}
}
\newcommand*{\fSubFocusRRuleNoLabel}{
\inferrule*[right=\fSubFocusRName]
  {\Gamma  ;   \emptyset   \vdash   \GRANULEnt{C}  \Downarrow\   \Rightarrow^{-}  \GRANULEnt{t}  \ |\  \Delta \\ \text{ C not atomic}}{\Gamma  ;    \emptyset   \Uparrow\   \vdash  \GRANULEnt{C}  \Rightarrow^{-}  \GRANULEnt{t}  \ |\  \Delta}
}
\newcommand*{\fSubFocusLRule}{
\inferrule*[right=\fSubFocusLName]
  {\Gamma  ;     \GRANULEmv{x}  :  \GRANULEnt{A}    \Downarrow\   \vdash  \GRANULEnt{C}  \Rightarrow^{-}  \GRANULEnt{t}  \ |\  \Delta}{\Gamma  \GRANULEsym{,}   \GRANULEmv{x}  :  \GRANULEnt{A}   ;    \emptyset   \Uparrow\   \vdash  \GRANULEnt{C}  \Rightarrow^{-}  \GRANULEnt{t}  \ |\  \Delta}
}
\newcommand*{\fSubPairIntroRule}{
    \inferrule*[right=\fSubPairIntroName,lab=RightSync]
    {\Gamma  ;   \emptyset   \vdash   \GRANULEnt{A}  \Downarrow\   \Rightarrow^{-}  \GRANULEnt{t_{{\mathrm{1}}}}  \ |\  \Delta_{{\mathrm{1}}} \\ \Delta_{{\mathrm{1}}}  ;   \emptyset   \vdash   \GRANULEnt{B}  \Downarrow\   \Rightarrow^{-}  \GRANULEnt{t_{{\mathrm{2}}}}  \ |\  \Delta_{{\mathrm{2}}}}{\Gamma  ;   \emptyset   \vdash     \GRANULEnt{A}  \, \otimes \,  \GRANULEnt{B}    \Downarrow\   \Rightarrow^{-}   ( \GRANULEnt{t_{{\mathrm{1}}}} ,  \GRANULEnt{t_{{\mathrm{2}}}} )   \ |\  \Delta_{{\mathrm{2}}}}
}
\newcommand*{\fSubPairIntroRuleNoLabel}{
    \inferrule*[right=\fSubPairIntroName]
    {\Gamma  ;   \emptyset   \vdash   \GRANULEnt{A}  \Downarrow\   \Rightarrow^{-}  \GRANULEnt{t_{{\mathrm{1}}}}  \ |\  \Delta_{{\mathrm{1}}} \\ \Delta_{{\mathrm{1}}}  ;   \emptyset   \vdash   \GRANULEnt{B}  \Downarrow\   \Rightarrow^{-}  \GRANULEnt{t_{{\mathrm{2}}}}  \ |\  \Delta_{{\mathrm{2}}}}{\Gamma  ;   \emptyset   \vdash     \GRANULEnt{A}  \, \otimes \,  \GRANULEnt{B}    \Downarrow\   \Rightarrow^{-}   ( \GRANULEnt{t_{{\mathrm{1}}}} ,  \GRANULEnt{t_{{\mathrm{2}}}} )   \ |\  \Delta_{{\mathrm{2}}}}
}
\newcommand*{\fSubSumIntroRuleL}{
    \inferrule*[right=\fAddSumIntroNameL]
    {\Gamma  ;   \emptyset   \vdash   \GRANULEnt{A}  \Downarrow\   \Rightarrow^{-}  \GRANULEnt{t}  \ |\  \Delta}{\Gamma  ;   \emptyset   \vdash     \GRANULEnt{A}  \, \oplus \,  \GRANULEnt{B}    \Downarrow\   \Rightarrow^{-}  \GRANULEkw{inl} \, \GRANULEnt{t}  \ |\  \Delta}
}
\newcommand*{\fSubSumIntroRuleR}{
    \inferrule*[right=\fAddSumIntroNameR]
    {\Gamma  ;   \emptyset   \vdash   \GRANULEnt{B}  \Downarrow\   \Rightarrow^{-}  \GRANULEnt{t}  \ |\  \Delta}{\Gamma  ;   \emptyset   \vdash     \GRANULEnt{A}  \, \oplus \,  \GRANULEnt{B}    \Downarrow\   \Rightarrow^{-}  \GRANULEkw{inr} \, \GRANULEnt{t}  \ |\  \Delta}
}
\newcommand*{\fSubBoxRule}{
    \inferrule*[right=\fSubBoxName]
    {\Gamma  ;   \emptyset   \vdash   \GRANULEnt{A}  \Uparrow\   \Rightarrow^{-}  \GRANULEnt{t}  \ |\  \Delta}{\Gamma  ;   \emptyset   \vdash     \Box_{  \GRANULEnt{r}  }  \GRANULEnt{A}    \Downarrow\   \Rightarrow^{-}  \GRANULEnt{t}  \ |\  \Gamma  \GRANULEsym{-}   \textcolor{coeffectColor}{ \GRANULEnt{r}   \textcolor{coeffectColor}{\,\cdot\,} }  \GRANULEsym{(}  \Gamma  \GRANULEsym{-}  \Delta  \GRANULEsym{)}}
}
\newcommand*{\fSubUnitIntroRule}{
    \inferrule*[right=\fSubUnitIntroName]
    {\quad}
    {\Gamma  \vdash   \mathsf{1}   \Rightarrow^-  \GRANULEsym{()} \ |\  \Gamma}
}
\newcommand*{\fSubRSyncTransitionRule}{
  \inferrule*[right=\fSubRSyncTransitionName]
  {\Gamma  ;   \emptyset   \vdash   \GRANULEnt{A}  \Uparrow\   \Rightarrow^{-}  \GRANULEnt{t}  \ |\  \Delta}{ \Gamma  ;   \emptyset   \vdash   \GRANULEnt{A}  \Downarrow\   \Rightarrow^{-}  \GRANULEnt{t}  \ |\  \Delta}
}
\newcommand*{\fSubAppRule}{
    \inferrule*[right=\fSubAppName,lab=LeftSync]
    {\Gamma  ;     \GRANULEmv{x_{{\mathrm{2}}}}  :  \GRANULEnt{B}    \Downarrow\   \vdash  \GRANULEnt{C}  \Rightarrow^{-}  \GRANULEnt{t_{{\mathrm{1}}}}  \ |\  \Delta_{{\mathrm{1}}} \qquad \GRANULEmv{x_{{\mathrm{2}}}}  \not\in | \Delta_{{\mathrm{1}}} | \qquad  \Delta_{{\mathrm{1}}}  ;   \emptyset   \vdash   \GRANULEnt{A}  \Downarrow\   \Rightarrow^{-}  \GRANULEnt{t_{{\mathrm{2}}}}  \ |\  \Delta_{{\mathrm{2}}}}{\Gamma  ;     \GRANULEmv{x_{{\mathrm{1}}}}  :   \GRANULEnt{A}  \multimap  \GRANULEnt{B}     \Downarrow\   \vdash  \GRANULEnt{C}  \Rightarrow^{-}   [  \GRANULEsym{(}  \GRANULEmv{x_{{\mathrm{1}}}} \, \GRANULEnt{t_{{\mathrm{2}}}}  \GRANULEsym{)}  /  \GRANULEmv{x_{{\mathrm{2}}}}  ]  \GRANULEnt{t_{{\mathrm{1}}}}   \ |\  \Delta_{{\mathrm{2}}}}
}
\newcommand*{\fSubAppRuleNoLabel}{
    \inferrule*[right=\fSubAppName]
    {\Gamma  ;     \GRANULEmv{x_{{\mathrm{2}}}}  :  \GRANULEnt{B}    \Downarrow\   \vdash  \GRANULEnt{C}  \Rightarrow^{-}  \GRANULEnt{t_{{\mathrm{1}}}}  \ |\  \Delta_{{\mathrm{1}}} \qquad \GRANULEmv{x_{{\mathrm{2}}}}  \not\in | \Delta_{{\mathrm{1}}} | \qquad  \Delta_{{\mathrm{1}}}  ;   \emptyset   \vdash   \GRANULEnt{A}  \Downarrow\   \Rightarrow^{-}  \GRANULEnt{t_{{\mathrm{2}}}}  \ |\  \Delta_{{\mathrm{2}}}}{\Gamma  ;     \GRANULEmv{x_{{\mathrm{1}}}}  :   \GRANULEnt{A}  \multimap  \GRANULEnt{B}     \Downarrow\   \vdash  \GRANULEnt{C}  \Rightarrow^{-}   [  \GRANULEsym{(}  \GRANULEmv{x_{{\mathrm{1}}}} \, \GRANULEnt{t_{{\mathrm{2}}}}  \GRANULEsym{)}  /  \GRANULEmv{x_{{\mathrm{2}}}}  ]  \GRANULEnt{t_{{\mathrm{1}}}}   \ |\  \Delta_{{\mathrm{2}}}}
}
\newcommand*{\fSubLinVarRule}{
    \inferrule*[right=\fSubLinVarName]
    {\quad}{\Gamma  ;     \GRANULEmv{x}  :  \GRANULEnt{A}    \Downarrow\   \vdash  \GRANULEnt{A}  \Rightarrow^{-}  \GRANULEmv{x}  \ |\  \Gamma}
}
\newcommand*{\fSubGrVarRule}{
    \inferrule*[right=\fSubGrVarName]
    {\exists  \GRANULEnt{s}  .\,   \GRANULEnt{r}  \sqsubseteq   \GRANULEnt{s}  \GRANULEsym{+}   1}{\Gamma  ;     \GRANULEmv{x}  :_{\textcolor{coeffectColor}{  \GRANULEnt{r}  } }   \GRANULEnt{A}    \Downarrow\   \vdash  \GRANULEnt{A}  \Rightarrow^{-}  \GRANULEmv{x}  \ |\  \Gamma  \GRANULEsym{,}   \GRANULEmv{x}  :_{\textcolor{coeffectColor}{  \GRANULEnt{s}  } }   \GRANULEnt{A}}
}
\newcommand*{\fSubLSyncTransitionRule}{
  \inferrule*[right=\fSubLSyncTransitionName]
  {\Gamma  ;     \GRANULEmv{x}  :  \GRANULEnt{A}    \Uparrow\   \vdash  \GRANULEnt{C}  \Rightarrow^{-}  \GRANULEnt{t}  \ |\  \Delta \\ \text{ A not atomic and not left sync}}{\Gamma  ;     \GRANULEmv{x}  :  \GRANULEnt{A}    \Downarrow\   \vdash  \GRANULEnt{C}  \Rightarrow^{-}  \GRANULEnt{t}  \ |\  \Delta}
}

% FOCUSED ADDITIVE RULES
\newcommand*{\fAddAbsRule}{
    \inferrule*[right=\fAddAbsName, lab=RightAsync]
    {\Gamma  ;  \Omega  \GRANULEsym{,}   \GRANULEmv{x}  :  \GRANULEnt{A}   \vdash   \GRANULEnt{B}  \Uparrow\   \Rightarrow^{+}  \GRANULEnt{t}  \ | \  \Delta  \GRANULEsym{,}   \GRANULEmv{x}  :  \GRANULEnt{A}}{\Gamma  ;  \Omega  \vdash     \GRANULEnt{A}  \multimap  \GRANULEnt{B}    \Uparrow\   \Rightarrow^{+}   \lambda  \GRANULEmv{x}  .  \GRANULEnt{t}   \ | \  \Delta}
}
\newcommand*{\fAddAbsRuleNoLabel}{
    \inferrule*[right=\fAddAbsName]
    {\Gamma  ;  \Omega  \GRANULEsym{,}   \GRANULEmv{x}  :  \GRANULEnt{A}   \vdash   \GRANULEnt{B}  \Uparrow\   \Rightarrow^{+}  \GRANULEnt{t}  \ | \  \Delta  \GRANULEsym{,}   \GRANULEmv{x}  :  \GRANULEnt{A}}{\Gamma  ;  \Omega  \vdash     \GRANULEnt{A}  \multimap  \GRANULEnt{B}    \Uparrow\   \Rightarrow^{+}   \lambda  \GRANULEmv{x}  .  \GRANULEnt{t}   \ | \  \Delta}
}
\newcommand*{\fAddRAsyncTransitionRule}{
  \inferrule*[right=\fAddRAsyncTransitionName]
  {\Gamma  ;   \Omega  \Uparrow\   \vdash  \GRANULEnt{C}  \Rightarrow^{+}  \GRANULEnt{t}  \ | \  \Delta \\ \GRANULEnt{C} \text{ not right async}}{\Gamma  ;  \Omega  \vdash   \GRANULEnt{C}  \Uparrow\   \Rightarrow^{+}  \GRANULEnt{t}  \ | \  \Delta}
}
\newcommand*{\fAddPairElimRule}{
    \inferrule*[right=\fAddPairElimName,lab=LeftAsync]
    {\Gamma  ;  \Omega  \GRANULEsym{,}   \GRANULEmv{x_{{\mathrm{1}}}}  :  \GRANULEnt{A}   \GRANULEsym{,}   \GRANULEmv{x_{{\mathrm{2}}}}  :  \GRANULEnt{B}   \vdash  \GRANULEnt{C}  \Rightarrow^{+}  \GRANULEnt{t_{{\mathrm{2}}}}  \ | \  \Delta  \GRANULEsym{,}   \GRANULEmv{x_{{\mathrm{1}}}}  :  \GRANULEnt{A}   \GRANULEsym{,}   \GRANULEmv{x_{{\mathrm{2}}}}  :  \GRANULEnt{B}}{\Gamma  ;  \Omega  \GRANULEsym{,}   \GRANULEmv{x_{{\mathrm{3}}}}  :   \GRANULEnt{A}  \, \otimes \,  \GRANULEnt{B}    \vdash  \GRANULEnt{C}  \Rightarrow^{+}   \textbf{let} \, ( \GRANULEmv{x_{{\mathrm{1}}}} ,  \GRANULEmv{x_{{\mathrm{2}}}} ) =  \GRANULEmv{x_{{\mathrm{3}}}}  \, \textbf{in} \,  \GRANULEnt{t_{{\mathrm{2}}}}   \ | \  \Delta  \GRANULEsym{,}   \GRANULEmv{x_{{\mathrm{3}}}}  :   \GRANULEnt{A}  \, \otimes \,  \GRANULEnt{B}}
}
\newcommand*{\fAddPairElimRuleNoLabel}{
    \inferrule*[right=\fAddPairElimName]
    {\Gamma  ;  \Omega  \GRANULEsym{,}   \GRANULEmv{x_{{\mathrm{1}}}}  :  \GRANULEnt{A}   \GRANULEsym{,}   \GRANULEmv{x_{{\mathrm{2}}}}  :  \GRANULEnt{B}   \vdash  \GRANULEnt{C}  \Rightarrow^{+}  \GRANULEnt{t_{{\mathrm{2}}}}  \ | \  \Delta  \GRANULEsym{,}   \GRANULEmv{x_{{\mathrm{1}}}}  :  \GRANULEnt{A}   \GRANULEsym{,}   \GRANULEmv{x_{{\mathrm{2}}}}  :  \GRANULEnt{B}}{\Gamma  ;  \Omega  \GRANULEsym{,}   \GRANULEmv{x_{{\mathrm{3}}}}  :   \GRANULEnt{A}  \, \otimes \,  \GRANULEnt{B}    \vdash  \GRANULEnt{C}  \Rightarrow^{+}   \textbf{let} \, ( \GRANULEmv{x_{{\mathrm{1}}}} ,  \GRANULEmv{x_{{\mathrm{2}}}} ) =  \GRANULEmv{x_{{\mathrm{3}}}}  \, \textbf{in} \,  \GRANULEnt{t_{{\mathrm{2}}}}   \ | \  \Delta  \GRANULEsym{,}   \GRANULEmv{x_{{\mathrm{3}}}}  :   \GRANULEnt{A}  \, \otimes \,  \GRANULEnt{B}}
}
\newcommand*{\fAddSumElimRule}{
    \inferrule*[right=\fAddSumElimName]
    {\Gamma  ;    \Omega  \GRANULEsym{,}   \GRANULEmv{x_{{\mathrm{2}}}}  :  \GRANULEnt{A}    \Uparrow\   \vdash  \GRANULEnt{C}  \Rightarrow^{+}  \GRANULEnt{t_{{\mathrm{1}}}}  \ | \  \Delta_{{\mathrm{1}}}  \GRANULEsym{,}   \GRANULEmv{x_{{\mathrm{2}}}}  :  \GRANULEnt{A} \\ \Gamma  ;    \Omega  \GRANULEsym{,}   \GRANULEmv{x_{{\mathrm{3}}}}  :  \GRANULEnt{B}    \Uparrow\   \vdash  \GRANULEnt{C}  \Rightarrow^{+}  \GRANULEnt{t_{{\mathrm{2}}}}  \ | \  \Delta_{{\mathrm{2}}}  \GRANULEsym{,}   \GRANULEmv{x_{{\mathrm{3}}}}  :  \GRANULEnt{B}}
  {\Gamma ; \Omega  \GRANULEsym{,}   \GRANULEmv{x_{{\mathrm{1}}}}  :   \GRANULEnt{A}  \, \oplus \,  \GRANULEnt{B}     \Uparrow\ \vdash C \Rightarrow^- \textbf{case} \ x_{1}\ \textbf{of}\ \textbf{inl}\ x_{2} \rightarrow t_{1};\ \textbf{inr}\ x_{3} \rightarrow t_{2} \ |\ \Delta_1 \sqcup \Delta_2, \GRANULEmv{x_{{\mathrm{1}}}}  :   \GRANULEnt{A}  \, \oplus \,  \GRANULEnt{B}}
}
\newcommand*{\fAddUnboxRule}{
    \inferrule*[right=\fAddUnboxName]
    {\Gamma  ;   \Omega  \GRANULEsym{,}     \GRANULEmv{x_{{\mathrm{2}}}}  :_{\textcolor{coeffectColor}{  \GRANULEnt{r}  } }   \GRANULEnt{A}   \Uparrow\     \vdash  \GRANULEnt{B}  \Rightarrow^{+}  \GRANULEnt{t}  \ | \  \Delta \\ \textit{if}\ \GRANULEmv{x_{{\mathrm{2}}}}  :_{\textcolor{coeffectColor}{  \GRANULEnt{s}  } }   \GRANULEnt{A}
                                                               \in \Delta\ \textit{then}\ \GRANULEnt{s}
                                                               \sqsubseteq \GRANULEnt{r}\ \textit{else}\  0
   \sqsubseteq \GRANULEnt{r}}
  {\Gamma  ;  \Omega  \GRANULEsym{,}   \GRANULEmv{x_{{\mathrm{1}}}}  :   \Box_{  \GRANULEnt{r}  }  \GRANULEnt{A}    \vdash  \GRANULEnt{B}  \Rightarrow^{+}   \textbf{let} \, [  \GRANULEmv{x_{{\mathrm{2}}}}  ] =  \GRANULEmv{x_{{\mathrm{1}}}}  \, \textbf{in} \,  \GRANULEnt{t}   \ | \  \GRANULEsym{(}   \Delta \!\setminus\!  \GRANULEmv{x_{{\mathrm{2}}}}   \GRANULEsym{)}  \GRANULEsym{,}   \GRANULEmv{x_{{\mathrm{1}}}}  :   \Box_{  \GRANULEnt{r}  }  \GRANULEnt{A}}
}
\newcommand*{\fAddUnitElimRule}{
    \inferrule*[right=\fAddUnitElimName]
    {\Gamma  ;   \emptyset   \vdash  \GRANULEnt{C}  \Rightarrow^{+}  \GRANULEnt{t}  \ | \  \Delta}
    {\Gamma  ;   \GRANULEmv{x}  :   \mathsf{1}    \vdash  \GRANULEnt{C}  \Rightarrow^{+}  \GRANULEkw{let} \, \GRANULEsym{()}  \GRANULEsym{=}  \GRANULEmv{x} \, \GRANULEkw{in} \, \GRANULEnt{t}  \ | \  \Delta  \GRANULEsym{,}   \GRANULEmv{x}  :   \mathsf{1}}
}
\newcommand*{\fAddDerRule}{
  \inferrule*[right=\fAddDerName]
  { \Gamma  ;     \GRANULEmv{x}  :_{\textcolor{coeffectColor}{  \GRANULEnt{s}  } }   \GRANULEnt{A}   \GRANULEsym{,}   \GRANULEmv{y}  :  \GRANULEnt{A}    \Uparrow\   \vdash  \GRANULEnt{B}  \Rightarrow^{+}  \GRANULEnt{t}  \ | \  \Delta  \GRANULEsym{,}   \GRANULEmv{y}  :  \GRANULEnt{A} }
  { \Gamma  ;    \GRANULEmv{x}  :_{\textcolor{coeffectColor}{  \GRANULEnt{s}  } }   \GRANULEnt{A}   \Uparrow\   \vdash  \GRANULEnt{B}  \Rightarrow^{+}   [  \GRANULEmv{x}  /  \GRANULEmv{y}  ]  \GRANULEnt{t}   \ | \  \Delta  \GRANULEsym{+}   \GRANULEmv{x}  :_{\textcolor{coeffectColor}{   1   } }   \GRANULEnt{A} }
}
\newcommand*{\fAddLAsyncTransitionRule}{
  \inferrule*[right=\fAddLAsyncTransitionName]
  {\Gamma  \GRANULEsym{,}   \GRANULEmv{x}  :  \GRANULEnt{A}   ;   \Omega  \Uparrow\   \vdash  \GRANULEnt{C}  \Rightarrow^{+}  \GRANULEnt{t}  \ | \  \Delta \\ \text{ A not left async}}{\Gamma  ;    \Omega  \GRANULEsym{,}   \GRANULEmv{x}  :  \GRANULEnt{A}    \Uparrow\   \vdash  \GRANULEnt{C}  \Rightarrow^{+}  \GRANULEnt{t}  \ | \  \Delta}
}
\newcommand*{\fAddFocusRRule}{
\inferrule*[right=\fAddFocusRName,lab=Focus]
  {\Gamma  ;   \emptyset   \vdash   \GRANULEnt{C}  \Downarrow\   \Rightarrow^{+}  \GRANULEnt{t}  \ | \  \Delta \\ \text{ C not atomic}}{\Gamma  ;    \emptyset   \Uparrow\   \vdash  \GRANULEnt{C}  \Rightarrow^{+}  \GRANULEnt{t}  \ | \  \Delta}
}
\newcommand*{\fAddFocusRRuleNoLabel}{
\inferrule*[right=\fAddFocusRName]
  {\Gamma  ;   \emptyset   \vdash   \GRANULEnt{C}  \Downarrow\   \Rightarrow^{+}  \GRANULEnt{t}  \ | \  \Delta \\ \text{ C not atomic}}{\Gamma  ;    \emptyset   \Uparrow\   \vdash  \GRANULEnt{C}  \Rightarrow^{+}  \GRANULEnt{t}  \ | \  \Delta}
}
\newcommand*{\fAddFocusLRule}{
\inferrule*[right=\fAddFocusLName]
  {\Gamma  ;     \GRANULEmv{x}  :  \GRANULEnt{A}    \Downarrow\   \vdash  \GRANULEnt{C}  \Rightarrow^{+}  \GRANULEnt{t}  \ | \  \Delta}{\Gamma  \GRANULEsym{,}   \GRANULEmv{x}  :  \GRANULEnt{A}   ;    \emptyset   \Uparrow\   \vdash  \GRANULEnt{C}  \Rightarrow^{+}  \GRANULEnt{t}  \ | \  \Delta}
}
\newcommand*{\fAddPairIntroRule}{
    \inferrule*[right=\fAddPairIntroName,lab=RightSync]
    {\Gamma  ;   \emptyset   \vdash   \GRANULEnt{A}  \Downarrow\   \Rightarrow^{+}  \GRANULEnt{t_{{\mathrm{1}}}}  \ | \  \Delta_{{\mathrm{1}}} \\ \Gamma  ;   \emptyset   \vdash   \GRANULEnt{B}  \Downarrow\   \Rightarrow^{+}  \GRANULEnt{t_{{\mathrm{2}}}}  \ | \  \Delta_{{\mathrm{2}}}}{\Gamma  ;   \emptyset   \vdash     \GRANULEnt{A}  \, \otimes \,  \GRANULEnt{B}    \Downarrow\   \Rightarrow^{+}   ( \GRANULEnt{t_{{\mathrm{1}}}} ,  \GRANULEnt{t_{{\mathrm{2}}}} )   \ | \  \Delta_{{\mathrm{1}}}  \GRANULEsym{+}  \Delta_{{\mathrm{2}}}}
}
\newcommand*{\fAddPairIntroRuleNoLabel}{
    \inferrule*[right=\fAddPairIntroName]
    {\Gamma  ;   \emptyset   \vdash   \GRANULEnt{A}  \Downarrow\   \Rightarrow^{+}  \GRANULEnt{t_{{\mathrm{1}}}}  \ | \  \Delta_{{\mathrm{1}}} \\ \Gamma  ;   \emptyset   \vdash   \GRANULEnt{B}  \Downarrow\   \Rightarrow^{+}  \GRANULEnt{t_{{\mathrm{2}}}}  \ | \  \Delta_{{\mathrm{2}}}}{\Gamma  ;   \emptyset   \vdash     \GRANULEnt{A}  \, \otimes \,  \GRANULEnt{B}    \Downarrow\   \Rightarrow^{+}   ( \GRANULEnt{t_{{\mathrm{1}}}} ,  \GRANULEnt{t_{{\mathrm{2}}}} )   \ | \  \Delta_{{\mathrm{1}}}  \GRANULEsym{+}  \Delta_{{\mathrm{2}}}}
}
\newcommand*{\fAddSumIntroRuleL}{
    \inferrule*[right=\fAddSumIntroNameL]
    {\Gamma  ;   \emptyset   \vdash   \GRANULEnt{A}  \Downarrow\   \Rightarrow^{+}  \GRANULEnt{t}  \ | \  \Delta}{\Gamma  ;   \emptyset   \vdash     \GRANULEnt{A}  \, \oplus \,  \GRANULEnt{B}    \Downarrow\   \Rightarrow^{+}  \GRANULEkw{inl} \, \GRANULEnt{t}  \ | \  \Delta}
}
\newcommand*{\fAddSumIntroRuleR}{
    \inferrule*[right=\fAddSumIntroNameR]
    {\Gamma  ;   \emptyset   \vdash   \GRANULEnt{B}  \Downarrow\   \Rightarrow^{+}  \GRANULEnt{t}  \ | \  \Delta}{\Gamma  ;   \emptyset   \vdash     \GRANULEnt{A}  \, \oplus \,  \GRANULEnt{B}    \Downarrow\   \Rightarrow^{+}  \GRANULEkw{inr} \, \GRANULEnt{t}  \ | \  \Delta}
}
\newcommand*{\fAddBoxRule}{
    \inferrule*[right=\fAddBoxName]
    {\Gamma  ;   \emptyset   \vdash   \GRANULEnt{A}  \Uparrow\   \Rightarrow^{+}  \GRANULEnt{t}  \ | \  \Delta}{\Gamma  ;   \emptyset   \vdash     \Box_{  \GRANULEnt{r}  }  \GRANULEnt{A}    \Downarrow\   \Rightarrow^{+}  \GRANULEsym{[}  \GRANULEnt{t}  \GRANULEsym{]}  \ | \   \textcolor{coeffectColor}{ \GRANULEnt{r}   \textcolor{coeffectColor}{\,\cdot\,} }  \Delta}
}
\newcommand*{\fAddUnitIntroRule}{
    \inferrule*[right=\fAddUnitIntroName]
    {\quad}
    {\Gamma  ;   \emptyset   \vdash   \mathsf{1}   \Rightarrow^{+}  \GRANULEsym{()}  \ | \   \emptyset}
}
\newcommand*{\fAddRSyncTransitionRule}{
  \inferrule*[right=\fAddRSyncTransitionName]
  {\Gamma  ;   \emptyset   \vdash   \GRANULEnt{A}  \Uparrow\   \Rightarrow^{+}  \GRANULEnt{t}  \ | \  \Delta}{ \Gamma  ;   \emptyset   \vdash   \GRANULEnt{A}  \Downarrow\   \Rightarrow^{+}  \GRANULEnt{t}  \ | \  \Delta}
}
\newcommand*{\fAddAppRule}{
    \inferrule*[right=\fAddAppName,lab=LeftSync]
    {\Gamma  ;     \GRANULEmv{x_{{\mathrm{2}}}}  :  \GRANULEnt{B}    \Downarrow\   \vdash  \GRANULEnt{C}  \Rightarrow^{+}  \GRANULEnt{t_{{\mathrm{1}}}}  \ | \  \Delta_{{\mathrm{1}}}  \GRANULEsym{,}   \GRANULEmv{x_{{\mathrm{2}}}}  :  \GRANULEnt{B} \\ \Gamma  ;   \emptyset   \vdash   \GRANULEnt{A}  \Downarrow\   \Rightarrow^{+}  \GRANULEnt{t_{{\mathrm{2}}}}  \ | \  \Delta_{{\mathrm{2}}}}{\Gamma  ;     \GRANULEmv{x_{{\mathrm{1}}}}  :   \GRANULEnt{A}  \multimap  \GRANULEnt{B}     \Downarrow\   \vdash  \GRANULEnt{C}  \Rightarrow^{+}   [  \GRANULEsym{(}  \GRANULEmv{x_{{\mathrm{1}}}} \, \GRANULEnt{t_{{\mathrm{2}}}}  \GRANULEsym{)}  /  \GRANULEmv{x_{{\mathrm{2}}}}  ]  \GRANULEnt{t_{{\mathrm{1}}}}   \ | \  \GRANULEsym{(}  \Delta_{{\mathrm{1}}}  \GRANULEsym{+}  \Delta_{{\mathrm{2}}}  \GRANULEsym{)}  \GRANULEsym{,}   \GRANULEmv{x_{{\mathrm{1}}}}  :   \GRANULEnt{A}  \multimap  \GRANULEnt{B}}
}
\newcommand*{\fAddAppRuleNoLabel}{
    \inferrule*[right=\fAddAppName]
    {\Gamma  ;     \GRANULEmv{x_{{\mathrm{2}}}}  :  \GRANULEnt{B}    \Downarrow\   \vdash  \GRANULEnt{C}  \Rightarrow^{+}  \GRANULEnt{t_{{\mathrm{1}}}}  \ | \  \Delta_{{\mathrm{1}}}  \GRANULEsym{,}   \GRANULEmv{x_{{\mathrm{2}}}}  :  \GRANULEnt{B} \\ \Gamma  ;   \emptyset   \vdash   \GRANULEnt{A}  \Downarrow\   \Rightarrow^{+}  \GRANULEnt{t_{{\mathrm{2}}}}  \ | \  \Delta_{{\mathrm{2}}}}{\Gamma  ;     \GRANULEmv{x_{{\mathrm{1}}}}  :   \GRANULEnt{A}  \multimap  \GRANULEnt{B}     \Downarrow\   \vdash  \GRANULEnt{C}  \Rightarrow^{+}   [  \GRANULEsym{(}  \GRANULEmv{x_{{\mathrm{1}}}} \, \GRANULEnt{t_{{\mathrm{2}}}}  \GRANULEsym{)}  /  \GRANULEmv{x_{{\mathrm{2}}}}  ]  \GRANULEnt{t_{{\mathrm{1}}}}   \ | \  \GRANULEsym{(}  \Delta_{{\mathrm{1}}}  \GRANULEsym{+}  \Delta_{{\mathrm{2}}}  \GRANULEsym{)}  \GRANULEsym{,}   \GRANULEmv{x_{{\mathrm{1}}}}  :   \GRANULEnt{A}  \multimap  \GRANULEnt{B}}
}
\newcommand*{\fAddLinVarRule}{
    \inferrule*[right=\fAddLinVarName]
  {\quad}{\Gamma  ;   \GRANULEmv{x}  :  \GRANULEnt{A}   \vdash  \GRANULEnt{A}  \Rightarrow^{+}  \GRANULEmv{x}  \ | \   \GRANULEmv{x}  :  \GRANULEnt{A}}
}
\newcommand*{\fAddGrVarRule}{
    \inferrule*[right=\fAddGrVarName]
    {\quad}{\Gamma  ;   \GRANULEmv{x}  :_{\textcolor{coeffectColor}{  \GRANULEnt{r}  } }   \GRANULEnt{A}   \vdash  \GRANULEnt{A}  \Rightarrow^{+}  \GRANULEmv{x}  \ | \   \GRANULEmv{x}  :_{\textcolor{coeffectColor}{   1   } }   \GRANULEnt{A}}
}
\newcommand*{\fAddLSyncTransitionRule}{
  \inferrule*[right=\fAddLSyncTransitionName]
  {\Gamma  ;     \GRANULEmv{x}  :  \GRANULEnt{A}    \Uparrow\   \vdash  \GRANULEnt{C}  \Rightarrow^{+}  \GRANULEnt{t}  \ | \  \Delta \\ \text{ A not atomic and not left sync
  }}{\Gamma  ;     \GRANULEmv{x}  :  \GRANULEnt{A}    \Downarrow\   \vdash  \GRANULEnt{C}  \Rightarrow^{+}  \GRANULEnt{t}  \ | \  \Delta}
}

% ADDITIVE PRUNING FOCUS RULES

\newcommand*{\fAddAltAppName}{$\multimap'^{+}_{L}$}
\newcommand*{\fAddAltPairIntroName}{$\otimes'^{+}_{R}$}

\newcommand*{\fAddAltAppRule}{
    \inferrule*[right=\fAddAltAppName]
    {\Gamma  ;   \GRANULEmv{x_{{\mathrm{2}}}}  :  \GRANULEnt{B}   \vdash  \GRANULEnt{C}  \Rightarrow^{+}  \GRANULEnt{t_{{\mathrm{1}}}}  \ | \  \Delta_{{\mathrm{1}}}  \GRANULEsym{,}   \GRANULEmv{x_{{\mathrm{2}}}}  :  \GRANULEnt{B} \\ \Gamma  \GRANULEsym{-}  \Delta_{{\mathrm{1}}}  ;   \emptyset   \vdash  \GRANULEnt{A}  \Rightarrow^{+}  \GRANULEnt{t_{{\mathrm{2}}}}  \ | \  \Delta_{{\mathrm{2}}}}{\Gamma  ;   \GRANULEmv{x_{{\mathrm{1}}}}  :   \GRANULEnt{A}  \multimap  \GRANULEnt{B}    \vdash  \GRANULEnt{C}  \Rightarrow^{+}   [  \GRANULEsym{(}  \GRANULEmv{x_{{\mathrm{1}}}} \, \GRANULEnt{t_{{\mathrm{2}}}}  \GRANULEsym{)}  /  \GRANULEmv{x_{{\mathrm{2}}}}  ]  \GRANULEnt{t_{{\mathrm{1}}}}   \ | \  \GRANULEsym{(}  \Delta_{{\mathrm{1}}}  \GRANULEsym{+}  \Delta_{{\mathrm{2}}}  \GRANULEsym{)}  \GRANULEsym{,}   \GRANULEmv{x_{{\mathrm{1}}}}  :   \GRANULEnt{A}  \multimap  \GRANULEnt{B} }
}
\newcommand*{\fAddAltPairIntroRule}{
    \inferrule*[right=\fAddAltPairIntroName]
    {\Gamma  ;   \emptyset   \vdash  \GRANULEnt{A}  \Rightarrow^{+}  \GRANULEnt{t_{{\mathrm{1}}}}  \ | \  \Delta_{{\mathrm{1}}} \\ \Gamma  \GRANULEsym{-}  \Delta_{{\mathrm{1}}}  ;   \emptyset   \vdash  \GRANULEnt{B}  \Rightarrow^{+}  \GRANULEnt{t_{{\mathrm{2}}}}  \ | \  \Delta_{{\mathrm{2}}}}
    {\Gamma  ;   \emptyset   \vdash   \GRANULEnt{A}  \, \otimes \,  \GRANULEnt{B}   \Rightarrow^{+}   ( \GRANULEnt{t_{{\mathrm{1}}}} ,  \GRANULEnt{t_{{\mathrm{2}}}} )   \ | \  \Delta_{{\mathrm{1}}}  \GRANULEsym{+}  \Delta_{{\mathrm{2}}}}
}






%%% GRADED BASE NAMES AND RULES %%%

% Typing renames
\renewcommand{\GRANULEdruleTyVarName}{$\textsc{Var}$}
\renewcommand{\GRANULEdruleTyAbsName}{$\textsc{Abs}$}
\renewcommand{\GRANULEdruleTyAppName}{$\textsc{App}$}
\renewcommand{\GRANULEdruleTyConName}{$\textsc{Con}$}
\renewcommand{\GRANULEdruleTyCaseName}{$\textsc{Case}$}
\renewcommand{\GRANULEdruleTyPrName}{$\textsc{Pr}$}
\renewcommand{\GRANULEdruleTyApproxName}{$\textsc{Approx}$}

% Pattern typing renames
\renewcommand{\GRANULEdrulePatWildName}{$\textsc{PWild}$}
\renewcommand{\GRANULEdrulePatVarName}{$\textsc{PVar}$}
\renewcommand{\GRANULEdrulePatBoxName}{$\textsc{PBox}$}
\renewcommand{\GRANULEdrulePatConName}{$\textsc{PCon}$}

% Synthesis rule names

% Focusing synhtesis rule names
\newcommand*{\fsynRAsyncTransName}{$\Uparrow_{\textsc{R}}$}
\newcommand*{\fsynLAsyncTransName}{$\Uparrow_{\textsc{L}}$}
\newcommand*{\fsynFocusRName}{$\textsc{Foc}_{\textsc{R}}$}
\newcommand*{\fsynFocusLName}{$\textsc{Foc}_{\textsc{L}}$}
\newcommand*{\fsynRSyncTransName}{$\Downarrow_{\textsc{R}}$}
\newcommand*{\fsynLSyncTransName}{$\Downarrow_{\textsc{L}}$}

% GRADED BASE TYPING RULES
\newcommand*{\tyVar}{\GRANULEdruleTyVar{}}
\newcommand*{\tyAbs}{\GRANULEdruleTyAbs{}}
\newcommand*{\tyApp}{\GRANULEdruleTyApp{}}
\newcommand*{\tyApprox}{\GRANULEdruleTyApprox{}}
\newcommand*{\tyPr}{\GRANULEdruleTyPr{}}
\newcommand*{\tyCon}{\GRANULEdruleTyCon{}}
\newcommand*{\tyCase}{\GRANULEdruleTyCase{}}


% GRADED BASE SYNTHESIS RULES
\newcommand*{\synVar}{\GRANULEdruleVar{}}
\newcommand*{\synAbs}{\GRANULEdruleAbs{}}
\newcommand*{\synApp}{\GRANULEdruleApp{}}
\newcommand*{\synCon}{\GRANULEdruleCon{}}
\newcommand*{\synConAlt}{\GRANULEdruleConAlt{}}
\newcommand*{\synCase}{\GRANULEdruleCase{}}
\newcommand*{\synCaseAlt}{\GRANULEdruleCaseAlt{}}
\newcommand*{\synCaseAltAlt}{\GRANULEdruleCaseAltAlt{}}
\newcommand*{\synBox}{\GRANULEdruleBox{}}
\newcommand*{\synUnbox}{\GRANULEdruleUnbox{}}
\newcommand*{\synCaseSub}{\GRANULEdruleCaseSub{}}


\newcommand*{\fsynVar}{\GRANULEdruleVarF{}}
\newcommand*{\fsynAbs}{\GRANULEdruleAbsF{}}
\newcommand*{\fsynUnbox}{\GRANULEdruleUnboxF{}}
\newcommand*{\fsynApp}{\GRANULEdruleAppF{}}
\newcommand*{\fsynCon}{\GRANULEdruleConF{}}
\newcommand*{\fsynCase}{\GRANULEdruleCaseF{}}
\newcommand*{\fsynBox}{\GRANULEdruleBoxF{}}

\newcommand*{\fsynRAsyncTrans}{
  \inferrule*[right=\fsynRAsyncTransName]
  {\Gamma  ;  \Omega  \Uparrow \vdash  \GRANULEnt{B}  \Rightarrow  \GRANULEnt{t}  \mid  \Delta \\ \GRANULEnt{B} \text{ not right async}}{\Gamma  ;  \Omega  \vdash  \GRANULEnt{B}  \Uparrow \Rightarrow  \GRANULEnt{t}  \mid  \Delta}
}
%
\newcommand*{\fsynLAsyncTrans}{
  \inferrule*[right=\fsynLAsyncTransName]
  {\Gamma  \GRANULEsym{,}   \GRANULEmv{x}  :  \GRANULEnt{A}   ;  \Omega  \Uparrow \vdash  \GRANULEnt{B}  \Rightarrow  \GRANULEnt{t}  \mid  \Delta \\ \text{A not left async}}{\Gamma  ;   \Omega  \GRANULEsym{,}   \GRANULEmv{x}  :  \GRANULEnt{A}    \Uparrow \vdash  \GRANULEnt{B}  \Rightarrow  \GRANULEnt{t}  \mid  \Delta}
}
%
\newcommand*{\fsynFocusR}{
\inferrule*[right=\fsynFocusRName,lab=Focus]
{\Gamma ; \emptyset \vdash \GRANULEnt{B} \Downarrow\ \Rightarrow \GRANULEnt{t} \mid\ \Delta
  \\ \text{B not atomic}}{\Gamma  ;    \emptyset   \Uparrow\   \vdash  \GRANULEnt{B}  \Rightarrow^{+}  \GRANULEnt{t}  \ | \  \Delta}
}
%
\newcommand*{\fsynFocusRNoLabel}{
\inferrule*[right=\fsynFocusRName]
{\Gamma ; \emptyset \vdash \GRANULEnt{B} \Downarrow\ \Rightarrow \GRANULEnt{t} \mid\ \Delta
  \\ \text{B not atomic}}{\Gamma  ;    \emptyset   \Uparrow\   \vdash  \GRANULEnt{B}  \Rightarrow^{+}  \GRANULEnt{t}  \ | \  \Delta}
}
%
\newcommand*{\fsynFocusL}{
\inferrule*[right=\fsynFocusLName]
  {\Gamma  ;    \GRANULEmv{x}  :  \GRANULEnt{A}    \Downarrow \vdash  \GRANULEnt{B}  \Rightarrow  \GRANULEnt{t}  \mid  \Delta}{\Gamma  \GRANULEsym{,}   \GRANULEmv{x}  :  \GRANULEnt{A}   ;   \emptyset   \Uparrow \vdash  \GRANULEnt{B}  \Rightarrow  \GRANULEnt{t}  \mid  \Delta}
}
%
\newcommand*{\fsynRSyncTrans}{
  \inferrule*[right=\fsynRSyncTransName]
  {\Gamma  ;   \emptyset   \vdash  \GRANULEnt{A}  \Uparrow \Rightarrow  \GRANULEnt{t}  \mid  \Delta}{ \texttt{\textcolor{red}{<<no parses (char 20):  G ; . \mbox{$\mid$}- A sync => ***t ; D  >>}}}
}
%
\newcommand*{\fsynLSyncTrans}{
  \inferrule*[right=\fsynLSyncTransName]
  {\Gamma  ;    \GRANULEmv{x}  :  \GRANULEnt{A}    \Uparrow \vdash  \GRANULEnt{B}  \Rightarrow  \GRANULEnt{t}  \mid  \Delta \\ \text{ A not atomic and not left sync
  }}{\Gamma  ;    \GRANULEmv{x}  :  \GRANULEnt{A}    \Downarrow \vdash  \GRANULEnt{B}  \Rightarrow  \GRANULEnt{t}  \mid  \Delta}
}

% \newcommand*{\fsynCaseSub}{\GRANULEdruleCaseSub{}}
% \newcommand*{\fsynCase}{\GRANULEdruleCase{}}
% \newcommand*{\synCaseAltAlt}{\GRANULEdruleCaseAltAlt{}}



\begin{document}
\normalsize
%Cover page information
\title{University of Kent\\Program Synthesis from Linear and Graded Types}
\author{Jack Hughes}
\subject{computer science}
\degree{PhD}

%thesis preface
\begin{preface}
\section{Abstract}
A type-directed program synthesis tool can leverage the information provided by
resourceful types (linear and graded types) to prune ill-resourced programs from
the search space of candidate programs. Therefore, barring any other
specification information, a synthesise tool will synthesise a target program
(if one exists) more quickly when given a type specification which includes
resourceful types than when given an equivalent non-resourceful type.

\section{Acknowledgements}
\end{preface}

\chapter{Introduction}
\label{chapter:intro}
\input{1_introduction/intro-ottput}

\chapter{Background}
\label{chapter:background}
This chapter provides the relevant background information. It largely



We explore two lineages of resourceful type systems. In the first, modern
resourceful type systems trace their roots to Girard's Linear Logic~\cite{}
which was one of the first treatments of data as a resource inside a program. Bounded Linear Logic (BLL)
developed this idea further, refining the coarse grained view of data as either
linear or non linear. Several subsequent works generalised BLL, coalescing into
the notion of a \textit{graded} type system in languages such as
Granule~\cite{}. This lineage treats these systems essentially as refinements of an
underlying linear structure.

At the same time that these systems were being studied, resourceful types were
also being approached from an entirely different perspective --- that of
computational effects. \jnote{more to go here}.

\begin{table}[htbp]
\centering
\caption{Timeline.}
\label{tbl:1}
\begin{tabular}{r l l}
  Year & \ \ Linear & \ \ Graded \\
  \toprule
  1986 & & \textasteriskcentered\ D.K. Gifford, J.M. Lucassen \\
       & & \ \ \textit{Integrating Functional and} \\
       & & \ \ \textit{Imperative Programming}  \\

  1987 & \textasteriskcentered\ J.Y. Girard \\
       & \ \ \textit{Linear Logic} & \\

  1990 & & \textasteriskcentered\ E. Moggi \\
       & & \ \ \textit{Notions of Computation and Monads} \\

  1991 & \textasteriskcentered\ J.Y. Girard et al. & \\
       & \ \ \textit{Bounded Linear Logic} & \\

  1994 & \textasteriskcentered\ J.S. Hodas, D. Miller &  \\
       & \ \ \textit{Logic Programming in a Fragment} & \\
       & \ \ \textit{of Intuitionistic Linear Logic} & \\

  2000 & & \textasteriskcentered\ P. Wadler, P. Thiemann \\
       & & \ \ \textit{Marriage of effects and monads} \\
  2011 & \textasteriskcentered\ U.D. Lago, M. Gaboardi &  \\
       & \ \ \textit{Linear Dependent Types} & \\
       & \ \ \textit{and Relative Completeness} & \\
  2013 & & \textasteriskcentered\ T. Petricek et al. \\
       & & \ \ \textit{Coeffects: Unified Static} \\
       & & \ \ \textit{Analysis of Context-Dependence} \\
  2014 & \textasteriskcentered\ D.R. Ghica, A. I. Smith & \textasteriskcentered\
  T. Petricek et al. \\
       & \ \ \textit{Bounded Linear Types in} & \ \ \textit{Coeffects: a
                                                Calculus of} \\
       & \ \ \textit{a Resource Semiring}  & \ \ \textit{Context-Dependent Computation} \\
       & \textasteriskcentered\ A. Brunel et al. & \\
       & \ \ \textit{A Core Quantative Coeffect Calculus} & \\
  2016 & \textasteriskcentered\ M. Gaboardi et al. & \textasteriskcentered\ C. McBride \\
       & \ \ \textit{Combining Effects and} & \ \ \textit{I Got Plenty o' Nuttin'} \\
       & \ \ \textit{Coeffects via Grading} & \\
  2017 & & \textasteriskcentered\ J.P. Bernardy et al. \\
       & & \ \ \textit{Linear Haskell} \\
  2018 & & \textasteriskcentered\ R. Atkey \\
       & & \ \ \textit{Syntax and Semantics of} \\
       & & \ \ \textit{Quantative Type Theory} \\
  2019 & \textasteriskcentered\ Orchard et al. & \\
       & \ \ \textit{Quantitative Program Reasoning} & \\
       & \ \ \textit{with Graded Modal Types} & \\
  % 2020 & LOPSTR &  \\
  % 2021 & & \textasteriskcentered\ A. Abel, J.P. Bernardy \\
  %      & & \ \ \textit{A Unified View of Modalities} \\
  %      & & \ \ \textit{in Type Systems} \\
  %      & & \textasteriskcentered\ P. Choudhury et al. \\
  %      & & \ \ \textit{A Graded Dependent Type System} \\
  %      & & \ \ \textit{with a Usage-Aware Semantics} \\
  %      & & \textasteriskcentered\ B. Moon et al. \\
  %      & & \ \ \textit{Graded Modal Dependent} \\
  %      & & \ \ \textit{Type Theory} \\
  %      & & \textasteriskcentered\ E. Brady \\
  %      & & \ \ \textit{Idris 2: Quantative Type} \\
  %      & & \ \ \textit{Theory in Practice} \\
  %
  \bottomrule
\end{tabular}
\end{table}


\paragraph{Terminology}
Before delving into linear and graded systems, we briefly frame the approach
we will take to discussing the relevant background material. Throughout this
chapter (and subsequent chapters) we will tend towards using a
\textit{types-and-programs} terminology rather than
\textit{propositions-and-proofs}. Through the lens of the Curry-Howard
correspondence, one can switch smoothly to viewing our approach to program
synthesis as proof search in logic.

The functional programming languages we discuss are presented as typed
calculi given by sets of \textit{types}, \textit{terms} (programs), and \textit{typing
 rules} that relate a term to its type. The most well-known typed
calculus is the simply-typed $\lambda$-calculus, which corresponds to
intuitionistic logic.

A \textit{judgment} defines the typing relation between a type and a term based on a
\textit{context}. In the simple typed $\lambda$-calculus, judgments have the
form: $\Gamma \vdash t : A$, stating that under some context of
\textit{assumptions} $\Gamma$ the program term $t$ can be assigned the type $A$.
An assumption is a name with an associated type, written $x : A$ and
corresponds to an in-scope variable in a program.

A term can be related to a type if we can derive a valid judgment through the
application of typing rules. The application of these rules forms a tree
structure known as a \textit{typing derivation}.


\section{Linear and substructural logics}
Linear logic~\cite{} was introduces by Girard as a way of being more descriptive
about the properties of a derivation in intuitionistic logic. In type systems
such as the simply typed $\lambda$-calculus, the properties of \textit{weakening},
\textit{contraction}, and \textit{exchange} are assumed implicitly. These are
typing rules which are \textit{structural} as they determine how the context
may be used rather than being directed by the syntax. Weakening is a rule which allows terms
that are not needed in a typing derivation to be discarded. Contraction works as a dual to weakening, allowing an
assumption in the context to be used more than once. Finally, exchange allows assumptions in a context to arbitrarily re-ordered.


\begin{figure}[H]
  \begin{align*}
  \begin{array}{c}
    \inferrule*[right=Weakening]{\Gamma \vdash t : B}{\Gamma, x : A \vdash t : B}
    \;\;\;\;
    \;\;\;\;
    \;\;\;\;
    \inferrule*[right=Contraction]{\Gamma, x : A, y : A \vdash t : B}{\Gamma, x : A \vdash t : B}
    \\\\
    \inferrule*[right=Exchange]{\Gamma_{1}, y : B, x : A, \Gamma_{2} \vdash t : C} {\Gamma_{1}, x : A, y : B, \Gamma_{2} \vdash t : C}
    \end{array}
  \end{align*}
\end{figure}

Linear logic is known as a \textit{substructural} logic because it lacks the weakening
and contraction rules, while permitting exchange.

The disallowance of these rules means that in order to construct of a typing
derivation, each assumption must be used exactly once --- arbitrarily copying or
discarding values is disallowed, excluding a vast number of programs from being
typeable in linear logic. Non-restricted usage of a value is recovered through
the modal operator $!$ (also called ``bang'', ``of-course'', or the
\textit{exponential} modality). Affixing $!$ to a type captures the notion that values of
that type may be used freely in a program.

providing a binary view of data as a
resource inside a program: values are either linear or completely unrestricted.

Bounded Linear Logic, took this idea further --- instead of a single modal
operator, $!$ is replaced with a family of modal operators indexed by terms
which provide an upper bound on usage~\cite{}. These terms provide an upper bound on
the usage of a values inside term, e.g. $!_{3}A$ is the type of $A$ values which
may be used up to 3 times.


\jnote{more about Hodas and Miller, ICFP and Granule etc}
The idea of data as a resource inside a program has been extended further by the
proliferation of graded type systems.


\subsection{A graded linear typing calculus}
\label{sec:linear-base}

Having seen the resourceful types from linear logic to graded type systems, we
are now in a position to define a full typing calculus, based on the linear
$\lambda$-calculus extended with a graded modal type. This calculus is
equivalent to the core calculus Granule,
\textsc{GrMini}\cite{}. Granule's full type system is an extension of this
graded linear core, with the addition of polymorphism, indexed-types, pattern
matching, and the ability to use multiple different graded modalities. We refer
to this system as the \textit{linear-base} calculus, reflecting the underlying
linear structure of the system.

The types of the linear-base calculus are defined as:
\begin{align*}
\hspace{-0.9em}\GRANULEnt{A} , \GRANULEnt{B} & ::=
       \GRANULEnt{A} \multimap \GRANULEnt{B}
  \mid \Box_{  \GRANULEnt{r}  }  \GRANULEnt{A}
{\small{\tag{types}}}
\end{align*}
The type $\Box_{r} A$ is an indexed family of type operators where $r$ is a
\textit{grade} ranging over the elements of a pre-ordered semiring
$({\mathcal{R}}, {*}, {1}, {+}, {0},
{\sqsubseteq})$ parameterising the calculus (where $\ast$ and $+$
are monotonic with respect to the pre-order $\sqsubseteq$).

The syntax of terms provides the elimination and introduction forms:
\begin{align*}
\hspace{-0.8em} \GRANULEnt{t} ::= \;
       & \GRANULEmv{x}
  \mid \lambda  \GRANULEmv{x}  .  \GRANULEnt{t}
  \mid \GRANULEnt{t_{{\mathrm{1}}}} \, \GRANULEnt{t_{{\mathrm{2}}}}
  \mid \GRANULEsym{[}  \GRANULEnt{t}  \GRANULEsym{]}
  \mid \textbf{let} \, [  \GRANULEmv{x}  ] =  \GRANULEnt{t_{{\mathrm{1}}}}  \, \textbf{in} \,  \GRANULEnt{t_{{\mathrm{2}}}}
{\small{\tag{terms}}}
\end{align*}
In addition the the terms of the linear $\lambda$-calculus, we also have the
construct $\GRANULEsym{[}  \GRANULEnt{t}  \GRANULEsym{]}$ which introduces a graded modal type $\Box_{  \GRANULEnt{r}  }  \GRANULEnt{A}$ by `promoting' a term $t$ to the graded modality, and it's dual $\textbf{let} \, [  \GRANULEmv{x}  ] =  \GRANULEnt{t_{{\mathrm{1}}}}  \, \textbf{in} \,  \GRANULEnt{t_{{\mathrm{2}}}}$ eliminates a graded modal value $\GRANULEnt{t_{{\mathrm{1}}}}$, binding a graded variable $x$
in scope of $\GRANULEnt{t_{{\mathrm{2}}}}$. The typing rules provide further understanding of the
behaviour of these terms.

Typing judgments are of the form $\Gamma  \vdash  \GRANULEnt{t}  :  \GRANULEnt{A}$, where $\Gamma$ ranges over contexts:
\begin{equation*}
  \Gamma ::= \emptyset
  \mid \Gamma  \GRANULEsym{,}   \GRANULEmv{x}  :  \GRANULEnt{A}
  \mid \Gamma  \GRANULEsym{,}   \GRANULEmv{x}  :_{\textcolor{coeffectColor}{  \GRANULEnt{r}  } }   \GRANULEnt{A}
\tag{contexts}
\end{equation*}

Thus, a context may be empty $\emptyset$, extended with a
linear assumption $\GRANULEmv{x}  :  \GRANULEnt{A}$ or extended with a graded assumption $\GRANULEmv{x}  :_{\textcolor{coeffectColor}{  \GRANULEnt{r}  } }   \GRANULEnt{A}$. For linear assumptions, structural rules of weakening
and contraction are disallowed. Graded assumptions may be used
non-linearly according to the constraints given by their grade, the
semiring element $r$. Throughout, comma denotes disjoint context
concatenation.

Various operations on contexts are used to capture non-linear data flow
via grading. Firstly, \emph{context addition}~\eqref{def:contextAdd} provides an
analogue to contraction, combining contexts that have come from
typing multiple subterms in a rule.
Context addition, written $\Gamma_{{\mathrm{1}}}  \GRANULEsym{+}  \Gamma_{{\mathrm{2}}}$, is undefined if $\Gamma_{{\mathrm{1}}}$ and $\Gamma_{{\mathrm{2}}}$
overlap in their linear assumptions. Otherwise graded assumptions appearing
in both contexts are combined via the semiring $+$ of their grades.

\begin{definition}[Context addition]\label{def:contextAdd}

\begin{align*}
    \setlength{\arraycolsep}{0.1em}
    \begin{array}{cc}
    \begin{array}{rl}
    \GRANULEsym{(}  \Gamma  \GRANULEsym{,}   \GRANULEmv{x}  :  \GRANULEnt{A}   \GRANULEsym{)}  \GRANULEsym{+}  \Gamma' & = \GRANULEsym{(}  \Gamma  \GRANULEsym{+}  \Gamma'  \GRANULEsym{)}  \GRANULEsym{,}   \GRANULEmv{x}  :  \GRANULEnt{A} \quad \text{iff} \,\; x
    \not\in | \Gamma' | \\
    \Gamma  \GRANULEsym{+}  \GRANULEsym{(}  \Gamma'  \GRANULEsym{,}   \GRANULEmv{x}  :  \GRANULEnt{A}   \GRANULEsym{)} & = \GRANULEsym{(}  \Gamma  \GRANULEsym{+}  \Gamma'  \GRANULEsym{)}  \GRANULEsym{,}   \GRANULEmv{x}  :  \GRANULEnt{A} \quad \text{iff} \,\; x \not\in | \Gamma | \\
    \GRANULEsym{(}  \Gamma  \GRANULEsym{,}   \GRANULEmv{x}  :_{\textcolor{coeffectColor}{  \GRANULEnt{r}  } }   \GRANULEnt{A}   \GRANULEsym{)}  \GRANULEsym{+}  \GRANULEsym{(}  \Gamma'  \GRANULEsym{,}   \GRANULEmv{x}  :_{\textcolor{coeffectColor}{  \GRANULEnt{s}  } }   \GRANULEnt{A}   \GRANULEsym{)} & = \GRANULEsym{(}  \Gamma  \GRANULEsym{+}  \Gamma'  \GRANULEsym{)}  \GRANULEsym{,}   \GRANULEmv{x}  :_{\textcolor{coeffectColor}{  \GRANULEsym{(}  \GRANULEnt{r}  \GRANULEsym{+}  \GRANULEnt{s}  \GRANULEsym{)}  } }   \GRANULEnt{A}
    \end{array}
      \;\;\quad & \quad\;\;
    \begin{array}{rl}
      \emptyset   \GRANULEsym{+}  \Gamma & = \Gamma \\
      \Gamma  \GRANULEsym{+}   \emptyset & = \Gamma \\ \\
    \end{array}
  \end{array}
  \end{align*}

Note that this is a declarative specification of context addition. Graded
assumptions may appear in any position in $\Gamma$ and $\Gamma'$ as witnessed by
the algorithmic specification where for all $\Gamma_{{\mathrm{1}}}, \Gamma_{{\mathrm{2}}}$
  \emph{context addition} is defined
as follows by ordered cases matching inductively on the structure of
$\Gamma_{{\mathrm{2}}}$:
\begin{align*}
\Gamma_{{\mathrm{1}}}  \GRANULEsym{+}  \Gamma_{{\mathrm{2}}} = \left\{\begin{matrix}
    \begin{array}{ll}
    \Gamma_{{\mathrm{1}}} &
     \Gamma_{{\mathrm{2}}} = \emptyset
             \\
      ((\Gamma'_{{\mathrm{1}}}, \Gamma''_{{\mathrm{1}}}) + \Gamma'_{{\mathrm{2}}}), \GRANULEmv{x}  :_{\textcolor{coeffectColor}{  \GRANULEsym{(}  \GRANULEnt{r}  \GRANULEsym{+}  \GRANULEnt{s}  \GRANULEsym{)}  } }   \GRANULEnt{A} \; &
\Gamma_{{\mathrm{2}}} = \Gamma'_{{\mathrm{2}}}  \GRANULEsym{,}   \GRANULEmv{x}  :_{\textcolor{coeffectColor}{  \GRANULEnt{s}  } }   \GRANULEnt{A} \wedge \Gamma_{{\mathrm{1}}} = \Gamma'_{{\mathrm{1}}}  \GRANULEsym{,}   \GRANULEmv{x}  :_{\textcolor{coeffectColor}{  \GRANULEnt{r}  } }   \GRANULEnt{A},\Gamma''_{{\mathrm{1}}}  \\
 (\Gamma_{{\mathrm{1}}} + \Gamma'_{{\mathrm{2}}}), \GRANULEmv{x}  :  \GRANULEnt{A} & \Gamma_{{\mathrm{2}}} = \Gamma'_{{\mathrm{2}}}  \GRANULEsym{,}   \GRANULEmv{x}  :  \GRANULEnt{A}\ \wedge\  \GRANULEmv{x}  :  \GRANULEnt{A} \notin \Gamma_{{\mathrm{1}}}
    \end{array}
  \end{matrix}\right.
\end{align*}
\end{definition}

\begin{figure}[H]
\hspace{-0.5em}
\begin{align*}
\hspace{-0.5em}
  \begin{array}{c}
  \inferrule*[right = Var]
  {\;}
  {\GRANULEmv{x}  :  \GRANULEnt{A}   \vdash  \GRANULEmv{x}  :  \GRANULEnt{A}}
\;\;
  \inferrule*[right = Abs]
  {\Gamma  \GRANULEsym{,}   \GRANULEmv{x}  :  \GRANULEnt{A}   \vdash  \GRANULEnt{t}  :  \GRANULEnt{B}}
  {\Gamma  \vdash   \lambda  \GRANULEmv{x}  .  \GRANULEnt{t}   :  \GRANULEnt{A}  \rightarrow  \GRANULEnt{B}}
\;\;
  \inferrule*[right = App]
  {\Gamma_{{\mathrm{1}}}  \vdash  \GRANULEnt{t_{{\mathrm{1}}}}  :  \GRANULEnt{A}  \rightarrow  \GRANULEnt{B} \;\;\;
   \Gamma_{{\mathrm{2}}}  \vdash  \GRANULEnt{t_{{\mathrm{2}}}}  :  \GRANULEnt{A} }
  {\Gamma_{{\mathrm{1}}}  \GRANULEsym{+}  \Gamma_{{\mathrm{2}}}  \vdash  \GRANULEnt{t_{{\mathrm{1}}}} \, \GRANULEnt{t_{{\mathrm{2}}}}  :  \GRANULEnt{B}}
\\[0.75em]
 \inferrule*[right = Weak]
  {\Gamma  \vdash  \GRANULEnt{t}  :  \GRANULEnt{A}}
  {\Gamma  \GRANULEsym{,}    [  \Delta  ]_{  0  }    \vdash  \GRANULEnt{t}  :  \GRANULEnt{A}}
\;\;\;
\inferrule*[right = Der]
  {\Gamma  \GRANULEsym{,}   \GRANULEmv{x}  :  \GRANULEnt{A}   \vdash  \GRANULEnt{t}  :  \GRANULEnt{B}}
  {\Gamma  \GRANULEsym{,}   \GRANULEmv{x}  :_{\textcolor{coeffectColor}{   1   } }   \GRANULEnt{A}   \vdash  \GRANULEnt{t}  :  \GRANULEnt{B}}
\;\;\;
\inferrule*[right = Approx]
{\Gamma  \GRANULEsym{,}   \GRANULEmv{x}  :_{\textcolor{coeffectColor}{  \GRANULEnt{r}  } }   \GRANULEnt{A}    \GRANULEsym{,}  \Gamma'  \vdash  \GRANULEnt{t}  :  \GRANULEnt{A} \quad r \sqsubseteq s }
{\Gamma  \GRANULEsym{,}   \GRANULEmv{x}  :_{\textcolor{coeffectColor}{  \GRANULEnt{s}  } }   \GRANULEnt{A}    \GRANULEsym{,}  \Gamma'  \vdash  \GRANULEnt{t}  :  \GRANULEnt{A}}
\\[0.75em]
\inferrule*[right = Pr]
  {\GRANULEsym{[}  \Gamma  \GRANULEsym{]}  \vdash  \GRANULEnt{t}  :  \GRANULEnt{A}}
  {\textcolor{coeffectColor}{ \GRANULEnt{r}   \textcolor{coeffectColor}{\,\cdot\,} }   \GRANULEsym{[}  \Gamma  \GRANULEsym{]}    \vdash  \GRANULEsym{[}  \GRANULEnt{t}  \GRANULEsym{]}  :   \Box_{  \GRANULEnt{r}  }  \GRANULEnt{A}}
\;\;\;
\inferrule*[right = Let$\Box$]
  {\Gamma_{{\mathrm{1}}}  \vdash  \GRANULEnt{t_{{\mathrm{1}}}}  :   \Box_{  \GRANULEnt{r}  }  \GRANULEnt{A} \;\;\;
   \Gamma_{{\mathrm{2}}}  \GRANULEsym{,}   \GRANULEmv{x}  :_{\textcolor{coeffectColor}{  \GRANULEnt{r}  } }   \GRANULEnt{A}   \vdash  \GRANULEnt{t_{{\mathrm{2}}}}  :  \GRANULEnt{B} }
    {\Gamma_{{\mathrm{1}}}  \GRANULEsym{+}  \Gamma_{{\mathrm{2}}}  \vdash   \textbf{let} \, [  \GRANULEmv{x}  ] =  \GRANULEnt{t_{{\mathrm{1}}}}  \, \textbf{in} \,  \GRANULEnt{t_{{\mathrm{2}}}}   :  \GRANULEnt{B}}
\end{array}
\end{align*}
\vspace{-1.25em}
  \caption{Typing rules of the graded linear $\lambda$-calculus}
\label{fig:typing}
\vspace{-0.65em}
 \end{figure}


Figure~\ref{fig:typing} defines the typing rules.
Linear variables are typed in a singleton context
(\textsc{Var}). Abstraction (\textsc{Abs}) and application (\textsc{App})
follow the rules of the linear $\lambda$-calculus.
The $\textsc{Weak}$ rule captures
weakening of assumptions graded by $0$ (where $[  \Delta  ]_{  0  }$ denotes a context
containing only graded assumptions graded by $0$). Context addition and
\textsc{Weak} together therefore provide the rules of substructural rules of contraction
and weakening.
Dereliction ($\textsc{Der}$),
allows a linear assumption to be converted to a graded assumption with grade
$1$. Grade approximation is captured by the $\textsc{Approx}$
rule, which allows a grade $s$ to be converted to another grade $r$,
providing that $r$ \textit{approximates} $s$, where the relation
$\sqsubseteq$ is the pre-order provided
with the semiring.
Introduction and elimination of the graded modality is provided by the
$\textsc{Pr}$ and $\textsc{Let}$ rules
respectively. The $\textsc{Pr}$ rule propagates the grade $r$ to the
assumptions through \emph{scalar multiplication} of $\Gamma$ by $r$ where
every assumption in $\Gamma$ must already be graded (written $\GRANULEsym{[}  \Gamma  \GRANULEsym{]}$ in the rule), given by definition~\eqref{def:scalar}.
%
%
\begin{definition}[Scalar context multiplication]
  \label{def:scalar}
 A context which consists solely of graded assumptions can be multiplied by a
 semiring grade $r \in \mathcal{R}$
\begin{align*}
   \textcolor{coeffectColor}{ \GRANULEnt{r}   \textcolor{coeffectColor}{\,\cdot\,} }   \emptyset = \emptyset
    \qquad\qquad
    \textcolor{coeffectColor}{ \GRANULEnt{r}   \textcolor{coeffectColor}{\,\cdot\,} }  \GRANULEsym{(}  \Gamma  \GRANULEsym{,}   \GRANULEmv{x}  :_{\textcolor{coeffectColor}{  \GRANULEnt{s}  } }   \GRANULEnt{A}   \GRANULEsym{)} = \GRANULEsym{(}   \textcolor{coeffectColor}{ \GRANULEnt{r}   \textcolor{coeffectColor}{\,\cdot\,} }  \Gamma   \GRANULEsym{)}  \GRANULEsym{,}   \GRANULEmv{x}  :_{\textcolor{coeffectColor}{  \GRANULEsym{(}  \GRANULEnt{r}  \textcolor{coeffectColor}{\,\cdot\,}  \GRANULEnt{s}  \GRANULEsym{)}  } }   \GRANULEnt{A}
\end{align*}
\end{definition}

The $\textsc{Let}$ rule eliminates a graded modal value $\Box_{  \GRANULEnt{r}  }  \GRANULEnt{A}$
into a graded assumption $\GRANULEmv{x}  :_{\textcolor{coeffectColor}{  \GRANULEnt{r}  } }   \GRANULEnt{A}$ with a matching
grade in the scope of the \textbf{let} body. This is also referred to as
``unboxing''.


We give an example of graded modalities using a graded modality indexed
by the semiring of natural numbers.

%SKI
\begin{example}
\label{ex:s-comb}
  The natural number semiring with discrete ordering
  $(\mathbb{N}, \ast, 1, +, 0, \equiv)$ provides a graded modality
  that counts exactly how many times non-linear values are used. As a
  simple example, the \emph{S} combinator from the SKI system of combinatory logic is typed and defined:
% then2 : forall {a b c : Type} . (a -> (b -> c)) -> (a -> b) -> (a
% [2] -> c)
\begin{align*}
s & : \GRANULEsym{(}  \GRANULEnt{A}  \rightarrow  \GRANULEsym{(}  \GRANULEnt{B}  \rightarrow  \GRANULEnt{C}  \GRANULEsym{)}  \GRANULEsym{)}  \rightarrow   \GRANULEsym{(}  \GRANULEnt{A}  \rightarrow  \GRANULEnt{B}  \GRANULEsym{)}  \rightarrow  \GRANULEsym{(}    \Box_{  \GRANULEsym{2}  }  \GRANULEnt{A}    \rightarrow  \GRANULEnt{C}  \GRANULEsym{)} \\
s & = \lambda  \GRANULEmv{x}  .    \lambda  \GRANULEmv{y}  .    \lambda  \GRANULEmv{z'}  .    \textbf{let} \, [  \GRANULEmv{z}  ] =  \GRANULEmv{z'}  \, \textbf{in} \,   \GRANULEsym{(}  \GRANULEmv{x} \, \GRANULEmv{z}  \GRANULEsym{)} \, \GRANULEsym{(}  \GRANULEmv{y} \, \GRANULEmv{z}  \GRANULEsym{)}
\end{align*}
The graded modal value $z'$ captures the `capability' for a value
of type $A$ to be used twice. This capability is made available by eliminating
$\Box$ (via \textbf{let}) to the variable $z$, which is
graded $z : [A]_2$ in the scope of the body.
\end{example}



\paragraph{Metatheory}
The admissibility of substitution is a key result that holds
for this language~\cite{DBLP:journals/pacmpl/OrchardLE19}, which is
leveraged in soundness of the synthesis calculi.
%
\begin{restatable}[Admissibility of substitution]{lemma}{linearSubst}
Let $\Delta  \vdash  \GRANULEnt{t'}  :  \GRANULEnt{A}$, then:
\label{lemma:substitution}
\begin{itemize}[leftmargin=1em]
\item (Linear) \hspace{0.04em} If $\Gamma  \GRANULEsym{,}   \GRANULEmv{x}  :  \GRANULEnt{A}    \GRANULEsym{,}  \,  \GRANULEsym{,}   \Gamma'   \vdash  \GRANULEnt{t}  :  \GRANULEnt{B}$ then $\Gamma  \GRANULEsym{+}  \Delta  \GRANULEsym{+}  \Gamma'  \vdash   [  \GRANULEnt{t'}  /  \GRANULEmv{x}  ]  \GRANULEnt{t}   :  \GRANULEnt{B}$
\item (Graded) If $\Gamma  \GRANULEsym{,}   \GRANULEmv{x}  :_{\textcolor{coeffectColor}{  \GRANULEnt{r}  } }   \GRANULEnt{A}    \GRANULEsym{,}  \,  \GRANULEsym{,}   \Gamma'   \vdash  \GRANULEnt{t}  :  \GRANULEnt{B}$
then $\Gamma  \GRANULEsym{+}  \GRANULEsym{(}   \textcolor{coeffectColor}{ \GRANULEnt{r}   \textcolor{coeffectColor}{\,\cdot\,} }  \Delta   \GRANULEsym{)}  \GRANULEsym{+}  \Gamma'  \vdash   [  \GRANULEnt{t'}  /  \GRANULEmv{x}  ]  \GRANULEnt{t}   :  \GRANULEnt{B}$
\end{itemize}
\end{restatable}

\section{Graded types: An alternative history}
\jnote{talk about lineage of graded type systems}

\subsection{Graded-base}
\label{sec:graded-base}

We now define a core calculus for a fully graded type system. This system is much closer
to those outlined above than the linear-base calculus, drawing from the coeffect calculus of \citet{petricek2014coeffects}, Quantitative Type Theory (QTT) by \citet{McBride2016} and refined further by \citet{quantitative-type-theory} (although we omit dependent types from our language), the calculus of \citet{DBLP:journals/pacmpl/AbelB20}, and other graded dependent type theories~\cite{quantitative-type-theory,DBLP:conf/esop/MoonEO21}. Similar
systems also form the basis of the core of the linear types extension to
Haskell~\cite{DBLP:journals/pacmpl/BernardyBNJS18}. We refer to this system as
the \textit{graded-base} calculus to differentiate it from linear-base.

The syntax of graded-base types is given by:
\begin{align*}
\hspace{-0.9em}\GRANULEnt{A} , \GRANULEnt{B} & ::=
       \GRANULEnt{A} ^ \GRANULEnt{r}  \rightarrow  \GRANULEnt{B}
  \mid \Box_{  \GRANULEnt{r}  }  \GRANULEnt{A}
{\small{\tag{\textit{types}}}}
\end{align*}
where the function space $\GRANULEnt{A} ^ \GRANULEnt{r}  \rightarrow  \GRANULEnt{B}$ annotates the input type with a \emph{grade} $\GRANULEnt{r}$
drawn from a pre-ordered semiring
$(\mathcal{R}, {\ast}, {1}, {+}, {0}, \sqsubseteq)$ which paramterising the
calculus as in linear-base.

The graded necessity modality $\Box_{  \GRANULEnt{r}  }  \GRANULEnt{A}$ is similarly annotated by the grade
$\GRANULEnt{r}$ being an element of the semiring.


The syntax of terms is given as:
%
\begin{align*}
\hspace{-0.8em} \GRANULEnt{t} ::= \;
       & \GRANULEmv{x}
  \mid \lambda  \GRANULEmv{x}  .  \GRANULEnt{t}
  \mid \GRANULEnt{t_{{\mathrm{1}}}} \, \GRANULEnt{t_{{\mathrm{2}}}}
  \mid \GRANULEsym{[}  \GRANULEnt{t}  \GRANULEsym{]}
  % \mid \textbf{case} \  \GRANULEnt{t}  \ \textbf{of} \   \GRANULEnt{p_{{\mathrm{1}}}}  \mapsto  \GRANULEnt{t_{{\mathrm{1}}}} ; ... ;  \GRANULEnt{p_{\GRANULEmv{n}}}  \mapsto  \GRANULEnt{t_{\GRANULEmv{n}}}
{\small{\tag{\textit{terms} }}}
\end{align*}
%
Similarly to linear-base, terms consist of a graded $\lambda$-calculus, extended
with a \textit{promotion} construct [t] which introduces a graded modality
explicitly.


\begin{definition}[Context addition]\label{def:contextAdd}

\begin{align*}
\Gamma_{{\mathrm{1}}}  \GRANULEsym{+}  \Gamma_{{\mathrm{2}}} = \left\{\begin{matrix}
    \begin{array}{ll}
    \Gamma_{{\mathrm{1}}} &
    \Gamma_{{\mathrm{2}}} = \emptyset
             \\
      ((\Gamma'_{{\mathrm{1}}}, \Gamma''_{{\mathrm{1}}}) + \Gamma'_{{\mathrm{2}}}), \GRANULEmv{x}  :_{\textcolor{coeffectColor}{  \GRANULEsym{(}  \GRANULEnt{r}  \GRANULEsym{+}  \GRANULEnt{s}  \GRANULEsym{)}  } }   \GRANULEnt{A} \; &
\Gamma_{{\mathrm{2}}} = \Gamma'_{{\mathrm{2}}}  \GRANULEsym{,}   \GRANULEmv{x}  :_{\textcolor{coeffectColor}{  \GRANULEnt{s}  } }   \GRANULEnt{A} \wedge \Gamma_{{\mathrm{1}}} = \Gamma'_{{\mathrm{1}}}  \GRANULEsym{,}   \GRANULEmv{x}  :_{\textcolor{coeffectColor}{  \GRANULEnt{r}  } }   \GRANULEnt{A},\Gamma''_{{\mathrm{1}}} \\
 \GRANULEsym{(}  \Gamma_{{\mathrm{1}}}  \GRANULEsym{+}  \Gamma'_{{\mathrm{2}}}  \GRANULEsym{)}  \GRANULEsym{,}   \GRANULEmv{x}  :_{\textcolor{coeffectColor}{  \GRANULEnt{s}  } }   \GRANULEnt{A} & \Gamma_{{\mathrm{2}}} = \Gamma'_{{\mathrm{2}}}  \GRANULEsym{,}   \GRANULEmv{x}  :_{\textcolor{coeffectColor}{  \GRANULEnt{s}  } }   \GRANULEnt{A} \wedge \GRANULEmv{x} \not\in \mathsf{dom}(\Gamma_{{\mathrm{1}}})
\end{array}
  \end{matrix}\right.
\end{align*}
\end{definition}

\begin{figure}[H]
\hspace{-0.5em}
\begin{align*}
\hspace{-0.5em}
\begin{array}{c}
\GRANULEdruleTyVar{}
\;\;\;
\GRANULEdruleTyAbs{}
\;\;\;
\GRANULEdruleTyApp{}
\\\\
\GRANULEdruleTyPr{}
\;\;\;
\GRANULEdruleTyApprox{}
\end{array}
\end{align*}
\vspace{-0.5em}
\caption{Typing rules for graded-base}
\label{fig:typing}
\vspace{-0.5em}
 \end{figure}

Figure~\ref{fig:typing} gives the full typing rules, which helps explain the meaning of
the syntax with reference to their static semantics.

Variables (rule \textsc{Var}) are
typed in a context where the variable $x$ has grade $1$ denoting its
single usage here. All other variable assumptions are given the grade
of the $0$ semiring element (providing \emph{weakening}),
using \textit{scalar multiplication} of contexts by a grade, re-using definition~\eqref{def:scalar}.


\section{Two typed calculi}


Having outlined the two major lineages of resourceful types, we
are now left with the question: what approach should we use as the basis of our
program synthesis tool? Both approaches pose there own challenges and questions
which influence the design of a synthesis calculus. The breadth of these
differences leaves omitting either approach an undesirable prospect. For this
reason, this thesis tackles languages based on both approaches:

\begin{enumerate}
        \item We begin with a simplified synthesis calculus for a core language
        based on the linear $\lambda$ calculus outlined in
        section~\eqref{sec:linear-base}. To better illustrate the practicality
        of the syntesis calculus, the language is extended with product
        and sum types, as well as a unit type.
        \item
\end{enumerate}



% The different approaches to graded types here gives rise to the question: how might the
% features of the target language influence the design of a synthesis tool?

This chapter introduced some of the key features of languages with resourceful
types. Linear and graded types embed usage constraints in the typing rules,
enforcing the notion that a well-typed program is also \textit{well-resourced}.

The next chapter focuses on the linear-base core calculus of
section~\ref{sec:linear-base}, extending this calculus with multiplicative and additive
types, as well as a unit type to form a more practical programming language. This
then comprises the target language of a synthesis algorithm. Likewise, the graded-base
calculus is revisited in chapter~\ref{chapter:graded-base}, where it is extended
with (G)ADTs, and recursion, providing a target language for a more in-depth and
featureful synthesis tool.


\chapter{A core synthesis calculus}
\label{chapter:core}
\input{3_linear-base/linear-base-ottput}

\chapter{Deriving graded combinators}
\label{chapter:deriving}
\input{4_deriving/deriving-ottput}

\chapter{An extended synthesis calculus}
\label{chapter:extended}
\input{5_graded-base/extended-ottput}

\chapter{Conclusion}
\label{chapter:conclusion}
\input{6_conclusion/conclusion-ottput}

\bibliography{references}{}

% \iffalse
\newpage
\appendix
\input{7_appendix/appendix-ottput}
\section{Linear-base Proofs}
\label{sec:linear-proofs}
This section gives the proofs of Lemma~\ref{lemma:subSynthSound} and
Lemma~\ref{lemma:addSynthSound}, along with soundness results for the
variant systems: additive pruning and subtractive division.

We first state and prove some intermediate results about context manipulations
which are needed for the main lemmas.

\begin{definition}[Context approximation]
For contexts $\Gamma_{{\mathrm{1}}}$, $\Gamma_{{\mathrm{2}}}$ then:
%
\begin{align*}
\begin{array}{c}
\dfrac{}{\emptyset  \  \textcolor{coeffectColor}{\sqsubseteq} \   \emptyset}
\qquad
\dfrac{\Gamma_{{\mathrm{1}}} \sqsubseteq \Gamma_{{\mathrm{2}}}}
      {\Gamma_{{\mathrm{1}}}  \GRANULEsym{,}   \GRANULEmv{x}  :  \GRANULEnt{A} \sqsubseteq \Gamma_{{\mathrm{2}}}  \GRANULEsym{,}   \GRANULEmv{x}  :  \GRANULEnt{A}}
\qquad \\[1.5em]
\dfrac{\Gamma_{{\mathrm{1}}} \sqsubseteq \Gamma_{{\mathrm{2}}} \qquad \GRANULEnt{r} \sqsubseteq \GRANULEnt{s}}
      {\Gamma_{{\mathrm{1}}}  \GRANULEsym{,}   \GRANULEmv{x}  :_{\textcolor{coeffectColor}{  \GRANULEnt{r}  } }   \GRANULEnt{A} \sqsubseteq \Gamma_{{\mathrm{2}}}  \GRANULEsym{,}   \GRANULEmv{x}  :_{\textcolor{coeffectColor}{  \GRANULEnt{s}  } }   \GRANULEnt{A}}
\qquad
\dfrac{ \Gamma_{{\mathrm{1}}} \sqsubseteq \Gamma_{{\mathrm{2}}} \qquad 0 \sqsubseteq \GRANULEnt{s}}
      { \Gamma_{{\mathrm{1}}} \sqsubseteq \Gamma_{{\mathrm{2}}}  \GRANULEsym{,}   \GRANULEmv{x}  :_{\textcolor{coeffectColor}{  \GRANULEnt{s}  } }   \GRANULEnt{A}}
\end{array}
\end{align*}
%
This is actioned in type checking by iterative application of $\textsc{Approx}$.
\end{definition}

\begin{restatable}[$\Gamma  \GRANULEsym{+}  \GRANULEsym{(}  \Gamma'  \GRANULEsym{-}  \Gamma''  \GRANULEsym{)} \sqsubseteq \GRANULEsym{(}  \Gamma  \GRANULEsym{+}  \Gamma'  \GRANULEsym{)}  \GRANULEsym{-}  \Gamma''$]{lemma}{contextLemma1}
  \label{lemma:contextLemma1}
\end{restatable}

\begin{proof}
  Induction over the structure of both $\Gamma'$ and $\Gamma''$. The possible forms of
  $\Gamma'$ and $\Gamma''$ are considered in turn:
  \begin{enumerate}
    \item $\Gamma'$ = $\emptyset$ and $\Gamma''$ = $\emptyset$\\
      We have:
      \begin{align*}
        (\Gamma + \emptyset) - \emptyset = \Gamma + (\emptyset - \emptyset)
      \end{align*}
      From definitions~\ref{def:contextAdd} and~\ref{def:contextSub}, we know that
      on the left hand side:
      \begin{align*}
        (\Gamma + \emptyset) - \emptyset &= \Gamma + \emptyset \\
                                &= \Gamma
      \end{align*}
      and on the right-hand side:
      \begin{align*}
        \Gamma + (\emptyset - \emptyset) &= \Gamma + \emptyset \\
                                &= \Gamma
      \end{align*}
      making both the left and right hand sides equivalent:
      \begin{align*}
        \Gamma = \Gamma
      \end{align*}
    \item $\Gamma'$ = $\Gamma'  \GRANULEsym{,}   \GRANULEmv{x}  :  \GRANULEnt{A}$ and $\Gamma''$ = $\emptyset$\\
      We have
      \begin{align*}
        (\Gamma + \Gamma'  \GRANULEsym{,}   \GRANULEmv{x}  :  \GRANULEnt{A}) - \emptyset = \Gamma + (\Gamma  \GRANULEsym{,}   \GRANULEmv{x}  :  \GRANULEnt{A} - \emptyset)
      \end{align*}
      From definitions~\ref{def:contextAdd} and~\ref{def:contextSub}, we know that
      on the left hand side we have:
      \begin{align*}
        (\Gamma + \Gamma'  \GRANULEsym{,}   \GRANULEmv{x}  :  \GRANULEnt{A}) - \emptyset &= (\Gamma, \Gamma'), \GRANULEmv{x}  :  \GRANULEnt{A} - \emptyset \\
                                        &= (\Gamma, \Gamma'), \GRANULEmv{x}  :  \GRANULEnt{A}
      \end{align*}
      and on the right hand side:
      \begin{align*}
        \Gamma + (\Gamma  \GRANULEsym{,}   \GRANULEmv{x}  :  \GRANULEnt{A} - \emptyset) &= \Gamma + \Gamma'  \GRANULEsym{,}   \GRANULEmv{x}  :  \GRANULEnt{A}\\
                                       &= (\Gamma, \Gamma', \GRANULEmv{x}  :  \GRANULEnt{A})
      \end{align*}
      making both the left and right hand sides equal:
      \begin{align*}
        (\Gamma,\Gamma'), \GRANULEmv{x}  :  \GRANULEnt{A} = (\Gamma,\Gamma'), \GRANULEmv{x}  :  \GRANULEnt{A}
      \end{align*}


    \item $\Gamma'$ = $\Gamma'  \GRANULEsym{,}   \GRANULEmv{x}  :  \GRANULEnt{A}$ and $\Gamma''$ = $\Gamma''  \GRANULEsym{,}   \GRANULEmv{x}  :  \GRANULEnt{A}$\\
      We have
      \begin{align*}
        (\Gamma + \Gamma'  \GRANULEsym{,}   \GRANULEmv{x}  :  \GRANULEnt{A}) - \Gamma''  \GRANULEsym{,}   \GRANULEmv{x}  :  \GRANULEnt{A} = \Gamma + (\Gamma'  \GRANULEsym{,}   \GRANULEmv{x}  :  \GRANULEnt{A} -
        \Gamma''  \GRANULEsym{,}   \GRANULEmv{x}  :  \GRANULEnt{A})
      \end{align*}
      From definitions~\ref{def:contextAdd} and~\ref{def:contextSub}, we know that
      on the left hand side we have:
      \begin{align*}
        (\Gamma + \Gamma'  \GRANULEsym{,}   \GRANULEmv{x}  :  \GRANULEnt{A}) - \Gamma''  \GRANULEsym{,}   \GRANULEmv{x}  :  \GRANULEnt{A} &= (\Gamma,\Gamma'), \GRANULEmv{x}  :  \GRANULEnt{A} -
                                                   \Gamma''  \GRANULEsym{,}   \GRANULEmv{x}  :  \GRANULEnt{A} \\
                                                 &= \Gamma, \Gamma' - \Gamma''
      \end{align*}
      and on the right hand side:
      \begin{align*}
        \Gamma + (\Gamma'  \GRANULEsym{,}   \GRANULEmv{x}  :  \GRANULEnt{A} - \Gamma''  \GRANULEsym{,}   \GRANULEmv{x}  :  \GRANULEnt{A}) &= \Gamma + (\Gamma' - \Gamma'') \\
                                                &= \Gamma, \Gamma' - \Gamma''
      \end{align*}
      making both the left and right hand sides equivalent:
      \begin{align*}
        \Gamma, \Gamma' - \Gamma'' = \Gamma, \Gamma' - \Gamma''
      \end{align*}

    \item $\Gamma'$ = $\Gamma'  \GRANULEsym{,}   \GRANULEmv{x}  :_{\textcolor{coeffectColor}{  \GRANULEnt{r}  } }   \GRANULEnt{A}$ and $\Gamma''$ = $\emptyset$\\
      We have
      \begin{align*}
        (\Gamma + \Gamma'  \GRANULEsym{,}   \GRANULEmv{x}  :_{\textcolor{coeffectColor}{  \GRANULEnt{r}  } }   \GRANULEnt{A}) - \emptyset = \Gamma + (\GRANULEmv{x}  :_{\textcolor{coeffectColor}{  \GRANULEnt{r}  } }   \GRANULEnt{A} - \emptyset)
      \end{align*}
      From definitions~\ref{def:contextAdd} and~\ref{def:contextSub}, we know that
      on the left hand side we have:
      \begin{align*}
        (\Gamma + \Gamma'  \GRANULEsym{,}   \GRANULEmv{x}  :_{\textcolor{coeffectColor}{  \GRANULEnt{r}  } }   \GRANULEnt{A}) - \emptyset &= (\Gamma + \Gamma'  \GRANULEsym{,}   \GRANULEmv{x}  :_{\textcolor{coeffectColor}{  \GRANULEnt{r}  } }   \GRANULEnt{A}) \\
                                            &= (\Gamma, \Gamma'), \GRANULEmv{x}  :_{\textcolor{coeffectColor}{  \GRANULEnt{r}  } }   \GRANULEnt{A}
      \end{align*}
      and on the right hand side:
      \begin{align*}
        \Gamma + (\Gamma'  \GRANULEsym{,}   \GRANULEmv{x}  :_{\textcolor{coeffectColor}{  \GRANULEnt{r}  } }   \GRANULEnt{A} - \emptyset) &= \Gamma + (\Gamma'  \GRANULEsym{,}   \GRANULEmv{x}  :_{\textcolor{coeffectColor}{  \GRANULEnt{r}  } }   \GRANULEnt{A})
                                            &= (\Gamma,\Gamma'),\GRANULEmv{x}  :_{\textcolor{coeffectColor}{  \GRANULEnt{r}  } }   \GRANULEnt{A}
      \end{align*}
      making both the left and right hand sides equivalent:
      \begin{align*}
        (\Gamma,\Gamma'),\GRANULEmv{x}  :_{\textcolor{coeffectColor}{  \GRANULEnt{r}  } }   \GRANULEnt{A} = (\Gamma,\Gamma'),\GRANULEmv{x}  :_{\textcolor{coeffectColor}{  \GRANULEnt{r}  } }   \GRANULEnt{A}
      \end{align*}


    \item $\Gamma'$ = $\Gamma'  \GRANULEsym{,}   \GRANULEmv{x}  :_{\textcolor{coeffectColor}{  \GRANULEnt{r}  } }   \GRANULEnt{A}$ and $\Gamma''$ = $\Gamma''  \GRANULEsym{,}   \GRANULEmv{x}  :_{\textcolor{coeffectColor}{  \GRANULEnt{s}  } }   \GRANULEnt{A}$\\

      Thus we have (for the LHS of the inequality term):
      %
       \begin{align*}
        \Gamma + (\Gamma'  \GRANULEsym{,}   \GRANULEmv{x}  :_{\textcolor{coeffectColor}{  \GRANULEnt{r}  } }   \GRANULEnt{A} - \Gamma''  \GRANULEsym{,}   \GRANULEmv{x}  :_{\textcolor{coeffectColor}{  \GRANULEnt{s}  } }   \GRANULEnt{A})
      \end{align*}
      %
    which by context subtraction yields:
      \begin{align*}
       \Gamma + (\Gamma'  \GRANULEsym{,}   \GRANULEmv{x}  :_{\textcolor{coeffectColor}{  \GRANULEnt{r}  } }   \GRANULEnt{A} - \Gamma''  \GRANULEsym{,}   \GRANULEmv{x}  :_{\textcolor{coeffectColor}{  \GRANULEnt{s}  } }   \GRANULEnt{A}) &= \Gamma + (\Gamma' -
                                                         \Gamma''), \GRANULEmv{x}  :_{\textcolor{coeffectColor}{  \GRANULEnt{q'}  } }   \GRANULEnt{A}
      \end{align*}
      where:
      \begin{align*}
        \exists q' . \GRANULEnt{r} \sqsupseteq \GRANULEnt{q'} + \GRANULEnt{s}
\quad \maximal{q'}{\hat{q'}}{r}{\hat{q'}+s} \qquad (2)
      \end{align*}
       %
       And for the LHS of the inequality, from
       definitions~\ref{def:contextAdd} and~\ref{def:contextSub}
       we have:
      \begin{align*}
        (\Gamma + \Gamma'  \GRANULEsym{,}   \GRANULEmv{x}  :_{\textcolor{coeffectColor}{  \GRANULEnt{r}  } }   \GRANULEnt{A}) - \Gamma''  \GRANULEsym{,}   \GRANULEmv{x}  :_{\textcolor{coeffectColor}{  \GRANULEnt{s}  } }   \GRANULEnt{A} &
= (\Gamma + \Gamma'), \GRANULEmv{x}  :_{\textcolor{coeffectColor}{  \GRANULEnt{r}  } }   \GRANULEnt{A} - \Gamma''  \GRANULEsym{,}   \GRANULEmv{x}  :_{\textcolor{coeffectColor}{  \GRANULEnt{s}  } }   \GRANULEnt{A} \\
  &= ((\Gamma + \Gamma') -  \Gamma''), \GRANULEmv{x}  :_{\textcolor{coeffectColor}{  \GRANULEnt{r}  } }   \GRANULEnt{A} - \GRANULEmv{x}  :_{\textcolor{coeffectColor}{  \GRANULEnt{s}  } }   \GRANULEnt{A} \\
  &= ((\Gamma + \Gamma') -  \Gamma''), \GRANULEmv{x}  :_{\textcolor{coeffectColor}{  \GRANULEnt{q}  } }   \GRANULEnt{A}
      \end{align*}
      where:
      \begin{align*}
        \exists q . \GRANULEnt{r} \sqsupseteq \GRANULEnt{q} + \GRANULEnt{s}
\quad \maximal{q}{\hat{q}}{r}{\hat{q}+s} \qquad (1)
      \end{align*}
    %
    %Applying maximality (1) to $q'$ yields that $\texttt{\textcolor{red}{<<multiple parses>>}}$
    Applying $\exists q . \GRANULEnt{r} \sqsupseteq \GRANULEnt{q} + \GRANULEnt{s}$ to
    maximality (2) (at $\hat{q'} = q$) then yields that $q \sqsubseteq q'$.

    Therefore, applying induction, we derive:
    %
     \begin{align*}
      \dfrac{\GRANULEsym{(}  \Gamma  \GRANULEsym{+}  \GRANULEsym{(}  \Gamma'  \GRANULEsym{-}  \Gamma''  \GRANULEsym{)}  \GRANULEsym{)} \sqsubseteq \GRANULEsym{(}  \GRANULEsym{(}  \Gamma  \GRANULEsym{+}  \Gamma'  \GRANULEsym{)}  \GRANULEsym{-}  \Gamma''  \GRANULEsym{)} \qquad \GRANULEnt{q} \sqsubseteq \GRANULEnt{q'}}
             {\GRANULEsym{(}  \Gamma  \GRANULEsym{+}  \GRANULEsym{(}  \Gamma'  \GRANULEsym{-}  \Gamma''  \GRANULEsym{)}  \GRANULEsym{)}  \GRANULEsym{,}   \GRANULEmv{x}  :_{\textcolor{coeffectColor}{  \GRANULEnt{q}  } }   \GRANULEnt{A} \sqsubseteq \GRANULEsym{(}  \GRANULEsym{(}  \Gamma  \GRANULEsym{+}  \Gamma'  \GRANULEsym{)}  \GRANULEsym{-}  \Gamma''  \GRANULEsym{)}  \GRANULEsym{,}   \GRANULEmv{x}  :_{\textcolor{coeffectColor}{  \GRANULEnt{q'}  } }   \GRANULEnt{A}}
     \end{align*}
    %
    satisfying the lemma statement.
    %\dnote{if partial order then $\GRANULEnt{q} = \GRANULEnt{q'}$, but we can be weaker
    %  and weaken Lemma 4 to be a context approximation.}
    %  making both the left and right hand sides equivalent:

  \end{enumerate}
\end{proof}

\begin{restatable}[$\GRANULEsym{(}  \Gamma  \GRANULEsym{-}  \Gamma'  \GRANULEsym{)}  \GRANULEsym{+}  \Gamma' \sqsubseteq \Gamma$]{lemma}{contextLemma2}
  \label{lemma:contextLemma2}
\end{restatable}
\begin{proof}
 The proof follows by induction over the structure of $\Gamma'$. The possible
 forms of $\Gamma'$ are considered in turn:
 \begin{enumerate}
     \item $\Gamma'$ = $\emptyset$\\
     We have:
     \begin{align*}
       (\Gamma - \emptyset) + \emptyset = \Gamma
     \end{align*}
     From definition~\ref{def:contextSub}, we know that:
     \begin{align*}
       \Gamma - \emptyset = \Gamma
     \end{align*}
     and from definition~\ref{def:contextAdd}, we know:
     \begin{align*}
       \Gamma + \emptyset = \Gamma
     \end{align*}
     giving us:
     \begin{align*}
       \Gamma = \Gamma
     \end{align*}


     \item $\Gamma'$ = $\Gamma''  \GRANULEsym{,}   \GRANULEmv{x}  :  \GRANULEnt{A}$\\
     and let $\Gamma = \Gamma'  \GRANULEsym{,}   \GRANULEmv{x}  :  \GRANULEnt{A}$.

     \begin{align*}
       (\Gamma'  \GRANULEsym{,}   \GRANULEmv{x}  :  \GRANULEnt{A} - \Gamma''  \GRANULEsym{,}   \GRANULEmv{x}  :  \GRANULEnt{A}) + \Gamma''  \GRANULEsym{,}   \GRANULEmv{x}  :  \GRANULEnt{A} = \Gamma
     \end{align*}
     From definition~\ref{def:contextAdd}, we know that:
     \begin{align*}
       (\Gamma'  \GRANULEsym{,}   \GRANULEmv{x}  :  \GRANULEnt{A} - \Gamma''  \GRANULEsym{,}   \GRANULEmv{x}  :  \GRANULEnt{A}) + \Gamma''  \GRANULEsym{,}   \GRANULEmv{x}  :  \GRANULEnt{A}
       & =  ((\Gamma'  \GRANULEsym{-}  \Gamma'') + \Gamma''), \GRANULEmv{x}  :  \GRANULEnt{A} \\
 \textit{induction}  & = \Gamma'  \GRANULEsym{,}   \GRANULEmv{x}  :  \GRANULEnt{A} \\
                     & = \Gamma
     \end{align*}
     thus satisfying the lemma statement by equality.

     \item $\Gamma'$ = $\Gamma''  \GRANULEsym{,}   \GRANULEmv{x}  :_{\textcolor{coeffectColor}{  \GRANULEnt{r}  } }   \GRANULEnt{A}$\\
       and let $\Gamma = \Gamma'  \GRANULEsym{,}   \GRANULEmv{x}  :_{\textcolor{coeffectColor}{  \GRANULEnt{s}  } }   \GRANULEnt{A}$.

     We have:
     \begin{align*}
       (\Gamma'  \GRANULEsym{,}   \GRANULEmv{x}  :_{\textcolor{coeffectColor}{  \GRANULEnt{s}  } }   \GRANULEnt{A} - \Gamma''  \GRANULEsym{,}   \GRANULEmv{x}  :_{\textcolor{coeffectColor}{  \GRANULEnt{r}  } }   \GRANULEnt{A}) + \Gamma''  \GRANULEsym{,}   \GRANULEmv{x}  :_{\textcolor{coeffectColor}{  \GRANULEnt{r}  } }   \GRANULEnt{A}
     \end{align*}
     From definition~\ref{def:contextSub}, we know that:
     %

       \begin{align*}
      & (\Gamma'  \GRANULEsym{,}   \GRANULEmv{x}  :_{\textcolor{coeffectColor}{  \GRANULEnt{s}  } }   \GRANULEnt{A} - \Gamma''  \GRANULEsym{,}   \GRANULEmv{x}  :_{\textcolor{coeffectColor}{  \GRANULEnt{r}  } }   \GRANULEnt{A}) + \Gamma''  \GRANULEsym{,}   \GRANULEmv{x}  :_{\textcolor{coeffectColor}{  \GRANULEnt{r}  } }   \GRANULEnt{A} \\
    =\ & \GRANULEsym{(}  \Gamma'  \GRANULEsym{-}  \Gamma''  \GRANULEsym{)}  \GRANULEsym{,}   \GRANULEmv{x}  :_{\textcolor{coeffectColor}{  \GRANULEnt{q}  } }   \GRANULEnt{A} + \Gamma''  \GRANULEsym{,}   \GRANULEmv{x}  :_{\textcolor{coeffectColor}{  \GRANULEnt{r}  } }   \GRANULEnt{A} \\
    =\ & \GRANULEsym{(}  \GRANULEsym{(}  \Gamma'  \GRANULEsym{-}  \Gamma''  \GRANULEsym{)}  \GRANULEsym{+}  \Gamma''  \GRANULEsym{)}  \GRANULEsym{,}   \GRANULEmv{x}  :_{\textcolor{coeffectColor}{   \GRANULEnt{q}  \GRANULEsym{+}  \GRANULEnt{r}   } }   \GRANULEnt{A}
       \end{align*}

     %
     where $\GRANULEnt{s}  \sqsupseteq  \GRANULEnt{q}  \GRANULEsym{+}  \GRANULEnt{r}$ and $\maximal{q}{q'}{s}{q' + r}$.

     Then by induction we derive the ordering:
     %
     \begin{align*}
       \dfrac{\GRANULEsym{(}  \GRANULEsym{(}  \Gamma'  \GRANULEsym{-}  \Gamma''  \GRANULEsym{)}  \GRANULEsym{+}  \Gamma''  \GRANULEsym{)} \sqsubseteq \Gamma' \qquad \GRANULEnt{q}  \GRANULEsym{+}  \GRANULEnt{r}  \sqsubseteq \GRANULEnt{s}}
            {\GRANULEsym{(}  \GRANULEsym{(}  \Gamma'  \GRANULEsym{-}  \Gamma''  \GRANULEsym{)}  \GRANULEsym{+}  \Gamma''  \GRANULEsym{)}  \GRANULEsym{,}   \GRANULEmv{x}  :_{\textcolor{coeffectColor}{   \GRANULEnt{q}  \GRANULEsym{+}  \GRANULEnt{r}   } }   \GRANULEnt{A} \sqsubseteq \Gamma'  \GRANULEsym{,}   \GRANULEmv{x}  :_{\textcolor{coeffectColor}{  \GRANULEnt{s}  } }   \GRANULEnt{A}}
     \end{align*}
     %
     which satifies the lemma statement.
 \end{enumerate}
\end{proof}

\begin{lemma}[Context negation]
\label{lemma:context-neg}
For all contexts $\Gamma$:
\begin{align*}
\emptyset \sqsubseteq \Gamma  \GRANULEsym{-}  \Gamma
\end{align*}
\end{lemma}

\begin{proof}
By induction on the structure of $\Gamma$:
%
\begin{itemize}
\item $\Gamma = \emptyset$ Trivial.

\item $\Gamma = \Gamma'  \GRANULEsym{,}   \GRANULEmv{x}  :  \GRANULEnt{A}$ then
$\GRANULEsym{(}  \Gamma'  \GRANULEsym{,}   \GRANULEmv{x}  :  \GRANULEnt{A}   \GRANULEsym{)}  \GRANULEsym{-}  \GRANULEsym{(}  \Gamma'  \GRANULEsym{,}   \GRANULEmv{x}  :  \GRANULEnt{A}   \GRANULEsym{)} = \Gamma'  \GRANULEsym{-}  \Gamma'$ so proceed by
induction.

\item $\Gamma = \Gamma'  \GRANULEsym{,}   \GRANULEmv{x}  :_{\textcolor{coeffectColor}{  \GRANULEnt{r}  } }   \GRANULEnt{A}$ then
$\exists q. $
$\GRANULEsym{(}  \Gamma'  \GRANULEsym{,}   \GRANULEmv{x}  :_{\textcolor{coeffectColor}{  \GRANULEnt{r}  } }   \GRANULEnt{A}   \GRANULEsym{)}  \GRANULEsym{-}  \GRANULEsym{(}  \Gamma'  \GRANULEsym{,}   \GRANULEmv{x}  :_{\textcolor{coeffectColor}{  \GRANULEnt{r}  } }   \GRANULEnt{A}   \GRANULEsym{)} = \GRANULEsym{(}  \Gamma  \GRANULEsym{-}  \Gamma'  \GRANULEsym{)}  \GRANULEsym{,}   \GRANULEmv{x}  :_{\textcolor{coeffectColor}{  \GRANULEnt{q}  } }   \GRANULEnt{A}$

such that $\GRANULEnt{r} \sqsupseteq \GRANULEnt{q}  \GRANULEsym{+}  \GRANULEnt{r}$ and
$\maximal{q}{q'}{r}{q'+r}$.

Instantiating maximality with $q' = 0$ and reflexivity then we have
$0 \sqsubseteq \GRANULEnt{q}$. From this, and the inductive hypothesis, we can construct:
%
\begin{align*}
\dfrac{ \emptyset \sqsubseteq \GRANULEsym{(}  \Gamma  \GRANULEsym{-}  \Gamma'  \GRANULEsym{)} \quad 0 \sqsubseteq \GRANULEnt{q}}
{ \emptyset \sqsubseteq \GRANULEsym{(}  \Gamma  \GRANULEsym{-}  \Gamma'  \GRANULEsym{)}  \GRANULEsym{,}   \GRANULEmv{x}  :_{\textcolor{coeffectColor}{  \GRANULEnt{q}  } }   \GRANULEnt{A} }
\end{align*}
%
\end{itemize}
%
\end{proof}

\begin{lemma}
\label{lemma:contexts-subsub}
For all contexts $\Gamma_{{\mathrm{1}}}$, $\Gamma_{{\mathrm{2}}}$, where
$\GRANULEsym{[}  \Gamma_{{\mathrm{2}}}  \GRANULEsym{]}$ (i.e., $\Gamma_{{\mathrm{2}}}$ is all graded)
then:
%
\begin{align*}
\Gamma_{{\mathrm{2}}} \sqsubseteq \Gamma_{{\mathrm{1}}}  \GRANULEsym{-}  \GRANULEsym{(}  \Gamma_{{\mathrm{1}}}  \GRANULEsym{-}  \Gamma_{{\mathrm{2}}}  \GRANULEsym{)}
\end{align*}
\end{lemma}

\begin{proof}
By induction on the structure of $\Gamma_{{\mathrm{2}}}$.

\begin{itemize}
\item $\Gamma_{{\mathrm{2}}} = \sqsubseteq$

Then $\Gamma_{{\mathrm{1}}}  \GRANULEsym{-}  \GRANULEsym{(}  \Gamma_{{\mathrm{1}}}  \GRANULEsym{-}   \emptyset   \GRANULEsym{)} = \Gamma_{{\mathrm{1}}}  \GRANULEsym{-}  \Gamma_{{\mathrm{1}}}$.

By Lemma~\ref{lemma:context-neg}, then $ \emptyset \sqsubseteq \GRANULEsym{(}  \Gamma_{{\mathrm{1}}}  \GRANULEsym{-}  \Gamma_{{\mathrm{1}}}  \GRANULEsym{)}$
satisfying this case.

\item $\Gamma_{{\mathrm{2}}} = \Gamma'_{{\mathrm{2}}}  \GRANULEsym{,}   \GRANULEmv{x}  :_{\textcolor{coeffectColor}{  \GRANULEnt{s}  } }   \GRANULEnt{A}$

By the premises $\Gamma_{{\mathrm{1}}} \sqsubseteq \Gamma_{{\mathrm{2}}}$ then we can
assume $\GRANULEmv{x} \in \Gamma_{{\mathrm{1}}}$ and thus (by context
rearrangement) $\Gamma'_{{\mathrm{1}}}  \GRANULEsym{,}   \GRANULEmv{x}  :_{\textcolor{coeffectColor}{  \GRANULEnt{r}  } }   \GRANULEnt{A}$.

Thus we consider $\GRANULEsym{(}  \Gamma'_{{\mathrm{1}}}  \GRANULEsym{,}   \GRANULEmv{x}  :_{\textcolor{coeffectColor}{  \GRANULEnt{r}  } }   \GRANULEnt{A}   \GRANULEsym{)}  \GRANULEsym{-}  \GRANULEsym{(}  \GRANULEsym{(}  \Gamma'_{{\mathrm{1}}}  \GRANULEsym{,}   \GRANULEmv{x}  :_{\textcolor{coeffectColor}{  \GRANULEnt{r}  } }   \GRANULEnt{A}   \GRANULEsym{)}  \GRANULEsym{-}  \GRANULEsym{(}  \Gamma'_{{\mathrm{2}}}  \GRANULEsym{,}   \GRANULEmv{x}  :_{\textcolor{coeffectColor}{  \GRANULEnt{s}  } }   \GRANULEnt{A}   \GRANULEsym{)}  \GRANULEsym{)}$.

\begin{align*}
  \; & \GRANULEsym{(}  \Gamma'_{{\mathrm{1}}}  \GRANULEsym{,}   \GRANULEmv{x}  :_{\textcolor{coeffectColor}{  \GRANULEnt{r}  } }   \GRANULEnt{A}   \GRANULEsym{)}  \GRANULEsym{-}  \GRANULEsym{(}  \GRANULEsym{(}  \Gamma'_{{\mathrm{1}}}  \GRANULEsym{,}   \GRANULEmv{x}  :_{\textcolor{coeffectColor}{  \GRANULEnt{r}  } }   \GRANULEnt{A}   \GRANULEsym{)}  \GRANULEsym{-}  \GRANULEsym{(}  \Gamma'_{{\mathrm{2}}}  \GRANULEsym{,}   \GRANULEmv{x}  :_{\textcolor{coeffectColor}{  \GRANULEnt{s}  } }   \GRANULEnt{A}   \GRANULEsym{)}  \GRANULEsym{)} \\
= \; & \GRANULEsym{(}  \Gamma'_{{\mathrm{1}}}  \GRANULEsym{,}   \GRANULEmv{x}  :_{\textcolor{coeffectColor}{  \GRANULEnt{r}  } }   \GRANULEnt{A}   \GRANULEsym{)}  \GRANULEsym{-}  \GRANULEsym{(}  \GRANULEsym{(}  \Gamma'_{{\mathrm{1}}}  \GRANULEsym{-}  \Gamma'_{{\mathrm{2}}}  \GRANULEsym{)}  \GRANULEsym{,}   \GRANULEmv{x}  :_{\textcolor{coeffectColor}{  \GRANULEnt{q}  } }   \GRANULEnt{A}   \GRANULEsym{)} \\
= \; & \GRANULEsym{(}  \Gamma'_{{\mathrm{1}}}  \GRANULEsym{-}  \GRANULEsym{(}  \Gamma'_{{\mathrm{1}}}  \GRANULEsym{-}  \Gamma'_{{\mathrm{2}}}  \GRANULEsym{)}  \GRANULEsym{)}  \GRANULEsym{,}   \GRANULEmv{x}  :_{\textcolor{coeffectColor}{  \GRANULEnt{q'}  } }   \GRANULEnt{A}
\end{align*}
%
where (1) $\exists  \GRANULEnt{q}  .\,   \GRANULEnt{r}  \sqsupseteq  \GRANULEnt{q}  \GRANULEsym{+}  \GRANULEnt{s}$ with
(2) $(\maximal{q}{\hat{q}}{r}{\hat{q}+s})$

and (3) $\exists  \GRANULEnt{q'}  .\,   \GRANULEnt{r}  \sqsupseteq  \GRANULEnt{q'}  \GRANULEsym{+}  \GRANULEnt{q}$
with (4) $(\maximal{q'}{\hat{q'}}{r}{\hat{q'}+s})$.

Apply (1) to (4) by letting $\hat{q'} = {s}$
and by commutativity of $+$ then we get that
$\GRANULEnt{q'}  \sqsupseteq  \GRANULEnt{s}$.

By induction we have that
%
\begin{align*}
\Gamma'_{{\mathrm{1}}} \sqsubseteq \Gamma'_{{\mathrm{1}}}  \GRANULEsym{-}  \GRANULEsym{(}  \Gamma'_{{\mathrm{1}}}  \GRANULEsym{-}  \Gamma'_{{\mathrm{2}}}  \GRANULEsym{)} \tag{ih}
\end{align*}
%
%
Thus we get that:
%
\begin{align*}
\dfrac{\GRANULEnt{s} \sqsubseteq \GRANULEnt{q'} \quad
\Gamma'_{{\mathrm{1}}} \sqsubseteq \Gamma'_{{\mathrm{1}}}  \GRANULEsym{-}  \GRANULEsym{(}  \Gamma'_{{\mathrm{1}}}  \GRANULEsym{-}  \Gamma'_{{\mathrm{2}}}  \GRANULEsym{)}}
{ \Gamma'_{{\mathrm{1}}}  \GRANULEsym{,}   \GRANULEmv{x}  :_{\textcolor{coeffectColor}{  \GRANULEnt{s}  } }   \GRANULEnt{A} \sqsubseteq \GRANULEsym{(}  \Gamma'_{{\mathrm{1}}}  \GRANULEsym{-}  \GRANULEsym{(}  \Gamma'_{{\mathrm{1}}}  \GRANULEsym{-}  \Gamma'_{{\mathrm{2}}}  \GRANULEsym{)}  \GRANULEsym{)}  \GRANULEsym{,}   \GRANULEmv{x}  :_{\textcolor{coeffectColor}{  \GRANULEnt{q'}  } }   \GRANULEnt{A} }
\end{align*}
%
\item $\Gamma_{{\mathrm{2}}} = \Gamma'_{{\mathrm{2}}}  \GRANULEsym{,}   \GRANULEmv{x}  :  \GRANULEnt{A}$ Trivial as it violates the grading
  condition of the premise.
\end{itemize}
\end{proof}

%%%%%%%%%%%%%%%%%%%%%%%%%%%%%%%%%%%%%%%%%%%%%
\subSynthSound*

\begin{proof}
Structural induction over the synthesis rules. Each of the possible synthesis
rules are considered in turn.

\begin{enumerate}[itemsep=1em]
  \item Case \subLinVarName \\
  In the case of linear variable synthesis, we have the derivation:
  \[
    \subLinVar{}
  \]
  %
  By the definition of context subtraction,
  $(\Gamma  \GRANULEsym{,}   \GRANULEmv{x}  :  \GRANULEnt{A}) - \Gamma = \GRANULEmv{x}  :  \GRANULEnt{A}$, thus we
  can construct the following typing derivation, matching the
  conclusion:
  \[
    \inferrule*[Right=\textsc{Var}]
    {\quad}
    {\GRANULEmv{x}  :  \GRANULEnt{A}   \vdash  \GRANULEmv{x}  :  \GRANULEnt{A}}
  \]

\item Case \subGrVarName{} \\
    Matching the form of the lemma, we have the derivation:
    \[
    \subGrVar{}
    \]
    %
    By the definition of context subtraction,
    $ \GRANULEsym{(}  \Gamma  \GRANULEsym{,}   \GRANULEmv{x}  :_{\textcolor{coeffectColor}{  \GRANULEnt{r}  } }   \GRANULEnt{A}   \GRANULEsym{)}  \GRANULEsym{-}  \GRANULEsym{(}  \Gamma  \GRANULEsym{,}   \GRANULEmv{x}  :_{\textcolor{coeffectColor}{  \GRANULEnt{s}  } }   \GRANULEnt{A}   \GRANULEsym{)} = \GRANULEmv{x}  :_{\textcolor{coeffectColor}{  \GRANULEnt{q}  } }   \GRANULEnt{A}$
    where (1) $\exists  \GRANULEnt{q}  .\,   \GRANULEnt{r}  \sqsupseteq  \GRANULEnt{q}  \GRANULEsym{+}  \GRANULEnt{s}$ and $\maximal{q}{q'}{r}{q' + s}$.

    Applying maximality (1) with $q = 1$ then we have that $1 \sqsubseteq \GRANULEnt{q}$ (*)

    Thus, from this we can construct the typing derivation, matching the conclusion:
    %
    \[
    \inferrule*[Right=\textsc{Approx}]
    {
      \inferrule*[Right=\textsc{Der}]
        {\inferrule*[Right=\textsc{Var}]
          {\quad}
          {\GRANULEmv{x}  :  \GRANULEnt{A}   \vdash  \GRANULEmv{x}  :  \GRANULEnt{A}}
        }
        {\GRANULEmv{x}  :_{\textcolor{coeffectColor}{   1   } }   \GRANULEnt{A}   \vdash  \GRANULEmv{x}  :  \GRANULEnt{A} \qquad 1 \sqsubseteq \GRANULEnt{q} \; (*)}
    }{
      \GRANULEmv{x}  :_{\textcolor{coeffectColor}{  \GRANULEnt{q}  } }   \GRANULEnt{A}   \vdash  \GRANULEmv{x}  :  \GRANULEnt{A}
    }
    \]

  \item Case \subAbsName \\
    We thus have the derivation:
    %
    \[
    \subAbs{}
    \]
    %
    By induction we then have that:
    %
    \[
      \GRANULEsym{(}  \Gamma  \GRANULEsym{,}   \GRANULEmv{x}  :  \GRANULEnt{A}   \GRANULEsym{)}  \GRANULEsym{-}  \Delta  \vdash  \GRANULEnt{t}  :  \GRANULEnt{B}
    \]
    %
    Since $\GRANULEmv{x}  \not\in | \Delta |$ then by the definition of context
    subtraction we have that $\GRANULEsym{(}  \Gamma  \GRANULEsym{,}   \GRANULEmv{x}  :  \GRANULEnt{A}   \GRANULEsym{)}  \GRANULEsym{-}  \Delta = \GRANULEsym{(}  \Gamma  \GRANULEsym{-}  \Delta  \GRANULEsym{)}  \GRANULEsym{,}   \GRANULEmv{x}  :  \GRANULEnt{A}$.
    From this, we can construct the following derivation, matching the
    conclusion:
    %
    \[
    \inferrule*[Right=Abs]
    {\GRANULEsym{(}  \Gamma  \GRANULEsym{-}  \Delta  \GRANULEsym{)}  \GRANULEsym{,}   \GRANULEmv{x}  :  \GRANULEnt{A}   \vdash  \GRANULEnt{t}  :  \GRANULEnt{B}}{\Gamma  \GRANULEsym{-}  \Delta  \vdash   \lambda  \GRANULEmv{x}  .  \GRANULEnt{t}   :   \GRANULEnt{A}  \multimap  \GRANULEnt{B}}
    \]


  \item Case \subAppName \\
    %
    Matching the form of the lemma, the application derivation is:
    \[
      \subApp{}
    \]
    %
    By induction, we have that:
    %
      \begin{align*}
        \GRANULEsym{(}  \Gamma  \GRANULEsym{,}   \GRANULEmv{x_{{\mathrm{2}}}}  :  \GRANULEnt{B}   \GRANULEsym{)}  \GRANULEsym{-}  \Delta_{{\mathrm{1}}}  \vdash  \GRANULEnt{t_{{\mathrm{1}}}}  :  \GRANULEnt{C} \tag{ih1} \\
        \Delta_{{\mathrm{1}}}  \GRANULEsym{-}  \Delta_{{\mathrm{2}}}  \vdash  \GRANULEnt{t_{{\mathrm{2}}}}  :  \GRANULEnt{A} \tag{ih2}
      \end{align*}
    %
    By the definition of context subtraction and since $\GRANULEmv{x_{{\mathrm{2}}}}  \not\in | \Delta_{{\mathrm{1}}} |$ then (ih1) is equal to:
    %
      \begin{align*}
        \GRANULEsym{(}  \Gamma  \GRANULEsym{-}  \Delta_{{\mathrm{1}}}  \GRANULEsym{)}  \GRANULEsym{,}   \GRANULEmv{x_{{\mathrm{2}}}}  :  \GRANULEnt{B}   \vdash  \GRANULEnt{t_{{\mathrm{1}}}}  :  \GRANULEnt{C} \tag{ih1'}
      \end{align*}
    %
    We can thus construct the following typing derivation, making use of
    of the admissibility of linear substitution
    (Lemma~\ref{lemma:substitution}):
    %
    {\scriptsize{
    \[
    \hspace{-8em}
    \inferrule*[Right=\textsc{app}]
    {\inferrule*[right=\textsc{abs}, leftskip=5em]
       {\GRANULEsym{(}  \Gamma  \GRANULEsym{-}  \Delta_{{\mathrm{1}}}  \GRANULEsym{)}  \GRANULEsym{,}   \GRANULEmv{x_{{\mathrm{2}}}}  :   \GRANULEnt{B}  \multimap  \GRANULEnt{C}    \vdash  \GRANULEnt{t_{{\mathrm{1}}}}  :  \GRANULEnt{C}}
       {\Gamma  \GRANULEsym{-}  \Delta_{{\mathrm{1}}}  \vdash   \lambda  \GRANULEmv{x_{{\mathrm{2}}}}  .  \GRANULEnt{t_{{\mathrm{1}}}}   :   \GRANULEnt{B}  \multimap  \GRANULEnt{C}} \\
    \inferrule*[right=\textsc{app}, rightskip=15em]
    {\inferrule*[right=\textsc{var}, leftskip=2em]
    {\quad}
    {\GRANULEmv{x_{{\mathrm{1}}}}  :   \GRANULEnt{A}  \multimap  \GRANULEnt{B}    \vdash  \GRANULEmv{x_{{\mathrm{1}}}}  :   \GRANULEnt{A}  \multimap  \GRANULEnt{B}} \\ {\Delta_{{\mathrm{1}}}  \GRANULEsym{-}  \Delta_{{\mathrm{2}}}  \vdash  \GRANULEnt{t_{{\mathrm{2}}}}  :  \GRANULEnt{A}}}  {\GRANULEsym{(}  \Delta_{{\mathrm{1}}}  \GRANULEsym{-}  \Delta_{{\mathrm{2}}}  \GRANULEsym{)}, \GRANULEmv{x_{{\mathrm{1}}}}  :   \GRANULEnt{A}  \multimap  \GRANULEnt{B}    \vdash  \GRANULEmv{x_{{\mathrm{1}}}} \, \GRANULEnt{t_{{\mathrm{2}}}}  :  \GRANULEnt{B}}}
    {\GRANULEsym{(}  \Gamma  \GRANULEsym{-}  \Delta_{{\mathrm{1}}}  \GRANULEsym{)}  \GRANULEsym{+}  \GRANULEsym{(}  \Delta_{{\mathrm{1}}}  \GRANULEsym{-}  \Delta_{{\mathrm{2}}}  \GRANULEsym{)} ,\GRANULEmv{x_{{\mathrm{1}}}}  :   \GRANULEnt{A}  \multimap  \GRANULEnt{B}    \vdash   [  \GRANULEsym{(}  \GRANULEmv{x_{{\mathrm{1}}}} \, \GRANULEnt{t_{{\mathrm{2}}}}  \GRANULEsym{)}  /  \GRANULEmv{x_{{\mathrm{2}}}}  ]  \GRANULEnt{t_{{\mathrm{1}}}}   :  \GRANULEnt{C}}
    \]
    }}

    From Lemma~\ref{lemma:contextLemma1}, we have that
    %
    \begin{align*}
      \GRANULEsym{(}  \GRANULEsym{(}  \Gamma  \GRANULEsym{-}  \Delta_{{\mathrm{1}}}  \GRANULEsym{)}  \GRANULEsym{+}  \GRANULEsym{(}  \Delta_{{\mathrm{1}}}  \GRANULEsym{-}  \Delta_{{\mathrm{2}}}  \GRANULEsym{)}  \GRANULEsym{)}  \GRANULEsym{,}   \GRANULEmv{x_{{\mathrm{1}}}}  :   \GRANULEnt{A}  \multimap  \GRANULEnt{B} \sqsubseteq \GRANULEsym{(}  \GRANULEsym{(}  \GRANULEsym{(}  \Gamma  \GRANULEsym{-}  \Delta_{{\mathrm{1}}}  \GRANULEsym{)}  \GRANULEsym{+}  \Delta_{{\mathrm{1}}}  \GRANULEsym{)}  \GRANULEsym{-}  \Delta_{{\mathrm{2}}}  \GRANULEsym{)}  \GRANULEsym{,}   \GRANULEmv{x_{{\mathrm{1}}}}  :   \GRANULEnt{A}  \multimap  \GRANULEnt{B}
    \end{align*}
    %
    and from Lemma~\ref{lemma:contextLemma2}, that:
    %
    \begin{align*}
      \GRANULEsym{(}  \GRANULEsym{(}  \GRANULEsym{(}  \Gamma  \GRANULEsym{-}  \Delta_{{\mathrm{1}}}  \GRANULEsym{)}  \GRANULEsym{+}  \Delta_{{\mathrm{1}}}  \GRANULEsym{)}  \GRANULEsym{-}  \Delta_{{\mathrm{2}}}  \GRANULEsym{)}  \GRANULEsym{,}   \GRANULEmv{x_{{\mathrm{1}}}}  :   \GRANULEnt{A}  \multimap  \GRANULEnt{B} \sqsubseteq \GRANULEsym{(}  \Gamma  \GRANULEsym{-}  \Delta_{{\mathrm{2}}}  \GRANULEsym{)}  \GRANULEsym{,}   \GRANULEmv{x_{{\mathrm{1}}}}  :   \GRANULEnt{A}  \multimap  \GRANULEnt{B}
    \end{align*}
    %
    %% JACK: this arguments needs strengthening, but it is true.
    %% easy to given a proof of output contexts being subsets of input
    which, since $\GRANULEmv{x_{{\mathrm{1}}}}$ is not in $\Delta_{{\mathrm{2}}}$ (as $\GRANULEmv{x_{{\mathrm{1}}}}$ is not
    in $\Gamma$)
    %%
    $\GRANULEsym{(}  \Gamma  \GRANULEsym{-}  \Delta_{{\mathrm{2}}}  \GRANULEsym{)}  \GRANULEsym{,}   \GRANULEmv{x_{{\mathrm{1}}}}  :   \GRANULEnt{A}  \multimap  \GRANULEnt{B} = \GRANULEsym{(}  \Gamma  \GRANULEsym{,}   \GRANULEmv{x_{{\mathrm{1}}}}  :   \GRANULEnt{A}  \multimap  \GRANULEnt{B}    \GRANULEsym{)}  \GRANULEsym{-}  \Delta_{{\mathrm{2}}}$. Applying
    these inequalities with \textsc{Approx} then yields the lemma's
    conclusion $\GRANULEsym{(}  \Gamma  \GRANULEsym{,}   \GRANULEmv{x_{{\mathrm{1}}}}  :   \GRANULEnt{A}  \multimap  \GRANULEnt{B}    \GRANULEsym{)}  \GRANULEsym{-}  \Delta_{{\mathrm{2}}}   \vdash   [  \GRANULEsym{(}  \GRANULEmv{x_{{\mathrm{1}}}} \, \GRANULEnt{t_{{\mathrm{2}}}}  \GRANULEsym{)}  /  \GRANULEmv{x_{{\mathrm{2}}}}  ]  \GRANULEnt{t_{{\mathrm{1}}}}   :  \GRANULEnt{C}$.

  \item Case \subBoxName{} \\
    %
    The synthesis rule for boxing can be constructed as:
    %
    \[
      \subBox{}
    \]
    %
    By induction on the premise we get:
    %
    \begin{align*}
      \Gamma  \GRANULEsym{-}  \Delta  \vdash  \GRANULEnt{t}  :  \GRANULEnt{A}
    \end{align*}
    %
    Since we apply scalar multipication ih the conclusion of the rule
    to $\Gamma  \GRANULEsym{-}  \Delta$ then we know that all of $\Gamma  \GRANULEsym{-}  \Delta$ must be
    graded assumptions.

    From this, we can construct the typing derivation:
    %
    \[
    \inferrule*[Right=\textsc{Pr}]
    {\GRANULEsym{[}   \Gamma  \GRANULEsym{-}  \Delta   \GRANULEsym{]}  \vdash  \GRANULEnt{t}  :  \GRANULEnt{A}}{\textcolor{coeffectColor}{ \GRANULEnt{r}   \textcolor{coeffectColor}{\,\cdot\,} }  \GRANULEsym{[}   \Gamma  \GRANULEsym{-}  \Delta   \GRANULEsym{]}   \vdash  \GRANULEsym{[}  \GRANULEnt{t}  \GRANULEsym{]}  :   \Box_{  \GRANULEnt{r}  }  \GRANULEnt{A}}
    \]
    Via Lemma~\ref{lemma:contexts-subsub}, we then have that
    $\GRANULEsym{(}   \textcolor{coeffectColor}{ \GRANULEnt{r}   \textcolor{coeffectColor}{\,\cdot\,} }   \Gamma  \GRANULEsym{-}  \Delta    \GRANULEsym{)} \sqsubseteq \GRANULEsym{(}  \Gamma  \GRANULEsym{-}  \GRANULEsym{(}  \Gamma  \GRANULEsym{-}  \GRANULEsym{(}   \textcolor{coeffectColor}{ \GRANULEnt{r}   \textcolor{coeffectColor}{\,\cdot\,} }  \GRANULEsym{(}  \Gamma  \GRANULEsym{-}  \Delta  \GRANULEsym{)}   \GRANULEsym{)}  \GRANULEsym{)}  \GRANULEsym{)}$ thus, we can
    derived:
    %
    \[
    \inferrule*[Right=\textsc{Approx}]
   {
    \inferrule*[Right=\textsc{Pr}]
    {\GRANULEsym{[}   \Gamma  \GRANULEsym{-}  \Delta   \GRANULEsym{]}  \vdash  \GRANULEnt{t}  :  \GRANULEnt{A}}{\textcolor{coeffectColor}{ \GRANULEnt{r}   \textcolor{coeffectColor}{\,\cdot\,} }  \GRANULEsym{[}   \Gamma  \GRANULEsym{-}  \Delta   \GRANULEsym{]}   \vdash  \GRANULEsym{[}  \GRANULEnt{t}  \GRANULEsym{]}  :   \Box_{  \GRANULEnt{r}  }  \GRANULEnt{A}
    \quad \text{Lem.~\ref{lemma:contexts-subsub}}}
   }{ \Gamma  \GRANULEsym{-}  \GRANULEsym{(}  \Gamma  \GRANULEsym{-}  \GRANULEsym{(}   \textcolor{coeffectColor}{ \GRANULEnt{r}   \textcolor{coeffectColor}{\,\cdot\,} }  \GRANULEsym{(}  \Gamma  \GRANULEsym{-}  \Delta  \GRANULEsym{)}   \GRANULEsym{)}  \GRANULEsym{)}  \vdash  \GRANULEsym{[}  \GRANULEnt{t}  \GRANULEsym{]}  :   \Box_{  \GRANULEnt{r}  }  \GRANULEnt{A}}
    \]
    %
    Satisfying the goal of the lemma.

  \item Case \subUnboxName \\
    The synthesis rule for unboxing has the form:
    \[
    \subUnbox{}
    \]
    %
    By induction on the premise we have that:
    %
      \begin{align*}
        \GRANULEsym{(}  \Gamma  \GRANULEsym{,}   \GRANULEmv{x_{{\mathrm{2}}}}  :_{\textcolor{coeffectColor}{  \GRANULEnt{r}  } }   \GRANULEnt{A}   \GRANULEsym{)}  \GRANULEsym{-}  \GRANULEsym{(}  \Delta  \GRANULEsym{,}   \GRANULEmv{x_{{\mathrm{2}}}}  :_{\textcolor{coeffectColor}{  \GRANULEnt{s}  } }   \GRANULEnt{A}   \GRANULEsym{)}  \vdash  \GRANULEnt{t}  :  \GRANULEnt{B}
      \end{align*}
   %
   By the definition of context subtraction we get that $\exists q$ and:
   %
     \begin{align*}
       \GRANULEsym{(}  \Gamma  \GRANULEsym{,}   \GRANULEmv{x_{{\mathrm{2}}}}  :_{\textcolor{coeffectColor}{  \GRANULEnt{r}  } }   \GRANULEnt{A}   \GRANULEsym{)}  \GRANULEsym{-}  \GRANULEsym{(}  \Delta  \GRANULEsym{,}   \GRANULEmv{x_{{\mathrm{2}}}}  :_{\textcolor{coeffectColor}{  \GRANULEnt{s}  } }   \GRANULEnt{A}   \GRANULEsym{)}
     = \GRANULEsym{(}  \Gamma  \GRANULEsym{-}  \Delta  \GRANULEsym{)}  \GRANULEsym{,}   \GRANULEmv{x_{{\mathrm{2}}}}  :_{\textcolor{coeffectColor}{  \GRANULEnt{q}  } }   \GRANULEnt{A}
       \end{align*}
   %
    such that $r = q + s$

    We also have that $0 \sqsubseteq \GRANULEnt{s}$.

    By monotonicity with $\GRANULEnt{q} \sqsubseteq \GRANULEnt{q}$ (reflexivity)
    and $0 \sqsubseteq \GRANULEnt{s}$ then $\GRANULEnt{q} \sqsubseteq \GRANULEnt{q}  \GRANULEsym{+}  \GRANULEnt{s}$.

    By context subtraction we have $r = q + s$ therefore
    $\GRANULEnt{q} \sqsubseteq \GRANULEnt{r}$ (*).

    From this, we can construct the typing derivation:
    %
    \[
    \inferrule*[Right=\textsc{Let}]
    {\inferrule*[right=\textsc{Var}]
                    {\quad}{\GRANULEmv{x_{{\mathrm{1}}}}  :   \Box_{  \GRANULEnt{r}  }  \GRANULEnt{A}    \vdash  \GRANULEmv{x_{{\mathrm{1}}}}  :   \Box_{  \GRANULEnt{r}  }  \GRANULEnt{A}}
           \\
      \inferrule*[right=\textsc{approx}]
         {\GRANULEsym{(}  \Gamma  \GRANULEsym{-}  \Delta  \GRANULEsym{)}  \GRANULEsym{,}   \GRANULEmv{x_{{\mathrm{2}}}}  :_{\textcolor{coeffectColor}{  \GRANULEnt{q}  } }   \GRANULEnt{A}   \vdash  \GRANULEnt{t}  :  \GRANULEnt{B} \quad (*)}
         {\GRANULEsym{(}  \Gamma  \GRANULEsym{-}  \Delta  \GRANULEsym{)}  \GRANULEsym{,}   \GRANULEmv{x_{{\mathrm{2}}}}  :_{\textcolor{coeffectColor}{  \GRANULEnt{r}  } }   \GRANULEnt{A}   \vdash  \GRANULEnt{t}  :  \GRANULEnt{B}}}
     {\GRANULEsym{(}  \Gamma  \GRANULEsym{-}  \Delta  \GRANULEsym{)}  \GRANULEsym{,}   \GRANULEmv{x_{{\mathrm{1}}}}  :   \Box_{  \GRANULEnt{r}  }  \GRANULEnt{A}    \vdash   \textbf{let} \, [  \GRANULEmv{x_{{\mathrm{2}}}}  ] =  \GRANULEmv{x_{{\mathrm{1}}}}  \, \textbf{in} \,  \GRANULEnt{t}   :  \GRANULEnt{B}}
    \]
    %
    Which matches the goal.

  \item Case \subPairIntroName \\

    The synthesis rule for pair introduction has the form:
    %
    \[
      \subPairIntro{}
    \]
    %
    By induction we get:
    %
    \begin{align*}
      \Gamma  \GRANULEsym{-}  \Delta_{{\mathrm{1}}}  \vdash  \GRANULEnt{t_{{\mathrm{1}}}}  :  \GRANULEnt{A} \tag{ih1} \\
      \Delta_{{\mathrm{1}}}  \GRANULEsym{-}  \Delta_{{\mathrm{2}}}  \vdash  \GRANULEnt{t_{{\mathrm{2}}}}  :  \GRANULEnt{B} \tag{ih2}
    \end{align*}
    %
    From this, we can construct the typing derivation:
    %
    \[
    \inferrule*[Right=\textsc{Pair}]
    {\Gamma  \GRANULEsym{-}  \Delta_{{\mathrm{1}}}  \vdash  \GRANULEnt{t_{{\mathrm{1}}}}  :  \GRANULEnt{A} \\ \Delta_{{\mathrm{1}}}  \GRANULEsym{-}  \Delta_{{\mathrm{2}}}  \vdash  \GRANULEnt{t_{{\mathrm{2}}}}  :  \GRANULEnt{B}}
    {\GRANULEsym{(}  \Gamma  \GRANULEsym{-}  \Delta_{{\mathrm{1}}}  \GRANULEsym{)}  \GRANULEsym{+}  \GRANULEsym{(}  \Delta_{{\mathrm{1}}}  \GRANULEsym{-}  \Delta_{{\mathrm{2}}}  \GRANULEsym{)}  \vdash   ( \GRANULEnt{t_{{\mathrm{1}}}} ,  \GRANULEnt{t_{{\mathrm{2}}}} )   :   \GRANULEnt{A}  \, \otimes \,  \GRANULEnt{B}}
    \]
    %
    From Lemma~\ref{lemma:contextLemma1}, we have that:
    \begin{align*}
      \GRANULEsym{(}  \Gamma  \GRANULEsym{-}  \Delta_{{\mathrm{1}}}  \GRANULEsym{)}  \GRANULEsym{+}  \GRANULEsym{(}  \Delta_{{\mathrm{1}}}  \GRANULEsym{-}  \Delta_{{\mathrm{2}}}  \GRANULEsym{)} \sqsubseteq \GRANULEsym{(}  \GRANULEsym{(}  \Gamma  \GRANULEsym{-}  \Delta_{{\mathrm{1}}}  \GRANULEsym{)}  \GRANULEsym{+}  \Delta_{{\mathrm{1}}}  \GRANULEsym{)}  \GRANULEsym{-}  \Delta_{{\mathrm{2}}}
    \end{align*}
    %
    and from Lemma~\ref{lemma:contextLemma2}, that:
    %
    \begin{align*}
      \GRANULEsym{(}  \GRANULEsym{(}  \Gamma  \GRANULEsym{-}  \Delta_{{\mathrm{1}}}  \GRANULEsym{)}  \GRANULEsym{+}  \Delta_{{\mathrm{1}}}  \GRANULEsym{)}  \GRANULEsym{-}  \Delta_{{\mathrm{2}}} \sqsubseteq \Gamma  \GRANULEsym{-}  \Delta_{{\mathrm{2}}}
    \end{align*}
    %
    From which we then apply \textsc{Approx} to the
    above derivation,
    yielding the goal $\Gamma  \GRANULEsym{-}  \Delta_{{\mathrm{2}}}  \vdash   ( \GRANULEnt{t_{{\mathrm{1}}}} ,  \GRANULEnt{t_{{\mathrm{2}}}} )   :   \GRANULEnt{A}  \, \otimes \,  \GRANULEnt{B}$.

  \item Case \subPairElimName \\
    The synthesis rule for pair elimination has the form:
    \[
      \subPairElim
    \]
    %
    By induction we get:
    %
    \begin{align*}
      \GRANULEsym{(}  \Gamma  \GRANULEsym{,}   \GRANULEmv{x_{{\mathrm{1}}}}  :  \GRANULEnt{A}   \GRANULEsym{,}   \GRANULEmv{x_{{\mathrm{2}}}}  :  \GRANULEnt{B}   \GRANULEsym{)}  \GRANULEsym{-}  \Delta  \vdash  \GRANULEnt{t_{{\mathrm{2}}}}  :  \GRANULEnt{C}
    \end{align*}
    %
    since $\GRANULEmv{x_{{\mathrm{1}}}}  \not\in | \Delta | \wedge \GRANULEmv{x_{{\mathrm{2}}}}  \not\in | \Delta |$ then
    $\GRANULEsym{(}  \Gamma  \GRANULEsym{,}   \GRANULEmv{x_{{\mathrm{1}}}}  :  \GRANULEnt{A}   \GRANULEsym{,}   \GRANULEmv{x_{{\mathrm{2}}}}  :  \GRANULEnt{B}   \GRANULEsym{)}  \GRANULEsym{-}  \Delta = \GRANULEsym{(}  \Gamma  \GRANULEsym{-}  \Delta  \GRANULEsym{)}  \GRANULEsym{,}   \GRANULEmv{x_{{\mathrm{1}}}}  :  \GRANULEnt{A}   \GRANULEsym{,}   \GRANULEmv{x_{{\mathrm{2}}}}  :  \GRANULEnt{B}$.

    From this, we can construct the following typing derivation,
    matching the conclusion:
    \[
    \inferrule*[Right=\textsc{Case}]
    {\inferrule*[right=\textsc{Var}] {\quad} {\GRANULEmv{x_{{\mathrm{3}}}}  :   \GRANULEnt{A}  \, \otimes \,  \GRANULEnt{B}    \vdash  \GRANULEmv{x_{{\mathrm{3}}}}  :   \GRANULEnt{A}  \, \otimes \,  \GRANULEnt{B}} \\ \GRANULEsym{(}  \Gamma  \GRANULEsym{-}  \Delta  \GRANULEsym{)}  \GRANULEsym{,}   \GRANULEmv{x_{{\mathrm{1}}}}  :  \GRANULEnt{A}   \GRANULEsym{,}   \GRANULEmv{x_{{\mathrm{2}}}}  :  \GRANULEnt{B}   \vdash  \GRANULEnt{t_{{\mathrm{2}}}}  :  \GRANULEnt{C}}
    {\GRANULEsym{(}  \Gamma  \GRANULEsym{-}  \Delta  \GRANULEsym{)}  \GRANULEsym{,}   \GRANULEmv{x_{{\mathrm{3}}}}  :   \GRANULEnt{A}  \, \otimes \,  \GRANULEnt{B}    \vdash   \textbf{let} \, ( \GRANULEmv{x_{{\mathrm{1}}}} ,  \GRANULEmv{x_{{\mathrm{2}}}} ) =  \GRANULEmv{x_{{\mathrm{3}}}}  \, \textbf{in} \,  \GRANULEnt{t_{{\mathrm{2}}}}   :  \GRANULEnt{C}}
    \]
    % JACK: tighten from context subset
    which matches the conclusion since $\GRANULEsym{(}  \Gamma  \GRANULEsym{-}  \Delta  \GRANULEsym{)}  \GRANULEsym{,}   \GRANULEmv{x_{{\mathrm{3}}}}  :   \GRANULEnt{A}  \, \otimes \,  \GRANULEnt{B} = \GRANULEsym{(}  \Gamma  \GRANULEsym{,}   \GRANULEmv{x_{{\mathrm{3}}}}  :   \GRANULEnt{A}  \, \otimes \,  \GRANULEnt{B}    \GRANULEsym{)}  \GRANULEsym{-}  \Delta$ since $\GRANULEmv{x_{{\mathrm{3}}}}  \not\in | \Delta |$ by its
    disjointness from $\Gamma$.

  \item Case \subSumIntroLname and \subSumIntroRname \\
    The synthesis rules for sum introduction are straightforward. For
     \subSumIntroLname we have the rule:
    \[
       \subSumIntroL
    \]
    By induction we have:
    %
      \begin{align*}
        \Gamma  \GRANULEsym{-}  \Delta  \vdash  \GRANULEnt{t}  :  \GRANULEnt{A} \tag{ih1}
       \end{align*}
    %
    from which we can construct the typing derivation, matching the
    conclusion:
    \[
    \inferrule*[Right=\subSumIntroLname]
    {\Gamma  \GRANULEsym{-}  \Delta  \vdash  \GRANULEnt{t}  :  \GRANULEnt{A}}
    {\Gamma  \GRANULEsym{-}  \Delta  \vdash  \GRANULEkw{inl} \, \GRANULEnt{t}  :   \GRANULEnt{A}  \, \oplus \,  \GRANULEnt{B}}
    \]
    Matching the goal. And likewise for \subSumIntroRname.

  \item Case \subSumElimName
      The synthesis rule for sum elimination has the form:
      \[
        \subSumElim
      \]
      %
      By induction:
        \begin{align*}
          \GRANULEsym{(}  \Gamma  \GRANULEsym{,}   \GRANULEmv{x_{{\mathrm{2}}}}  :  \GRANULEnt{A}   \GRANULEsym{)}  \GRANULEsym{-}  \Delta_{{\mathrm{1}}}  \vdash  \GRANULEnt{t_{{\mathrm{1}}}}  :  \GRANULEnt{C} \tag{ih}
\\        \GRANULEsym{(}  \Gamma  \GRANULEsym{,}   \GRANULEmv{x_{{\mathrm{3}}}}  :  \GRANULEnt{B}   \GRANULEsym{)}  \GRANULEsym{-}  \Delta_{{\mathrm{2}}}  \vdash  \GRANULEnt{t_{{\mathrm{2}}}}  :  \GRANULEnt{C} \tag{ih}
        \end{align*}
      %
      From this we can construct the typing derivation, matching the conclusion:
      \[
      \inferrule*[Right=Case]
      {\inferrule*[right=Var,leftskip=10em]{\quad}{\GRANULEmv{x_{{\mathrm{1}}}}  :   \GRANULEnt{A}  \, \oplus \,  \GRANULEnt{B}    \vdash  \GRANULEnt{t_{{\mathrm{1}}}}  :   \GRANULEnt{A}  \, \oplus \,  \GRANULEnt{B}} \\ \GRANULEsym{(}  \Gamma  \GRANULEsym{-}  \Delta_{{\mathrm{1}}}  \GRANULEsym{)}  \GRANULEsym{,}   \GRANULEmv{x_{{\mathrm{2}}}}  :  \GRANULEnt{A}   \vdash  \GRANULEnt{t_{{\mathrm{2}}}}  :  \GRANULEnt{C} \\ \GRANULEsym{(}  \Gamma  \GRANULEsym{-}  \Delta_{{\mathrm{2}}}  \GRANULEsym{)}  \GRANULEsym{,}   \GRANULEmv{x_{{\mathrm{3}}}}  :  \GRANULEnt{B}   \vdash  \GRANULEnt{t_{{\mathrm{3}}}}  :  \GRANULEnt{C}}{\GRANULEsym{(}  \Gamma  \GRANULEsym{,}   \GRANULEmv{x_{{\mathrm{1}}}}  :   \GRANULEnt{A}  \, \oplus \,  \GRANULEnt{B}    \GRANULEsym{)} - (\Delta_{{\mathrm{1}}} \sqcap \Delta_{{\mathrm{2}}}) \vdash  \textbf{case} \ x_{1}\ \textbf{of}\ \textbf{inl}\ x_{2} \rightarrow t_{1};\ \textbf{inr}\ x_{3} \rightarrow t_{2} : C}
      \]

      \item Case \subUnitIntroName

          \begin{align*}
            \subUnitIntro{}
           \end{align*}
         %
         By Lemma~\ref{lemma:context-neg} we have that $\emptyset \sqsubseteq \Gamma  \GRANULEsym{-}  \Gamma$
         then we have:
         %
           \begin{align*}
             \inferrule*[right = \textsc{Approx}]
             {\inferrule*[right = 1]
             {\quad}{\emptyset   \vdash  \GRANULEsym{()}  :   \mathsf{1}}}
             {\Gamma  \GRANULEsym{-}  \Gamma  \vdash  \GRANULEsym{()}  :   \mathsf{1}}
           \end{align*}
         %
         Matching the goal

     \item Case \subUnitElimName
          \begin{align*}
            \subUnitElim{}
           \end{align*}
         %
         By induction we have:
           \begin{align*}
             \Gamma  \GRANULEsym{-}  \Delta  \vdash  \GRANULEnt{t}  :  \GRANULEnt{C} \tag{ih}
            \end{align*}
         %
         Then we make the derivation:
         %
           \begin{align*}
             \inferrule*[right = Let$1$]
             {\inferrule*[right = Var]{\quad}{\GRANULEmv{x}  :   \mathsf{1}    \vdash  \GRANULEmv{x}  :   \mathsf{1}}
             \\ \Gamma  \GRANULEsym{-}  \Delta  \vdash  \GRANULEnt{t}  :  \GRANULEnt{C}}
             {\GRANULEsym{(}  \Gamma  \GRANULEsym{-}  \Delta  \GRANULEsym{)}  \GRANULEsym{,}   \GRANULEmv{x}  :   \mathsf{1}    \vdash  \GRANULEkw{let} \, \GRANULEsym{()}  \GRANULEsym{=}  \GRANULEmv{x} \, \GRANULEkw{in} \, \GRANULEnt{t}  :  \GRANULEnt{C}}
           \end{align*}
         where the context is equal to $\GRANULEsym{(}  \Gamma  \GRANULEsym{,}   \GRANULEmv{x}  :   \mathsf{1}    \GRANULEsym{)}  \GRANULEsym{-}  \Delta$.

  \item Case \subDerName

      \begin{align*}
        \subDer
      \end{align*}
     %
     By induction:
     %
       \begin{align*}
         \GRANULEsym{(}  \Gamma  \GRANULEsym{,}   \GRANULEmv{x}  :_{\textcolor{coeffectColor}{  \GRANULEnt{s}  } }   \GRANULEnt{A}   \GRANULEsym{,}   \GRANULEmv{y}  :  \GRANULEnt{A}   \GRANULEsym{)}  \GRANULEsym{-}  \GRANULEsym{(}  \Delta  \GRANULEsym{,}   \GRANULEmv{x}  :_{\textcolor{coeffectColor}{  \GRANULEnt{s'}  } }   \GRANULEnt{A}   \GRANULEsym{)}  \vdash  \GRANULEnt{t}  :  \GRANULEnt{B} \tag{ih}
       \end{align*}
     %
     By the definition of context subtraction we have (since also $\GRANULEmv{y}  \not\in | \Delta |$)
       \begin{align*}
     &  \GRANULEsym{(}  \Gamma  \GRANULEsym{,}   \GRANULEmv{x}  :_{\textcolor{coeffectColor}{  \GRANULEnt{s}  } }   \GRANULEnt{A}   \GRANULEsym{,}   \GRANULEmv{y}  :  \GRANULEnt{A}   \GRANULEsym{)}  \GRANULEsym{-}  \GRANULEsym{(}  \Delta  \GRANULEsym{,}   \GRANULEmv{x}  :_{\textcolor{coeffectColor}{  \GRANULEnt{s'}  } }   \GRANULEnt{A}   \GRANULEsym{)} \\
   =\ &  \GRANULEsym{(}  \Gamma  \GRANULEsym{-}  \Delta  \GRANULEsym{)}  \GRANULEsym{,}   \GRANULEmv{x}  :_{\textcolor{coeffectColor}{  \GRANULEnt{q}  } }   \GRANULEnt{A}   \GRANULEsym{,}   \GRANULEmv{y}  :  \GRANULEnt{A}
       \end{align*}
     where $\exists  \GRANULEnt{q}  .\,   \GRANULEnt{s}  \sqsupseteq  \GRANULEnt{q}  \GRANULEsym{+}  \GRANULEnt{s'}$ (1) and
     $\maximal{q}{\hat{q}}{s}{\hat{q} + s'}$ (2)

       The goal context is computed by:
       %
       \begin{align*}
      & \GRANULEsym{(}  \Gamma  \GRANULEsym{,}   \GRANULEmv{x}  :_{\textcolor{coeffectColor}{  \GRANULEnt{r}  } }   \GRANULEnt{A}   \GRANULEsym{)}  \GRANULEsym{-}  \GRANULEsym{(}  \Delta  \GRANULEsym{,}   \GRANULEmv{x}  :_{\textcolor{coeffectColor}{  \GRANULEnt{s'}  } }   \GRANULEnt{A}   \GRANULEsym{)} \\
    =\ & \GRANULEsym{(}  \Gamma  \GRANULEsym{-}  \Delta  \GRANULEsym{)}  \GRANULEsym{,}   \GRANULEmv{x}  :_{\textcolor{coeffectColor}{  \GRANULEnt{q'}  } }   \GRANULEnt{A}
       \end{align*}
       where $\GRANULEnt{r}  \sqsupseteq   \GRANULEnt{q'}  \GRANULEsym{+}  \GRANULEnt{s'}$ (3)
       and $\maximal{q'}{\hat{q'}}{r}{\hat{q'} + s'}$ (4)

       From the premise of \subDerName we have
       $\GRANULEnt{r}  \sqsupseteq  \GRANULEsym{(}  \GRANULEnt{s}  \GRANULEsym{+}   1   \GRANULEsym{)}$.

       \begin{align*}
      \begin{array}{rll}
        \text{congruence of + and (1)} & \implies \GRANULEnt{s}  \GRANULEsym{+}   1    \sqsupseteq    \GRANULEnt{q}  \GRANULEsym{+}  \GRANULEnt{s'}   \GRANULEsym{+}   1 & (5) \\
        \text{transitivity with \subDerName premise and (5)} & \implies
                 \GRANULEnt{r}  \sqsupseteq    \GRANULEnt{q}  \GRANULEsym{+}  \GRANULEnt{s'}   \GRANULEsym{+}   1 & (6) \\
        \text{+ assoc./comm. on (6)} & \implies \GRANULEnt{r}  \sqsupseteq    \GRANULEnt{q}  \GRANULEsym{+}   1    \GRANULEsym{+}  \GRANULEnt{s'} & (7) \\
       \text{apply (8) to (4) with $\hat{q'} = q + 1$} & \implies
                                                    \GRANULEnt{q'}  \sqsupseteq  \GRANULEnt{q}  \GRANULEsym{+}   1
                                                     & (8)
      \end{array}
       \end{align*}
       %
       Using this last result we derive:
       \begin{align*}
         \inferrule*[right = approx]
        {
         \inferrule*[right = contraction]
         {
         \inferrule*[right = Der]
         {\GRANULEsym{(}  \Gamma  \GRANULEsym{-}  \Delta  \GRANULEsym{)}  \GRANULEsym{,}   \GRANULEmv{x}  :_{\textcolor{coeffectColor}{  \GRANULEnt{q}  } }   \GRANULEnt{A}   \GRANULEsym{,}   \GRANULEmv{y}  :  \GRANULEnt{A}   \vdash  \GRANULEnt{t}  :  \GRANULEnt{B}}
         {\GRANULEsym{(}  \Gamma  \GRANULEsym{-}  \Delta  \GRANULEsym{)}  \GRANULEsym{,}   \GRANULEmv{x}  :_{\textcolor{coeffectColor}{  \GRANULEnt{q}  } }   \GRANULEnt{A}   \GRANULEsym{,}   \GRANULEmv{y}  :_{\textcolor{coeffectColor}{   1   } }   \GRANULEnt{A}   \vdash  \GRANULEnt{t}  :  \GRANULEnt{B}}
         }
         {\GRANULEsym{(}  \Gamma  \GRANULEsym{-}  \Delta  \GRANULEsym{)}  \GRANULEsym{,}   \GRANULEmv{x}  :_{\textcolor{coeffectColor}{   \GRANULEnt{q}  \GRANULEsym{+}   1    } }   \GRANULEnt{A}   \vdash   [  \GRANULEmv{x}  /  \GRANULEmv{y}  ]  \GRANULEnt{t}   :  \GRANULEnt{B}}
         \quad (8)}
        {\GRANULEsym{(}  \Gamma  \GRANULEsym{-}  \Delta  \GRANULEsym{)}  \GRANULEsym{,}   \GRANULEmv{x}  :_{\textcolor{coeffectColor}{  \GRANULEnt{q'}  } }   \GRANULEnt{A}   \vdash   [  \GRANULEmv{x}  /  \GRANULEmv{y}  ]  \GRANULEnt{t}   :  \GRANULEnt{B}}
         \end{align*}
       Which matches the goal.

\end{enumerate}
\end{proof}

\addSynthSound*
\begin{proof}

  \begin{enumerate}
    \item Case \addLinVarName \\
  In the case of linear variable synthesis, we have the derivation:
  \[
    \addLinVar
  \]
  Therefore we can construct the following typing derivation, matching the conclusion:
  \[
    \inferrule*[Right=\textsc{var}]
    {\quad}
    {\GRANULEmv{x}  :  \GRANULEnt{A}   \vdash  \GRANULEmv{x}  :  \GRANULEnt{A}}
  \]
    \item Case \addGrVarName\\
    Matching the form of the lemma, we have the derivation:
    \[
      \addGrVar
    \]
    From this we can construct the typing derivation, matching the conclusion:
    \[
      \inferrule*[Right=\textsc{Der}]
        {\inferrule*[Right=\textsc{Var}]
          {\quad}
          {\GRANULEmv{x}  :  \GRANULEnt{A}   \vdash  \GRANULEmv{x}  :  \GRANULEnt{A}}
        }
        {\GRANULEmv{x}  :_{\textcolor{coeffectColor}{   1   } }   \GRANULEnt{A}   \vdash  \GRANULEmv{x}  :  \GRANULEnt{A}}
    \]
    \item Case \addAbsName\\
    We thus have the derivation:
    \[
      \addAbs
    \]
    %
    By induction on the premise we then have:
    \[
      \Delta  \GRANULEsym{,}   \GRANULEmv{x}  :  \GRANULEnt{A}   \vdash  \GRANULEnt{t}  :  \GRANULEnt{B}
    \]
    From this, we can construct the typing derivation, matching the conclusion:
    \[
    \inferrule*[Right=\textsc{abs}]
    {\Delta  \GRANULEsym{,}   \GRANULEmv{x}  :  \GRANULEnt{A}   \vdash  \GRANULEnt{t}  :  \GRANULEnt{B}}{\Delta  \vdash   \lambda  \GRANULEmv{x}  .  \GRANULEnt{t}   :   \GRANULEnt{A}  \multimap  \GRANULEnt{B}}
    \]
    \item Case \addAppName\\
    Matching the form of the lemma, the application derivation can be
    constructed as:
    \[
      \addApp
    \]
    %
    By induction on the premises we then have the following typing
    judgments:
    %
    \begin{align*}
      \Delta_{{\mathrm{1}}}  \GRANULEsym{,}   \GRANULEmv{x_{{\mathrm{2}}}}  :  \GRANULEnt{B}   \vdash  \GRANULEnt{t_{{\mathrm{1}}}}  :  \GRANULEnt{C} \\
      \Delta_{{\mathrm{2}}}  \vdash  \GRANULEnt{t_{{\mathrm{2}}}}  :  \GRANULEnt{A}
    \end{align*}
    %
    We can thus construct the following typing derivation, making use
    of the admissibility of linear substitution
    (Lemma~\ref{lemma:substitution}):
    \[
    \inferrule*[Right=(L.~\ref{lemma:substitution})]
    {\inferrule*[right=\textsc{app}, leftskip=2em]
    {\inferrule*[right=\textsc{var}]
    {\quad}
    {\GRANULEmv{x_{{\mathrm{1}}}}  :   \GRANULEnt{A}  \multimap  \GRANULEnt{B}    \vdash  \GRANULEmv{x_{{\mathrm{1}}}}  :   \GRANULEnt{A}  \multimap  \GRANULEnt{B}} \\ {\Delta_{{\mathrm{2}}}  \vdash  \GRANULEnt{t_{{\mathrm{2}}}}  :  \GRANULEnt{A}}}
  {\Delta_{{\mathrm{2}}}  \GRANULEsym{,}   \GRANULEmv{x_{{\mathrm{1}}}}  :   \GRANULEnt{A}  \multimap  \GRANULEnt{B}    \vdash  \GRANULEmv{x_{{\mathrm{1}}}} \, \GRANULEnt{t_{{\mathrm{2}}}}  :  \GRANULEnt{B}}
    \\ \Delta_{{\mathrm{1}}}  \GRANULEsym{,}   \GRANULEmv{x_{{\mathrm{2}}}}  :  \GRANULEnt{B}   \vdash  \GRANULEnt{t_{{\mathrm{1}}}}  :  \GRANULEnt{C}}
    {\GRANULEsym{(}  \Delta_{{\mathrm{1}}}  \GRANULEsym{+}  \Delta_{{\mathrm{2}}}  \GRANULEsym{)}  \GRANULEsym{,}   \GRANULEmv{x_{{\mathrm{1}}}}  :   \GRANULEnt{A}  \multimap  \GRANULEnt{B}    \vdash   [  \GRANULEsym{(}  \GRANULEmv{x_{{\mathrm{1}}}} \, \GRANULEnt{t_{{\mathrm{2}}}}  \GRANULEsym{)}  /  \GRANULEmv{x_{{\mathrm{2}}}}  ]  \GRANULEnt{t_{{\mathrm{1}}}}   :  \GRANULEnt{C}}
    \]
    \item Case \addBoxName\\
    The synthesis rule for boxing can be constructed as:
    \[
      \addBox
    \]
    By induction we then have:
    %
    \[
      \Delta  \vdash  \GRANULEnt{t}  :  \GRANULEnt{A}
    \]
    %
    In the conclusion of the above derivation we know that $\textcolor{coeffectColor}{ \GRANULEnt{r}   \textcolor{coeffectColor}{\,\cdot\,} }  \Delta$ is defined, therefore it must be that all of $\Delta$ are
    graded assumptions, i.e., we have that $\GRANULEsym{[}  \Delta  \GRANULEsym{]}$ holds.
    We can thus construct the following typing derivation, matching the conclusion:
    \[
    \inferrule*[Right=\textsc{Pr}]
    {\GRANULEsym{[}  \Delta  \GRANULEsym{]}  \vdash  \GRANULEnt{t}  :  \GRANULEnt{A}}{\textcolor{coeffectColor}{ \GRANULEnt{r}   \textcolor{coeffectColor}{\,\cdot\,} }  \GRANULEsym{[}  \Delta  \GRANULEsym{]}   \vdash  \GRANULEsym{[}  \GRANULEnt{t}  \GRANULEsym{]}  :   \Box_{  \GRANULEnt{r}  }  \GRANULEnt{A}}
    \]
    \item Case \addDerName\\
    From the dereliction rule we have:
    \[
      \addDer
    \]
    %
    By induction we get:
    %
      \begin{align*}
        \Delta  \GRANULEsym{,}   \GRANULEmv{y}  :  \GRANULEnt{A}   \vdash  \GRANULEnt{t}  :  \GRANULEnt{B} \tag{ih}
      \end{align*}
    %
    Case on $x \in \Delta$
    \begin{itemize}
      \item $x \in \Delta$, i.e., $\Delta = \Delta'  \GRANULEsym{,}   \GRANULEmv{x}  :_{\textcolor{coeffectColor}{  \GRANULEnt{s'}  } }   \GRANULEnt{A}$.

        Then by admissibility of contraction we can derive:
        %
        \begin{align*}
          \inferrule*{
            \inferrule*[Right=\textsc{Der}]
             {\Delta'  \GRANULEsym{,}   \GRANULEmv{x}  :_{\textcolor{coeffectColor}{  \GRANULEnt{s'}  } }   \GRANULEnt{A}   \GRANULEsym{,}   \GRANULEmv{y}  :  \GRANULEnt{A}   \vdash  \GRANULEnt{t}  :  \GRANULEnt{B}}{\Delta'  \GRANULEsym{,}   \GRANULEmv{x}  :_{\textcolor{coeffectColor}{  \GRANULEnt{s'}  } }   \GRANULEnt{A}   \GRANULEsym{,}   \GRANULEmv{y}  :_{\textcolor{coeffectColor}{   1   } }   \GRANULEnt{A}   \vdash  \GRANULEnt{t}  :  \GRANULEnt{B}}
            }
            { \GRANULEsym{(}  \Delta'  \GRANULEsym{,}   \GRANULEmv{x}  :_{\textcolor{coeffectColor}{  \GRANULEnt{s'}  } }   \GRANULEnt{A}   \GRANULEsym{)}  \GRANULEsym{+}   \GRANULEmv{x}  :_{\textcolor{coeffectColor}{   1   } }   \GRANULEnt{A}   \vdash   [  \GRANULEmv{x}  /  \GRANULEmv{y}  ]  \GRANULEnt{t}   :  \GRANULEnt{B} }
        \end{align*}
        %
        Satisfying the lemma statment.

     \item $x \not\in \Delta$. Then
      again from the admissiblity of contraction, we derive the
      typing:
      %
        \begin{align*}
          \inferrule*{
            \inferrule*[Right=\textsc{Der}]
             {\Delta  \GRANULEsym{,}   \GRANULEmv{y}  :  \GRANULEnt{A}   \vdash  \GRANULEnt{t}  :  \GRANULEnt{B}}{\Delta  \GRANULEsym{,}   \GRANULEmv{y}  :_{\textcolor{coeffectColor}{   1   } }   \GRANULEnt{A}   \vdash  \GRANULEnt{t}  :  \GRANULEnt{B}}
            }
            { \Delta  \GRANULEsym{+}   \GRANULEmv{x}  :_{\textcolor{coeffectColor}{   1   } }   \GRANULEnt{A}   \vdash   [  \GRANULEmv{x}  /  \GRANULEmv{y}  ]  \GRANULEnt{t}   :  \GRANULEnt{B} }
        \end{align*}
        %
        which is well defined as $x \not\in \Delta$ and gives the
        lemma conclusion.
        \end{itemize}

    \item Case \addUnboxName\\
    The synthesis rule for unboxing has the form:
    \[
      \addUnbox
    \]
    %
    By induction we have that:
    %
    \[
      \Delta  \vdash  \GRANULEnt{t}  :  \GRANULEnt{B} \tag{ih}
    \]
    %
    Case on $\GRANULEmv{x_{{\mathrm{2}}}}  :_{\textcolor{coeffectColor}{  \GRANULEnt{s}  } }   \GRANULEnt{A} \in \Delta$
    \begin{itemize}
        \item $\GRANULEmv{x_{{\mathrm{2}}}}  :_{\textcolor{coeffectColor}{  \GRANULEnt{s}  } }   \GRANULEnt{A} \in \Delta$, i.e., $\GRANULEnt{s} \sqsubseteq \GRANULEnt{r}$. \\
        From this, we can construct the typing derivation, matching the conclusion:
          \[
            \inferrule*[Right=\textsc{let}$\square$]
            {\inferrule*[right=\textsc{var}]{\quad}{\GRANULEmv{x_{{\mathrm{1}}}}  :   \Box_{  \GRANULEnt{r}  }  \GRANULEnt{A}    \vdash  \GRANULEmv{x_{{\mathrm{1}}}}  :   \Box_{  \GRANULEnt{r}  }  \GRANULEnt{A}} \\ \Delta  \GRANULEsym{,}   \GRANULEmv{x_{{\mathrm{2}}}}  :_{\textcolor{coeffectColor}{  \GRANULEnt{r}  } }   \GRANULEnt{A}   \vdash  \GRANULEnt{t}  :  \GRANULEnt{B}}{\Delta  \GRANULEsym{,}   \GRANULEmv{x_{{\mathrm{1}}}}  :   \Box_{  \GRANULEnt{r}  }  \GRANULEnt{A}    \vdash   \textbf{let} \, [  \GRANULEmv{x_{{\mathrm{2}}}}  ] =  \GRANULEmv{x_{{\mathrm{1}}}}  \, \textbf{in} \,  \GRANULEnt{t}   :  \GRANULEnt{B}}
          \]
        \item $\GRANULEmv{x_{{\mathrm{2}}}}  :_{\textcolor{coeffectColor}{  \GRANULEnt{s}  } }   \GRANULEnt{A} \notin \Delta$, i.e., $0 \sqsubseteq \GRANULEnt{r}$. \\
        From this, we can construct the typing derivation, matching the conclusion:
          \[
            \inferrule*[right=\textsc{let}$\square$]
            {\inferrule*[right=\textsc{var}]{\quad}{\GRANULEmv{x_{{\mathrm{1}}}}  :   \Box_{  \GRANULEnt{r}  }  \GRANULEnt{A}    \vdash  \GRANULEmv{x_{{\mathrm{1}}}}  :   \Box_{  \GRANULEnt{r}  }  \GRANULEnt{A}} \\ \inferrule*[right=\textsc{Approx}, rightskip=5em]{\inferrule*[right=Weak]{\Delta  \vdash  \GRANULEnt{t}  :  \GRANULEnt{B}}{\Delta  \GRANULEsym{,}   \GRANULEmv{x_{{\mathrm{2}}}}  :_{\textcolor{coeffectColor}{   0   } }   \GRANULEnt{A}   \vdash  \GRANULEnt{t}  :  \GRANULEnt{B}} \\ 0 \sqsubseteq \GRANULEnt{r}}{\Delta  \GRANULEsym{,}   \GRANULEmv{x_{{\mathrm{2}}}}  :_{\textcolor{coeffectColor}{  \GRANULEnt{r}  } }   \GRANULEnt{A}   \vdash  \GRANULEnt{t}  :  \GRANULEnt{B}}}{\Delta  \GRANULEsym{,}   \GRANULEmv{x_{{\mathrm{1}}}}  :   \Box_{  \GRANULEnt{r}  }  \GRANULEnt{A}    \vdash   \textbf{let} \, [  \GRANULEmv{x_{{\mathrm{2}}}}  ] =  \GRANULEmv{x_{{\mathrm{1}}}}  \, \textbf{in} \,  \GRANULEnt{t}   :  \GRANULEnt{B}}
          \]
    \end{itemize}
  \item Case \addPairIntroName\\

    The synthesis rule for pair introduction has the form:

    \[
      \addPairIntro
    \]
    %
    By induction on the premises we have that:
    %
    \begin{align*}
      \Delta_{{\mathrm{1}}}  \vdash  \GRANULEnt{t_{{\mathrm{1}}}}  :  \GRANULEnt{A} \tag{ih1}\\
      \Delta_{{\mathrm{2}}}  \vdash  \GRANULEnt{t_{{\mathrm{2}}}}  :  \GRANULEnt{B} \tag{ih2}
    \end{align*}
    %
    From this, we can construct the typing derivation, matching the conclusion:
    %
    \[
    \inferrule*[Right=\textsc{pair}]
    {\Delta_{{\mathrm{1}}}  \vdash  \GRANULEnt{t_{{\mathrm{1}}}}  :  \GRANULEnt{A} \\ \Delta_{{\mathrm{2}}}  \vdash  \GRANULEnt{t_{{\mathrm{2}}}}  :  \GRANULEnt{B}}
    {\Delta_{{\mathrm{1}}}  \GRANULEsym{+}  \Delta_{{\mathrm{2}}}  \vdash   ( \GRANULEnt{t_{{\mathrm{1}}}} ,  \GRANULEnt{t_{{\mathrm{2}}}} )   :   \GRANULEnt{A}  \, \otimes \,  \GRANULEnt{B}}
    \]

  \item Case \addPairElimName\\
    The synthesis rule for pair elimination has the form:
    \[
      \addPairElim
    \]
      By induction on the premises we have that:
    \begin{align*}
      \Delta_{{\mathrm{1}}}  \vdash  \GRANULEnt{t_{{\mathrm{1}}}}  :  \GRANULEnt{A} \tag{ih1} \\
      \Delta_{{\mathrm{2}}}  \vdash  \GRANULEnt{t_{{\mathrm{2}}}}  :  \GRANULEnt{B} \tag{ih2}
    \end{align*}
    From this, we can construct the typing derivation, matching the conclusion:
    \[
    \inferrule*[Right=\textsc{LetPair}]
    {\inferrule*[right=\textsc{Var}] {\quad} {\GRANULEmv{x_{{\mathrm{3}}}}  :   \GRANULEnt{A}  \, \otimes \,  \GRANULEnt{B}    \vdash  \GRANULEmv{x_{{\mathrm{3}}}}  :   \GRANULEnt{A}  \, \otimes \,  \GRANULEnt{B}} \\ \Delta  \GRANULEsym{,}   \GRANULEmv{x_{{\mathrm{1}}}}  :  \GRANULEnt{A}   \GRANULEsym{,}   \GRANULEmv{x_{{\mathrm{2}}}}  :  \GRANULEnt{B}   \vdash  \GRANULEnt{t_{{\mathrm{2}}}}  :  \GRANULEnt{C}}
    {\Delta  \GRANULEsym{,}   \GRANULEmv{x_{{\mathrm{3}}}}  :   \GRANULEnt{A}  \, \otimes \,  \GRANULEnt{B}    \vdash   \textbf{let} \, ( \GRANULEmv{x_{{\mathrm{1}}}} ,  \GRANULEmv{x_{{\mathrm{2}}}} ) =  \GRANULEmv{x_{{\mathrm{3}}}}  \, \textbf{in} \,  \GRANULEnt{t_{{\mathrm{2}}}}   :  \GRANULEnt{C}}
    \]

  \item Case \addSumIntroLName and \addSumIntroRName\\
    The synthesis rules for sum introduction are straightforward. For
    \addSumIntroLName we have the rule:
    \[
      \addSumIntroL
    \]
    By induction on the premises we have that:
    \begin{align*}
      \Delta  \vdash  \GRANULEnt{t}  :  \GRANULEnt{A} \tag{ih}
    \end{align*}
    From this, we can construct the typing derivation, matching the conclusion:
    \[
    \inferrule*[Right=\textsc{Inl}]
    {\Delta  \vdash  \GRANULEnt{t}  :  \GRANULEnt{A}}
    {\Delta  \vdash  \GRANULEkw{inl} \, \GRANULEnt{t}  :   \GRANULEnt{A}  \, \oplus \,  \GRANULEnt{B}}
    \]
    Likewise, for the \addSumIntroRName we have the
    synthesis rule:
    \[
      \addSumIntroR
    \]
    By induction on the premises we have that:
    \begin{align*}
      \Delta  \vdash  \GRANULEnt{t}  :  \GRANULEnt{B} \tag{ih}
    \end{align*}
    From this, we can construct the typing derivation, matching the conclusion:
    \[
    \inferrule*[Right=\textsc{Inr}]
    {\Delta  \vdash  \GRANULEnt{t}  :  \GRANULEnt{B}}
    {\Delta  \vdash  \GRANULEkw{inl} \, \GRANULEnt{t}  :   \GRANULEnt{A}  \, \oplus \,  \GRANULEnt{B}}
    \]

    \item Case \addSumElimName\\
      The synthesis rule for sum elimination has the form:
      \[
      \addSumElim
      \]
    By induction on the premises we have that:
    \begin{align*}
      \Delta_{{\mathrm{1}}}  \GRANULEsym{,}   \GRANULEmv{x_{{\mathrm{2}}}}  :  \GRANULEnt{A}   \vdash  \GRANULEnt{t_{{\mathrm{1}}}}  :  \GRANULEnt{C} \tag{ih1}\\
      \Delta_{{\mathrm{2}}}  \GRANULEsym{,}   \GRANULEmv{x_{{\mathrm{3}}}}  :  \GRANULEnt{B}   \vdash  \GRANULEnt{t_{{\mathrm{2}}}}  :  \GRANULEnt{C} \tag{ih2}
    \end{align*}
      From this, we can construct the typing derivation, matching the
      conclusion:
      \[
      \inferrule*[Right=\textsc{Case}]
      {\inferrule*[right=\textsc{Var}, leftskip=5em]{\quad}{\GRANULEmv{x_{{\mathrm{1}}}}  :   \GRANULEnt{A}  \, \oplus \,  \GRANULEnt{B}    \vdash  \GRANULEmv{x_{{\mathrm{1}}}}  :   \GRANULEnt{A}  \, \oplus \,  \GRANULEnt{B}} \\ \Delta_{{\mathrm{1}}}  \GRANULEsym{,}   \GRANULEmv{x_{{\mathrm{2}}}}  :  \GRANULEnt{A}   \vdash  \GRANULEnt{t_{{\mathrm{1}}}}  :  \GRANULEnt{C} \\ \Delta_{{\mathrm{2}}}  \GRANULEsym{,}   \GRANULEmv{x_{{\mathrm{3}}}}  :  \GRANULEnt{B}   \vdash  \GRANULEnt{t_{{\mathrm{2}}}}  :  \GRANULEnt{C}}{(\Delta_{{\mathrm{1}}} \sqcup \Delta_{{\mathrm{2}}} ), \GRANULEmv{x_{{\mathrm{1}}}}  :   \GRANULEnt{A}  \, \oplus \,  \GRANULEnt{B} \vdash \textbf{case} \ x_{1}\ \textbf{of}\ \textbf{inl}\ x_{2} \rightarrow t_{1};\ \textbf{inr}\ x_{3} \rightarrow t_{2} : C}
      \]

  \item Case \addUnitIntroName\\
    The synthesis rule for unit introduction has the form:
    \[
      \addUnitIntro
    \]
    From this, we can construct the typing derivation, matching the conclusion:
    \[
    \inferrule*[Right=\textsc{1}]
    {\quad}
    {\emptyset   \vdash  \GRANULEsym{()}  :   \mathsf{1}}
    \]

  \item Case \addUnitElimName\\
    The synthesis rule for unit elimination has the form:
    \[
      \addUnitElim
    \]
    By induction on the premises we have that:
    \begin{align*}
      \Delta  \vdash  \GRANULEnt{t}  :  \GRANULEnt{C} \tag{ih}
    \end{align*}
    From this, we can construct the typing derivation, matching the
    conclusion:
    \[
    \inferrule*[Right=\textsc{Let1}]
    { \inferrule*[Right=\textsc{Var}]{\quad}{\GRANULEmv{x}  :   \mathsf{1}    \vdash  \GRANULEmv{x}  :   \mathsf{1}} \\ \Delta  \vdash  \GRANULEnt{t}  :  \GRANULEnt{C}}
    {\Delta  \GRANULEsym{,}   \GRANULEmv{x}  :   \mathsf{1}    \vdash  \GRANULEkw{let} \, \GRANULEsym{()}  \GRANULEsym{=}  \GRANULEmv{x} \, \GRANULEkw{in} \, \GRANULEnt{t}  :  \GRANULEnt{C}}
    \]
    %

  \end{enumerate}
\end{proof}
\addPruningSynthSound*
\begin{proof}
  The cases for the rules in the additive pruning synthesis calculus are equivalent to lemma \eqref{lemma:addSynthSound}, except for the cases of the \addPruningAppName and \addPruningPairIntroName rules which we consider here:
  \begin{enumerate}
    \item Case \addPruningAppName\\
    Matching the form of the lemma, the application derivation can be
    constructed as:
    \[
      \addPruneApp
    \]
    %
    By induction on the premises we then have the following typing
    judgments:
    %
    \begin{align*}
      \Delta_{{\mathrm{1}}}  \GRANULEsym{,}   \GRANULEmv{x_{{\mathrm{2}}}}  :  \GRANULEnt{B}   \vdash  \GRANULEnt{t_{{\mathrm{1}}}}  :  \GRANULEnt{C} \\
      \Delta_{{\mathrm{2}}}  \vdash  \GRANULEnt{t_{{\mathrm{2}}}}  :  \GRANULEnt{A}
    \end{align*}
    %
    We can thus construct the following typing derivation, making use
    of the admissibility of linear substitution
    (Lemma~\ref{lemma:substitution}):
    \[
    \inferrule*[Right=(L.~\ref{lemma:substitution})]
    {\inferrule*[right=\textsc{app}, leftskip=2em]
    {\inferrule*[right=\textsc{var}]
    {\quad}
    {\GRANULEmv{x_{{\mathrm{1}}}}  :   \GRANULEnt{A}  \multimap  \GRANULEnt{B}    \vdash  \GRANULEmv{x_{{\mathrm{1}}}}  :   \GRANULEnt{A}  \multimap  \GRANULEnt{B}} \\ {\Delta_{{\mathrm{2}}}  \vdash  \GRANULEnt{t_{{\mathrm{2}}}}  :  \GRANULEnt{A}}}
  {\Delta_{{\mathrm{2}}}  \GRANULEsym{,}   \GRANULEmv{x_{{\mathrm{1}}}}  :   \GRANULEnt{A}  \multimap  \GRANULEnt{B}    \vdash  \GRANULEmv{x_{{\mathrm{1}}}} \, \GRANULEnt{t_{{\mathrm{2}}}}  :  \GRANULEnt{B}}
    \\ \Delta_{{\mathrm{1}}}  \GRANULEsym{,}   \GRANULEmv{x_{{\mathrm{2}}}}  :  \GRANULEnt{B}   \vdash  \GRANULEnt{t_{{\mathrm{1}}}}  :  \GRANULEnt{C}}
    {\GRANULEsym{(}  \Delta_{{\mathrm{1}}}  \GRANULEsym{+}  \Delta_{{\mathrm{2}}}  \GRANULEsym{)}  \GRANULEsym{,}   \GRANULEmv{x_{{\mathrm{1}}}}  :   \GRANULEnt{A}  \multimap  \GRANULEnt{B}    \vdash   [  \GRANULEsym{(}  \GRANULEmv{x_{{\mathrm{1}}}} \, \GRANULEnt{t_{{\mathrm{2}}}}  \GRANULEsym{)}  /  \GRANULEmv{x_{{\mathrm{2}}}}  ]  \GRANULEnt{t_{{\mathrm{1}}}}   :  \GRANULEnt{C}}
    \]

  \item Case \addPruningPairIntroName\\

    The synthesis rule for the pruning alternative for pair introduction has the form:

    \[
      \addPrunePairIntro
    \]
    By induction on the premises we have that:
    \begin{align*}
      \Delta_{{\mathrm{1}}}  \vdash  \GRANULEnt{t_{{\mathrm{1}}}}  :  \GRANULEnt{A} \tag{ih1} \\
      \Delta_{{\mathrm{2}}}  \vdash  \GRANULEnt{t_{{\mathrm{2}}}}  :  \GRANULEnt{B} \tag{ih2}
    \end{align*}

    From this, we can construct the typing derivation, matching the conclusion:

    \[
    \inferrule*[Right=\textsc{pair}]
    {\Delta_{{\mathrm{1}}}  \vdash  \GRANULEnt{t_{{\mathrm{1}}}}  :  \GRANULEnt{A} \\ \Delta_{{\mathrm{2}}}  \vdash  \GRANULEnt{t_{{\mathrm{2}}}}  :  \GRANULEnt{B}}
    {\Delta_{{\mathrm{1}}}  \GRANULEsym{+}  \Delta_{{\mathrm{2}}}  \vdash   ( \GRANULEnt{t_{{\mathrm{1}}}} ,  \GRANULEnt{t_{{\mathrm{2}}}} )   :   \GRANULEnt{A}  \, \otimes \,  \GRANULEnt{B}}
    \]


  \end{enumerate}
\end{proof}

%\focusSoundSub*
\begin{restatable}[Soundness of focusing for subtractive synthesis]{lemma}{focusSoundSub}
For all contexts $\Gamma$, $\Omega$ and types $\GRANULEnt{A}$
then:
\begin{align*}
\begin{array}{lll}
 1.\ Right\ Async: & \Gamma  ;  \Omega  \vdash   \GRANULEnt{A}  \Uparrow\   \Rightarrow^{-}  \GRANULEnt{t}  \ |\  \Delta \quad &\implies \quad \Gamma  \GRANULEsym{,}  \Omega  \vdash  \GRANULEnt{A}  \Rightarrow^-  \GRANULEnt{t} \ |\  \Delta\\
 2.\ Left\ Async: & \Gamma  ;   \Omega  \Uparrow\   \vdash  \GRANULEnt{C}  \Rightarrow^{-}  \GRANULEnt{t}  \ |\  \Delta \quad &\implies \quad \Gamma  \GRANULEsym{,}  \Omega  \vdash  \GRANULEnt{C}  \Rightarrow^-  \GRANULEnt{t} \ |\  \Delta\\
 3.\ Right\ Sync: & \Gamma  ;   \emptyset   \vdash   \GRANULEnt{A}  \Downarrow\   \Rightarrow^{-}  \GRANULEnt{t}  \ |\  \Delta \quad &\implies \quad \Gamma  \vdash  \GRANULEnt{A}  \Rightarrow^-  \GRANULEnt{t} \ |\  \Delta\\
 4.\ Left\ Sync: & \Gamma  ;     \GRANULEmv{x}  :  \GRANULEnt{A}    \Downarrow\   \vdash  \GRANULEnt{C}  \Rightarrow^{-}  \GRANULEnt{t}  \ |\  \Delta \quad &\implies \quad \Gamma  \GRANULEsym{,}   \GRANULEmv{x}  :  \GRANULEnt{A}   \vdash  \GRANULEnt{C}  \Rightarrow^-  \GRANULEnt{t} \ |\  \Delta\\
 5.\ Focus\ Right: & \Gamma  ;   \Omega  \Uparrow\   \vdash  \GRANULEnt{C}  \Rightarrow^{-}  \GRANULEnt{t}  \ |\  \Delta \quad &\implies \quad \Gamma  \vdash  \GRANULEnt{C}  \Rightarrow^-  \GRANULEnt{t} \ |\  \Delta\\
 6.\ Focus\ Left: & \Gamma  \GRANULEsym{,}   \GRANULEmv{x}  :  \GRANULEnt{A}   ;   \Omega  \Uparrow\   \vdash  \GRANULEnt{C}  \Rightarrow^{-}  \GRANULEnt{t}  \ |\  \Delta \quad &\implies \quad \Gamma  \vdash  \GRANULEnt{C}  \Rightarrow^-  \GRANULEnt{t} \ |\  \Delta
\end{array}
\end{align*}
\end{restatable}
\begin{proof}
  \begin{enumerate}
      \item Case 1. Right Async: \\
      \begin{enumerate}
        \item Case \subAbsName \\
          In the case of the right asynchronous rule for abstraction introduction, the synthesis rule has the form:
          \[
          \fSubAbsRuleNoLabel
          \]
          By induction on the first premise, we have that:
          \[
            \GRANULEsym{(}  \Gamma  \GRANULEsym{,}  \Omega  \GRANULEsym{)}  \GRANULEsym{,}   \GRANULEmv{x}  :  \GRANULEnt{A}   \vdash  \GRANULEnt{A}  \Rightarrow^-  \GRANULEnt{t} \ |\  \Delta \tag{ih}
          \]
          from case 1 of the lemma. From which, we can construct the following instantiation of the \subAbsName synthesis rule in the non-focusing calculus:
          \[
          \inferrule*[right=\subAbsName]
          {\GRANULEsym{(}  \Gamma  \GRANULEsym{,}  \Omega  \GRANULEsym{)}  \GRANULEsym{,}   \GRANULEmv{x}  :  \GRANULEnt{A}   \vdash  \GRANULEnt{B}  \Rightarrow^-  \GRANULEnt{t} \ |\  \Delta \quad\; \GRANULEmv{x}  \not\in | \Delta |}{\Gamma  \GRANULEsym{,}  \Omega  \vdash   \GRANULEnt{A}  \multimap  \GRANULEnt{B}   \Rightarrow^-   \lambda  \GRANULEmv{x}  .  \GRANULEnt{t}  \ |\  \Delta}
          \]
    \item Case \fSubRAsyncTransitionName \\
          In the case of the right asynchronous rule for transition to a left asynchronous judgement, the synthesis rule has the form:
          \[
            \fSubRAsyncTransitionRule
          \]
          By induction on the first premise, we have that:
          \[
            \Gamma  \GRANULEsym{,}  \Omega  \vdash  \GRANULEnt{C}  \Rightarrow^-  \GRANULEnt{t} \ |\  \Delta
          \]
          from case 2 of the lemma.
    \end{enumerate}
    \item Case 2. Left Async: \\
      \begin{enumerate}
        \item Case \subPairElimName \\
          In the case of the left asynchronous rule for pair elimination, the synthesis rule has the form:
          \[
          \fSubPairElimRuleNoLabel
          \]
          By induction on the first premise, we have that:
            \[
            \GRANULEsym{(}  \Gamma  \GRANULEsym{,}  \Omega  \GRANULEsym{)}  \GRANULEsym{,}   \GRANULEmv{x_{{\mathrm{1}}}}  :  \GRANULEnt{A}   \GRANULEsym{,}   \GRANULEmv{x_{{\mathrm{2}}}}  :  \GRANULEnt{B}   \vdash  \GRANULEnt{C}  \Rightarrow^-  \GRANULEnt{t} \ |\  \Delta \tag{ih}
            \]
          from From which, we can construct the following instantiation of the \subPairIntroName\ synthesis rule in the non-focusing calculus:
          \[
          \inferrule*[right=\subPairElimName]
          {\GRANULEsym{(}  \Gamma  \GRANULEsym{,}  \Omega  \GRANULEsym{)}  \GRANULEsym{,}   \GRANULEmv{x_{{\mathrm{1}}}}  :  \GRANULEnt{A}   \GRANULEsym{,}   \GRANULEmv{x_{{\mathrm{2}}}}  :  \GRANULEnt{B}   \vdash  \GRANULEnt{C}  \Rightarrow^-  \GRANULEnt{t} \ |\  \Delta \\ \GRANULEmv{x_{{\mathrm{1}}}}  \not\in | \Delta | \\ \GRANULEmv{x_{{\mathrm{2}}}}  \not\in | \Delta |}{\Gamma  \GRANULEsym{,}  \GRANULEsym{(}  \Omega  \GRANULEsym{,}   \GRANULEmv{x_{{\mathrm{3}}}}  :   \GRANULEnt{A}  \, \otimes \,  \GRANULEnt{B}    \GRANULEsym{)}  \vdash  \GRANULEnt{C}  \Rightarrow^-   \textbf{let} \, ( \GRANULEmv{x_{{\mathrm{1}}}} ,  \GRANULEmv{x_{{\mathrm{2}}}} ) =  \GRANULEmv{x_{{\mathrm{3}}}}  \, \textbf{in} \,  \GRANULEnt{t}  \ |\  \Delta_{{\mathrm{2}}}}
          \]
        \item Case \subSumElimName \\
          In the case of the left asynchronous rule for sum elimination, the synthesis rule has the form:
          \[
          \fSubSumElimRule
          \]
          By induction on the first and second premises, we have that:
          \[
            \GRANULEsym{(}  \Gamma  \GRANULEsym{,}  \Omega  \GRANULEsym{)}  \GRANULEsym{,}   \GRANULEmv{x_{{\mathrm{2}}}}  :  \GRANULEnt{A}   \vdash  \GRANULEnt{C}  \Rightarrow^-  \GRANULEnt{t_{{\mathrm{1}}}} \ |\  \Delta_{{\mathrm{1}}} \tag{ih1}\\
          \]
          \[
            \GRANULEsym{(}  \Gamma  \GRANULEsym{,}  \Omega  \GRANULEsym{)}  \GRANULEsym{,}   \GRANULEmv{x_{{\mathrm{3}}}}  :  \GRANULEnt{B}   \vdash  \GRANULEnt{C}  \Rightarrow^-  \GRANULEnt{t_{{\mathrm{2}}}} \ |\  \Delta_{{\mathrm{2}}} \tag{ih2}
          \]
          from case 2 of the lemma. From which, we can construct the following instantiation of the \subSumElimName\ synthesis rule in the non-focusing calculus:
          \[
    \inferrule*[right=\subSumElimName]
      {\GRANULEsym{(}  \Gamma  \GRANULEsym{,}  \Omega  \GRANULEsym{)}  \GRANULEsym{,}   \GRANULEmv{x_{{\mathrm{2}}}}  :  \GRANULEnt{A}   \vdash  \GRANULEnt{C}  \Rightarrow^-  \GRANULEnt{t_{{\mathrm{1}}}} \ |\  \Delta_{{\mathrm{1}}} \quad\,
       \GRANULEsym{(}  \Gamma  \GRANULEsym{,}  \Omega  \GRANULEsym{)}  \GRANULEsym{,}   \GRANULEmv{x_{{\mathrm{3}}}}  :  \GRANULEnt{B}   \vdash  \GRANULEnt{C}  \Rightarrow^-  \GRANULEnt{t_{{\mathrm{2}}}} \ |\  \Delta_{{\mathrm{2}}} \quad\, \GRANULEmv{x_{{\mathrm{2}}}}  \not\in | \Delta_{{\mathrm{1}}} | \quad \GRANULEmv{x_{{\mathrm{3}}}}  \not\in | \Delta_{{\mathrm{2}}} |}
     {\Gamma  \GRANULEsym{,}  \GRANULEsym{(}  \Omega  \GRANULEsym{,}   \GRANULEmv{x_{{\mathrm{1}}}}  :   \GRANULEnt{A}  \, \oplus \,  \GRANULEnt{B}    \GRANULEsym{)} \vdash C \Rightarrow^{-}  \textbf{case} \ x_{1}\ \textbf{of}\ \textbf{inl}\ x_{2} \rightarrow t_{1};\ \textbf{inr}\ x_{3} \rightarrow t_{2} \Delta_{{\mathrm{1}}} \sqcap \Delta_{{\mathrm{2}}}}
          \]
        \item Case \subUnitElimName \\
          In the case of the left asynchronous rule for unit elimination, the synthesis rule has the form:
          \[
          \fSubUnitElimRule
          \]
          By induction on the premise, we have that:
          \[
            \Gamma  \vdash  \GRANULEnt{C}  \Rightarrow^-  \GRANULEnt{t} \ |\  \Delta \tag{ih}
          \]
          from case 2 of the lemma. From which, we can construct the following instantiation of the \subUnitElimName\ synthesis rule in the non-focusing calculus matching the conclusion:
          \[
    \inferrule*[right=\subUnitElimName]
    {\Gamma  \vdash  \GRANULEnt{C}  \Rightarrow^-  \GRANULEnt{t} \ |\  \Delta}
    {\Gamma  \GRANULEsym{,}   \GRANULEmv{x}  :   \mathsf{1}    \vdash  \GRANULEnt{C}  \Rightarrow^-  \GRANULEkw{let} \, \GRANULEsym{()}  \GRANULEsym{=}  \GRANULEmv{x} \, \GRANULEkw{in} \, \GRANULEnt{t} \ |\  \Delta}
          \]
        \item Case \subUnboxName \\
          In the case of the left asynchronous rule for graded modality elimination, the synthesis rule has the form:
          \[
          \fSubUnboxRule
          \]
          By induction on the first premise, we have that:
          \[
            \GRANULEsym{(}  \Gamma  \GRANULEsym{,}  \Omega  \GRANULEsym{)}  \GRANULEsym{,}   \GRANULEmv{x_{{\mathrm{2}}}}  :_{\textcolor{coeffectColor}{  \GRANULEnt{r}  } }   \GRANULEnt{A}   \vdash  \GRANULEnt{B}  \Rightarrow^-  \GRANULEnt{t} \ |\  \Delta  \GRANULEsym{,}   \GRANULEmv{x_{{\mathrm{2}}}}  :_{\textcolor{coeffectColor}{  \GRANULEnt{s}  } }   \GRANULEnt{A}  \tag{ih}\\
          \]
          from case 2 of the lemma. From which, we can construct the following instatiation of the \subUnboxName synthesis rule in the non-focusing calculus:
          \[
  \inferrule*[right=\subUnboxName]
    {\GRANULEsym{(}  \Gamma  \GRANULEsym{,}  \Omega  \GRANULEsym{)}  \GRANULEsym{,}   \GRANULEmv{x_{{\mathrm{2}}}}  :_{\textcolor{coeffectColor}{  \GRANULEnt{r}  } }   \GRANULEnt{A}   \vdash  \GRANULEnt{B}  \Rightarrow^-  \GRANULEnt{t} \ |\  \Delta  \GRANULEsym{,}   \GRANULEmv{x_{{\mathrm{2}}}}  :_{\textcolor{coeffectColor}{  \GRANULEnt{s}  } }   \GRANULEnt{A}  \\ 0 \sqsubseteq \GRANULEnt{s}}{\Gamma  \GRANULEsym{,}  \GRANULEsym{(}  \Omega  \GRANULEsym{,}   \GRANULEmv{x_{{\mathrm{1}}}}  :   \Box_{  \GRANULEnt{r}  }  \GRANULEnt{A}    \GRANULEsym{)}  \vdash  \GRANULEnt{B}  \Rightarrow^-   \textbf{let} \, [  \GRANULEmv{x_{{\mathrm{2}}}}  ] =  \GRANULEmv{x_{{\mathrm{1}}}}  \, \textbf{in} \,  \GRANULEnt{t}  \ |\  \Delta}
          \]
        \item Case \subDerName \\
          In the case of the left asynchronous rule for dereliction, the synthesis rule has the form:
          \[
          \fSubDerRule
          \]
          By induction on the first premise, we have that:
          \[
            \Gamma  \GRANULEsym{,}   \GRANULEmv{x}  :_{\textcolor{coeffectColor}{  \GRANULEnt{s}  } }   \GRANULEnt{A}   \GRANULEsym{,}   \GRANULEmv{y}  :  \GRANULEnt{A}   \vdash  \GRANULEnt{B}  \Rightarrow^-  \GRANULEnt{t} \ |\  \Delta  \GRANULEsym{,}   \GRANULEmv{x}  :_{\textcolor{coeffectColor}{  \GRANULEnt{s'}  } }   \GRANULEnt{A}  \tag{ih}\\
          \]
          from case 2 of the lemma. From which, we can construct the following instatiation of the \subDerName synthesis rule in the non-focusing calculus:
          \[
      \inferrule*[right=\subDerName]
{\Gamma  \GRANULEsym{,}   \GRANULEmv{x}  :_{\textcolor{coeffectColor}{  \GRANULEnt{s}  } }   \GRANULEnt{A}   \GRANULEsym{,}   \GRANULEmv{y}  :  \GRANULEnt{A}   \vdash  \GRANULEnt{B}  \Rightarrow^-  \GRANULEnt{t} \ |\  \Delta  \GRANULEsym{,}   \GRANULEmv{x}  :_{\textcolor{coeffectColor}{  \GRANULEnt{s'}  } }   \GRANULEnt{A} \\
\GRANULEmv{y}  \not\in | \Delta | \\
\exists  \GRANULEnt{s}  .\,   \GRANULEnt{r}  \sqsupseteq  \GRANULEnt{s}  \GRANULEsym{+}   1
}
{\Gamma  \GRANULEsym{,}   \GRANULEmv{x}  :_{\textcolor{coeffectColor}{  \GRANULEnt{r}  } }   \GRANULEnt{A}   \vdash  \GRANULEnt{B}  \Rightarrow^-   [  \GRANULEmv{x}  /  \GRANULEmv{y}  ]  \GRANULEnt{t}  \ |\  \Delta  \GRANULEsym{,}   \GRANULEmv{x}  :_{\textcolor{coeffectColor}{  \GRANULEnt{s'}  } }   \GRANULEnt{A}}
          \]

        \item Case \fSubLAsyncTransitionName \\
          In the case of the left asynchronous rule for transitioning an assumption from the focusing context $\Omega$ to the non-focusing context $\Gamma$, the synthesis rule has the form:
          \[
            \fSubLAsyncTransitionRule
          \]
          By induction on the first premise, we have that:
          \[
            \Gamma  \GRANULEsym{,}   \GRANULEmv{x}  :  \GRANULEnt{A}    \GRANULEsym{,}  \Omega  \vdash  \GRANULEnt{C}  \Rightarrow^-  \GRANULEnt{t} \ |\  \Delta \tag{ih}
          \]
          from case 2 of the lemma.
      \end{enumerate}
    \item Case 3. Right Sync: \\
      \begin{enumerate}
        \item Case \subPairIntroName \\
          In the case of the right synchronous rule for pair introduction, the synthesis rule has the form:
          \[
          \fSubPairIntroRuleNoLabel
          \]
          By induction on the first and second premises, we have that:
          \[
            \Gamma  \vdash  \GRANULEnt{A}  \Rightarrow^-  \GRANULEnt{t_{{\mathrm{1}}}} \ |\  \Delta_{{\mathrm{1}}}  \tag{ih1}
          \]
          \[
            \Delta_{{\mathrm{1}}}  \vdash  \GRANULEnt{B}  \Rightarrow^-  \GRANULEnt{t_{{\mathrm{2}}}} \ |\  \Delta_{{\mathrm{2}}} \tag{ih2}
          \]
          from case 3 of the lemma. From which, we can construct the following instatiation of the \subPairIntroName\ synthesis rule in the non-focusing calculus:
          \[
    \inferrule*[right=\subPairIntroName]
    {\Gamma  \vdash  \GRANULEnt{A}  \Rightarrow^-  \GRANULEnt{t_{{\mathrm{1}}}} \ |\  \Delta_{{\mathrm{1}}} \\ \Delta_{{\mathrm{1}}}  \vdash  \GRANULEnt{B}  \Rightarrow^-  \GRANULEnt{t_{{\mathrm{2}}}} \ |\  \Delta_{{\mathrm{2}}}}{\Gamma  \vdash   \GRANULEnt{A}  \, \otimes \,  \GRANULEnt{B}   \Rightarrow^-   ( \GRANULEnt{t_{{\mathrm{1}}}} ,  \GRANULEnt{t_{{\mathrm{2}}}} )  \ |\  \Delta_{{\mathrm{2}}}}
          \]
        \item Case \subSumIntroLname\ and \subSumIntroRname\\
          In the case of the right synchronous rules for sum introduction, the synthesis rules has the form:
          \[
          \fSubSumIntroRuleL
          \]
          \[
          \fSubSumIntroRuleR
          \]
          By induction on the premises of these rules, we have that:
          \[
            \Gamma  \vdash  \GRANULEnt{A}  \Rightarrow^-  \GRANULEnt{t} \ |\  \Delta  \tag{ih1}
          \]
          \[
            \Gamma  \vdash  \GRANULEnt{B}  \Rightarrow^-  \GRANULEnt{t} \ |\  \Delta \tag{ih2}
          \]
          from case 3 of the lemma. From which, we can construct the following instatiations of the \subSumIntroLname\ and \subSumIntroRname\ rule in the non-focusing calculus, respectively:
          \[
    \inferrule*[right=\subSumIntroLname]
    {\Gamma  \vdash  \GRANULEnt{A}  \Rightarrow^-  \GRANULEnt{t} \ |\  \Delta}
    {\Gamma  \vdash   \GRANULEnt{A}  \, \oplus \,  \GRANULEnt{B}   \Rightarrow^-  \GRANULEkw{inl} \, \GRANULEnt{t} \ |\  \Delta}
          \]
          \[
    \inferrule*[right=\subSumIntroRname]
    {\Gamma  \vdash  \GRANULEnt{B}  \Rightarrow^-  \GRANULEnt{t} \ |\  \Delta}
    {\Gamma  \vdash   \GRANULEnt{A}  \, \oplus \,  \GRANULEnt{B}   \Rightarrow^-  \GRANULEkw{inr} \, \GRANULEnt{t} \ |\  \Delta}
          \]
        \item Case \subUnitIntroName \\
          In the case of the right synchronous rule for unit introduction, the synthesis rule has the form:
          \[
          \fSubUnitIntroRule
          \]
          From which, we can construct the following instatiation of the \subUnitIntroName\ synthesis rule in the non-focusing calculus:
          \[
    \inferrule*[right=\subUnitIntroName]
    {\quad}
    {\Gamma  \GRANULEsym{,}  \Omega  \vdash   \mathsf{1}   \Rightarrow^-  \GRANULEsym{()} \ |\  \Gamma}
          \]
        \item Case \subBoxName \\
          In the case of the right synchronous rule for graded modality introduction, the synthesis rule has the form:
          \[
          \fSubBoxRule
          \]
          By induction on the premise, we have that:
          \[
            \Gamma  \vdash  \GRANULEnt{A}  \Rightarrow^-  \GRANULEnt{t} \ |\  \Delta \tag{ih}
          \]
          from case 1 of the lemma. From which, we can construct the following instatiation of the \subBoxName synthesis rule in the non-focusing calculus:
          \[
  \inferrule*[right=\subBoxName]
  {\Gamma  \vdash  \GRANULEnt{A}  \Rightarrow^-  \GRANULEnt{t} \ |\  \Delta}{\Gamma  \vdash   \Box_{  \GRANULEnt{r}  }  \GRANULEnt{A}   \Rightarrow^-  \GRANULEsym{[}  \GRANULEnt{t}  \GRANULEsym{]} \ |\  \Gamma  \GRANULEsym{-}   \textcolor{coeffectColor}{ \GRANULEnt{r}   \textcolor{coeffectColor}{\,\cdot\,} }  \GRANULEsym{(}  \Gamma  \GRANULEsym{-}  \Delta  \GRANULEsym{)}}
          \]
      \item Case \fSubRSyncTransitionName \\
          In the case of the right synchronous rule for transitioning back to an asynchronous judgement, the synthesis rule has the form:
          \[
            \fSubRSyncTransitionRule
          \]
          By induction on the premise, we have that:
          \[
            \Gamma  \vdash  \GRANULEnt{A}  \Rightarrow^-  \GRANULEnt{t} \ |\  \Delta \tag{ih}
          \]
          from case 1 of the lemma.
      \end{enumerate}
    \item Case 4. Left Sync \\
      \begin{enumerate}
          \item Case \subAppName \\
          In the case of the left synchronous rule for application, the synthesis rule has the form:
          \[
          \fSubAppRuleNoLabel
          \]
          By induction on the first premise, we have that:
          \[
            \Gamma  \GRANULEsym{,}   \GRANULEmv{x_{{\mathrm{2}}}}  :  \GRANULEnt{B}   \vdash  \GRANULEnt{C}  \Rightarrow^-  \GRANULEnt{t_{{\mathrm{1}}}} \ |\  \Delta_{{\mathrm{1}}} \tag{ih1}
          \]
          from case 4 of the lemma. By induction on the third premise, we have that:
          \[
            \Delta_{{\mathrm{1}}}  \vdash  \GRANULEnt{A}  \Rightarrow^-  \GRANULEnt{t_{{\mathrm{2}}}} \ |\  \Delta_{{\mathrm{2}}} \tag{ih2}
          \]
          from case 3 of the lemma. From which, we can construct the following instatiation of the \subAppName synthesis rule in the non-focusing calculus:
          \[
  \inferrule*[right=\subAppName]
  {\Gamma  \GRANULEsym{,}   \GRANULEmv{x_{{\mathrm{2}}}}  :  \GRANULEnt{B}   \vdash  \GRANULEnt{C}  \Rightarrow^-  \GRANULEnt{t_{{\mathrm{1}}}} \ |\  \Delta_{{\mathrm{1}}} \qquad \GRANULEmv{x_{{\mathrm{2}}}}  \not\in | \Delta_{{\mathrm{1}}} | \qquad \Delta_{{\mathrm{1}}}  \vdash  \GRANULEnt{A}  \Rightarrow^-  \GRANULEnt{t_{{\mathrm{2}}}} \ |\  \Delta_{{\mathrm{2}}}}{\Gamma  \GRANULEsym{,}   \GRANULEmv{x_{{\mathrm{1}}}}  :   \GRANULEnt{A}  \multimap  \GRANULEnt{B}    \vdash  \GRANULEnt{C}  \Rightarrow^-   [  \GRANULEsym{(}  \GRANULEmv{x_{{\mathrm{1}}}} \, \GRANULEnt{t_{{\mathrm{2}}}}  \GRANULEsym{)}  /  \GRANULEmv{x_{{\mathrm{2}}}}  ]  \GRANULEnt{t_{{\mathrm{1}}}}  \ |\  \Delta_{{\mathrm{2}}}}
          \]
          \item Case \subLinVarName \\
          In the case of the left synchronous rule for linear variable synthesis, the synthesis rule has the form:
          \[
          \fSubLinVarRule
          \]
          From which, we can construct the following instatiation of the \subLinVarName\  synthesis rule in the non-focusing calculus:
          \[
                             \inferrule*[right=\subLinVarName]
                             {\quad}{\Gamma  \GRANULEsym{,}   \GRANULEmv{x}  :  \GRANULEnt{A}   \vdash  \GRANULEnt{A}  \Rightarrow^-  \GRANULEmv{x} \ |\  \Gamma}
          \]
          \item Case \subGrVarName \\
          In the case of the left synchronous rule for graded variable synthesis, the synthesis rule has the form:
          \[
          \fSubGrVarRule
          \]
          From which, we can construct the following instatiation of the \subGrVarName\  synthesis rule in the non-focusing calculus:
          \[
      \inferrule*[right=\subGrVarName]
  {\exists  \GRANULEnt{s}  .\,   \GRANULEnt{r}  \sqsubseteq   \GRANULEnt{s}  \GRANULEsym{+}   1}{\Gamma  \GRANULEsym{,}   \GRANULEmv{x}  :_{\textcolor{coeffectColor}{  \GRANULEnt{r}  } }   \GRANULEnt{A}   \vdash  \GRANULEnt{A}  \Rightarrow^-  \GRANULEmv{x} \ |\  \Gamma  \GRANULEsym{,}   \GRANULEmv{x}  :_{\textcolor{coeffectColor}{  \GRANULEnt{s}  } }   \GRANULEnt{A}}
          \]
      \item Case \fSubLSyncTransitionName \\
          In the case of the left synchronous rule for transitioning back to an asynchronous judgement, the synthesis rule has the form:
          \[
            \fSubLSyncTransitionRule
          \]
          By induction on the premise, we have that:
          \[
            \Gamma  \GRANULEsym{,}   \GRANULEmv{x}  :  \GRANULEnt{A}   \vdash  \GRANULEnt{C}  \Rightarrow^-  \GRANULEnt{t} \ |\  \Delta \tag{ih}
          \]
          from case 2 of the lemma.
      \end{enumerate}
        \item Case 5. Focus Right: \fSubFocusRName \\
          In the case of the focusing rule for transitioning from a left asynchronous judgement to a right synchronous judgement, the synthesis rule has the form:
          \[
            \fSubFocusRRuleNoLabel
          \]
          By induction on the first premise, we have that:
          \[
            \Gamma  \vdash  \GRANULEnt{C}  \Rightarrow^-  \GRANULEnt{t} \ |\  \Delta \tag{ih}
          \]
          from case 2 of the lemma.
        \item Case 6. Focus Left \fSubFocusLName \\
          In the case of the focusing rule for transitioning from a left asynchronous judgement to a left synchronous judgement, the synthesis rule has the form:
          \[
            \fSubFocusLRule
          \]
          By induction on the first premise, we have that:
          \[
            \Gamma  \GRANULEsym{,}   \GRANULEmv{x}  :  \GRANULEnt{A}   \vdash  \GRANULEnt{C}  \Rightarrow^-  \GRANULEnt{t} \ |\  \Delta \tag{ih}
          \]
          from case 2 of the lemma.
  \end{enumerate}
\end{proof}

%\focusSoundAdd*
\begin{restatable}[Soundness of focusing for additive synthesis]{lemma}{focusSoundAdd}
  \label{lemma:fAddSynthSound}
For all contexts $\Gamma$, $\Omega$ and types $\GRANULEnt{A}$
then:
\begin{align*}
\begin{array}{lll}
 1.\ Right\ Async: & \Gamma  ;  \Omega  \vdash   \GRANULEnt{A}  \Uparrow\   \Rightarrow^{+}  \GRANULEnt{t}  \ | \  \Delta \quad &\implies \quad \Gamma  \GRANULEsym{,}  \,  \GRANULEsym{,}  \Omega  \vdash  \GRANULEnt{A}  \Rightarrow^+  \GRANULEnt{t}  ;\,  \Delta\\
 2.\ Left\ Async: & \Gamma  ;   \Omega  \Uparrow\   \vdash  \GRANULEnt{C}  \Rightarrow^{+}  \GRANULEnt{t}  \ | \  \Delta \quad &\implies \quad \Gamma  \GRANULEsym{,}  \,  \GRANULEsym{,}  \Omega  \vdash  \GRANULEnt{C}  \Rightarrow^+  \GRANULEnt{t}  ;\,  \Delta\\
 3.\ Right\ Sync: & \Gamma  ;   \emptyset   \vdash   \GRANULEnt{A}  \Downarrow\   \Rightarrow^{+}  \GRANULEnt{t}  \ | \  \Delta \quad &\implies \quad \Gamma  \vdash  \GRANULEnt{A}  \Rightarrow^+  \GRANULEnt{t}  ;\,  \Delta\\
 4.\ Left\ Sync: & \Gamma  ;     \GRANULEmv{x}  :  \GRANULEnt{A}    \Downarrow\   \vdash  \GRANULEnt{C}  \Rightarrow^{+}  \GRANULEnt{t}  \ | \  \Delta \quad &\implies \quad \Gamma  \GRANULEsym{,}   \GRANULEmv{x}  :  \GRANULEnt{A}   \vdash  \GRANULEnt{C}  \Rightarrow^+  \GRANULEnt{t}  ;\,  \Delta\\
 5.\ Focus\ Right: & \Gamma  ;   \Omega  \Uparrow\   \vdash  \GRANULEnt{C}  \Rightarrow^{+}  \GRANULEnt{t}  \ | \  \Delta \quad &\implies \quad \Gamma  \vdash  \GRANULEnt{C}  \Rightarrow^+  \GRANULEnt{t}  ;\,  \Delta\\
 6.\ Focus\ Left: & \Gamma  \GRANULEsym{,}   \GRANULEmv{x}  :  \GRANULEnt{A}   ;   \Omega  \Uparrow\   \vdash  \GRANULEnt{C}  \Rightarrow^{+}  \GRANULEnt{t}  \ | \  \Delta \quad &\implies \quad \Gamma  \vdash  \GRANULEnt{C}  \Rightarrow^+  \GRANULEnt{t}  ;\,  \Delta
\end{array}
\end{align*}
\end{restatable}
\begin{proof}
  \begin{enumerate}
      \item Case 1. Right Async: \\
      \begin{enumerate}
        \item Case \addAbsName \\
          In the case of the right asynchronous rule for abstraction introduction, the synthesis rule has the form:
          \[
          \fAddAbsRuleNoLabel
          \]
          By induction on the premise, we have that:
          \[
            \GRANULEsym{(}  \Gamma  \GRANULEsym{,}  \Omega  \GRANULEsym{)}  \GRANULEsym{,}   \GRANULEmv{x}  :  \GRANULEnt{A}   \vdash  \GRANULEnt{B}  \Rightarrow^+  \GRANULEnt{t}  ;\,  \Delta  \GRANULEsym{,}   \GRANULEmv{x}  :  \GRANULEnt{A}   \tag{ih}
          \]
          from case 1 of the lemma. From which, we can construct the following instatiation of the \addAbsName\ synthesis rule in the non-focusing calculus:
          \[
    \inferrule*[right=R$\multimap^{+}$]
    {\GRANULEsym{(}  \Gamma  \GRANULEsym{,}  \Omega  \GRANULEsym{)}  \GRANULEsym{,}   \GRANULEmv{x}  :  \GRANULEnt{A}   \vdash  \GRANULEnt{B}  \Rightarrow^+  \GRANULEnt{t}  ;\,  \Delta  \GRANULEsym{,}   \GRANULEmv{x}  :  \GRANULEnt{A}}{\Gamma  \GRANULEsym{,}  \Omega  \vdash   \GRANULEnt{A}  \multimap  \GRANULEnt{B}   \Rightarrow^+   \lambda  \GRANULEmv{x}  .  \GRANULEnt{t}   ;\,  \Delta}
          \]
          \item Case \fAddRAsyncTransitionName
          In the case of the right asynchronous rule for transition to a left asynchronous judgement, the synthesis rule has the form:
          \[
            \fAddRAsyncTransitionRule
          \]
          By induction on the first premise, we have that:
          \[
            \Gamma  \GRANULEsym{,}  \Omega  \vdash  \GRANULEnt{C}  \Rightarrow^+  \GRANULEnt{t}  ;\,  \Delta
          \]
          from case 2 of the lemma.
      \end{enumerate}
    \item Case 2. Left Async: \\
      \begin{enumerate}
        \item Case \addPairElimName \\
          In the case of the left asynchronous rule for pair elimination, the synthesis rule has the form:
          \[
          \fAddPairElimRuleNoLabel
          \]
          By induction on the premise, we have that:
          \[
            \GRANULEsym{(}  \Gamma  \GRANULEsym{,}  \Omega  \GRANULEsym{)}  \GRANULEsym{,}   \GRANULEmv{x_{{\mathrm{1}}}}  :  \GRANULEnt{A}   \GRANULEsym{,}   \GRANULEmv{x_{{\mathrm{2}}}}  :  \GRANULEnt{B}   \vdash  \GRANULEnt{C}  \Rightarrow^+  \GRANULEnt{t_{{\mathrm{2}}}}  ;\,  \Delta  \GRANULEsym{,}   \GRANULEmv{x_{{\mathrm{1}}}}  :  \GRANULEnt{A}   \GRANULEsym{,}   \GRANULEmv{x_{{\mathrm{2}}}}  :  \GRANULEnt{B}   \tag{ih}
          \]
          from case 2 of the lemma. From which, we can construct the following instatiation of the \addPairElimName\ synthesis rule in the non-focusing calculus:
          \[
    \inferrule*[right=L$\otimes^{+}$]
    {\GRANULEsym{(}  \Gamma  \GRANULEsym{,}  \Omega  \GRANULEsym{)}  \GRANULEsym{,}   \GRANULEmv{x_{{\mathrm{1}}}}  :  \GRANULEnt{A}   \GRANULEsym{,}   \GRANULEmv{x_{{\mathrm{2}}}}  :  \GRANULEnt{B}   \vdash  \GRANULEnt{C}  \Rightarrow^+  \GRANULEnt{t_{{\mathrm{2}}}}  ;\,  \Delta  \GRANULEsym{,}   \GRANULEmv{x_{{\mathrm{1}}}}  :  \GRANULEnt{A}   \GRANULEsym{,}   \GRANULEmv{x_{{\mathrm{2}}}}  :  \GRANULEnt{B}}
    {\Gamma  \GRANULEsym{,}  \GRANULEsym{(}  \Omega  \GRANULEsym{,}   \GRANULEmv{x_{{\mathrm{3}}}}  :   \GRANULEnt{A}  \, \otimes \,  \GRANULEnt{B}    \GRANULEsym{)}  \vdash  \GRANULEnt{C}  \Rightarrow^+   \textbf{let} \, ( \GRANULEmv{x_{{\mathrm{1}}}} ,  \GRANULEmv{x_{{\mathrm{2}}}} ) =  \GRANULEmv{x_{{\mathrm{3}}}}  \, \textbf{in} \,  \GRANULEnt{t_{{\mathrm{2}}}}   ;\,  \Delta  \GRANULEsym{,}   \GRANULEmv{x_{{\mathrm{3}}}}  :   \GRANULEnt{A}  \, \otimes \,  \GRANULEnt{B}}
          \]
        \item Case \addSumElimName \\
          In the case of the left asynchronous rule for sum elimination, the synthesis rule has the form:
          \[
          \fAddSumElimRule
          \]
          By induction on the premises, we have that:
          \[
           \GRANULEsym{(}  \Gamma  \GRANULEsym{,}  \Omega  \GRANULEsym{)}  \GRANULEsym{,}   \GRANULEmv{x_{{\mathrm{2}}}}  :  \GRANULEnt{A}   \vdash  \GRANULEnt{C}  \Rightarrow^+  \GRANULEnt{t_{{\mathrm{1}}}}  ;\,  \Delta_{{\mathrm{1}}}  \GRANULEsym{,}   \GRANULEmv{x_{{\mathrm{2}}}}  :  \GRANULEnt{A}   \tag{ih1}
          \]
          \[
           \GRANULEsym{(}  \Gamma  \GRANULEsym{,}  \Omega  \GRANULEsym{)}  \GRANULEsym{,}   \GRANULEmv{x_{{\mathrm{3}}}}  :  \GRANULEnt{B}   \vdash  \GRANULEnt{C}  \Rightarrow^+  \GRANULEnt{t_{{\mathrm{2}}}}  ;\,  \Delta_{{\mathrm{2}}}  \GRANULEsym{,}   \GRANULEmv{x_{{\mathrm{3}}}}  :  \GRANULEnt{B}   \tag{ih2}
          \]
          from case 2 of the lemma. From which, we can construct the following instatiation of the \addSumElimName\ synthesis rule in the non-focusing calculus:
          \[
    \inferrule*[right=L$\oplus^{+}$]
    {\GRANULEsym{(}  \Gamma  \GRANULEsym{,}  \Omega  \GRANULEsym{)}  \GRANULEsym{,}   \GRANULEmv{x_{{\mathrm{2}}}}  :  \GRANULEnt{A}   \vdash  \GRANULEnt{C}  \Rightarrow^+  \GRANULEnt{t_{{\mathrm{1}}}}  ;\,  \Delta_{{\mathrm{1}}}  \GRANULEsym{,}   \GRANULEmv{x_{{\mathrm{2}}}}  :  \GRANULEnt{A} \\ \GRANULEsym{(}  \Gamma  \GRANULEsym{,}  \Omega  \GRANULEsym{)}  \GRANULEsym{,}   \GRANULEmv{x_{{\mathrm{3}}}}  :  \GRANULEnt{B}   \vdash  \GRANULEnt{C}  \Rightarrow^+  \GRANULEnt{t_{{\mathrm{2}}}}  ;\,  \Delta_{{\mathrm{2}}}  \GRANULEsym{,}   \GRANULEmv{x_{{\mathrm{3}}}}  :  \GRANULEnt{B}}{\Gamma  \GRANULEsym{,}  \GRANULEsym{(}  \Omega  \GRANULEsym{,}   \GRANULEmv{x_{{\mathrm{1}}}}  :   \GRANULEnt{A}  \, \oplus \,  \GRANULEnt{B}    \GRANULEsym{)} \vdash C \Rightarrow^{+} \textbf{case} \ x_{1}\ \textbf{of}\ \textbf{inl}\ x_{2} \rightarrow t_{1};\ \textbf{inr}\ x_{3} \rightarrow t_{2}\ |\  (\Delta_{{\mathrm{1}}} \sqcup \Delta_{{\mathrm{2}}}), \GRANULEmv{x_{{\mathrm{1}}}}  :   \GRANULEnt{A}  \, \oplus \,  \GRANULEnt{B}}
          \]
        \item Case \addUnitElimName \\
          In the case of the left asynchronous rule for unit elimination, the synthesis rule has the form:
          \[
          \fAddUnitElimRule
          \]
          By induction on the premise, we have that:
          \[
           \Gamma  \vdash  \GRANULEnt{C}  \Rightarrow^+  \GRANULEnt{t}  ;\,  \Delta   \tag{ih}
          \]
          from case 2 of the lemma. From which, we can construct the following instatiation of the \addUnitElimName\ synthesis rule in the non-focusing calculus:
          \[
    \inferrule*[right=L1$^{+}$]
    {\Gamma  \vdash  \GRANULEnt{C}  \Rightarrow^+  \GRANULEnt{t}  ;\,  \Delta}
    {\Gamma  \GRANULEsym{,}   \GRANULEmv{x}  :   \mathsf{1}    \vdash  \GRANULEnt{C}  \Rightarrow^+  \GRANULEkw{let} \, \GRANULEsym{()}  \GRANULEsym{=}  \GRANULEmv{x} \, \GRANULEkw{in} \, \GRANULEnt{t}  ;\,  \Delta  \GRANULEsym{,}   \GRANULEmv{x}  :   \mathsf{1}}
          \]
        \item Case \addUnboxName \\
          In the case of the left asynchronous rule for graded modality elimination, the synthesis rule has the form:
          \[
          \fAddUnboxRule
          \]
          By induction on the first premise, we have that:
          \[
            \GRANULEsym{(}  \Gamma  \GRANULEsym{,}  \Omega  \GRANULEsym{)}  \GRANULEsym{,}   \GRANULEmv{x_{{\mathrm{2}}}}  :_{\textcolor{coeffectColor}{  \GRANULEnt{r}  } }   \GRANULEnt{A}   \vdash  \GRANULEnt{B}  \Rightarrow^+  \GRANULEnt{t}  ;\,  \Delta \tag{ih}
          \]
          from case 2 of the lemma. From which, we can construct the following instatiation of the \addUnboxName\ synthesis rule in the non-focusing calculus:
          \[
    \inferrule*[right=L$\square^{+}$]
    {\GRANULEsym{(}  \Gamma  \GRANULEsym{,}  \Omega  \GRANULEsym{)}  \GRANULEsym{,}   \GRANULEmv{x_{{\mathrm{2}}}}  :_{\textcolor{coeffectColor}{  \GRANULEnt{r}  } }   \GRANULEnt{A}   \vdash  \GRANULEnt{B}  \Rightarrow^+  \GRANULEnt{t}  ;\,  \Delta \\ \textit{if}\ \GRANULEmv{x_{{\mathrm{2}}}}  :_{\textcolor{coeffectColor}{  \GRANULEnt{s}  } }   \GRANULEnt{A} \in
      \Delta\ \textit{then}\ \GRANULEnt{s} \sqsubseteq \GRANULEnt{r}\ \textit{else}\ 0 \sqsubseteq \GRANULEnt{r}}{\Gamma  \GRANULEsym{,}  \GRANULEsym{(}  \Omega  \GRANULEsym{,}   \GRANULEmv{x_{{\mathrm{1}}}}  :   \Box_{  \GRANULEnt{r}  }  \GRANULEnt{A}    \GRANULEsym{)}  \vdash  \GRANULEnt{B}  \Rightarrow^+   \textbf{let} \, [  \GRANULEmv{x_{{\mathrm{2}}}}  ] =  \GRANULEmv{x_{{\mathrm{1}}}}  \, \textbf{in} \,  \GRANULEnt{t}   ;\,  \GRANULEsym{(}   \Delta \!\setminus\!  \GRANULEmv{x_{{\mathrm{2}}}}   \GRANULEsym{)}  \GRANULEsym{,}   \GRANULEmv{x_{{\mathrm{1}}}}  :   \Box_{  \GRANULEnt{r}  }  \GRANULEnt{A}}
          \]
        \item Case \addDerName \\
          In the case of the left asynchronous rule for dereliction, the synthesis rule has the form:
          \[
          \fAddDerRule
          \]
          By induction on the premise, we have that:
          \[
           \Gamma  \GRANULEsym{,}   \GRANULEmv{x}  :_{\textcolor{coeffectColor}{  \GRANULEnt{s}  } }   \GRANULEnt{A}   \GRANULEsym{,}   \GRANULEmv{y}  :  \GRANULEnt{A}   \vdash  \GRANULEnt{B}  \Rightarrow^+  \GRANULEnt{t}  ;\,  \Delta  \GRANULEsym{,}   \GRANULEmv{y}  :  \GRANULEnt{A}   \tag{ih}
          \]
          from case 2 of the lemma. From which, we can construct the following instantiation of the \addDerName\ synthesis rule in the non-focusing calculus:
          \[
\inferrule*[right=der$^{+}$]
{ \Gamma  \GRANULEsym{,}   \GRANULEmv{x}  :_{\textcolor{coeffectColor}{  \GRANULEnt{s}  } }   \GRANULEnt{A}   \GRANULEsym{,}   \GRANULEmv{y}  :  \GRANULEnt{A}   \vdash  \GRANULEnt{B}  \Rightarrow^+  \GRANULEnt{t}  ;\,  \Delta  \GRANULEsym{,}   \GRANULEmv{y}  :  \GRANULEnt{A} }
{ \Gamma  \GRANULEsym{,}   \GRANULEmv{x}  :_{\textcolor{coeffectColor}{  \GRANULEnt{s}  } }   \GRANULEnt{A}   \vdash  \GRANULEnt{B}  \Rightarrow^+   [  \GRANULEmv{x}  /  \GRANULEmv{y}  ]  \GRANULEnt{t}   ;\,  \Delta  \GRANULEsym{+}   \GRANULEmv{x}  :_{\textcolor{coeffectColor}{   1   } }   \GRANULEnt{A} }
          \]
        \item Case \fAddLAsyncTransitionName \\
          In the case of the left asynchronous rule for transitioning an assumption from the focusing context $\Omega$ to the non-focusing context $\Gamma$, the synthesis rule has the form:
          \[
            \fAddLAsyncTransitionRule
          \]
          By induction on the first premise, we have that:
          \[
            \Gamma  \GRANULEsym{,}   \GRANULEmv{x}  :  \GRANULEnt{A}    \GRANULEsym{,}  \Omega  \vdash  \GRANULEnt{C}  \Rightarrow^+  \GRANULEnt{t}  ;\,  \Delta \tag{ih}
          \]
          from case 2 of the lemma.
      \end{enumerate}
    \item Case 3. Right Sync: \\
      \begin{enumerate}
        \item Case \addPairIntroName \\
          In the case of the right synchronous rule for pair introduction, the synthesis rule has the form:
          \[
          \fAddPairIntroRuleNoLabel
          \]
          By induction on the premises, we have that:
          \[
           \Gamma  \vdash  \GRANULEnt{A}  \Rightarrow^+  \GRANULEnt{t_{{\mathrm{1}}}}  ;\,  \Delta_{{\mathrm{1}}}   \tag{ih1}
          \]
          \[
           \Gamma  \vdash  \GRANULEnt{B}  \Rightarrow^+  \GRANULEnt{t_{{\mathrm{2}}}}  ;\,  \Delta_{{\mathrm{2}}}  \tag{ih2}
          \]
          from case 3 of the lemma. From which, we can construct the following instantiation of the \addPairIntroName\ synthesis rule in the non-focusing calculus:
          \[
    \inferrule*[right=R$\otimes^{+}$]
    {\Gamma  \vdash  \GRANULEnt{A}  \Rightarrow^+  \GRANULEnt{t_{{\mathrm{1}}}}  ;\,  \Delta_{{\mathrm{1}}} \\ \Gamma  \vdash  \GRANULEnt{B}  \Rightarrow^+  \GRANULEnt{t_{{\mathrm{2}}}}  ;\,  \Delta_{{\mathrm{2}}}}
    {\Gamma  \vdash   \GRANULEnt{A}  \, \otimes \,  \GRANULEnt{B}   \Rightarrow^+   ( \GRANULEnt{t_{{\mathrm{1}}}} ,  \GRANULEnt{t_{{\mathrm{2}}}} )   ;\,  \Delta_{{\mathrm{1}}}  \GRANULEsym{+}  \Delta_{{\mathrm{2}}}}
          \]
        \item Case \addSumIntroLName\ and \addSumIntroRName\\
          In the case of the right synchronous rules for sum introduction, the synthesis rules have the form:
          \[
          \fAddSumIntroRuleL
          \]
          \[
          \fAddSumIntroRuleR
          \]
          By induction on the premises of the rules, we have that:
          \[
           \Gamma  \vdash  \GRANULEnt{A}  \Rightarrow^+  \GRANULEnt{t}  ;\,  \Delta   \tag{ih1}
          \]
          \[
           \Gamma  \vdash  \GRANULEnt{B}  \Rightarrow^+  \GRANULEnt{t}  ;\,  \Delta  \tag{ih2}
          \]
          from case 3 of the lemma. From which, we can construct the following instantiations of the \addSumIntroLName\ and \addSumIntroRName\ synthesis rules in the non-focusing calculus, respectively:
          \[
    \inferrule*[right=R$\oplus_{1}^{+}$]
    {\Gamma  \vdash  \GRANULEnt{A}  \Rightarrow^+  \GRANULEnt{t}  ;\,  \Delta}
    {\Gamma  \vdash   \GRANULEnt{A}  \, \oplus \,  \GRANULEnt{B}   \Rightarrow^+  \GRANULEkw{inl} \, \GRANULEnt{t}  ;\,  \Delta}
          \]
          \[
    \inferrule*[right=R$\oplus_{2}^{+}$]
    {\Gamma  \vdash  \GRANULEnt{B}  \Rightarrow^+  \GRANULEnt{t}  ;\,  \Delta}
    {\Gamma  \vdash   \GRANULEnt{A}  \, \oplus \,  \GRANULEnt{B}   \Rightarrow^+  \GRANULEkw{inr} \, \GRANULEnt{t}  ;\,  \Delta}
          \]
        \item Case \addUnitIntroName \\
          In the case of the right synchronous rule for unit introduction, the synthesis rule has the form:
          \[
          \fAddUnitIntroRule
          \]
          From which, we can construct the following instantiation of the \addUnitIntroName\  synthesis rule in the non-focusing calculus:
          \[
    \inferrule*[right=R1$^{+}$]
    {\quad}
    {\Gamma  \vdash   \mathsf{1}   \Rightarrow^+  \GRANULEsym{()}  ;\,   \emptyset}
          \]
        \item Case \addBoxName \\
          In the case of the right synchronous rule for graded modality introduction, the synthesis rule has the form:
          \[
          \fAddBoxRule
          \]
          By induction on the premise, we have that:
          \[
           \Gamma  \vdash  \GRANULEnt{A}  \Rightarrow^+  \GRANULEnt{t}  ;\,  \Delta   \tag{ih}
          \]
          from case 1 of the lemma. From which, we can construct the following instantiation of the \addBoxName\ synthesis rule in the non-focusing calculus:
          \[
    \inferrule*[right=R$\square^{+}$]
    {\Gamma  \vdash  \GRANULEnt{A}  \Rightarrow^+  \GRANULEnt{t}  ;\,  \Delta}{\Gamma  \vdash   \Box_{  \GRANULEnt{r}  }  \GRANULEnt{A}   \Rightarrow^+  \GRANULEsym{[}  \GRANULEnt{t}  \GRANULEsym{]}  ;\,   \textcolor{coeffectColor}{ \GRANULEnt{r}   \textcolor{coeffectColor}{\,\cdot\,} }  \Delta}
          \]
      \item Case \fAddRSyncTransitionName \\
          In the case of the right synchronous rule for transitioning back to an asynchronous judgement, the synthesis rule has the form:
          \[
            \fAddRSyncTransitionRule
          \]
          By induction on the premise, we have that:
          \[
            \Gamma  \vdash  \GRANULEnt{A}  \Rightarrow^+  \GRANULEnt{t}  ;\,  \Delta \tag{ih}
          \]
          from case 1 of the lemma.
      \end{enumerate}
    \item Case 4. Left Sync \\
      \begin{enumerate}
          \item Case \addAppName \\
          In the case of the left synchronous rule for application, the synthesis rule has the form:
          \[
          \fAddAppRuleNoLabel
          \]
          By induction on the first premise, we have that:
          \[
            \Gamma  \GRANULEsym{,}   \GRANULEmv{x_{{\mathrm{2}}}}  :  \GRANULEnt{B}   \vdash  \GRANULEnt{C}  \Rightarrow^+  \GRANULEnt{t_{{\mathrm{1}}}}  ;\,  \Delta_{{\mathrm{1}}}  \GRANULEsym{,}   \GRANULEmv{x_{{\mathrm{2}}}}  :  \GRANULEnt{B} \tag{ih1}
          \]
          from case 4 of the lemma. By induction on the second premise, we have that:
          \[
            \Gamma  \vdash  \GRANULEnt{A}  \Rightarrow^+  \GRANULEnt{t_{{\mathrm{2}}}}  ;\,  \Delta_{{\mathrm{2}}} \tag{ih2}
          \]
          from case 3 of the lemma. From which, we can construct the following instantiation of the \addAppName synthesis rule in the non-focusing calculus:
          \[
    \inferrule*[right=L$\multimap^{+}$]
    {\Gamma  \GRANULEsym{,}   \GRANULEmv{x_{{\mathrm{2}}}}  :  \GRANULEnt{B}   \vdash  \GRANULEnt{C}  \Rightarrow^+  \GRANULEnt{t_{{\mathrm{1}}}}  ;\,  \Delta_{{\mathrm{1}}}  \GRANULEsym{,}   \GRANULEmv{x_{{\mathrm{2}}}}  :  \GRANULEnt{B} \\ \Gamma  \vdash  \GRANULEnt{A}  \Rightarrow^+  \GRANULEnt{t_{{\mathrm{2}}}}  ;\,  \Delta_{{\mathrm{2}}}}{\Gamma  \GRANULEsym{,}   \GRANULEmv{x_{{\mathrm{1}}}}  :   \GRANULEnt{A}  \multimap  \GRANULEnt{B}    \vdash  \GRANULEnt{C}  \Rightarrow^+   [  \GRANULEsym{(}  \GRANULEmv{x_{{\mathrm{1}}}} \, \GRANULEnt{t_{{\mathrm{2}}}}  \GRANULEsym{)}  /  \GRANULEmv{x_{{\mathrm{2}}}}  ]  \GRANULEnt{t_{{\mathrm{1}}}}   ;\,  \GRANULEsym{(}  \Delta_{{\mathrm{1}}}  \GRANULEsym{+}  \Delta_{{\mathrm{2}}}  \GRANULEsym{)}  \GRANULEsym{,}   \GRANULEmv{x_{{\mathrm{1}}}}  :   \GRANULEnt{A}  \multimap  \GRANULEnt{B} }
          \]
          \item Case \addLinVarName \\
          In the case of the left synchronous rule for linear variable synthesis, the synthesis rule has the form:
          \[
          \fAddLinVarRule
          \]
          From which, we can construct the following instantiation of the \addLinVarName\ in the non-focusing calculus:
          \[
    \inferrule*[right=LinVar$^{+}$]
    {\quad}
    {\Gamma  \GRANULEsym{,}   \GRANULEmv{x}  :  \GRANULEnt{A}   \vdash  \GRANULEnt{A}  \Rightarrow^+  \GRANULEmv{x}  ;\,   \GRANULEmv{x}  :  \GRANULEnt{A}}
          \]
          \item Case \addGrVarName \\
          In the case of the left synchronous rule for graded variable synthesis, the synthesis rule has the form:
          \[
          \fAddGrVarRule
          \]
          From which, we can construct the following instantiation of the \addGrVarName\ synthesis rule in the non-focusing calculus:
          \[
    \inferrule*[right=GrVar$^{+}$]
    {\quad}
      {\Gamma  \GRANULEsym{,}   \GRANULEmv{x}  :_{\textcolor{coeffectColor}{  \GRANULEnt{r}  } }   \GRANULEnt{A}   \vdash  \GRANULEnt{A}  \Rightarrow^+  \GRANULEmv{x}  ;\,   \GRANULEmv{x}  :_{\textcolor{coeffectColor}{   1   } }   \GRANULEnt{A} }
          \]
      \item Case \fAddLSyncTransitionName \\
          In the case of the left synchronous rule for transitioning back to an asynchronous judgement, the synthesis rule has the form:
          \[
            \fAddLSyncTransitionRule
          \]
          By induction on the premise, we have that:
          \[
            \Gamma  \GRANULEsym{,}   \GRANULEmv{x}  :  \GRANULEnt{A}   \vdash  \GRANULEnt{C}  \Rightarrow^+  \GRANULEnt{t}  ;\,  \Delta \tag{ih}
          \]
          from case 2 of the lemma.
      \end{enumerate}
    \item Case 5. Focus Right: \fAddFocusRName \\
          In the case of the focusing rule for transitioning from a left asynchronous judgement to a right synchronous judgement, the synthesis rule has the form:
          \[
            \fAddFocusRRuleNoLabel
          \]
          By induction on the first premise, we have that:
          \[
            \Gamma  \vdash  \GRANULEnt{C}  \Rightarrow^+  \GRANULEnt{t}  ;\,  \Delta \tag{ih}
          \]
          from case 2 of the lemma.
    \item Case 6. Focus Left: \fAddFocusLName \\
          In the case of the focusing rule for transitioning from a left asynchronous judgement to a left synchronous judgement, the synthesis rule has the form:
          \[
            \fAddFocusLRule
          \]
          By induction on the first premise, we have that:
          \[
            \Gamma  \GRANULEsym{,}   \GRANULEmv{x}  :  \GRANULEnt{A}   \vdash  \GRANULEnt{C}  \Rightarrow^+  \GRANULEnt{t}  ;\,  \Delta \tag{ih}
          \]
          from case 2 of the lemma.

  \end{enumerate}
\end{proof}

%\focusSoundAddPruning*
\begin{restatable}[Soundness of focusing for additive pruning synthesis]{lemma}{focusSoundAddPruning}
For all contexts $\Gamma$, $\Omega$ and types $\GRANULEnt{A}$
then:
\begin{align*}
\begin{array}{lll}
 1.\ Right\ Async: & \Gamma  ;  \Omega  \vdash   \GRANULEnt{A}  \Uparrow\   \Rightarrow^{+}  \GRANULEnt{t}  \ | \  \Delta \quad &\implies \quad \Gamma  \GRANULEsym{,}  \,  \GRANULEsym{,}  \Omega  \vdash  \GRANULEnt{A}  \Rightarrow^+  \GRANULEnt{t}  ;\,  \Delta\\
 2.\ Left\ Async: & \Gamma  ;   \Omega  \Uparrow\   \vdash  \GRANULEnt{C}  \Rightarrow^{+}  \GRANULEnt{t}  \ | \  \Delta \quad &\implies \quad \Gamma  \GRANULEsym{,}  \,  \GRANULEsym{,}  \Omega  \vdash  \GRANULEnt{C}  \Rightarrow^+  \GRANULEnt{t}  ;\,  \Delta\\
 3.\ Right\ Sync: & \Gamma  ;   \emptyset   \vdash   \GRANULEnt{A}  \Downarrow\   \Rightarrow^{+}  \GRANULEnt{t}  \ | \  \Delta \quad &\implies \quad \Gamma  \vdash  \GRANULEnt{A}  \Rightarrow^+  \GRANULEnt{t}  ;\,  \Delta\\
 4.\ Left\ Sync: & \Gamma  ;     \GRANULEmv{x}  :  \GRANULEnt{A}    \Downarrow\   \vdash  \GRANULEnt{C}  \Rightarrow^{+}  \GRANULEnt{t}  \ | \  \Delta \quad &\implies \quad \Gamma  \GRANULEsym{,}   \GRANULEmv{x}  :  \GRANULEnt{A}   \vdash  \GRANULEnt{C}  \Rightarrow^+  \GRANULEnt{t}  ;\,  \Delta\\
 5.\ Focus\ Right: & \Gamma  ;   \Omega  \Uparrow\   \vdash  \GRANULEnt{C}  \Rightarrow^{+}  \GRANULEnt{t}  \ | \  \Delta \quad &\implies \quad \Gamma  \vdash  \GRANULEnt{C}  \Rightarrow^+  \GRANULEnt{t}  ;\,  \Delta\\
 6.\ Focus\ Left: & \Gamma  \GRANULEsym{,}   \GRANULEmv{x}  :  \GRANULEnt{A}   ;   \Omega  \Uparrow\   \vdash  \GRANULEnt{C}  \Rightarrow^{+}  \GRANULEnt{t}  \ | \  \Delta \quad &\implies \quad \Gamma  \vdash  \GRANULEnt{C}  \Rightarrow^+  \GRANULEnt{t}  ;\,  \Delta
\end{array}
\end{align*}
\end{restatable}
\begin{proof}
  \begin{enumerate}
      \item Case: 1. Right Async: The proofs for right asynchronous rules are equivalent to those of lemma  \eqref{lemma:fAddSynthSound}\\
    \item Case 2. Left Async: The proofs for left asynchronous rules are equivalent to those of lemma \eqref{lemma:fAddSynthSound}\\
    \item Case 3. Right Sync: The proofs for right synchronous rules are equivalent to those of lemma \eqref{lemma:fAddSynthSound}, except for the case of the \addPruningPairIntroName rule:\\
      \begin{enumerate}
        \item Case \addPruningPairIntroName \\
          In the case of the right synchronous rule for pair introduction, the synthesis rule has the form:
          \[
          \fAddAltPairIntroRule
          \]
          By induction on the premises, we have that:
          \[
           \Gamma  \vdash  \GRANULEnt{A}  \Rightarrow^+  \GRANULEnt{t_{{\mathrm{1}}}}  ;\,  \Delta_{{\mathrm{1}}}   \tag{ih1}
          \]
          \[
           \Gamma  \GRANULEsym{-}  \Delta_{{\mathrm{1}}}  \vdash  \GRANULEnt{B}  \Rightarrow^+  \GRANULEnt{t_{{\mathrm{2}}}}  ;\,  \Delta_{{\mathrm{2}}}   \tag{ih2}
          \]
          from case 3 of the lemma. From which, we can construct the following instantiation of the \addPruningPairIntroName\ synthesis rule in the non-focusing calculus:
          \[
    \inferrule*[right=R$^{\prime}{\otimes^{+}}$]
    {\Gamma  \vdash  \GRANULEnt{A}  \Rightarrow^+  \GRANULEnt{t_{{\mathrm{1}}}}  ;\,  \Delta_{{\mathrm{1}}} \\ \Gamma  \GRANULEsym{-}  \Delta_{{\mathrm{1}}}  \vdash  \GRANULEnt{B}  \Rightarrow^+  \GRANULEnt{t_{{\mathrm{2}}}}  ;\,  \Delta_{{\mathrm{2}}}}
    {\Gamma  \vdash   \GRANULEnt{A}  \, \otimes \,  \GRANULEnt{B}   \Rightarrow^+   ( \GRANULEnt{t_{{\mathrm{1}}}} ,  \GRANULEnt{t_{{\mathrm{2}}}} )   ;\,  \Delta_{{\mathrm{1}}}  \GRANULEsym{+}  \Delta_{{\mathrm{2}}}}
          \]
      \end{enumerate}
    \item Case 4. Left Sync: The proofs for left synchronous rules are equivalent to those of lemma  \eqref{lemma:fAddSynthSound}, except for the case of the \addPruningAppName\  rule:\\\\
      \begin{enumerate}
          \item Case \addPruningAppName \\
          In the case of the left synchronous rule for application, the synthesis rule has the form:
          \[
          \fAddAltAppRule
          \]
          By induction on the first premise, we have that:
          \[
            \Gamma  \GRANULEsym{,}   \GRANULEmv{x_{{\mathrm{2}}}}  :  \GRANULEnt{B}   \vdash  \GRANULEnt{C}  \Rightarrow^+  \GRANULEnt{t_{{\mathrm{1}}}}  ;\,  \Delta_{{\mathrm{1}}}  \GRANULEsym{,}   \GRANULEmv{x_{{\mathrm{2}}}}  :  \GRANULEnt{B} \tag{ih1}
          \]
          from case 4 of the lemma. By induction on the second premise, we have that:
          \[
            \Gamma  \vdash  \GRANULEnt{A}  \Rightarrow^+  \GRANULEnt{t_{{\mathrm{2}}}}  ;\,  \Delta_{{\mathrm{2}}} \tag{ih2}
          \]
          from case 3 of the lemma. From which, we can construct the following instantiation of the \addPruningAppName\ synthesis rule in the non-focusing calculus:
          \[
\inferrule*[right=L$^{\prime}\multimap^{+}$]
    {\Gamma  \GRANULEsym{,}   \GRANULEmv{x_{{\mathrm{2}}}}  :  \GRANULEnt{B}   \vdash  \GRANULEnt{C}  \Rightarrow^+  \GRANULEnt{t_{{\mathrm{1}}}}  ;\,  \Delta_{{\mathrm{1}}}  \GRANULEsym{,}   \GRANULEmv{x_{{\mathrm{2}}}}  :  \GRANULEnt{B} \\ \Gamma  \GRANULEsym{-}  \Delta_{{\mathrm{1}}}  \vdash  \GRANULEnt{A}  \Rightarrow^+  \GRANULEnt{t_{{\mathrm{2}}}}  ;\,  \Delta_{{\mathrm{2}}}}{\Gamma  \GRANULEsym{,}   \GRANULEmv{x_{{\mathrm{1}}}}  :   \GRANULEnt{A}  \multimap  \GRANULEnt{B}    \vdash  \GRANULEnt{C}  \Rightarrow^+   [  \GRANULEsym{(}  \GRANULEmv{x_{{\mathrm{1}}}} \, \GRANULEnt{t_{{\mathrm{2}}}}  \GRANULEsym{)}  /  \GRANULEmv{x_{{\mathrm{2}}}}  ]  \GRANULEnt{t_{{\mathrm{1}}}}   ;\,  \GRANULEsym{(}  \Delta_{{\mathrm{1}}}  \GRANULEsym{+}  \Delta_{{\mathrm{2}}}  \GRANULEsym{)}  \GRANULEsym{,}   \GRANULEmv{x_{{\mathrm{1}}}}  :   \GRANULEnt{A}  \multimap  \GRANULEnt{B} }
          \]
      \end{enumerate}
    \item Case 5. Right Focus: \fAddFocusRName\ - The proof for right focusing rule is equivalent to that of lemma \eqref{lemma:fAddSynthSound}\\
    \item Case 6. Left Focus: \fAddFocusLName\ - The proof for left focusing rule is equivalent to that of lemma \eqref{lemma:fAddSynthSound}\\
  \end{enumerate}
\end{proof}

% \synthSound*
\begin{proof}
Induction on the synthesis rules
\begin{enumerate}
\item Case $\textsc{Var}$ \\
        For synthesis of a variable term, we have the derivation:
        \begin{align*}
          \synVar
        \end{align*}
        from which we can construct the following typing derivation, matching the above conclusion:
        \begin{align*}
          \tyVar
        \end{align*}
\item Case $\multimap_{R}$\\
        For synthesis of an abstraction term, we have the derivation:
        \begin{align*}
          \synAbs
        \end{align*}
        By induction on the premise, we have:
        \begin{align*}
          \Delta  \GRANULEsym{,}   \GRANULEmv{x}  :_{\textcolor{coeffectColor}{  \GRANULEnt{r}  } }   \GRANULEnt{A}   \vdash  \GRANULEnt{t}  :  \GRANULEnt{B}
        \end{align*}
        and that:
        \begin{align*}
          \GRANULEnt{r}  \, \textcolor{coeffectColor}{\sqsubseteq} \,  \GRANULEnt{q}
        \end{align*}
        from which we can construct the following typing derivation, matching the conclusion:
        \begin{align*}
          \inferrule*[Right=Abs]
            {\inferrule*[Right=Approx]{\Delta  \GRANULEsym{,}   \GRANULEmv{x}  :_{\textcolor{coeffectColor}{  \GRANULEnt{r}  } }   \GRANULEnt{A}   \vdash  \GRANULEnt{t}  :  \GRANULEnt{B} \\ \GRANULEnt{r}  \, \textcolor{coeffectColor}{\sqsubseteq} \,  \GRANULEnt{q}}{\Delta  \GRANULEsym{,}   \GRANULEmv{x}  :_{\textcolor{coeffectColor}{  \GRANULEnt{q}  } }   \GRANULEnt{A}   \vdash  \GRANULEnt{t}  :  \GRANULEnt{B}}}
            {\Delta  \vdash   \lambda  \GRANULEmv{x}  .  \GRANULEnt{t}   :   \GRANULEnt{A} ^ \GRANULEnt{q}  \rightarrow  \GRANULEnt{B}}
        \end{align*}
\item Case $\multimap_{L}$\\
        For synthesising an application, we have the derivation:
        \begin{align*}
          \synApp
        \end{align*}
        By induction on the premises, we obtain the following typing judgements:
        \begin{align*}
          \Delta_{{\mathrm{1}}}  \GRANULEsym{,}   \GRANULEmv{x_{{\mathrm{1}}}}  :_{\textcolor{coeffectColor}{  \GRANULEnt{s_{{\mathrm{1}}}}  } }    \GRANULEnt{A} ^ \GRANULEnt{q}  \rightarrow  \GRANULEnt{B}    \GRANULEsym{,}   \GRANULEmv{x_{{\mathrm{2}}}}  :_{\textcolor{coeffectColor}{  \GRANULEnt{s_{{\mathrm{2}}}}  } }   \GRANULEnt{B}   \vdash  \GRANULEnt{t_{{\mathrm{1}}}}  :  \GRANULEnt{C} \\
          \Delta_{{\mathrm{2}}}  \GRANULEsym{,}   \GRANULEmv{x_{{\mathrm{1}}}}  :_{\textcolor{coeffectColor}{  \GRANULEnt{s_{{\mathrm{3}}}}  } }    \GRANULEnt{A} ^ \GRANULEnt{q}  \rightarrow  \GRANULEnt{B}    \vdash  \GRANULEnt{t_{{\mathrm{2}}}}  :  \GRANULEnt{A}
        \end{align*}
        from which we can construct the following derivation, making use of the admissibility of substitution:
        \begin{align*}
          \inferrule*[Right=Subst]{
          {\inferrule*[Right=App]{\inferrule*[Right=Var]{\quad}{\GRANULEmv{x_{{\mathrm{1}}}}  :_{\textcolor{coeffectColor}{   1   } }    \GRANULEnt{A} ^ \GRANULEnt{q}  \rightarrow  \GRANULEnt{B}    \vdash  \GRANULEmv{x_{{\mathrm{1}}}}  :   \GRANULEnt{A} ^ \GRANULEnt{q}  \rightarrow  \GRANULEnt{B}} \\ \Delta_{{\mathrm{2}}}  \GRANULEsym{,}   \GRANULEmv{x_{{\mathrm{1}}}}  :_{\textcolor{coeffectColor}{  \GRANULEnt{s_{{\mathrm{3}}}}  } }    \GRANULEnt{A} ^ \GRANULEnt{q}  \rightarrow  \GRANULEnt{B}    \vdash  \GRANULEnt{t_{{\mathrm{2}}}}  :  \GRANULEnt{A}}{ \textcolor{coeffectColor}{ \GRANULEnt{q}   \textcolor{coeffectColor}{\,\cdot\,} }  \Delta_{{\mathrm{2}}}    \GRANULEsym{,}   \GRANULEmv{x_{{\mathrm{1}}}}  :_{\textcolor{coeffectColor}{    1   \GRANULEsym{+}  \GRANULEsym{(}  \GRANULEnt{q}  \textcolor{coeffectColor}{\,\cdot\,}  \GRANULEnt{s_{{\mathrm{3}}}}  \GRANULEsym{)}   } }    \GRANULEnt{A} ^ \GRANULEnt{q}  \rightarrow  \GRANULEnt{B}    \vdash  \GRANULEmv{x_{{\mathrm{1}}}} \, \GRANULEnt{t_{{\mathrm{2}}}}  :  \GRANULEnt{B}}} \\ \Delta_{{\mathrm{1}}}  \GRANULEsym{,}   \GRANULEmv{x_{{\mathrm{1}}}}  :_{\textcolor{coeffectColor}{  \GRANULEnt{s_{{\mathrm{1}}}}  } }    \GRANULEnt{A} ^ \GRANULEnt{q}  \rightarrow  \GRANULEnt{B}    \GRANULEsym{,}   \GRANULEmv{x_{{\mathrm{2}}}}  :_{\textcolor{coeffectColor}{  \GRANULEnt{s_{{\mathrm{2}}}}  } }   \GRANULEnt{B}   \vdash  \GRANULEnt{t_{{\mathrm{1}}}}  :  \GRANULEnt{C}}
          {\GRANULEsym{(}  \Delta_{{\mathrm{1}}}  \GRANULEsym{+}   \textcolor{coeffectColor}{ \GRANULEnt{s_{{\mathrm{2}}}}   \textcolor{coeffectColor}{\,\cdot\,} }    \textcolor{coeffectColor}{ \GRANULEnt{q}   \textcolor{coeffectColor}{\,\cdot\,} }  \Delta_{{\mathrm{2}}}     \GRANULEsym{)}  \GRANULEsym{,}   \GRANULEmv{x_{{\mathrm{1}}}}  :_{\textcolor{coeffectColor}{   \GRANULEnt{s_{{\mathrm{2}}}}  \GRANULEsym{+}   \GRANULEnt{s_{{\mathrm{1}}}}  \GRANULEsym{+}  \GRANULEsym{(}  \GRANULEnt{s_{{\mathrm{2}}}}  \textcolor{coeffectColor}{\,\cdot\,}   \GRANULEnt{q}  \textcolor{coeffectColor}{\,\cdot\,}  \GRANULEnt{s_{{\mathrm{3}}}}   \GRANULEsym{)}    } }    \GRANULEnt{A} ^ \GRANULEnt{q}  \rightarrow  \GRANULEnt{B}    \vdash    [  \GRANULEsym{(}  \GRANULEmv{x_{{\mathrm{1}}}} \, \GRANULEnt{t_{{\mathrm{2}}}}  \GRANULEsym{)}  /  \GRANULEmv{x_{{\mathrm{2}}}}  ]  \GRANULEnt{t_{{\mathrm{1}}}}    :  \GRANULEnt{C}}
        \end{align*}
        making use of the distributivity property of semirings, along with unitality of $1$
        and commutativity of $+$, such that
        $\GRANULEnt{s_{{\mathrm{1}}}}  \GRANULEsym{+}  \GRANULEnt{s_{{\mathrm{2}}}}  \textcolor{coeffectColor}{\,\cdot\,}  \GRANULEsym{(}   1   \GRANULEsym{+}  \GRANULEsym{(}  \GRANULEnt{q}  \textcolor{coeffectColor}{\,\cdot\,}  \GRANULEnt{s_{{\mathrm{3}}}}  \GRANULEsym{)}  \GRANULEsym{)} =  \GRANULEnt{s_{{\mathrm{1}}}}  \GRANULEsym{+}  \GRANULEsym{(}  \GRANULEnt{s_{{\mathrm{2}}}}  \textcolor{coeffectColor}{\,\cdot\,}   1   \GRANULEsym{)}  \GRANULEsym{+}  \GRANULEsym{(}  \GRANULEnt{s_{{\mathrm{2}}}}  \textcolor{coeffectColor}{\,\cdot\,}  \GRANULEnt{q}  \textcolor{coeffectColor}{\,\cdot\,}  \GRANULEnt{s_{{\mathrm{3}}}}  \GRANULEsym{)} = \GRANULEnt{s_{{\mathrm{2}}}}  \GRANULEsym{+}  \GRANULEnt{s_{{\mathrm{1}}}}  \GRANULEsym{+}  \GRANULEsym{(}  \GRANULEnt{s_{{\mathrm{2}}}}  \textcolor{coeffectColor}{\,\cdot\,}  \GRANULEnt{q}  \textcolor{coeffectColor}{\,\cdot\,}  \GRANULEnt{s_{{\mathrm{3}}}}  \GRANULEsym{)}$.

\item Case $\textsc{Con}_{R}$\\
        For synthesising a constructor introduction, we have the derivation:
        \begin{align*}
          \synCon
        \end{align*}
        By induction on the premises, we obtain the following typing judgements:
        \begin{align*}
          \Delta_{{\mathrm{1}}}  \vdash  \GRANULEnt{t_{{\mathrm{1}}}}  :  \GRANULEnt{B_{{\mathrm{1}}}} \quad , ..., \quad \Delta_{\GRANULEmv{n}}  \vdash  \GRANULEnt{t_{\GRANULEmv{n}}}  :  \GRANULEnt{B_{\GRANULEmv{n}}}
        \end{align*}
        from which we can construct the following derivation, matching the above conclusion:
        \begin{align*}
          \inferrule*[Right=App]{\inferrule*[Right=App]{\inferrule*[right=App, vdots=1em]{\inferrule*[Right=Con]{(  C  :  \GRANULEnt{B_{{\mathrm{1}}}} ^ \GRANULEnt{q_{{\mathrm{1}}}}  \rightarrow ... \rightarrow  \GRANULEnt{B_{\GRANULEmv{n}}} ^ \GRANULEnt{q_{\GRANULEmv{n}}}  \rightarrow     K  \, \GRANULEnt{A}   ) \in  D}{\textcolor{coeffectColor}{  0    \textcolor{coeffectColor}{\,\cdot\,} }  \Gamma   \vdash   C   :   \GRANULEnt{B_{{\mathrm{1}}}} ^{q_1}  \rightarrow \! ... \! \rightarrow   \GRANULEnt{B_{\GRANULEmv{n}}} ^{q_n}  \rightarrow     K  \, \GRANULEnt{A}} \\ \Delta_{{\mathrm{1}}}  \vdash  \GRANULEnt{t_{{\mathrm{1}}}}  :  \GRANULEnt{B_{{\mathrm{1}}}}}{{\textcolor{coeffectColor}{  0    \textcolor{coeffectColor}{\,\cdot\,} }  \Gamma    \GRANULEsym{+}   \textcolor{coeffectColor}{ \GRANULEnt{q_{{\mathrm{1}}}}   \textcolor{coeffectColor}{\,\cdot\,} }  \Delta_{{\mathrm{1}}}   \vdash  C \, \GRANULEnt{t_{{\mathrm{1}}}}  :    \GRANULEnt{B_{{\mathrm{2}}}} ^{q_1}  \rightarrow \! ... \! \rightarrow   \GRANULEnt{B_{\GRANULEmv{n}}} ^{q_n}  \rightarrow     K  \, \GRANULEnt{A}}}}{\textcolor{coeffectColor}{  0    \textcolor{coeffectColor}{\,\cdot\,} }  \Gamma    \GRANULEsym{+}   \textcolor{coeffectColor}{ \GRANULEnt{q_{{\mathrm{1}}}}   \textcolor{coeffectColor}{\,\cdot\,} }  \Delta_{{\mathrm{1}}}   \GRANULEsym{+} \, ... \, \GRANULEsym{+}   \textcolor{coeffectColor}{ \GRANULEnt{q_{{\GRANULEmv{n}-1}}}   \textcolor{coeffectColor}{\,\cdot\,} }  \Delta_{{\GRANULEmv{n}-1}}   \vdash  C \, \GRANULEnt{t_{{\mathrm{1}}}} \, ... \, \GRANULEnt{t_{{\GRANULEmv{n}-1}}}  :   \GRANULEnt{B_{\GRANULEmv{n}}} ^ \GRANULEnt{q_{\GRANULEmv{n}}}  \rightarrow    K  \, \GRANULEnt{A}} \\ \Delta_{\GRANULEmv{n}}  \vdash  \GRANULEnt{t_{\GRANULEmv{n}}}  :  \GRANULEnt{B_{\GRANULEmv{n}}}}{{\textcolor{coeffectColor}{  0    \textcolor{coeffectColor}{\,\cdot\,} }  \Gamma    \GRANULEsym{+}   \textcolor{coeffectColor}{ \GRANULEnt{q_{{\mathrm{1}}}}   \textcolor{coeffectColor}{\,\cdot\,} }  \Delta_{{\mathrm{1}}}+ ...+  \textcolor{coeffectColor}{ \GRANULEnt{q_{\GRANULEmv{n}}}   \textcolor{coeffectColor}{\,\cdot\,} }  \Delta_{\GRANULEmv{n}}   \vdash  C \, \GRANULEnt{t_{{\mathrm{1}}}} \, ... \, \GRANULEnt{t_{\GRANULEmv{n}}}  :   K  \, \GRANULEnt{A}}}
        \end{align*}

\item Case $\textsc{Con}_{L}$ \\
        For synthesising a case statement, we have the derivation:
        \begin{align*}
          \synCaseAlt
        \end{align*}
        By induction on the premises we obtain the following typing judgements:
        \begin{align*}
          \Delta_{\GRANULEmv{i}}  \GRANULEsym{,}   \GRANULEmv{x}  :_{\textcolor{coeffectColor}{   \GRANULEnt{r_{\GRANULEmv{i}}}   } }    K  \, \GRANULEnt{A}    \GRANULEsym{,}   {  \GRANULEmv{y} ^  i  _  1   }:_{\textcolor{coeffectColor}{    \GRANULEnt{s} ^  i  _  1     } }   \GRANULEnt{B_{{\mathrm{1}}}}    , ... ,   {  \GRANULEmv{y} ^  i  _  n   }:_{\textcolor{coeffectColor}{    \GRANULEnt{s} ^  i  _  n     } }   \GRANULEnt{B_{\GRANULEmv{n}}}    \vdash  \GRANULEnt{t_{\GRANULEmv{i}}}  :  \GRANULEnt{B}
        \end{align*}
        % By monotonicty of $\sqcup$ we have that $ \GRANULEnt{s} _  i     \, \textcolor{coeffectColor}{\sqsubseteq} \,      \GRANULEnt{s} _  1     \sqcup ... \sqcup    \GRANULEnt{s} _  n$
        % and by $*_{1}$ that $\GRANULEnt{s} _  i     =      \GRANULEnt{s} ^  i  _  1     \sqcup ... \sqcup    \GRANULEnt{s} ^  i  _  n$. Therefore, by monotonicty of $\sqcup$
        % we have that $\GRANULEnt{s} ^  i  _  j     \, \textcolor{coeffectColor}{\sqsubseteq} \,      \GRANULEnt{s} _  1     \sqcup ... \sqcup    \GRANULEnt{s} _  n$
        % from which we can construct the following
        % derivation, matching the above conclusion:
        We have by the definition of $\sqcup$:
        %
        \begin{enumerate}
                \item $ \Delta_{\GRANULEmv{i}} \sqsubseteq \GRANULEsym{(}   \Delta_{{\mathrm{1}}}  \sqcup ...  \sqcup  \Delta_{\GRANULEmv{m}}   \GRANULEsym{)}$
                \item $ \GRANULEnt{r} _  i     \, \textcolor{coeffectColor}{\sqsubseteq} \,      \GRANULEnt{r} _  1     \sqcup ... \sqcup    \GRANULEnt{r} _  m$
                % \item $\GRANULEnt{s} _  i     =      \GRANULEnt{s} ^  i  _  1     \sqcup ... \sqcup    \GRANULEnt{s} ^  i  _  n$
                % \item $\GRANULEnt{s} ^  i  _  j     \, \textcolor{coeffectColor}{\sqsubseteq} \,      \GRANULEnt{s} _  1     \sqcup ... \sqcup    \GRANULEnt{s} _  n$
        \end{enumerate}

        and from the premises of the synthesis rule:
        \begin{enumerate}[resume]
                \item $\GRANULEnt{s'} ^  i  _  j     \, \textcolor{coeffectColor}{\sqsubseteq} \,      \GRANULEnt{s} _  1     \sqcup ... \sqcup    \GRANULEnt{s} _  m$
                \item $ \GRANULEnt{s} ^  i  _  j     \, \textcolor{coeffectColor}{\sqsubseteq} \,    \GRANULEnt{s'} ^  i  _  j     \textcolor{coeffectColor}{\,\cdot\,}    \GRANULEnt{q} ^  i  _  j$
                \item $ |    K  \, \GRANULEnt{A}   | > 1 \Rightarrow 1 \sqsubseteq      \GRANULEnt{s} _  1     \sqcup ... \sqcup    \GRANULEnt{s} _  m $
        \end{enumerate}

        We then construct the following two derivations towards the goal:
        %
        \begin{equation}
          \label{soundConLSub1}
          \inferrule*[Right=PCon]{\inferrule*[Right=PVar]{\quad}{ \GRANULEnt{q} ^  i  _  j     \textcolor{coeffectColor}{\,\cdot\,}      \GRANULEnt{s} _  1     \sqcup ... \sqcup    \GRANULEnt{s} _  m       \vdash \,   \GRANULEmv{y} ^i_j   :  \GRANULEnt{B_{\GRANULEmv{j}}}  \, \rhd \,   {  \GRANULEmv{y} ^  i  _  j   }:_{\textcolor{coeffectColor}{    \GRANULEnt{q} ^  i  _  j     \textcolor{coeffectColor}{\,\cdot\,}      \GRANULEnt{s} _  1     \sqcup ... \sqcup    \GRANULEnt{s} _  m       } }   \GRANULEnt{B_{\GRANULEmv{j}}} } \\  (e) }{\GRANULEnt{s} _  1     \sqcup ... \sqcup    \GRANULEnt{s} _  m       \vdash \,   C_{\GRANULEmv{i}}  \  \GRANULEmv{y} ^  i  _  1   ...  \GRANULEmv{y} ^  i  _  n    :   K  \, \GRANULEnt{A}  \, \rhd \,    {  \GRANULEmv{y} ^  i  _  j   }:_{\textcolor{coeffectColor}{    \GRANULEnt{q} ^  i  _  j     \textcolor{coeffectColor}{\,\cdot\,}      \GRANULEnt{s} _  1     \sqcup ... \sqcup    \GRANULEnt{s} _  m       } }   \GRANULEnt{B_{\GRANULEmv{j}}}   , ... ,   {  \GRANULEmv{y} ^  i  _  n   }:_{\textcolor{coeffectColor}{    \GRANULEnt{q} ^  i  _  n     \textcolor{coeffectColor}{\,\cdot\,}      \GRANULEnt{s} _  1     \sqcup ... \sqcup    \GRANULEnt{s} _  m       } }   \GRANULEnt{B_{\GRANULEmv{n}}}}\\
        \end{equation}
        and
        \begin{equation}
          \label{soundConLSub2}
          \hspace{-7em}\inferrule*[Right=Approx, rightskip=5em]{ \inferrule*[Right=Approx]{ \inferrule*[Right=induction]{\quad}{\Delta_{\GRANULEmv{i}}  \GRANULEsym{,}   \GRANULEmv{x}  :_{\textcolor{coeffectColor}{   \GRANULEnt{r_{\GRANULEmv{i}}}   } }    K  \, \GRANULEnt{A}    \GRANULEsym{,}   {  \GRANULEmv{y} ^  i  _  1   }:_{\textcolor{coeffectColor}{    \GRANULEnt{s} ^  i  _  1     } }   \GRANULEnt{B_{{\mathrm{1}}}}    , ... ,   {  \GRANULEmv{y} ^  i  _  n   }:_{\textcolor{coeffectColor}{    \GRANULEnt{s} ^  i  _  n     } }   \GRANULEnt{B_{\GRANULEmv{n}}}    \vdash  \GRANULEnt{t_{\GRANULEmv{i}}}  :  \GRANULEnt{B}} \qquad \\ (d) }{\inferrule*[Right=Approx]{ \Delta_{\GRANULEmv{i}}  \GRANULEsym{,}   \GRANULEmv{x}  :_{\textcolor{coeffectColor}{    \GRANULEnt{r} _  i     } }     K  \, \GRANULEnt{A}    \GRANULEsym{,}   {  \GRANULEmv{y} ^  i  _  1   }:_{\textcolor{coeffectColor}{    \GRANULEnt{q} ^  i  _  1     \textcolor{coeffectColor}{\,\cdot\,}    \GRANULEnt{s'} ^  i  _  j     } }   \GRANULEnt{B_{{\mathrm{1}}}}    , ... ,   {  \GRANULEmv{y} ^  i  _  n   }:_{\textcolor{coeffectColor}{    \GRANULEnt{q} ^  i  _  n     \textcolor{coeffectColor}{\,\cdot\,}    \GRANULEnt{s'} ^  i  _  j     } }   \GRANULEnt{B_{\GRANULEmv{n}}}    \vdash  \GRANULEnt{t_{\GRANULEmv{i}}}  :  \GRANULEnt{B} \\ (c)}{\inferrule*[Right=Approx]{ \Delta_{\GRANULEmv{i}}  \GRANULEsym{,}   \GRANULEmv{x}  :_{\textcolor{coeffectColor}{    \GRANULEnt{r} _  i     } }     K  \, \GRANULEnt{A}    \GRANULEsym{,}   {  \GRANULEmv{y} ^  i  _  1   }:_{\textcolor{coeffectColor}{    \GRANULEnt{q} ^  i  _  1     \textcolor{coeffectColor}{\,\cdot\,}      \GRANULEnt{s} _  1     \sqcup ... \sqcup    \GRANULEnt{s} _  m       } }   \GRANULEnt{B_{{\mathrm{1}}}}    , ... ,   {  \GRANULEmv{y} ^  i  _  n   }:_{\textcolor{coeffectColor}{    \GRANULEnt{q} ^  i  _  n     \textcolor{coeffectColor}{\,\cdot\,}      \GRANULEnt{s} _  1     \sqcup ... \sqcup    \GRANULEnt{s} _  m       } }   \GRANULEnt{B_{\GRANULEmv{n}}}    \vdash  \GRANULEnt{t_{\GRANULEmv{i}}}  :  \GRANULEnt{B} \\ (b)}{\Delta_{\GRANULEmv{i}}  \GRANULEsym{,}   \GRANULEmv{x}  :_{\textcolor{coeffectColor}{   \GRANULEsym{(}     \GRANULEnt{r} _  1     \sqcup ... \sqcup    \GRANULEnt{r} _  m      \GRANULEsym{)}   } }     K  \, \GRANULEnt{A}    \GRANULEsym{,}   {  \GRANULEmv{y} ^  i  _  1   }:_{\textcolor{coeffectColor}{    \GRANULEnt{q} ^  i  _  1     \textcolor{coeffectColor}{\,\cdot\,}      \GRANULEnt{s} _  1     \sqcup ... \sqcup    \GRANULEnt{s} _  m       } }   \GRANULEnt{B_{{\mathrm{1}}}}    , ... ,   {  \GRANULEmv{y} ^  i  _  n   }:_{\textcolor{coeffectColor}{    \GRANULEnt{q} ^  i  _  n     \textcolor{coeffectColor}{\,\cdot\,}      \GRANULEnt{s} _  1     \sqcup ... \sqcup    \GRANULEnt{s} _  m       } }   \GRANULEnt{B_{\GRANULEmv{n}}}    \vdash  \GRANULEnt{t_{\GRANULEmv{i}}}  :  \GRANULEnt{B} \\ (a)} }}}{ \GRANULEsym{(}   \Delta_{{\mathrm{1}}}  \sqcup ...  \sqcup  \Delta_{\GRANULEmv{m}}   \GRANULEsym{)}  \GRANULEsym{,}   \GRANULEmv{x}  :_{\textcolor{coeffectColor}{   \GRANULEsym{(}     \GRANULEnt{r} _  1     \sqcup ... \sqcup    \GRANULEnt{r} _  m      \GRANULEsym{)}   } }     K  \, \GRANULEnt{A}    \GRANULEsym{,}   {  \GRANULEmv{y} ^  i  _  1   }:_{\textcolor{coeffectColor}{    \GRANULEnt{q} ^  i  _  1     \textcolor{coeffectColor}{\,\cdot\,}      \GRANULEnt{s} _  1     \sqcup ... \sqcup    \GRANULEnt{s} _  m       } }   \GRANULEnt{B_{{\mathrm{1}}}}    , ... ,   {  \GRANULEmv{y} ^  i  _  n   }:_{\textcolor{coeffectColor}{    \GRANULEnt{q} ^  i  _  n     \textcolor{coeffectColor}{\,\cdot\,}      \GRANULEnt{s} _  1     \sqcup ... \sqcup    \GRANULEnt{s} _  m       } }   \GRANULEnt{B_{\GRANULEmv{n}}}    \vdash  \GRANULEnt{t_{\GRANULEmv{i}}}  :  \GRANULEnt{B} }
        \end{equation}


        \begin{align*}
          \inferrule*[Right=Case]{ \inferrule*[Right=Var,leftskip=5em]{\quad}{\GRANULEmv{x}  :_{\textcolor{coeffectColor}{   1   } }     K  \, \GRANULEnt{A}    \vdash  \GRANULEmv{x}  :    K  \, \GRANULEnt{A}} \quad \\ \eqref{soundConLSub1} \\ \eqref{soundConLSub2} }
          { \inferrule*[Right=$\equiv$] {\GRANULEsym{(}  \GRANULEsym{(}   \Delta_{{\mathrm{1}}}  \sqcup ...  \sqcup  \Delta_{\GRANULEmv{m}}   \GRANULEsym{)}  \GRANULEsym{,}    \GRANULEmv{x}  :_{\textcolor{coeffectColor}{   \GRANULEsym{(}     \GRANULEnt{r} _  1     \sqcup ... \sqcup    \GRANULEnt{r} _  m      \GRANULEsym{)}   } }     K  \, \GRANULEnt{A}     \GRANULEsym{)}  \GRANULEsym{+}   \GRANULEmv{x}  :_{\textcolor{coeffectColor}{   \GRANULEsym{(}      \GRANULEnt{s} _  1     \sqcup ... \sqcup    \GRANULEnt{s} _  m       \textcolor{coeffectColor}{\,\cdot\,}   1   \GRANULEsym{)}   } }     K  \, \GRANULEnt{A}    \vdash   \textbf{case} \  \GRANULEmv{x}  \ \textbf{of} \   \overline{   C_{\GRANULEmv{i}}  \  \GRANULEmv{y} ^  i  _  1   ...  \GRANULEmv{y} ^  i  _  n    \mapsto  \GRANULEnt{t_{\GRANULEmv{i}}} }    :  \GRANULEnt{B}} {\GRANULEsym{(}   \Delta_{{\mathrm{1}}}  \sqcup ...  \sqcup  \Delta_{\GRANULEmv{m}}   \GRANULEsym{)}  \GRANULEsym{,}    \GRANULEmv{x}  :_{\textcolor{coeffectColor}{   \GRANULEsym{(}     \GRANULEnt{r} _  1     \sqcup ... \sqcup    \GRANULEnt{r} _  m      \GRANULEsym{)}   \GRANULEsym{+}   \GRANULEsym{(}      \GRANULEnt{s} _  1     \sqcup ... \sqcup    \GRANULEnt{s} _  m       \GRANULEsym{)}   } }     K  \, \GRANULEnt{A}     \vdash   \textbf{case} \  \GRANULEmv{x}  \ \textbf{of} \   \overline{   C_{\GRANULEmv{i}}  \  \GRANULEmv{y} ^  i  _  1   ...  \GRANULEmv{y} ^  i  _  n    \mapsto  \GRANULEnt{t_{\GRANULEmv{i}}} }    :  \GRANULEnt{B}}}
        \end{align*}



% \item Case $\textsc{Con}_{L}$ \\
%         For synthesising a case statement, we have the derivation:
%         \begin{align*}
%           \synCaseAltAlt
%         \end{align*}
%         By induction on the premises we obtain the following typing judgements:
%         \begin{align*}
%           \Delta_{\GRANULEmv{i}}  \GRANULEsym{,}   \GRANULEmv{x}  :_{\textcolor{coeffectColor}{   \GRANULEnt{r_{\GRANULEmv{i}}}   } }    K  \, \GRANULEnt{A}    \GRANULEsym{,}   {  \GRANULEmv{y} ^  i  _  1   }:_{\textcolor{coeffectColor}{    \GRANULEnt{s} ^  i  _  1     } }   \GRANULEnt{B_{{\mathrm{1}}}}    , ... ,   {  \GRANULEmv{y} ^  i  _  n   }:_{\textcolor{coeffectColor}{    \GRANULEnt{s} ^  i  _  n     } }   \GRANULEnt{B_{\GRANULEmv{n}}}    \vdash  \GRANULEnt{t_{\GRANULEmv{i}}}  :  \GRANULEnt{B}
%         \end{align*}
%         % By monotonicty of $\sqcup$ we have that $ \GRANULEnt{s} _  i     \, \textcolor{coeffectColor}{\sqsubseteq} \,      \GRANULEnt{s} _  1     \sqcup ... \sqcup    \GRANULEnt{s} _  n$
%         % and by $*_{1}$ that $\GRANULEnt{s} _  i     =      \GRANULEnt{s} ^  i  _  1     \sqcup ... \sqcup    \GRANULEnt{s} ^  i  _  n$. Therefore, by monotonicty of $\sqcup$
%         % we have that $\GRANULEnt{s} ^  i  _  j     \, \textcolor{coeffectColor}{\sqsubseteq} \,      \GRANULEnt{s} _  1     \sqcup ... \sqcup    \GRANULEnt{s} _  n$
%         % from which we can construct the following
%         % derivation, matching the above conclusion:
%         We have:

%         \begin{enumerate}
%                 \item $ \Delta_{\GRANULEmv{i}} \sqsubseteq \GRANULEsym{(}   \Delta_{{\mathrm{1}}}  \sqcup ...  \sqcup  \Delta_{\GRANULEmv{n}}   \GRANULEsym{)}$
%                 \item $ \GRANULEnt{r} _  i     \, \textcolor{coeffectColor}{\sqsubseteq} \,      \GRANULEnt{r} _  1     \sqcup ... \sqcup    \GRANULEnt{r} _  n$
%                 % \item $\GRANULEnt{s} _  i     =      \GRANULEnt{s} ^  i  _  1     \sqcup ... \sqcup    \GRANULEnt{s} ^  i  _  n$
%                 % \item $\GRANULEnt{s} ^  i  _  j     \, \textcolor{coeffectColor}{\sqsubseteq} \,      \GRANULEnt{s} _  1     \sqcup ... \sqcup    \GRANULEnt{s} _  n$
%                 \item $\GRANULEnt{s'} _  i     \, \textcolor{coeffectColor}{\sqsubseteq} \,      \GRANULEnt{s'} _  1     \sqcup ... \sqcup    \GRANULEnt{s'} _  n$
%                 \item $ \GRANULEnt{s} _  i     \, \textcolor{coeffectColor}{\sqsubseteq} \,    \GRANULEnt{s'} _  i     \textcolor{coeffectColor}{\,\cdot\,}    \GRANULEnt{q} ^  i  _  j$
%                 \item $\GRANULEnt{s} ^  i  _  j     \, \textcolor{coeffectColor}{\sqsubseteq} \,    \GRANULEnt{s} _  i$
%         \end{enumerate}
%         \begin{equation}
%           \label{soundConLAltSub1}
%           \inferrule*[Right=PCon]{\inferrule*[Right=PVar]{\quad}{ \GRANULEnt{q} ^  i  _  j     \textcolor{coeffectColor}{\,\cdot\,}      \GRANULEnt{s'} _  1     \sqcup ... \sqcup    \GRANULEnt{s'} _  n       \vdash \,   \GRANULEmv{y} ^i_j   :  \GRANULEnt{B_{\GRANULEmv{j}}}  \, \rhd \,   {  \GRANULEmv{y} ^  i  _  j   }:_{\textcolor{coeffectColor}{    \GRANULEnt{q} ^  i  _  j     \textcolor{coeffectColor}{\,\cdot\,}      \GRANULEnt{s'} _  1     \sqcup ... \sqcup    \GRANULEnt{s'} _  n       } }   \GRANULEnt{B_{\GRANULEmv{j}}}  }}{\GRANULEnt{s'} _  1     \sqcup ... \sqcup    \GRANULEnt{s'} _  n       \vdash \,   C_{\GRANULEmv{i}}  \  \GRANULEmv{y} ^  i  _  1   ...  \GRANULEmv{y} ^  i  _  n    :   K  \, \GRANULEnt{A}  \, \rhd \,    {  \GRANULEmv{y} ^  i  _  1   }:_{\textcolor{coeffectColor}{    \GRANULEnt{q} ^  i  _  1     \textcolor{coeffectColor}{\,\cdot\,}      \GRANULEnt{s'} _  1     \sqcup ... \sqcup    \GRANULEnt{s'} _  n       } }   \GRANULEnt{B_{{\mathrm{1}}}}   , ... ,   {  \GRANULEmv{y} ^  i  _  n   }:_{\textcolor{coeffectColor}{    \GRANULEnt{q} ^  i  _  n     \textcolor{coeffectColor}{\,\cdot\,}      \GRANULEnt{s'} _  1     \sqcup ... \sqcup    \GRANULEnt{s'} _  n       } }   \GRANULEnt{B_{\GRANULEmv{n}}}}\\
%         \end{equation}

%         \begin{equation}
%           \label{soundConLAltSub2}
%           \inferrule*[Right=Approx, rightskip=5em]{ \inferrule*[Right=Approx]{ \Delta_{\GRANULEmv{i}}  \GRANULEsym{,}   \GRANULEmv{x}  :_{\textcolor{coeffectColor}{   \GRANULEnt{r_{\GRANULEmv{i}}}   } }    K  \, \GRANULEnt{A}    \GRANULEsym{,}   {  \GRANULEmv{y} ^  i  _  1   }:_{\textcolor{coeffectColor}{    \GRANULEnt{s} ^  i  _  1     } }   \GRANULEnt{B_{{\mathrm{1}}}}    , ... ,   {  \GRANULEmv{y} ^  i  _  n   }:_{\textcolor{coeffectColor}{    \GRANULEnt{s} ^  i  _  n     } }   \GRANULEnt{B_{\GRANULEmv{n}}}    \vdash  \GRANULEnt{t_{\GRANULEmv{i}}}  :  \GRANULEnt{B} \\ (e)}{\inferrule*[Right=Approx]{\Delta_{\GRANULEmv{i}}  \GRANULEsym{,}   \GRANULEmv{x}  :_{\textcolor{coeffectColor}{   \GRANULEnt{r_{\GRANULEmv{i}}}   } }    K  \, \GRANULEnt{A}    \GRANULEsym{,}   {  \GRANULEmv{y} ^  i  _  1   }:_{\textcolor{coeffectColor}{    \GRANULEnt{s} ^  i  _  1     } }   \GRANULEnt{B_{{\mathrm{1}}}}    , ... ,   {  \GRANULEmv{y} ^  i  _  n   }:_{\textcolor{coeffectColor}{    \GRANULEnt{s} _  i     } }   \GRANULEnt{B_{\GRANULEmv{n}}}    \vdash  \GRANULEnt{t_{\GRANULEmv{i}}}  :  \GRANULEnt{B} \\ (e)}{\inferrule*[Right=Approx]{\Delta_{\GRANULEmv{i}}  \GRANULEsym{,}   \GRANULEmv{x}  :_{\textcolor{coeffectColor}{    \GRANULEnt{r} _  i     } }     K  \, \GRANULEnt{A}    \GRANULEsym{,}   {  \GRANULEmv{y} ^  i  _  1   }:_{\textcolor{coeffectColor}{    \GRANULEnt{s} _  i     } }   \GRANULEnt{B_{{\mathrm{1}}}}    , ... ,   {  \GRANULEmv{y} ^  i  _  n   }:_{\textcolor{coeffectColor}{    \GRANULEnt{s} _  i     } }   \GRANULEnt{B_{\GRANULEmv{n}}}    \vdash  \GRANULEnt{t_{\GRANULEmv{i}}}  :  \GRANULEnt{B}  \\ (d)  }{\inferrule*[Right=Approx]{ \Delta_{\GRANULEmv{i}}  \GRANULEsym{,}   \GRANULEmv{x}  :_{\textcolor{coeffectColor}{    \GRANULEnt{r} _  i     } }     K  \, \GRANULEnt{A}    \GRANULEsym{,}   {  \GRANULEmv{y} ^  i  _  1   }:_{\textcolor{coeffectColor}{    \GRANULEnt{q} ^  i  _  1     \textcolor{coeffectColor}{\,\cdot\,}    \GRANULEnt{s'} _  i     } }   \GRANULEnt{B_{{\mathrm{1}}}}    , ... ,   {  \GRANULEmv{y} ^  i  _  n   }:_{\textcolor{coeffectColor}{    \GRANULEnt{q} ^  i  _  n     \textcolor{coeffectColor}{\,\cdot\,}    \GRANULEnt{s'} _  i     } }   \GRANULEnt{B_{\GRANULEmv{n}}}    \vdash  \GRANULEnt{t_{\GRANULEmv{i}}}  :  \GRANULEnt{B} \\ (c)}{\inferrule*[Right=Approx]{ \Delta_{\GRANULEmv{i}}  \GRANULEsym{,}   \GRANULEmv{x}  :_{\textcolor{coeffectColor}{    \GRANULEnt{r} _  i     } }     K  \, \GRANULEnt{A}    \GRANULEsym{,}   {  \GRANULEmv{y} ^  i  _  1   }:_{\textcolor{coeffectColor}{    \GRANULEnt{q} ^  i  _  1     \textcolor{coeffectColor}{\,\cdot\,}      \GRANULEnt{s} _  1     \sqcup ... \sqcup    \GRANULEnt{s} _  n       } }   \GRANULEnt{B_{{\mathrm{1}}}}    , ... ,   {  \GRANULEmv{y} ^  i  _  n   }:_{\textcolor{coeffectColor}{    \GRANULEnt{q} ^  i  _  n     \textcolor{coeffectColor}{\,\cdot\,}      \GRANULEnt{s'} _  1     \sqcup ... \sqcup    \GRANULEnt{s'} _  n       } }   \GRANULEnt{B_{\GRANULEmv{n}}}    \vdash  \GRANULEnt{t_{\GRANULEmv{i}}}  :  \GRANULEnt{B} \\ (b)}{\Delta_{\GRANULEmv{i}}  \GRANULEsym{,}   \GRANULEmv{x}  :_{\textcolor{coeffectColor}{   \GRANULEsym{(}     \GRANULEnt{r} _  1     \sqcup ... \sqcup    \GRANULEnt{r} _  n      \GRANULEsym{)}   } }     K  \, \GRANULEnt{A}    \GRANULEsym{,}   {  \GRANULEmv{y} ^  i  _  1   }:_{\textcolor{coeffectColor}{    \GRANULEnt{q} ^  i  _  1     \textcolor{coeffectColor}{\,\cdot\,}      \GRANULEnt{s'} _  1     \sqcup ... \sqcup    \GRANULEnt{s'} _  n       } }   \GRANULEnt{B_{{\mathrm{1}}}}    , ... ,   {  \GRANULEmv{y} ^  i  _  n   }:_{\textcolor{coeffectColor}{    \GRANULEnt{q} ^  i  _  n     \textcolor{coeffectColor}{\,\cdot\,}      \GRANULEnt{s'} _  1     \sqcup ... \sqcup    \GRANULEnt{s'} _  n       } }   \GRANULEnt{B_{\GRANULEmv{n}}}    \vdash  \GRANULEnt{t_{\GRANULEmv{i}}}  :  \GRANULEnt{B} \\ (a)} }}}}}{ \GRANULEsym{(}   \Delta_{{\mathrm{1}}}  \sqcup ...  \sqcup  \Delta_{\GRANULEmv{n}}   \GRANULEsym{)}  \GRANULEsym{,}   \GRANULEmv{x}  :_{\textcolor{coeffectColor}{   \GRANULEsym{(}     \GRANULEnt{r} _  1     \sqcup ... \sqcup    \GRANULEnt{r} _  n      \GRANULEsym{)}   } }     K  \, \GRANULEnt{A}    \GRANULEsym{,}   {  \GRANULEmv{y} ^  i  _  1   }:_{\textcolor{coeffectColor}{    \GRANULEnt{q} ^  i  _  1     \textcolor{coeffectColor}{\,\cdot\,}      \GRANULEnt{s'} _  1     \sqcup ... \sqcup    \GRANULEnt{s'} _  n       } }   \GRANULEnt{B_{{\mathrm{1}}}}    , ... ,   {  \GRANULEmv{y} ^  i  _  n   }:_{\textcolor{coeffectColor}{    \GRANULEnt{q} ^  i  _  n     \textcolor{coeffectColor}{\,\cdot\,}      \GRANULEnt{s'} _  1     \sqcup ... \sqcup    \GRANULEnt{s'} _  n       } }   \GRANULEnt{B_{\GRANULEmv{n}}}    \vdash  \GRANULEnt{t_{\GRANULEmv{i}}}  :  \GRANULEnt{B} }
%         \end{equation}


%         \begin{align*}
%           \inferrule*[Right=Case]{ \inferrule*[Right=Var,leftskip=5em]{\quad}{\GRANULEmv{x}  :_{\textcolor{coeffectColor}{   1   } }     K  \, \GRANULEnt{A}    \vdash  \GRANULEmv{x}  :    K  \, \GRANULEnt{A}} \\ \eqref{soundConLAltSub1} \\ \eqref{soundConLAltSub2} }
%           {\GRANULEsym{(}   \Delta_{{\mathrm{1}}}  \sqcup ...  \sqcup  \Delta_{\GRANULEmv{n}}   \GRANULEsym{)}  \GRANULEsym{,}    \GRANULEmv{x}  :_{\textcolor{coeffectColor}{   \GRANULEsym{(}     \GRANULEnt{r} _  1     \sqcup ... \sqcup    \GRANULEnt{r} _  n      \GRANULEsym{)}   \GRANULEsym{+}   \GRANULEsym{(}      \GRANULEnt{s'} _  1     \sqcup ... \sqcup    \GRANULEnt{s'} _  n       \textcolor{coeffectColor}{\,\cdot\,}   1   \GRANULEsym{)}   } }     K  \, \GRANULEnt{A}     \vdash   \textbf{case} \  \GRANULEmv{x}  \ \textbf{of} \   \overline{   C_{\GRANULEmv{i}}  \  \GRANULEmv{y} ^  i  _  1   ...  \GRANULEmv{y} ^  i  _  n    \mapsto  \GRANULEnt{t_{\GRANULEmv{i}}} }    :  \GRANULEnt{B}}
        % \end{align*}

\item Case $\Box_{R}$ \\
        For synthesising a promotion, we have the derivation:
        \begin{align*}
          \synBox
        \end{align*}
        By induction on the premise we have:
        \begin{align*}
          \Delta  \vdash  \GRANULEnt{t}  :  \GRANULEnt{A}
        \end{align*}
        From which we can construct the following derivation, matching the above conclusion:
        \begin{align*}
          \inferrule*[Right=Pr]{\Delta  \vdash  \GRANULEnt{t}  :  \GRANULEnt{A}}{ \textcolor{coeffectColor}{ \GRANULEnt{r}   \textcolor{coeffectColor}{\,\cdot\,} }  \Delta   \vdash  \GRANULEsym{[}  \GRANULEnt{t}  \GRANULEsym{]}  :   \Box_{  \GRANULEnt{r}  }  \GRANULEnt{A}}
        \end{align*}
  \item Case $\Box_{L}$ \\
        For synthesising an unboxing, we have the derivation:
        \begin{align*}
          \synUnbox
        \end{align*}
        By induction on the premise we have:
        \begin{align*}
          \Delta  \GRANULEsym{,}   \GRANULEmv{y}  :_{\textcolor{coeffectColor}{  \GRANULEnt{s_{{\mathrm{1}}}}  } }   \GRANULEnt{A}    \GRANULEsym{,}   \GRANULEmv{x}  :_{\textcolor{coeffectColor}{  \GRANULEnt{s_{{\mathrm{2}}}}  } }    \Box_{  \GRANULEnt{q}  }  \GRANULEnt{A}    \vdash  \GRANULEnt{t}  :  \GRANULEnt{B}
        \end{align*}
        and that:
        \begin{align*}
          \GRANULEnt{s_{{\mathrm{1}}}}  \, \textcolor{coeffectColor}{\sqsubseteq} \,   \GRANULEnt{s_{{\mathrm{3}}}}  \textcolor{coeffectColor}{\,\cdot\,}  \GRANULEnt{q}
        \end{align*}
        From this we can construct the following derivation, matching the above conclusion:
        \begin{align*}
          \inferrule*[Right=Case]{\inferrule*[Right=Var, leftskip=5em]{\quad}{\GRANULEmv{x}  :_{\textcolor{coeffectColor}{   1   } }    \Box_{  \GRANULEnt{q}  }  \GRANULEnt{A}    \vdash  \GRANULEmv{x}  :   \Box_{  \GRANULEnt{q}  }  \GRANULEnt{A}} \\ \inferrule*[Right=PBox, leftskip=0em]{\inferrule*[Right=PVar]{\quad}{\GRANULEnt{s_{{\mathrm{3}}}}  \textcolor{coeffectColor}{\,\cdot\,}  \GRANULEnt{q}  \vdash \,  \GRANULEmv{y}  :  \GRANULEnt{A}  \, \rhd \,   \GRANULEmv{y}  :_{\textcolor{coeffectColor}{   \GRANULEnt{s_{{\mathrm{3}}}}  \textcolor{coeffectColor}{\,\cdot\,}  \GRANULEnt{q}   } }   \GRANULEnt{A}}}{\GRANULEnt{s_{{\mathrm{3}}}}  \vdash \,  \GRANULEsym{[}  \GRANULEmv{y}  \GRANULEsym{]}  :   \Box_{  \GRANULEnt{q}  }  \GRANULEnt{A}   \, \rhd \,   \GRANULEmv{y}  :_{\textcolor{coeffectColor}{   \GRANULEnt{s_{{\mathrm{3}}}}  \textcolor{coeffectColor}{\,\cdot\,}  \GRANULEnt{q}   } }   \GRANULEnt{A}} \\ \inferrule*[Right=Approx, rightskip=4.5em] {\Delta  \GRANULEsym{,}   \GRANULEmv{y}  :_{\textcolor{coeffectColor}{  \GRANULEnt{s_{{\mathrm{1}}}}  } }   \GRANULEnt{A}   \GRANULEsym{,}   \GRANULEmv{x}  :_{\textcolor{coeffectColor}{  \GRANULEnt{s_{{\mathrm{2}}}}  } }    \Box_{  \GRANULEnt{q}  }  \GRANULEnt{A}    \vdash  \GRANULEnt{t}  :  \GRANULEnt{B} \\ \GRANULEnt{s_{{\mathrm{1}}}}  \, \textcolor{coeffectColor}{\sqsubseteq} \,   \GRANULEnt{s_{{\mathrm{3}}}}  \textcolor{coeffectColor}{\,\cdot\,}  \GRANULEnt{q}} { \Delta  \GRANULEsym{,}   \GRANULEmv{y}  :_{\textcolor{coeffectColor}{   \GRANULEnt{s_{{\mathrm{3}}}}  \textcolor{coeffectColor}{\,\cdot\,}  \GRANULEnt{q}   } }   \GRANULEnt{A}   \GRANULEsym{,}   \GRANULEmv{x}  :_{\textcolor{coeffectColor}{  \GRANULEnt{s_{{\mathrm{2}}}}  } }    \Box_{  \GRANULEnt{q}  }  \GRANULEnt{A}    \vdash  \GRANULEnt{t}  :  \GRANULEnt{B}} }{\Delta  \GRANULEsym{,}   \GRANULEmv{x}  :_{\textcolor{coeffectColor}{   \GRANULEnt{s_{{\mathrm{3}}}}  \GRANULEsym{+}  \GRANULEnt{s_{{\mathrm{2}}}}   } }    \Box_{  \GRANULEnt{q}  }  \GRANULEnt{A}    \vdash   \textbf{case} \  \GRANULEmv{x}  \ \textbf{of} \  \GRANULEsym{[}  \GRANULEmv{y}  \GRANULEsym{]}  \rightarrow  \GRANULEnt{t}   :  \GRANULEnt{B}}
        \end{align*}

\end{enumerate}

\end{proof}

% \focusSynthSound*
% \begin{proof}

% \end{proof}
%

\begin{restatable}[Soundness of focusing for graded-base synthesis]{lemma}{gradedBaseFocusingSoundness}
For all contexts $\Gamma$, $\Omega$ and types $\GRANULEnt{A}$
then:
\begin{align*}
\begin{array}{lll}
 1.\ Right\ Async: & \Gamma  ;  \Omega  \vdash  \GRANULEnt{A}  \Uparrow \Rightarrow  \GRANULEnt{t}  \mid  \Delta \quad &\implies \quad \Gamma  \GRANULEsym{,}  \,  \GRANULEsym{,}  \Omega  \vdash  \GRANULEnt{A}  \Rightarrow  \GRANULEnt{t}  \mid  \Delta\\
 2.\ Left\ Async: & \Gamma  ;  \Omega  \Uparrow \vdash  \GRANULEnt{B}  \Rightarrow  \GRANULEnt{t}  \mid  \Delta \quad &\implies \quad \Gamma  \GRANULEsym{,}  \,  \GRANULEsym{,}  \Omega  \vdash  \GRANULEnt{B}  \Rightarrow  \GRANULEnt{t}  \mid  \Delta\\
 3.\ Right\ Sync: & \Gamma ; \emptyset \vdash \GRANULEnt{A} \Downarrow\ \Rightarrow \GRANULEnt{t} \mid\  \Delta \quad &\implies \quad \Gamma  \vdash  \GRANULEnt{A}  \Rightarrow  \GRANULEnt{t}  \mid  \Delta\\
 4.\ Left\ Sync: & \Gamma  ;    \GRANULEmv{x}  :  \GRANULEnt{A}    \Downarrow \vdash  \GRANULEnt{B}  \Rightarrow  \GRANULEnt{t}  \mid  \Delta \quad &\implies \quad \Gamma  \GRANULEsym{,}   \GRANULEmv{x}  :  \GRANULEnt{A}   \vdash  \GRANULEnt{B}  \Rightarrow  \GRANULEnt{t}  \mid  \Delta\\
 5.\ Focus\ Right: & \Gamma  ;  \Omega  \Uparrow \vdash  \GRANULEnt{B}  \Rightarrow  \GRANULEnt{t}  \mid  \Delta \quad &\implies \quad \Gamma  \vdash  \GRANULEnt{B}  \Rightarrow  \GRANULEnt{t}  \mid  \Delta\\
 6.\ Focus\ Left: & \Gamma  \GRANULEsym{,}   \GRANULEmv{x}  :  \GRANULEnt{A}   ;  \Omega  \Uparrow \vdash  \GRANULEnt{B}  \Rightarrow  \GRANULEnt{t}  \mid  \Delta \quad &\implies \quad \Gamma  \vdash  \GRANULEnt{B}  \Rightarrow  \GRANULEnt{t}  \mid  \Delta
\end{array}
\end{align*}
\end{restatable}
\begin{proof}
\begin{enumerate}
\item Case: 1. Right Async: \\
    \begin{enumerate}
      \item Case $\multimap_{R}$\\
          In the case of the right asynchronous rule for abstraction introduction, the synthesis rule has the form:
          \[
            \fsynAbs
          \]
          By induction on the premise, we have that:
          \[
            \GRANULEsym{(}  \Gamma  \GRANULEsym{,}  \Omega  \GRANULEsym{)}  \GRANULEsym{,}   \GRANULEmv{x}  :_{\textcolor{coeffectColor}{  \GRANULEnt{q}  } }   \GRANULEnt{A}   \vdash  \GRANULEnt{B}  \Rightarrow  \GRANULEnt{t}  \mid  \Delta  \GRANULEsym{,}   \GRANULEmv{x}  :_{\textcolor{coeffectColor}{  \GRANULEnt{r}  } }   \GRANULEnt{A}   \tag{ih}
          \]
          from case 1 of the lemma. From which, we can construct the following instatiation of the $\multimap_{\textsc{R}}$ synthesis rule in the non-focusing calculus:
          \[
    \inferrule*[right=$\multimap_{R}$]
    {\GRANULEsym{(}  \Gamma  \GRANULEsym{,}  \Omega  \GRANULEsym{)}  \GRANULEsym{,}   \GRANULEmv{x}  :_{\textcolor{coeffectColor}{  \GRANULEnt{q}  } }   \GRANULEnt{A}   \vdash  \GRANULEnt{B}  \Rightarrow  \GRANULEnt{t}  \mid  \Delta  \GRANULEsym{,}   \GRANULEmv{x}  :_{\textcolor{coeffectColor}{  \GRANULEnt{r}  } }   \GRANULEnt{A}    \quad\;\;   \GRANULEnt{r}  \, \textcolor{coeffectColor}{\sqsubseteq} \,  \GRANULEnt{q}}{\Gamma  \GRANULEsym{,}  \Omega  \vdash  \GRANULEnt{A}  \rightarrow  \GRANULEnt{B}  \Rightarrow   \lambda  \GRANULEmv{x}  .  \GRANULEnt{t}   \mid  \Delta}
          \]
          \item Case \fsynRAsyncTransName\ \\
          In the case of the right asynchronous rule for transition to a left asynchronous judgement, the synthesis rule has the form:
          \[
            \fsynRAsyncTrans
          \]
          By induction on the first premise, we have that:
          \[
            \Gamma  \GRANULEsym{,}  \Omega  \vdash  \GRANULEnt{B}  \Rightarrow^+  \GRANULEnt{t}  ;\,  \Delta
          \]
          from case 2 of the lemma.
    \end{enumerate}
\item Case 2. Left Async: \\
    \begin{enumerate}
      \item Case $\textsc{Con}_{L}$\\
        In the case of the left asynchronous rule for constructor elimination, the synthesis rule has the form:
            \[
            \fsynCase
            \]
            By induction on the second premise, we have that:
            \[
\GRANULEsym{(}  \Gamma  \GRANULEsym{,}  \Omega  \GRANULEsym{)}  \GRANULEsym{,}       \GRANULEmv{x}  :_{\textcolor{coeffectColor}{   \GRANULEnt{r}   } }    K  \,   \vec{ \GRANULEnt{A} }      \GRANULEsym{,}   {  \GRANULEmv{y} ^  i  _  1   }:_{\textcolor{coeffectColor}{   \GRANULEnt{r}  \textcolor{coeffectColor}{\,\cdot\,}    \GRANULEnt{q} ^  i  _  1      } }   \GRANULEnt{B_{{\mathrm{1}}}}    , ... ,   {  \GRANULEmv{y} ^  i  _  n   }:_{\textcolor{coeffectColor}{   \GRANULEnt{r}  \textcolor{coeffectColor}{\,\cdot\,}    \GRANULEnt{q} ^  i  _  1      } }   \GRANULEnt{B_{\GRANULEmv{n}}}      \Uparrow\   \vdash  \GRANULEnt{B}  \Rightarrow  \GRANULEnt{t_{\GRANULEmv{i}}}  \mid     \Delta_{\GRANULEmv{i}}  \GRANULEsym{,}   \GRANULEmv{x}  :_{\textcolor{coeffectColor}{   \GRANULEnt{r_{\GRANULEmv{i}}}   } }    K  \,   \vec{ \GRANULEnt{A} }      \GRANULEsym{,}   {  \GRANULEmv{y} ^  i  _  1   }:_{\textcolor{coeffectColor}{    \GRANULEnt{s} ^  i  _  1     } }   \GRANULEnt{B_{{\mathrm{1}}}}    , ... ,   {  \GRANULEmv{y} ^  i  _  n   }:_{\textcolor{coeffectColor}{    \GRANULEnt{s} ^  i  _  n     } }   \GRANULEnt{B_{\GRANULEmv{n}}}
            \]
            from case 2 of the lemma. From which we can construct the following instantiation of the $\textsc{Con}_{L}$ rule in the non-focusing calculus:
            \begin{align*}
            \inferrule*[right=$\textsc{Con}_{\textsc{L}}$]
            {
            (  C_{\GRANULEmv{i}}  :  \GRANULEnt{B_{{\mathrm{1}}}} ^{q_1^i} \rightarrow ... \rightarrow  \GRANULEnt{B_{\GRANULEmv{n}}} ^{q_n^i} \rightarrow     K  \,   \vec{ \GRANULEnt{A} }     ) \in  D\\\\
            \GRANULEsym{(}  \Gamma  \GRANULEsym{,}  \Omega  \GRANULEsym{)}  \GRANULEsym{,}   \GRANULEmv{x}  :_{\textcolor{coeffectColor}{   \GRANULEnt{r}   } }    K  \,   \vec{ \GRANULEnt{A} }      \GRANULEsym{,}   {  \GRANULEmv{y} ^  i  _  1   }:_{\textcolor{coeffectColor}{   \GRANULEnt{r}  \textcolor{coeffectColor}{\,\cdot\,}    \GRANULEnt{q} ^  i  _  1      } }   \GRANULEnt{B_{{\mathrm{1}}}}    , ... ,   {  \GRANULEmv{y} ^  i  _  n   }:_{\textcolor{coeffectColor}{   \GRANULEnt{r}  \textcolor{coeffectColor}{\,\cdot\,}    \GRANULEnt{q} ^  i  _  1      } }   \GRANULEnt{B_{\GRANULEmv{n}}}    \vdash  \GRANULEnt{B}  \Rightarrow  \GRANULEnt{t_{\GRANULEmv{i}}}  \mid     \Delta_{\GRANULEmv{i}}  \GRANULEsym{,}   \GRANULEmv{x}  :_{\textcolor{coeffectColor}{   \GRANULEnt{r_{\GRANULEmv{i}}}   } }    K  \,   \vec{ \GRANULEnt{A} }      \GRANULEsym{,}   {  \GRANULEmv{y} ^  i  _  1   }:_{\textcolor{coeffectColor}{    \GRANULEnt{s} ^  i  _  1     } }   \GRANULEnt{B_{{\mathrm{1}}}}    , ... ,   {  \GRANULEmv{y} ^  i  _  n   }:_{\textcolor{coeffectColor}{    \GRANULEnt{s} ^  i  _  n     } }   \GRANULEnt{B_{\GRANULEmv{n}}}\\\\
            \exists    \GRANULEnt{s'} ^  i  _  j     .\,     \GRANULEnt{s} ^  i  _  j     \sqsubseteq    \GRANULEnt{s'} ^  i  _  j     \textcolor{coeffectColor}{\,\cdot\,}    \GRANULEnt{q} ^  i  _  j     \sqsubseteq  \GRANULEnt{r}  \textcolor{coeffectColor}{\,\cdot\,}    \GRANULEnt{q} ^  i  _  j\\\\
            \GRANULEnt{s} _  i     =      \GRANULEnt{s'} ^  i  _  1     \sqcup ... \sqcup    \GRANULEnt{s'} ^  i  _  n\\\\
            |    K  \,   \vec{ \GRANULEnt{A} }     | > 1 \Rightarrow 1 \sqsubseteq      \GRANULEnt{s} _  1     \sqcup ... \sqcup    \GRANULEnt{s} _  m
            }
            {\GRANULEsym{(}  \Gamma  \GRANULEsym{,}  \Omega  \GRANULEsym{)}  \GRANULEsym{,}   \GRANULEmv{x}  :_{\textcolor{coeffectColor}{  \GRANULEnt{r}  } }     K  \,   \vec{ \GRANULEnt{A} }      \vdash  \GRANULEnt{B}  \Rightarrow   \textbf{case} \  \GRANULEmv{x}  \ \textbf{of} \   \overline{   C_{\GRANULEmv{i}}  \  \GRANULEmv{y} ^  i  _  1   ...  \GRANULEmv{y} ^  i  _  n    \mapsto  \GRANULEnt{t_{\GRANULEmv{i}}} }    \mid  \GRANULEsym{(}   \Delta_{{\mathrm{1}}}  \sqcup ...  \sqcup  \Delta_{\GRANULEmv{m}}   \GRANULEsym{)}  \GRANULEsym{,}    \GRANULEmv{x}  :_{\textcolor{coeffectColor}{   \GRANULEsym{(}     \GRANULEnt{r} _  1     \sqcup ... \sqcup    \GRANULEnt{r} _  m      \GRANULEsym{)}   \GRANULEsym{+}   \GRANULEsym{(}     \GRANULEnt{s} _  1     \sqcup ... \sqcup    \GRANULEnt{s} _  m      \GRANULEsym{)}   } }     K  \,   \vec{ \GRANULEnt{A} }
            }
            \end{align*}


      \item Case $\Box_{L}$ \\
            In the case of the left asynchronous rule for graded modality elimination, the synthesis rule has the form:
            \[
            \fsynUnbox
            \]
            By induction on the first premise, we have that:
            \[
            \GRANULEsym{(}  \Gamma  \GRANULEsym{,}  \Omega  \GRANULEsym{)}  \GRANULEsym{,}   \GRANULEmv{y}  :_{\textcolor{coeffectColor}{   \GRANULEnt{r}  \textcolor{coeffectColor}{\,\cdot\,}  \GRANULEnt{q}   } }   \GRANULEnt{A}    \GRANULEsym{,}   \GRANULEmv{x}  :_{\textcolor{coeffectColor}{  \GRANULEnt{r}  } }    \Box_{  \GRANULEnt{q}  }  \GRANULEnt{A}     \vdash  \GRANULEnt{B}  \Rightarrow  \GRANULEnt{t}  \mid  \Delta  \GRANULEsym{,}   \GRANULEmv{y}  :_{\textcolor{coeffectColor}{   \GRANULEnt{s_{{\mathrm{1}}}}   } }   \GRANULEnt{A}   \GRANULEsym{,}   \GRANULEmv{x}  :_{\textcolor{coeffectColor}{  \GRANULEnt{s_{{\mathrm{2}}}}  } }    \Box_{  \GRANULEnt{q}  }  \GRANULEnt{A}
            \]
            from case 2 of the lemma. From which, we can construct the following instantiation of the $\Box_{L}$ synthesis rule in the non focusing calculus:
            \[
            \inferrule*[right=$\Box_{\textsc{L}}$]
            {
            \GRANULEsym{(}  \Gamma  \GRANULEsym{,}  \Omega  \GRANULEsym{)}  \GRANULEsym{,}   \GRANULEmv{y}  :_{\textcolor{coeffectColor}{   \GRANULEnt{r}  \textcolor{coeffectColor}{\,\cdot\,}  \GRANULEnt{q}   } }   \GRANULEnt{A}    \GRANULEsym{,}   \GRANULEmv{x}  :_{\textcolor{coeffectColor}{  \GRANULEnt{r}  } }    \Box_{  \GRANULEnt{q}  }  \GRANULEnt{A}     \vdash  \GRANULEnt{B}  \Rightarrow  \GRANULEnt{t}  \mid  \Delta  \GRANULEsym{,}   \GRANULEmv{y}  :_{\textcolor{coeffectColor}{   \GRANULEnt{s_{{\mathrm{1}}}}   } }   \GRANULEnt{A}   \GRANULEsym{,}   \GRANULEmv{x}  :_{\textcolor{coeffectColor}{  \GRANULEnt{s_{{\mathrm{2}}}}  } }    \Box_{  \GRANULEnt{q}  }  \GRANULEnt{A} \\
            \exists  \GRANULEnt{s_{{\mathrm{3}}}}  .\,   \GRANULEnt{s_{{\mathrm{1}}}}  \sqsubseteq   \GRANULEnt{s_{{\mathrm{3}}}}  \textcolor{coeffectColor}{\,\cdot\,}  \GRANULEnt{q}   \sqsubseteq   \GRANULEnt{r}  \textcolor{coeffectColor}{\,\cdot\,}  \GRANULEnt{q}
            }
            {
            \GRANULEsym{(}  \Gamma  \GRANULEsym{,}  \Omega  \GRANULEsym{)}  \GRANULEsym{,}   \GRANULEmv{x}  :_{\textcolor{coeffectColor}{  \GRANULEnt{r}  } }    \Box_{  \GRANULEnt{q}  }  \GRANULEnt{A}     \vdash  \GRANULEnt{B}  \Rightarrow   \textbf{case} \  \GRANULEmv{x}  \ \textbf{of} \  \GRANULEsym{[}  \GRANULEmv{y}  \GRANULEsym{]}  \rightarrow  \GRANULEnt{t}   \mid   \Delta   \GRANULEsym{,}   \GRANULEmv{x}  :_{\textcolor{coeffectColor}{   \GRANULEnt{s_{{\mathrm{3}}}}  \GRANULEsym{+}  \GRANULEnt{s_{{\mathrm{2}}}}   } }    \Box_{  \GRANULEnt{q}  }  \GRANULEnt{A}
            }
            \]

        \item Case \fsynLAsyncTransName \\
          In the case of the left asynchronous rule for transitioning an assumption from the focusing context $\Omega$ to the non-focusing context $\Gamma$, the synthesis rule has the form:
          \[
            \fsynLAsyncTrans
          \]
          By induction on the first premise, we have that:
          \[
            \Gamma  \GRANULEsym{,}   \GRANULEmv{x}  :  \GRANULEnt{A}    \GRANULEsym{,}  \Omega  \vdash  \GRANULEnt{C}  \Rightarrow  \GRANULEnt{t}  \mid  \Delta \tag{ih}
          \]
          from case 2 of the lemma.
    \end{enumerate}
\item Case 3. Right Sync: \\
    \begin{enumerate}
      \item Case $\textsc{Con}_{\textsc{R}}$\\
          In the case of the right synchronous rule for constructor introduction, the synthesis rule has the form:
            \[
            \inferrule*[Right=$\textsc{Con}_{\textsc{R}}$]
            { (  C  :  \GRANULEnt{B_{{\mathrm{1}}}} ^1 \rightarrow ... \rightarrow  \GRANULEnt{B_{\GRANULEmv{n}}} ^1 \rightarrow     K  \,   \vec{ \GRANULEnt{A} }     ) \in  D \\
             \Gamma ; \emptyset \vdash \GRANULEnt{B_{\GRANULEmv{i}}} \Downarrow\ \Rightarrow \GRANULEnt{t_{\GRANULEmv{i}}} \mid\ \Delta_{\GRANULEmv{i}}}
            {\Gamma ; \emptyset \vdash K  \,   \vec{ \GRANULEnt{A} } \Downarrow\ \Rightarrow C \, \GRANULEnt{t_{{\mathrm{1}}}} \, ... \, \GRANULEnt{t_{\GRANULEmv{n}}} \mid\ \Delta_{{\mathrm{1}}}  \GRANULEsym{+} \, ... \, \GRANULEsym{+}  \Delta_{\GRANULEmv{n}}}
            \]
          By induction on the second premise, we have that:
            \[
            \Gamma  \vdash  \GRANULEnt{B_{\GRANULEmv{i}}}  \Rightarrow  \GRANULEnt{t_{\GRANULEmv{i}}}  \mid  \Delta_{\GRANULEmv{i}}
            \]
          from case 3 of the lemma. From which, we can construct the following instantiation of the $\textsc{Con}_{\textsc{R}}$\ synthesis rule in the non-focusing calculus:
            \[
            \inferrule*[]
            {
            (  C  :  \GRANULEnt{B_{{\mathrm{1}}}} ^ \GRANULEnt{q_{{\mathrm{1}}}}  \rightarrow ... \rightarrow  \GRANULEnt{B_{\GRANULEmv{n}}} ^ \GRANULEnt{q_{\GRANULEmv{n}}}  \rightarrow     K  \,   \vec{ \GRANULEnt{A} }     ) \in  D \\
            \Gamma  \vdash  \GRANULEnt{B_{\GRANULEmv{i}}}  \Rightarrow  \GRANULEnt{t_{\GRANULEmv{i}}}  \mid  \Delta_{\GRANULEmv{i}}
            }
            {\Gamma  \vdash   K  \,   \vec{ \GRANULEnt{A} }    \Rightarrow  C \, \GRANULEnt{t_{{\mathrm{1}}}} \, ... \, \GRANULEnt{t_{\GRANULEmv{n}}}  \mid    \textcolor{coeffectColor}{  0    \textcolor{coeffectColor}{\,\cdot\,} }  \Gamma    \GRANULEsym{+}   \GRANULEsym{(}   \textcolor{coeffectColor}{ \GRANULEnt{q_{{\mathrm{1}}}}   \textcolor{coeffectColor}{\,\cdot\,} }  \Delta_{{\mathrm{1}}}   \GRANULEsym{)}  \GRANULEsym{+} \, ... \, \GRANULEsym{+}  \GRANULEsym{(}   \textcolor{coeffectColor}{ \GRANULEnt{q_{\GRANULEmv{n}}}   \textcolor{coeffectColor}{\,\cdot\,} }  \Delta_{\GRANULEmv{n}}   \GRANULEsym{)}}
            \]
      \item Case $\Box_{R}$ \\

          In the case of the right synchronous rule for graded modality introduction, the synthesis rule has the form:
            \[
            \inferrule*[Right=$\Box_{\textsc{R}}$]
            { \Gamma ; \emptyset \vdash \GRANULEnt{A} \Downarrow\ \Rightarrow \GRANULEnt{t} \mid\ \Delta}
            { \Gamma ; \emptyset \vdash \Box_{  \GRANULEnt{r}  }  \GRANULEnt{A} \Downarrow\ \Rightarrow \GRANULEsym{[}  \GRANULEnt{t}  \GRANULEsym{]} \mid\ \textcolor{coeffectColor}{ \GRANULEnt{r}   \textcolor{coeffectColor}{\,\cdot\,} }  \Delta}
            \]
          By induction on the premises, we have that:
            \[
            \Gamma  \vdash  \GRANULEnt{A}  \Rightarrow  \GRANULEnt{t}  \mid  \Delta
            \]
          from case 3 of the lemma. From which, we can construct the following instantiation of the $\Box_{\textsc{R}}$\ synthesis rule in the non-focusing calculus:
            \[
            \synBox
            \]
    \end{enumerate}
\item Case 4. Left Sync: \\
    \begin{enumerate}
      \item Case $\multimap_{L}$\\
            In the case of the left synchronous rule for application, the synthesis rule has the form:
            \[
            \inferrule*[]{
            \Gamma  \GRANULEsym{,}   \GRANULEmv{x_{{\mathrm{1}}}}  :_{\textcolor{coeffectColor}{  \GRANULEnt{r_{{\mathrm{1}}}}  } }    \GRANULEnt{A} ^ \GRANULEnt{q}  \rightarrow  \GRANULEnt{B}    ;    \GRANULEmv{x_{{\mathrm{2}}}}  :_{\textcolor{coeffectColor}{  \GRANULEnt{r_{{\mathrm{1}}}}  } }   \GRANULEnt{B}    \Downarrow \vdash  \GRANULEnt{C}  \Rightarrow  \GRANULEnt{t_{{\mathrm{1}}}}  \mid  \Delta_{{\mathrm{1}}}  \GRANULEsym{,}   \GRANULEmv{x_{{\mathrm{1}}}}  :_{\textcolor{coeffectColor}{  \GRANULEnt{s_{{\mathrm{1}}}}  } }    \GRANULEnt{A} ^ \GRANULEnt{q}  \rightarrow  \GRANULEnt{B}    \GRANULEsym{,}   \GRANULEmv{x_{{\mathrm{2}}}}  :_{\textcolor{coeffectColor}{  \GRANULEnt{s_{{\mathrm{2}}}}  } }   \GRANULEnt{B} \\
            \Gamma  \GRANULEsym{,}   \GRANULEmv{x_{{\mathrm{1}}}}  :_{\textcolor{coeffectColor}{  \GRANULEnt{r_{{\mathrm{1}}}}  } }    \GRANULEnt{A} ^ \GRANULEnt{q}  \rightarrow  \GRANULEnt{B} ; \emptyset \vdash \GRANULEnt{A} \Downarrow\ \Rightarrow \GRANULEnt{t_{{\mathrm{2}}}} \mid\ \Delta_{{\mathrm{2}}}  \GRANULEsym{,}   \GRANULEmv{x_{{\mathrm{1}}}}  :_{\textcolor{coeffectColor}{  \GRANULEnt{s_{{\mathrm{3}}}}  } }    \GRANULEnt{A} ^ \GRANULEnt{q}  \rightarrow  \GRANULEnt{B}
            }
            {
            \Gamma  ;    \GRANULEmv{x_{{\mathrm{1}}}}  :_{\textcolor{coeffectColor}{  \GRANULEnt{r_{{\mathrm{1}}}}  } }    \GRANULEnt{A} ^ \GRANULEnt{q}  \rightarrow  \GRANULEnt{B}     \Downarrow \vdash  \GRANULEnt{C}  \Rightarrow    [  \GRANULEsym{(}  \GRANULEmv{x_{{\mathrm{1}}}} \, \GRANULEnt{t_{{\mathrm{2}}}}  \GRANULEsym{)}  /  \GRANULEmv{x_{{\mathrm{2}}}}  ]  \GRANULEnt{t_{{\mathrm{1}}}}    \mid  \GRANULEsym{(}  \Delta_{{\mathrm{1}}}  \GRANULEsym{+}   \textcolor{coeffectColor}{ \GRANULEnt{s_{{\mathrm{2}}}}   \textcolor{coeffectColor}{\,\cdot\,} }    \textcolor{coeffectColor}{ \GRANULEnt{q}   \textcolor{coeffectColor}{\,\cdot\,} }  \Delta_{{\mathrm{2}}}     \GRANULEsym{)}  \GRANULEsym{,}   \GRANULEmv{x_{{\mathrm{1}}}}  :_{\textcolor{coeffectColor}{   \GRANULEnt{s_{{\mathrm{2}}}}  \GRANULEsym{+}   \GRANULEnt{s_{{\mathrm{1}}}}  \GRANULEsym{+}   \GRANULEsym{(}  \GRANULEnt{s_{{\mathrm{2}}}}  \textcolor{coeffectColor}{\,\cdot\,}   \GRANULEnt{q}  \textcolor{coeffectColor}{\,\cdot\,}  \GRANULEnt{s_{{\mathrm{3}}}}   \GRANULEsym{)}     } }    \GRANULEnt{A} ^ \GRANULEnt{q}  \rightarrow  \GRANULEnt{B}
            }
            \]
            By induction on the first premise, we have that:
            \[
\Gamma  \GRANULEsym{,}   \GRANULEmv{x_{{\mathrm{1}}}}  :_{\textcolor{coeffectColor}{  \GRANULEnt{r_{{\mathrm{1}}}}  } }    \GRANULEnt{A} ^ \GRANULEnt{q}  \rightarrow  \GRANULEnt{B}    \GRANULEsym{,}    \GRANULEmv{x_{{\mathrm{2}}}}  :_{\textcolor{coeffectColor}{  \GRANULEnt{r_{{\mathrm{1}}}}  } }   \GRANULEnt{B}    \vdash  \GRANULEnt{C}  \Rightarrow  \GRANULEnt{t_{{\mathrm{1}}}}  \mid  \Delta_{{\mathrm{1}}}  \GRANULEsym{,}   \GRANULEmv{x_{{\mathrm{1}}}}  :_{\textcolor{coeffectColor}{  \GRANULEnt{s_{{\mathrm{1}}}}  } }    \GRANULEnt{A} ^ \GRANULEnt{q}  \rightarrow  \GRANULEnt{B}    \GRANULEsym{,}   \GRANULEmv{x_{{\mathrm{2}}}}  :_{\textcolor{coeffectColor}{  \GRANULEnt{s_{{\mathrm{2}}}}  } }   \GRANULEnt{B}
            \]
            from case 4 of the lemma. By induction on the second premise, we have that:
            \[
            \Gamma  \GRANULEsym{,}   \GRANULEmv{x_{{\mathrm{1}}}}  :_{\textcolor{coeffectColor}{  \GRANULEnt{r_{{\mathrm{1}}}}  } }    \GRANULEnt{A} ^ \GRANULEnt{q}  \rightarrow  \GRANULEnt{B} \vdash \GRANULEnt{A} \Rightarrow \GRANULEnt{t_{{\mathrm{2}}}} \mid\ \Delta_{{\mathrm{2}}}  \GRANULEsym{,}   \GRANULEmv{x_{{\mathrm{1}}}}  :_{\textcolor{coeffectColor}{  \GRANULEnt{s_{{\mathrm{3}}}}  } }    \GRANULEnt{A} ^ \GRANULEnt{q}  \rightarrow  \GRANULEnt{B}
            \]
            from case 3 of the lemma. From which, we can construc the following instantiation of the $\multimap_{\textsc{L}}$ synthesis rule in the non-focusing calculus:
            \[
            \inferrule*[right=$\multimap_{\textsc{L}}$]
            {
            \Gamma  \GRANULEsym{,}   \GRANULEmv{x_{{\mathrm{1}}}}  :_{\textcolor{coeffectColor}{  \GRANULEnt{r_{{\mathrm{1}}}}  } }    \GRANULEnt{A} ^ \GRANULEnt{q}  \rightarrow  \GRANULEnt{B}    \GRANULEsym{,}   \GRANULEmv{x_{{\mathrm{2}}}}  :_{\textcolor{coeffectColor}{  \GRANULEnt{r_{{\mathrm{1}}}}  } }   \GRANULEnt{B}   \vdash  \GRANULEnt{C}  \Rightarrow  \GRANULEnt{t_{{\mathrm{1}}}}  \mid  \Delta_{{\mathrm{1}}}  \GRANULEsym{,}   \GRANULEmv{x_{{\mathrm{1}}}}  :_{\textcolor{coeffectColor}{  \GRANULEnt{s_{{\mathrm{1}}}}  } }    \GRANULEnt{A} ^ \GRANULEnt{q}  \rightarrow  \GRANULEnt{B}    \GRANULEsym{,}   \GRANULEmv{x_{{\mathrm{2}}}}  :_{\textcolor{coeffectColor}{  \GRANULEnt{s_{{\mathrm{2}}}}  } }   \GRANULEnt{B}\\
            \Gamma  \GRANULEsym{,}   \GRANULEmv{x_{{\mathrm{1}}}}  :_{\textcolor{coeffectColor}{  \GRANULEnt{r_{{\mathrm{1}}}}  } }    \GRANULEnt{A} ^ \GRANULEnt{q}  \rightarrow  \GRANULEnt{B}    \vdash  \GRANULEnt{A}  \Rightarrow  \GRANULEnt{t_{{\mathrm{2}}}}  \mid  \Delta_{{\mathrm{2}}}  \GRANULEsym{,}   \GRANULEmv{x_{{\mathrm{1}}}}  :_{\textcolor{coeffectColor}{  \GRANULEnt{s_{{\mathrm{3}}}}  } }    \GRANULEnt{A} ^ \GRANULEnt{q}  \rightarrow  \GRANULEnt{B}
            }
            {\Gamma  \GRANULEsym{,}   \GRANULEmv{x_{{\mathrm{1}}}}  :_{\textcolor{coeffectColor}{  \GRANULEnt{r_{{\mathrm{1}}}}  } }    \GRANULEnt{A} ^ \GRANULEnt{q}  \rightarrow  \GRANULEnt{B}    \vdash  \GRANULEnt{C}  \Rightarrow    [  \GRANULEsym{(}  \GRANULEmv{x_{{\mathrm{1}}}} \, \GRANULEnt{t_{{\mathrm{2}}}}  \GRANULEsym{)}  /  \GRANULEmv{x_{{\mathrm{2}}}}  ]  \GRANULEnt{t_{{\mathrm{1}}}}    \mid  \GRANULEsym{(}  \Delta_{{\mathrm{1}}}  \GRANULEsym{+}   \textcolor{coeffectColor}{ \GRANULEnt{s_{{\mathrm{2}}}}   \textcolor{coeffectColor}{\,\cdot\,} }    \textcolor{coeffectColor}{ \GRANULEnt{q}   \textcolor{coeffectColor}{\,\cdot\,} }  \Delta_{{\mathrm{2}}}     \GRANULEsym{)}  \GRANULEsym{,}   \GRANULEmv{x_{{\mathrm{1}}}}  :_{\textcolor{coeffectColor}{   \GRANULEnt{s_{{\mathrm{2}}}}  \GRANULEsym{+}   \GRANULEnt{s_{{\mathrm{1}}}}  \GRANULEsym{+}   \GRANULEsym{(}  \GRANULEnt{s_{{\mathrm{2}}}}  \textcolor{coeffectColor}{\,\cdot\,}   \GRANULEnt{q}  \textcolor{coeffectColor}{\,\cdot\,}  \GRANULEnt{s_{{\mathrm{3}}}}   \GRANULEsym{)}     } }    \GRANULEnt{A} ^ \GRANULEnt{q}  \rightarrow  \GRANULEnt{B}}
            \]

      \item Case $\textsc{Var}$ \\
        In the case of the left synchronous rule for variable synthesis, the synthesis rule has the form:
          \[
          \fsynVar
          \]
          From which, we can construct the following instantiation of the \textsc{Var}\ synthesis rule in the non-focusing calculus:
          \[
            \synVar
          \]
    \end{enumerate}
\item Case 5. Right Focus: \\
          In the case of the focusing rule for transitioning from a left asynchronous judgement to a right synchronous judgement, the synthesis rule has the form:
          \[
            \fsynFocusRNoLabel
          \]
          By induction on the first premise, we have that:
          \[
            \Gamma  \vdash  \GRANULEnt{C}  \Rightarrow  \GRANULEnt{t}  \mid  \Delta \tag{ih}
          \]
          from case 2 of the lemma.
\item Case 6. Left Focus: \\
          In the case of the focusing rule for transitioning from a left asynchronous judgement to a left synchronous judgement, the synthesis rule has the form:
          \[
            \fsynFocusL
          \]
          By induction on the first premise, we have that:
          \[
            \Gamma  \GRANULEsym{,}   \GRANULEmv{x}  :  \GRANULEnt{A}   \vdash  \GRANULEnt{C}  \Rightarrow  \GRANULEnt{t}  \mid  \Delta \tag{ih}
          \]
          from case 2 of the lemma.
\end{enumerate}
\end{proof}

% \fi
\end{document}
