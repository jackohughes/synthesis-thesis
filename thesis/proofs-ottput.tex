\section{Linear-base Proofs}
\label{sec:linear-proofs}
This section gives the proofs of Lemma~\ref{lemma:subSynthSound} and
Lemma~\ref{lemma:addSynthSound}, along with soundness results for the
variant systems: additive pruning and subtractive division.

We first state and prove some intermediate results about context manipulations
which are needed for the main lemmas.

\begin{definition}[Context approximation]
For contexts $\Gamma_{{\mathrm{1}}}$, $\Gamma_{{\mathrm{2}}}$ then:
%
\begin{align*}
\begin{array}{c}
\dfrac{}{\emptyset  \  \textcolor{coeffectColor}{\sqsubseteq} \   \emptyset}
\qquad
\dfrac{\Gamma_{{\mathrm{1}}} \sqsubseteq \Gamma_{{\mathrm{2}}}}
      {\Gamma_{{\mathrm{1}}}  \GRANULEsym{,}   \GRANULEmv{x}  :  \GRANULEnt{A} \sqsubseteq \Gamma_{{\mathrm{2}}}  \GRANULEsym{,}   \GRANULEmv{x}  :  \GRANULEnt{A}}
\qquad \\[1.5em]
\dfrac{\Gamma_{{\mathrm{1}}} \sqsubseteq \Gamma_{{\mathrm{2}}} \qquad \GRANULEnt{r} \sqsubseteq \GRANULEnt{s}}
      {\Gamma_{{\mathrm{1}}}  \GRANULEsym{,}   \GRANULEmv{x}  :_{\textcolor{coeffectColor}{  \GRANULEnt{r}  } }   \GRANULEnt{A} \sqsubseteq \Gamma_{{\mathrm{2}}}  \GRANULEsym{,}   \GRANULEmv{x}  :_{\textcolor{coeffectColor}{  \GRANULEnt{s}  } }   \GRANULEnt{A}}
\qquad
\dfrac{ \Gamma_{{\mathrm{1}}} \sqsubseteq \Gamma_{{\mathrm{2}}} \qquad 0 \sqsubseteq \GRANULEnt{s}}
      { \Gamma_{{\mathrm{1}}} \sqsubseteq \Gamma_{{\mathrm{2}}}  \GRANULEsym{,}   \GRANULEmv{x}  :_{\textcolor{coeffectColor}{  \GRANULEnt{s}  } }   \GRANULEnt{A}}
\end{array}
\end{align*}
%
This is actioned in type checking by iterative application of $\textsc{Approx}$.
\end{definition}

\begin{restatable}[$\Gamma  \GRANULEsym{+}  \GRANULEsym{(}  \Gamma'  \GRANULEsym{-}  \Gamma''  \GRANULEsym{)} \sqsubseteq \GRANULEsym{(}  \Gamma  \GRANULEsym{+}  \Gamma'  \GRANULEsym{)}  \GRANULEsym{-}  \Gamma''$]{lemma}{contextLemma1}
  \label{lemma:contextLemma1}
\end{restatable}

\begin{proof}
  Induction over the structure of both $\Gamma'$ and $\Gamma''$. The possible forms of
  $\Gamma'$ and $\Gamma''$ are considered in turn:
  \begin{enumerate}
    \item $\Gamma'$ = $\emptyset$ and $\Gamma''$ = $\emptyset$\\
      We have:
      \begin{align*}
        (\Gamma + \emptyset) - \emptyset = \Gamma + (\emptyset - \emptyset)
      \end{align*}
      From definitions~\ref{def:contextAdd} and~\ref{def:contextSub}, we know that
      on the left hand side:
      \begin{align*}
        (\Gamma + \emptyset) - \emptyset &= \Gamma + \emptyset \\
                                &= \Gamma
      \end{align*}
      and on the right-hand side:
      \begin{align*}
        \Gamma + (\emptyset - \emptyset) &= \Gamma + \emptyset \\
                                &= \Gamma
      \end{align*}
      making both the left and right hand sides equivalent:
      \begin{align*}
        \Gamma = \Gamma
      \end{align*}
    \item $\Gamma'$ = $\Gamma'  \GRANULEsym{,}   \GRANULEmv{x}  :  \GRANULEnt{A}$ and $\Gamma''$ = $\emptyset$\\
      We have
      \begin{align*}
        (\Gamma + \Gamma'  \GRANULEsym{,}   \GRANULEmv{x}  :  \GRANULEnt{A}) - \emptyset = \Gamma + (\Gamma  \GRANULEsym{,}   \GRANULEmv{x}  :  \GRANULEnt{A} - \emptyset)
      \end{align*}
      From definitions~\ref{def:contextAdd} and~\ref{def:contextSub}, we know that
      on the left hand side we have:
      \begin{align*}
        (\Gamma + \Gamma'  \GRANULEsym{,}   \GRANULEmv{x}  :  \GRANULEnt{A}) - \emptyset &= (\Gamma, \Gamma'), \GRANULEmv{x}  :  \GRANULEnt{A} - \emptyset \\
                                        &= (\Gamma, \Gamma'), \GRANULEmv{x}  :  \GRANULEnt{A}
      \end{align*}
      and on the right hand side:
      \begin{align*}
        \Gamma + (\Gamma  \GRANULEsym{,}   \GRANULEmv{x}  :  \GRANULEnt{A} - \emptyset) &= \Gamma + \Gamma'  \GRANULEsym{,}   \GRANULEmv{x}  :  \GRANULEnt{A}\\
                                       &= (\Gamma, \Gamma', \GRANULEmv{x}  :  \GRANULEnt{A})
      \end{align*}
      making both the left and right hand sides equal:
      \begin{align*}
        (\Gamma,\Gamma'), \GRANULEmv{x}  :  \GRANULEnt{A} = (\Gamma,\Gamma'), \GRANULEmv{x}  :  \GRANULEnt{A}
      \end{align*}


    \item $\Gamma'$ = $\Gamma'  \GRANULEsym{,}   \GRANULEmv{x}  :  \GRANULEnt{A}$ and $\Gamma''$ = $\Gamma''  \GRANULEsym{,}   \GRANULEmv{x}  :  \GRANULEnt{A}$\\
      We have
      \begin{align*}
        (\Gamma + \Gamma'  \GRANULEsym{,}   \GRANULEmv{x}  :  \GRANULEnt{A}) - \Gamma''  \GRANULEsym{,}   \GRANULEmv{x}  :  \GRANULEnt{A} = \Gamma + (\Gamma'  \GRANULEsym{,}   \GRANULEmv{x}  :  \GRANULEnt{A} -
        \Gamma''  \GRANULEsym{,}   \GRANULEmv{x}  :  \GRANULEnt{A})
      \end{align*}
      From definitions~\ref{def:contextAdd} and~\ref{def:contextSub}, we know that
      on the left hand side we have:
      \begin{align*}
        (\Gamma + \Gamma'  \GRANULEsym{,}   \GRANULEmv{x}  :  \GRANULEnt{A}) - \Gamma''  \GRANULEsym{,}   \GRANULEmv{x}  :  \GRANULEnt{A} &= (\Gamma,\Gamma'), \GRANULEmv{x}  :  \GRANULEnt{A} -
                                                   \Gamma''  \GRANULEsym{,}   \GRANULEmv{x}  :  \GRANULEnt{A} \\
                                                 &= \Gamma, \Gamma' - \Gamma''
      \end{align*}
      and on the right hand side:
      \begin{align*}
        \Gamma + (\Gamma'  \GRANULEsym{,}   \GRANULEmv{x}  :  \GRANULEnt{A} - \Gamma''  \GRANULEsym{,}   \GRANULEmv{x}  :  \GRANULEnt{A}) &= \Gamma + (\Gamma' - \Gamma'') \\
                                                &= \Gamma, \Gamma' - \Gamma''
      \end{align*}
      making both the left and right hand sides equivalent:
      \begin{align*}
        \Gamma, \Gamma' - \Gamma'' = \Gamma, \Gamma' - \Gamma''
      \end{align*}

    \item $\Gamma'$ = $\Gamma'  \GRANULEsym{,}   \GRANULEmv{x}  :_{\textcolor{coeffectColor}{  \GRANULEnt{r}  } }   \GRANULEnt{A}$ and $\Gamma''$ = $\emptyset$\\
      We have
      \begin{align*}
        (\Gamma + \Gamma'  \GRANULEsym{,}   \GRANULEmv{x}  :_{\textcolor{coeffectColor}{  \GRANULEnt{r}  } }   \GRANULEnt{A}) - \emptyset = \Gamma + (\GRANULEmv{x}  :_{\textcolor{coeffectColor}{  \GRANULEnt{r}  } }   \GRANULEnt{A} - \emptyset)
      \end{align*}
      From definitions~\ref{def:contextAdd} and~\ref{def:contextSub}, we know that
      on the left hand side we have:
      \begin{align*}
        (\Gamma + \Gamma'  \GRANULEsym{,}   \GRANULEmv{x}  :_{\textcolor{coeffectColor}{  \GRANULEnt{r}  } }   \GRANULEnt{A}) - \emptyset &= (\Gamma + \Gamma'  \GRANULEsym{,}   \GRANULEmv{x}  :_{\textcolor{coeffectColor}{  \GRANULEnt{r}  } }   \GRANULEnt{A}) \\
                                            &= (\Gamma, \Gamma'), \GRANULEmv{x}  :_{\textcolor{coeffectColor}{  \GRANULEnt{r}  } }   \GRANULEnt{A}
      \end{align*}
      and on the right hand side:
      \begin{align*}
        \Gamma + (\Gamma'  \GRANULEsym{,}   \GRANULEmv{x}  :_{\textcolor{coeffectColor}{  \GRANULEnt{r}  } }   \GRANULEnt{A} - \emptyset) &= \Gamma + (\Gamma'  \GRANULEsym{,}   \GRANULEmv{x}  :_{\textcolor{coeffectColor}{  \GRANULEnt{r}  } }   \GRANULEnt{A})
                                            &= (\Gamma,\Gamma'),\GRANULEmv{x}  :_{\textcolor{coeffectColor}{  \GRANULEnt{r}  } }   \GRANULEnt{A}
      \end{align*}
      making both the left and right hand sides equivalent:
      \begin{align*}
        (\Gamma,\Gamma'),\GRANULEmv{x}  :_{\textcolor{coeffectColor}{  \GRANULEnt{r}  } }   \GRANULEnt{A} = (\Gamma,\Gamma'),\GRANULEmv{x}  :_{\textcolor{coeffectColor}{  \GRANULEnt{r}  } }   \GRANULEnt{A}
      \end{align*}


    \item $\Gamma'$ = $\Gamma'  \GRANULEsym{,}   \GRANULEmv{x}  :_{\textcolor{coeffectColor}{  \GRANULEnt{r}  } }   \GRANULEnt{A}$ and $\Gamma''$ = $\Gamma''  \GRANULEsym{,}   \GRANULEmv{x}  :_{\textcolor{coeffectColor}{  \GRANULEnt{s}  } }   \GRANULEnt{A}$\\

      Thus we have (for the LHS of the inequality term):
      %
       \begin{align*}
        \Gamma + (\Gamma'  \GRANULEsym{,}   \GRANULEmv{x}  :_{\textcolor{coeffectColor}{  \GRANULEnt{r}  } }   \GRANULEnt{A} - \Gamma''  \GRANULEsym{,}   \GRANULEmv{x}  :_{\textcolor{coeffectColor}{  \GRANULEnt{s}  } }   \GRANULEnt{A})
      \end{align*}
      %
    which by context subtraction yields:
      \begin{align*}
       \Gamma + (\Gamma'  \GRANULEsym{,}   \GRANULEmv{x}  :_{\textcolor{coeffectColor}{  \GRANULEnt{r}  } }   \GRANULEnt{A} - \Gamma''  \GRANULEsym{,}   \GRANULEmv{x}  :_{\textcolor{coeffectColor}{  \GRANULEnt{s}  } }   \GRANULEnt{A}) &= \Gamma + (\Gamma' -
                                                         \Gamma''), \GRANULEmv{x}  :_{\textcolor{coeffectColor}{  \GRANULEnt{q'}  } }   \GRANULEnt{A}
      \end{align*}
      where:
      \begin{align*}
        \exists q' . \GRANULEnt{r} \sqsupseteq \GRANULEnt{q'} + \GRANULEnt{s}
\quad \maximal{q'}{\hat{q'}}{r}{\hat{q'}+s} \qquad (2)
      \end{align*}
       %
       And for the LHS of the inequality, from
       definitions~\ref{def:contextAdd} and~\ref{def:contextSub}
       we have:
      \begin{align*}
        (\Gamma + \Gamma'  \GRANULEsym{,}   \GRANULEmv{x}  :_{\textcolor{coeffectColor}{  \GRANULEnt{r}  } }   \GRANULEnt{A}) - \Gamma''  \GRANULEsym{,}   \GRANULEmv{x}  :_{\textcolor{coeffectColor}{  \GRANULEnt{s}  } }   \GRANULEnt{A} &
= (\Gamma + \Gamma'), \GRANULEmv{x}  :_{\textcolor{coeffectColor}{  \GRANULEnt{r}  } }   \GRANULEnt{A} - \Gamma''  \GRANULEsym{,}   \GRANULEmv{x}  :_{\textcolor{coeffectColor}{  \GRANULEnt{s}  } }   \GRANULEnt{A} \\
  &= ((\Gamma + \Gamma') -  \Gamma''), \GRANULEmv{x}  :_{\textcolor{coeffectColor}{  \GRANULEnt{r}  } }   \GRANULEnt{A} - \GRANULEmv{x}  :_{\textcolor{coeffectColor}{  \GRANULEnt{s}  } }   \GRANULEnt{A} \\
  &= ((\Gamma + \Gamma') -  \Gamma''), \GRANULEmv{x}  :_{\textcolor{coeffectColor}{  \GRANULEnt{q}  } }   \GRANULEnt{A}
      \end{align*}
      where:
      \begin{align*}
        \exists q . \GRANULEnt{r} \sqsupseteq \GRANULEnt{q} + \GRANULEnt{s}
\quad \maximal{q}{\hat{q}}{r}{\hat{q}+s} \qquad (1)
      \end{align*}
    %
    %Applying maximality (1) to $q'$ yields that $\texttt{\textcolor{red}{<<multiple parses>>}}$
    Applying $\exists q . \GRANULEnt{r} \sqsupseteq \GRANULEnt{q} + \GRANULEnt{s}$ to
    maximality (2) (at $\hat{q'} = q$) then yields that $q \sqsubseteq q'$.

    Therefore, applying induction, we derive:
    %
     \begin{align*}
      \dfrac{\GRANULEsym{(}  \Gamma  \GRANULEsym{+}  \GRANULEsym{(}  \Gamma'  \GRANULEsym{-}  \Gamma''  \GRANULEsym{)}  \GRANULEsym{)} \sqsubseteq \GRANULEsym{(}  \GRANULEsym{(}  \Gamma  \GRANULEsym{+}  \Gamma'  \GRANULEsym{)}  \GRANULEsym{-}  \Gamma''  \GRANULEsym{)} \qquad \GRANULEnt{q} \sqsubseteq \GRANULEnt{q'}}
             {\GRANULEsym{(}  \Gamma  \GRANULEsym{+}  \GRANULEsym{(}  \Gamma'  \GRANULEsym{-}  \Gamma''  \GRANULEsym{)}  \GRANULEsym{)}  \GRANULEsym{,}   \GRANULEmv{x}  :_{\textcolor{coeffectColor}{  \GRANULEnt{q}  } }   \GRANULEnt{A} \sqsubseteq \GRANULEsym{(}  \GRANULEsym{(}  \Gamma  \GRANULEsym{+}  \Gamma'  \GRANULEsym{)}  \GRANULEsym{-}  \Gamma''  \GRANULEsym{)}  \GRANULEsym{,}   \GRANULEmv{x}  :_{\textcolor{coeffectColor}{  \GRANULEnt{q'}  } }   \GRANULEnt{A}}
     \end{align*}
    %
    satisfying the lemma statement.
    %\dnote{if partial order then $\GRANULEnt{q} = \GRANULEnt{q'}$, but we can be weaker
    %  and weaken Lemma 4 to be a context approximation.}
    %  making both the left and right hand sides equivalent:

  \end{enumerate}
\end{proof}

\begin{restatable}[$\GRANULEsym{(}  \Gamma  \GRANULEsym{-}  \Gamma'  \GRANULEsym{)}  \GRANULEsym{+}  \Gamma' \sqsubseteq \Gamma$]{lemma}{contextLemma2}
  \label{lemma:contextLemma2}
\end{restatable}
\begin{proof}
 The proof follows by induction over the structure of $\Gamma'$. The possible
 forms of $\Gamma'$ are considered in turn:
 \begin{enumerate}
     \item $\Gamma'$ = $\emptyset$\\
     We have:
     \begin{align*}
       (\Gamma - \emptyset) + \emptyset = \Gamma
     \end{align*}
     From definition~\ref{def:contextSub}, we know that:
     \begin{align*}
       \Gamma - \emptyset = \Gamma
     \end{align*}
     and from definition~\ref{def:contextAdd}, we know:
     \begin{align*}
       \Gamma + \emptyset = \Gamma
     \end{align*}
     giving us:
     \begin{align*}
       \Gamma = \Gamma
     \end{align*}


     \item $\Gamma'$ = $\Gamma''  \GRANULEsym{,}   \GRANULEmv{x}  :  \GRANULEnt{A}$\\
     and let $\Gamma = \Gamma'  \GRANULEsym{,}   \GRANULEmv{x}  :  \GRANULEnt{A}$.

     \begin{align*}
       (\Gamma'  \GRANULEsym{,}   \GRANULEmv{x}  :  \GRANULEnt{A} - \Gamma''  \GRANULEsym{,}   \GRANULEmv{x}  :  \GRANULEnt{A}) + \Gamma''  \GRANULEsym{,}   \GRANULEmv{x}  :  \GRANULEnt{A} = \Gamma
     \end{align*}
     From definition~\ref{def:contextAdd}, we know that:
     \begin{align*}
       (\Gamma'  \GRANULEsym{,}   \GRANULEmv{x}  :  \GRANULEnt{A} - \Gamma''  \GRANULEsym{,}   \GRANULEmv{x}  :  \GRANULEnt{A}) + \Gamma''  \GRANULEsym{,}   \GRANULEmv{x}  :  \GRANULEnt{A}
       & =  ((\Gamma'  \GRANULEsym{-}  \Gamma'') + \Gamma''), \GRANULEmv{x}  :  \GRANULEnt{A} \\
 \textit{induction}  & = \Gamma'  \GRANULEsym{,}   \GRANULEmv{x}  :  \GRANULEnt{A} \\
                     & = \Gamma
     \end{align*}
     thus satisfying the lemma statement by equality.

     \item $\Gamma'$ = $\Gamma''  \GRANULEsym{,}   \GRANULEmv{x}  :_{\textcolor{coeffectColor}{  \GRANULEnt{r}  } }   \GRANULEnt{A}$\\
       and let $\Gamma = \Gamma'  \GRANULEsym{,}   \GRANULEmv{x}  :_{\textcolor{coeffectColor}{  \GRANULEnt{s}  } }   \GRANULEnt{A}$.

     We have:
     \begin{align*}
       (\Gamma'  \GRANULEsym{,}   \GRANULEmv{x}  :_{\textcolor{coeffectColor}{  \GRANULEnt{s}  } }   \GRANULEnt{A} - \Gamma''  \GRANULEsym{,}   \GRANULEmv{x}  :_{\textcolor{coeffectColor}{  \GRANULEnt{r}  } }   \GRANULEnt{A}) + \Gamma''  \GRANULEsym{,}   \GRANULEmv{x}  :_{\textcolor{coeffectColor}{  \GRANULEnt{r}  } }   \GRANULEnt{A}
     \end{align*}
     From definition~\ref{def:contextSub}, we know that:
     %

       \begin{align*}
      & (\Gamma'  \GRANULEsym{,}   \GRANULEmv{x}  :_{\textcolor{coeffectColor}{  \GRANULEnt{s}  } }   \GRANULEnt{A} - \Gamma''  \GRANULEsym{,}   \GRANULEmv{x}  :_{\textcolor{coeffectColor}{  \GRANULEnt{r}  } }   \GRANULEnt{A}) + \Gamma''  \GRANULEsym{,}   \GRANULEmv{x}  :_{\textcolor{coeffectColor}{  \GRANULEnt{r}  } }   \GRANULEnt{A} \\
    =\ & \GRANULEsym{(}  \Gamma'  \GRANULEsym{-}  \Gamma''  \GRANULEsym{)}  \GRANULEsym{,}   \GRANULEmv{x}  :_{\textcolor{coeffectColor}{  \GRANULEnt{q}  } }   \GRANULEnt{A} + \Gamma''  \GRANULEsym{,}   \GRANULEmv{x}  :_{\textcolor{coeffectColor}{  \GRANULEnt{r}  } }   \GRANULEnt{A} \\
    =\ & \GRANULEsym{(}  \GRANULEsym{(}  \Gamma'  \GRANULEsym{-}  \Gamma''  \GRANULEsym{)}  \GRANULEsym{+}  \Gamma''  \GRANULEsym{)}  \GRANULEsym{,}   \GRANULEmv{x}  :_{\textcolor{coeffectColor}{   \GRANULEnt{q}  \GRANULEsym{+}  \GRANULEnt{r}   } }   \GRANULEnt{A}
       \end{align*}

     %
     where $\GRANULEnt{s}  \sqsupseteq  \GRANULEnt{q}  \GRANULEsym{+}  \GRANULEnt{r}$ and $\maximal{q}{q'}{s}{q' + r}$.

     Then by induction we derive the ordering:
     %
     \begin{align*}
       \dfrac{\GRANULEsym{(}  \GRANULEsym{(}  \Gamma'  \GRANULEsym{-}  \Gamma''  \GRANULEsym{)}  \GRANULEsym{+}  \Gamma''  \GRANULEsym{)} \sqsubseteq \Gamma' \qquad \GRANULEnt{q}  \GRANULEsym{+}  \GRANULEnt{r}  \sqsubseteq \GRANULEnt{s}}
            {\GRANULEsym{(}  \GRANULEsym{(}  \Gamma'  \GRANULEsym{-}  \Gamma''  \GRANULEsym{)}  \GRANULEsym{+}  \Gamma''  \GRANULEsym{)}  \GRANULEsym{,}   \GRANULEmv{x}  :_{\textcolor{coeffectColor}{   \GRANULEnt{q}  \GRANULEsym{+}  \GRANULEnt{r}   } }   \GRANULEnt{A} \sqsubseteq \Gamma'  \GRANULEsym{,}   \GRANULEmv{x}  :_{\textcolor{coeffectColor}{  \GRANULEnt{s}  } }   \GRANULEnt{A}}
     \end{align*}
     %
     which satifies the lemma statement.
 \end{enumerate}
\end{proof}

\begin{lemma}[Context negation]
\label{lemma:context-neg}
For all contexts $\Gamma$:
\begin{align*}
\emptyset \sqsubseteq \Gamma  \GRANULEsym{-}  \Gamma
\end{align*}
\end{lemma}

\begin{proof}
By induction on the structure of $\Gamma$:
%
\begin{itemize}
\item $\Gamma = \emptyset$ Trivial.

\item $\Gamma = \Gamma'  \GRANULEsym{,}   \GRANULEmv{x}  :  \GRANULEnt{A}$ then
$\GRANULEsym{(}  \Gamma'  \GRANULEsym{,}   \GRANULEmv{x}  :  \GRANULEnt{A}   \GRANULEsym{)}  \GRANULEsym{-}  \GRANULEsym{(}  \Gamma'  \GRANULEsym{,}   \GRANULEmv{x}  :  \GRANULEnt{A}   \GRANULEsym{)} = \Gamma'  \GRANULEsym{-}  \Gamma'$ so proceed by
induction.

\item $\Gamma = \Gamma'  \GRANULEsym{,}   \GRANULEmv{x}  :_{\textcolor{coeffectColor}{  \GRANULEnt{r}  } }   \GRANULEnt{A}$ then
$\exists q. $
$\GRANULEsym{(}  \Gamma'  \GRANULEsym{,}   \GRANULEmv{x}  :_{\textcolor{coeffectColor}{  \GRANULEnt{r}  } }   \GRANULEnt{A}   \GRANULEsym{)}  \GRANULEsym{-}  \GRANULEsym{(}  \Gamma'  \GRANULEsym{,}   \GRANULEmv{x}  :_{\textcolor{coeffectColor}{  \GRANULEnt{r}  } }   \GRANULEnt{A}   \GRANULEsym{)} = \GRANULEsym{(}  \Gamma  \GRANULEsym{-}  \Gamma'  \GRANULEsym{)}  \GRANULEsym{,}   \GRANULEmv{x}  :_{\textcolor{coeffectColor}{  \GRANULEnt{q}  } }   \GRANULEnt{A}$

such that $\GRANULEnt{r} \sqsupseteq \GRANULEnt{q}  \GRANULEsym{+}  \GRANULEnt{r}$ and
$\maximal{q}{q'}{r}{q'+r}$.

Instantiating maximality with $q' = 0$ and reflexivity then we have
$0 \sqsubseteq \GRANULEnt{q}$. From this, and the inductive hypothesis, we can construct:
%
\begin{align*}
\dfrac{ \emptyset \sqsubseteq \GRANULEsym{(}  \Gamma  \GRANULEsym{-}  \Gamma'  \GRANULEsym{)} \quad 0 \sqsubseteq \GRANULEnt{q}}
{ \emptyset \sqsubseteq \GRANULEsym{(}  \Gamma  \GRANULEsym{-}  \Gamma'  \GRANULEsym{)}  \GRANULEsym{,}   \GRANULEmv{x}  :_{\textcolor{coeffectColor}{  \GRANULEnt{q}  } }   \GRANULEnt{A} }
\end{align*}
%
\end{itemize}
%
\end{proof}

\begin{lemma}
\label{lemma:contexts-subsub}
For all contexts $\Gamma_{{\mathrm{1}}}$, $\Gamma_{{\mathrm{2}}}$, where
$\GRANULEsym{[}  \Gamma_{{\mathrm{2}}}  \GRANULEsym{]}$ (i.e., $\Gamma_{{\mathrm{2}}}$ is all graded)
then:
%
\begin{align*}
\Gamma_{{\mathrm{2}}} \sqsubseteq \Gamma_{{\mathrm{1}}}  \GRANULEsym{-}  \GRANULEsym{(}  \Gamma_{{\mathrm{1}}}  \GRANULEsym{-}  \Gamma_{{\mathrm{2}}}  \GRANULEsym{)}
\end{align*}
\end{lemma}

\begin{proof}
By induction on the structure of $\Gamma_{{\mathrm{2}}}$.

\begin{itemize}
\item $\Gamma_{{\mathrm{2}}} = \sqsubseteq$

Then $\Gamma_{{\mathrm{1}}}  \GRANULEsym{-}  \GRANULEsym{(}  \Gamma_{{\mathrm{1}}}  \GRANULEsym{-}   \emptyset   \GRANULEsym{)} = \Gamma_{{\mathrm{1}}}  \GRANULEsym{-}  \Gamma_{{\mathrm{1}}}$.

By Lemma~\ref{lemma:context-neg}, then $ \emptyset \sqsubseteq \GRANULEsym{(}  \Gamma_{{\mathrm{1}}}  \GRANULEsym{-}  \Gamma_{{\mathrm{1}}}  \GRANULEsym{)}$
satisfying this case.

\item $\Gamma_{{\mathrm{2}}} = \Gamma'_{{\mathrm{2}}}  \GRANULEsym{,}   \GRANULEmv{x}  :_{\textcolor{coeffectColor}{  \GRANULEnt{s}  } }   \GRANULEnt{A}$

By the premises $\Gamma_{{\mathrm{1}}} \sqsubseteq \Gamma_{{\mathrm{2}}}$ then we can
assume $\GRANULEmv{x} \in \Gamma_{{\mathrm{1}}}$ and thus (by context
rearrangement) $\Gamma'_{{\mathrm{1}}}  \GRANULEsym{,}   \GRANULEmv{x}  :_{\textcolor{coeffectColor}{  \GRANULEnt{r}  } }   \GRANULEnt{A}$.

Thus we consider $\GRANULEsym{(}  \Gamma'_{{\mathrm{1}}}  \GRANULEsym{,}   \GRANULEmv{x}  :_{\textcolor{coeffectColor}{  \GRANULEnt{r}  } }   \GRANULEnt{A}   \GRANULEsym{)}  \GRANULEsym{-}  \GRANULEsym{(}  \GRANULEsym{(}  \Gamma'_{{\mathrm{1}}}  \GRANULEsym{,}   \GRANULEmv{x}  :_{\textcolor{coeffectColor}{  \GRANULEnt{r}  } }   \GRANULEnt{A}   \GRANULEsym{)}  \GRANULEsym{-}  \GRANULEsym{(}  \Gamma'_{{\mathrm{2}}}  \GRANULEsym{,}   \GRANULEmv{x}  :_{\textcolor{coeffectColor}{  \GRANULEnt{s}  } }   \GRANULEnt{A}   \GRANULEsym{)}  \GRANULEsym{)}$.

\begin{align*}
  \; & \GRANULEsym{(}  \Gamma'_{{\mathrm{1}}}  \GRANULEsym{,}   \GRANULEmv{x}  :_{\textcolor{coeffectColor}{  \GRANULEnt{r}  } }   \GRANULEnt{A}   \GRANULEsym{)}  \GRANULEsym{-}  \GRANULEsym{(}  \GRANULEsym{(}  \Gamma'_{{\mathrm{1}}}  \GRANULEsym{,}   \GRANULEmv{x}  :_{\textcolor{coeffectColor}{  \GRANULEnt{r}  } }   \GRANULEnt{A}   \GRANULEsym{)}  \GRANULEsym{-}  \GRANULEsym{(}  \Gamma'_{{\mathrm{2}}}  \GRANULEsym{,}   \GRANULEmv{x}  :_{\textcolor{coeffectColor}{  \GRANULEnt{s}  } }   \GRANULEnt{A}   \GRANULEsym{)}  \GRANULEsym{)} \\
= \; & \GRANULEsym{(}  \Gamma'_{{\mathrm{1}}}  \GRANULEsym{,}   \GRANULEmv{x}  :_{\textcolor{coeffectColor}{  \GRANULEnt{r}  } }   \GRANULEnt{A}   \GRANULEsym{)}  \GRANULEsym{-}  \GRANULEsym{(}  \GRANULEsym{(}  \Gamma'_{{\mathrm{1}}}  \GRANULEsym{-}  \Gamma'_{{\mathrm{2}}}  \GRANULEsym{)}  \GRANULEsym{,}   \GRANULEmv{x}  :_{\textcolor{coeffectColor}{  \GRANULEnt{q}  } }   \GRANULEnt{A}   \GRANULEsym{)} \\
= \; & \GRANULEsym{(}  \Gamma'_{{\mathrm{1}}}  \GRANULEsym{-}  \GRANULEsym{(}  \Gamma'_{{\mathrm{1}}}  \GRANULEsym{-}  \Gamma'_{{\mathrm{2}}}  \GRANULEsym{)}  \GRANULEsym{)}  \GRANULEsym{,}   \GRANULEmv{x}  :_{\textcolor{coeffectColor}{  \GRANULEnt{q'}  } }   \GRANULEnt{A}
\end{align*}
%
where (1) $\exists  \GRANULEnt{q}  .\,   \GRANULEnt{r}  \sqsupseteq  \GRANULEnt{q}  \GRANULEsym{+}  \GRANULEnt{s}$ with
(2) $(\maximal{q}{\hat{q}}{r}{\hat{q}+s})$

and (3) $\exists  \GRANULEnt{q'}  .\,   \GRANULEnt{r}  \sqsupseteq  \GRANULEnt{q'}  \GRANULEsym{+}  \GRANULEnt{q}$
with (4) $(\maximal{q'}{\hat{q'}}{r}{\hat{q'}+s})$.

Apply (1) to (4) by letting $\hat{q'} = {s}$
and by commutativity of $+$ then we get that
$\GRANULEnt{q'}  \sqsupseteq  \GRANULEnt{s}$.

By induction we have that
%
\begin{align*}
\Gamma'_{{\mathrm{1}}} \sqsubseteq \Gamma'_{{\mathrm{1}}}  \GRANULEsym{-}  \GRANULEsym{(}  \Gamma'_{{\mathrm{1}}}  \GRANULEsym{-}  \Gamma'_{{\mathrm{2}}}  \GRANULEsym{)} \tag{ih}
\end{align*}
%
%
Thus we get that:
%
\begin{align*}
\dfrac{\GRANULEnt{s} \sqsubseteq \GRANULEnt{q'} \quad
\Gamma'_{{\mathrm{1}}} \sqsubseteq \Gamma'_{{\mathrm{1}}}  \GRANULEsym{-}  \GRANULEsym{(}  \Gamma'_{{\mathrm{1}}}  \GRANULEsym{-}  \Gamma'_{{\mathrm{2}}}  \GRANULEsym{)}}
{ \Gamma'_{{\mathrm{1}}}  \GRANULEsym{,}   \GRANULEmv{x}  :_{\textcolor{coeffectColor}{  \GRANULEnt{s}  } }   \GRANULEnt{A} \sqsubseteq \GRANULEsym{(}  \Gamma'_{{\mathrm{1}}}  \GRANULEsym{-}  \GRANULEsym{(}  \Gamma'_{{\mathrm{1}}}  \GRANULEsym{-}  \Gamma'_{{\mathrm{2}}}  \GRANULEsym{)}  \GRANULEsym{)}  \GRANULEsym{,}   \GRANULEmv{x}  :_{\textcolor{coeffectColor}{  \GRANULEnt{q'}  } }   \GRANULEnt{A} }
\end{align*}
%
\item $\Gamma_{{\mathrm{2}}} = \Gamma'_{{\mathrm{2}}}  \GRANULEsym{,}   \GRANULEmv{x}  :  \GRANULEnt{A}$ Trivial as it violates the grading
  condition of the premise.
\end{itemize}
\end{proof}

%%%%%%%%%%%%%%%%%%%%%%%%%%%%%%%%%%%%%%%%%%%%%
\subSynthSound*

\begin{proof}
Structural induction over the synthesis rules. Each of the possible synthesis
rules are considered in turn.

\begin{enumerate}[itemsep=1em]
  \item Case \subLinVarName \\
  In the case of linear variable synthesis, we have the derivation:
  \[
    \subLinVar{}
  \]
  %
  By the definition of context subtraction,
  $(\Gamma  \GRANULEsym{,}   \GRANULEmv{x}  :  \GRANULEnt{A}) - \Gamma = \GRANULEmv{x}  :  \GRANULEnt{A}$, thus we
  can construct the following typing derivation, matching the
  conclusion:
  \[
    \inferrule*[Right=\textsc{Var}]
    {\quad}
    {\GRANULEmv{x}  :  \GRANULEnt{A}   \vdash  \GRANULEmv{x}  :  \GRANULEnt{A}}
  \]

\item Case \subGrVarName{} \\
    Matching the form of the lemma, we have the derivation:
    \[
    \subGrVar{}
    \]
    %
    By the definition of context subtraction,
    $ \GRANULEsym{(}  \Gamma  \GRANULEsym{,}   \GRANULEmv{x}  :_{\textcolor{coeffectColor}{  \GRANULEnt{r}  } }   \GRANULEnt{A}   \GRANULEsym{)}  \GRANULEsym{-}  \GRANULEsym{(}  \Gamma  \GRANULEsym{,}   \GRANULEmv{x}  :_{\textcolor{coeffectColor}{  \GRANULEnt{s}  } }   \GRANULEnt{A}   \GRANULEsym{)} = \GRANULEmv{x}  :_{\textcolor{coeffectColor}{  \GRANULEnt{q}  } }   \GRANULEnt{A}$
    where (1) $\exists  \GRANULEnt{q}  .\,   \GRANULEnt{r}  \sqsupseteq  \GRANULEnt{q}  \GRANULEsym{+}  \GRANULEnt{s}$ and $\maximal{q}{q'}{r}{q' + s}$.

    Applying maximality (1) with $q = 1$ then we have that $1 \sqsubseteq \GRANULEnt{q}$ (*)

    Thus, from this we can construct the typing derivation, matching the conclusion:
    %
    \[
    \inferrule*[Right=\textsc{Approx}]
    {
      \inferrule*[Right=\textsc{Der}]
        {\inferrule*[Right=\textsc{Var}]
          {\quad}
          {\GRANULEmv{x}  :  \GRANULEnt{A}   \vdash  \GRANULEmv{x}  :  \GRANULEnt{A}}
        }
        {\GRANULEmv{x}  :_{\textcolor{coeffectColor}{   1   } }   \GRANULEnt{A}   \vdash  \GRANULEmv{x}  :  \GRANULEnt{A} \qquad 1 \sqsubseteq \GRANULEnt{q} \; (*)}
    }{
      \GRANULEmv{x}  :_{\textcolor{coeffectColor}{  \GRANULEnt{q}  } }   \GRANULEnt{A}   \vdash  \GRANULEmv{x}  :  \GRANULEnt{A}
    }
    \]

  \item Case \subAbsName \\
    We thus have the derivation:
    %
    \[
    \subAbs{}
    \]
    %
    By induction we then have that:
    %
    \[
      \GRANULEsym{(}  \Gamma  \GRANULEsym{,}   \GRANULEmv{x}  :  \GRANULEnt{A}   \GRANULEsym{)}  \GRANULEsym{-}  \Delta  \vdash  \GRANULEnt{t}  :  \GRANULEnt{B}
    \]
    %
    Since $\GRANULEmv{x}  \not\in | \Delta |$ then by the definition of context
    subtraction we have that $\GRANULEsym{(}  \Gamma  \GRANULEsym{,}   \GRANULEmv{x}  :  \GRANULEnt{A}   \GRANULEsym{)}  \GRANULEsym{-}  \Delta = \GRANULEsym{(}  \Gamma  \GRANULEsym{-}  \Delta  \GRANULEsym{)}  \GRANULEsym{,}   \GRANULEmv{x}  :  \GRANULEnt{A}$.
    From this, we can construct the following derivation, matching the
    conclusion:
    %
    \[
    \inferrule*[Right=Abs]
    {\GRANULEsym{(}  \Gamma  \GRANULEsym{-}  \Delta  \GRANULEsym{)}  \GRANULEsym{,}   \GRANULEmv{x}  :  \GRANULEnt{A}   \vdash  \GRANULEnt{t}  :  \GRANULEnt{B}}{\Gamma  \GRANULEsym{-}  \Delta  \vdash   \lambda  \GRANULEmv{x}  .  \GRANULEnt{t}   :   \GRANULEnt{A}  \multimap  \GRANULEnt{B}}
    \]


  \item Case \subAppName \\
    %
    Matching the form of the lemma, the application derivation is:
    \[
      \subApp{}
    \]
    %
    By induction, we have that:
    %
      \begin{align*}
        \GRANULEsym{(}  \Gamma  \GRANULEsym{,}   \GRANULEmv{x_{{\mathrm{2}}}}  :  \GRANULEnt{B}   \GRANULEsym{)}  \GRANULEsym{-}  \Delta_{{\mathrm{1}}}  \vdash  \GRANULEnt{t_{{\mathrm{1}}}}  :  \GRANULEnt{C} \tag{ih1} \\
        \Delta_{{\mathrm{1}}}  \GRANULEsym{-}  \Delta_{{\mathrm{2}}}  \vdash  \GRANULEnt{t_{{\mathrm{2}}}}  :  \GRANULEnt{A} \tag{ih2}
      \end{align*}
    %
    By the definition of context subtraction and since $\GRANULEmv{x_{{\mathrm{2}}}}  \not\in | \Delta_{{\mathrm{1}}} |$ then (ih1) is equal to:
    %
      \begin{align*}
        \GRANULEsym{(}  \Gamma  \GRANULEsym{-}  \Delta_{{\mathrm{1}}}  \GRANULEsym{)}  \GRANULEsym{,}   \GRANULEmv{x_{{\mathrm{2}}}}  :  \GRANULEnt{B}   \vdash  \GRANULEnt{t_{{\mathrm{1}}}}  :  \GRANULEnt{C} \tag{ih1'}
      \end{align*}
    %
    We can thus construct the following typing derivation, making use of
    of the admissibility of linear substitution
    (Lemma~\ref{lemma:substitution}):
    %
    {\scriptsize{
    \[
    \hspace{-8em}
    \inferrule*[Right=\textsc{app}]
    {\inferrule*[right=\textsc{abs}, leftskip=5em]
       {\GRANULEsym{(}  \Gamma  \GRANULEsym{-}  \Delta_{{\mathrm{1}}}  \GRANULEsym{)}  \GRANULEsym{,}   \GRANULEmv{x_{{\mathrm{2}}}}  :   \GRANULEnt{B}  \multimap  \GRANULEnt{C}    \vdash  \GRANULEnt{t_{{\mathrm{1}}}}  :  \GRANULEnt{C}}
       {\Gamma  \GRANULEsym{-}  \Delta_{{\mathrm{1}}}  \vdash   \lambda  \GRANULEmv{x_{{\mathrm{2}}}}  .  \GRANULEnt{t_{{\mathrm{1}}}}   :   \GRANULEnt{B}  \multimap  \GRANULEnt{C}} \\
    \inferrule*[right=\textsc{app}, rightskip=15em]
    {\inferrule*[right=\textsc{var}, leftskip=2em]
    {\quad}
    {\GRANULEmv{x_{{\mathrm{1}}}}  :   \GRANULEnt{A}  \multimap  \GRANULEnt{B}    \vdash  \GRANULEmv{x_{{\mathrm{1}}}}  :   \GRANULEnt{A}  \multimap  \GRANULEnt{B}} \\ {\Delta_{{\mathrm{1}}}  \GRANULEsym{-}  \Delta_{{\mathrm{2}}}  \vdash  \GRANULEnt{t_{{\mathrm{2}}}}  :  \GRANULEnt{A}}}  {\GRANULEsym{(}  \Delta_{{\mathrm{1}}}  \GRANULEsym{-}  \Delta_{{\mathrm{2}}}  \GRANULEsym{)}, \GRANULEmv{x_{{\mathrm{1}}}}  :   \GRANULEnt{A}  \multimap  \GRANULEnt{B}    \vdash  \GRANULEmv{x_{{\mathrm{1}}}} \, \GRANULEnt{t_{{\mathrm{2}}}}  :  \GRANULEnt{B}}}
    {\GRANULEsym{(}  \Gamma  \GRANULEsym{-}  \Delta_{{\mathrm{1}}}  \GRANULEsym{)}  \GRANULEsym{+}  \GRANULEsym{(}  \Delta_{{\mathrm{1}}}  \GRANULEsym{-}  \Delta_{{\mathrm{2}}}  \GRANULEsym{)} ,\GRANULEmv{x_{{\mathrm{1}}}}  :   \GRANULEnt{A}  \multimap  \GRANULEnt{B}    \vdash   [  \GRANULEsym{(}  \GRANULEmv{x_{{\mathrm{1}}}} \, \GRANULEnt{t_{{\mathrm{2}}}}  \GRANULEsym{)}  /  \GRANULEmv{x_{{\mathrm{2}}}}  ]  \GRANULEnt{t_{{\mathrm{1}}}}   :  \GRANULEnt{C}}
    \]
    }}

    From Lemma~\ref{lemma:contextLemma1}, we have that
    %
    \begin{align*}
      \GRANULEsym{(}  \GRANULEsym{(}  \Gamma  \GRANULEsym{-}  \Delta_{{\mathrm{1}}}  \GRANULEsym{)}  \GRANULEsym{+}  \GRANULEsym{(}  \Delta_{{\mathrm{1}}}  \GRANULEsym{-}  \Delta_{{\mathrm{2}}}  \GRANULEsym{)}  \GRANULEsym{)}  \GRANULEsym{,}   \GRANULEmv{x_{{\mathrm{1}}}}  :   \GRANULEnt{A}  \multimap  \GRANULEnt{B} \sqsubseteq \GRANULEsym{(}  \GRANULEsym{(}  \GRANULEsym{(}  \Gamma  \GRANULEsym{-}  \Delta_{{\mathrm{1}}}  \GRANULEsym{)}  \GRANULEsym{+}  \Delta_{{\mathrm{1}}}  \GRANULEsym{)}  \GRANULEsym{-}  \Delta_{{\mathrm{2}}}  \GRANULEsym{)}  \GRANULEsym{,}   \GRANULEmv{x_{{\mathrm{1}}}}  :   \GRANULEnt{A}  \multimap  \GRANULEnt{B}
    \end{align*}
    %
    and from Lemma~\ref{lemma:contextLemma2}, that:
    %
    \begin{align*}
      \GRANULEsym{(}  \GRANULEsym{(}  \GRANULEsym{(}  \Gamma  \GRANULEsym{-}  \Delta_{{\mathrm{1}}}  \GRANULEsym{)}  \GRANULEsym{+}  \Delta_{{\mathrm{1}}}  \GRANULEsym{)}  \GRANULEsym{-}  \Delta_{{\mathrm{2}}}  \GRANULEsym{)}  \GRANULEsym{,}   \GRANULEmv{x_{{\mathrm{1}}}}  :   \GRANULEnt{A}  \multimap  \GRANULEnt{B} \sqsubseteq \GRANULEsym{(}  \Gamma  \GRANULEsym{-}  \Delta_{{\mathrm{2}}}  \GRANULEsym{)}  \GRANULEsym{,}   \GRANULEmv{x_{{\mathrm{1}}}}  :   \GRANULEnt{A}  \multimap  \GRANULEnt{B}
    \end{align*}
    %
    %% JACK: this arguments needs strengthening, but it is true.
    %% easy to given a proof of output contexts being subsets of input
    which, since $\GRANULEmv{x_{{\mathrm{1}}}}$ is not in $\Delta_{{\mathrm{2}}}$ (as $\GRANULEmv{x_{{\mathrm{1}}}}$ is not
    in $\Gamma$)
    %%
    $\GRANULEsym{(}  \Gamma  \GRANULEsym{-}  \Delta_{{\mathrm{2}}}  \GRANULEsym{)}  \GRANULEsym{,}   \GRANULEmv{x_{{\mathrm{1}}}}  :   \GRANULEnt{A}  \multimap  \GRANULEnt{B} = \GRANULEsym{(}  \Gamma  \GRANULEsym{,}   \GRANULEmv{x_{{\mathrm{1}}}}  :   \GRANULEnt{A}  \multimap  \GRANULEnt{B}    \GRANULEsym{)}  \GRANULEsym{-}  \Delta_{{\mathrm{2}}}$. Applying
    these inequalities with \textsc{Approx} then yields the lemma's
    conclusion $\GRANULEsym{(}  \Gamma  \GRANULEsym{,}   \GRANULEmv{x_{{\mathrm{1}}}}  :   \GRANULEnt{A}  \multimap  \GRANULEnt{B}    \GRANULEsym{)}  \GRANULEsym{-}  \Delta_{{\mathrm{2}}}   \vdash   [  \GRANULEsym{(}  \GRANULEmv{x_{{\mathrm{1}}}} \, \GRANULEnt{t_{{\mathrm{2}}}}  \GRANULEsym{)}  /  \GRANULEmv{x_{{\mathrm{2}}}}  ]  \GRANULEnt{t_{{\mathrm{1}}}}   :  \GRANULEnt{C}$.

  \item Case \subBoxName{} \\
    %
    The synthesis rule for boxing can be constructed as:
    %
    \[
      \subBox{}
    \]
    %
    By induction on the premise we get:
    %
    \begin{align*}
      \Gamma  \GRANULEsym{-}  \Delta  \vdash  \GRANULEnt{t}  :  \GRANULEnt{A}
    \end{align*}
    %
    Since we apply scalar multipication ih the conclusion of the rule
    to $\Gamma  \GRANULEsym{-}  \Delta$ then we know that all of $\Gamma  \GRANULEsym{-}  \Delta$ must be
    graded assumptions.

    From this, we can construct the typing derivation:
    %
    \[
    \inferrule*[Right=\textsc{Pr}]
    {\GRANULEsym{[}   \Gamma  \GRANULEsym{-}  \Delta   \GRANULEsym{]}  \vdash  \GRANULEnt{t}  :  \GRANULEnt{A}}{\textcolor{coeffectColor}{ \GRANULEnt{r}   \textcolor{coeffectColor}{\,\cdot\,} }  \GRANULEsym{[}   \Gamma  \GRANULEsym{-}  \Delta   \GRANULEsym{]}   \vdash  \GRANULEsym{[}  \GRANULEnt{t}  \GRANULEsym{]}  :   \Box_{  \GRANULEnt{r}  }  \GRANULEnt{A}}
    \]
    Via Lemma~\ref{lemma:contexts-subsub}, we then have that
    $\GRANULEsym{(}   \textcolor{coeffectColor}{ \GRANULEnt{r}   \textcolor{coeffectColor}{\,\cdot\,} }   \Gamma  \GRANULEsym{-}  \Delta    \GRANULEsym{)} \sqsubseteq \GRANULEsym{(}  \Gamma  \GRANULEsym{-}  \GRANULEsym{(}  \Gamma  \GRANULEsym{-}  \GRANULEsym{(}   \textcolor{coeffectColor}{ \GRANULEnt{r}   \textcolor{coeffectColor}{\,\cdot\,} }  \GRANULEsym{(}  \Gamma  \GRANULEsym{-}  \Delta  \GRANULEsym{)}   \GRANULEsym{)}  \GRANULEsym{)}  \GRANULEsym{)}$ thus, we can
    derived:
    %
    \[
    \inferrule*[Right=\textsc{Approx}]
   {
    \inferrule*[Right=\textsc{Pr}]
    {\GRANULEsym{[}   \Gamma  \GRANULEsym{-}  \Delta   \GRANULEsym{]}  \vdash  \GRANULEnt{t}  :  \GRANULEnt{A}}{\textcolor{coeffectColor}{ \GRANULEnt{r}   \textcolor{coeffectColor}{\,\cdot\,} }  \GRANULEsym{[}   \Gamma  \GRANULEsym{-}  \Delta   \GRANULEsym{]}   \vdash  \GRANULEsym{[}  \GRANULEnt{t}  \GRANULEsym{]}  :   \Box_{  \GRANULEnt{r}  }  \GRANULEnt{A}
    \quad \text{Lem.~\ref{lemma:contexts-subsub}}}
   }{ \Gamma  \GRANULEsym{-}  \GRANULEsym{(}  \Gamma  \GRANULEsym{-}  \GRANULEsym{(}   \textcolor{coeffectColor}{ \GRANULEnt{r}   \textcolor{coeffectColor}{\,\cdot\,} }  \GRANULEsym{(}  \Gamma  \GRANULEsym{-}  \Delta  \GRANULEsym{)}   \GRANULEsym{)}  \GRANULEsym{)}  \vdash  \GRANULEsym{[}  \GRANULEnt{t}  \GRANULEsym{]}  :   \Box_{  \GRANULEnt{r}  }  \GRANULEnt{A}}
    \]
    %
    Satisfying the goal of the lemma.

  \item Case \subUnboxName \\
    The synthesis rule for unboxing has the form:
    \[
    \subUnbox{}
    \]
    %
    By induction on the premise we have that:
    %
      \begin{align*}
        \GRANULEsym{(}  \Gamma  \GRANULEsym{,}   \GRANULEmv{x_{{\mathrm{2}}}}  :_{\textcolor{coeffectColor}{  \GRANULEnt{r}  } }   \GRANULEnt{A}   \GRANULEsym{)}  \GRANULEsym{-}  \GRANULEsym{(}  \Delta  \GRANULEsym{,}   \GRANULEmv{x_{{\mathrm{2}}}}  :_{\textcolor{coeffectColor}{  \GRANULEnt{s}  } }   \GRANULEnt{A}   \GRANULEsym{)}  \vdash  \GRANULEnt{t}  :  \GRANULEnt{B}
      \end{align*}
   %
   By the definition of context subtraction we get that $\exists q$ and:
   %
     \begin{align*}
       \GRANULEsym{(}  \Gamma  \GRANULEsym{,}   \GRANULEmv{x_{{\mathrm{2}}}}  :_{\textcolor{coeffectColor}{  \GRANULEnt{r}  } }   \GRANULEnt{A}   \GRANULEsym{)}  \GRANULEsym{-}  \GRANULEsym{(}  \Delta  \GRANULEsym{,}   \GRANULEmv{x_{{\mathrm{2}}}}  :_{\textcolor{coeffectColor}{  \GRANULEnt{s}  } }   \GRANULEnt{A}   \GRANULEsym{)}
     = \GRANULEsym{(}  \Gamma  \GRANULEsym{-}  \Delta  \GRANULEsym{)}  \GRANULEsym{,}   \GRANULEmv{x_{{\mathrm{2}}}}  :_{\textcolor{coeffectColor}{  \GRANULEnt{q}  } }   \GRANULEnt{A}
       \end{align*}
   %
    such that $r = q + s$

    We also have that $0 \sqsubseteq \GRANULEnt{s}$.

    By monotonicity with $\GRANULEnt{q} \sqsubseteq \GRANULEnt{q}$ (reflexivity)
    and $0 \sqsubseteq \GRANULEnt{s}$ then $\GRANULEnt{q} \sqsubseteq \GRANULEnt{q}  \GRANULEsym{+}  \GRANULEnt{s}$.

    By context subtraction we have $r = q + s$ therefore
    $\GRANULEnt{q} \sqsubseteq \GRANULEnt{r}$ (*).

    From this, we can construct the typing derivation:
    %
    \[
    \inferrule*[Right=\textsc{Let}]
    {\inferrule*[right=\textsc{Var}]
                    {\quad}{\GRANULEmv{x_{{\mathrm{1}}}}  :   \Box_{  \GRANULEnt{r}  }  \GRANULEnt{A}    \vdash  \GRANULEmv{x_{{\mathrm{1}}}}  :   \Box_{  \GRANULEnt{r}  }  \GRANULEnt{A}}
           \\
      \inferrule*[right=\textsc{approx}]
         {\GRANULEsym{(}  \Gamma  \GRANULEsym{-}  \Delta  \GRANULEsym{)}  \GRANULEsym{,}   \GRANULEmv{x_{{\mathrm{2}}}}  :_{\textcolor{coeffectColor}{  \GRANULEnt{q}  } }   \GRANULEnt{A}   \vdash  \GRANULEnt{t}  :  \GRANULEnt{B} \quad (*)}
         {\GRANULEsym{(}  \Gamma  \GRANULEsym{-}  \Delta  \GRANULEsym{)}  \GRANULEsym{,}   \GRANULEmv{x_{{\mathrm{2}}}}  :_{\textcolor{coeffectColor}{  \GRANULEnt{r}  } }   \GRANULEnt{A}   \vdash  \GRANULEnt{t}  :  \GRANULEnt{B}}}
     {\GRANULEsym{(}  \Gamma  \GRANULEsym{-}  \Delta  \GRANULEsym{)}  \GRANULEsym{,}   \GRANULEmv{x_{{\mathrm{1}}}}  :   \Box_{  \GRANULEnt{r}  }  \GRANULEnt{A}    \vdash   \textbf{let} \, [  \GRANULEmv{x_{{\mathrm{2}}}}  ] =  \GRANULEmv{x_{{\mathrm{1}}}}  \, \textbf{in} \,  \GRANULEnt{t}   :  \GRANULEnt{B}}
    \]
    %
    Which matches the goal.

  \item Case \subPairIntroName \\

    The synthesis rule for pair introduction has the form:
    %
    \[
      \subPairIntro{}
    \]
    %
    By induction we get:
    %
    \begin{align*}
      \Gamma  \GRANULEsym{-}  \Delta_{{\mathrm{1}}}  \vdash  \GRANULEnt{t_{{\mathrm{1}}}}  :  \GRANULEnt{A} \tag{ih1} \\
      \Delta_{{\mathrm{1}}}  \GRANULEsym{-}  \Delta_{{\mathrm{2}}}  \vdash  \GRANULEnt{t_{{\mathrm{2}}}}  :  \GRANULEnt{B} \tag{ih2}
    \end{align*}
    %
    From this, we can construct the typing derivation:
    %
    \[
    \inferrule*[Right=\textsc{Pair}]
    {\Gamma  \GRANULEsym{-}  \Delta_{{\mathrm{1}}}  \vdash  \GRANULEnt{t_{{\mathrm{1}}}}  :  \GRANULEnt{A} \\ \Delta_{{\mathrm{1}}}  \GRANULEsym{-}  \Delta_{{\mathrm{2}}}  \vdash  \GRANULEnt{t_{{\mathrm{2}}}}  :  \GRANULEnt{B}}
    {\GRANULEsym{(}  \Gamma  \GRANULEsym{-}  \Delta_{{\mathrm{1}}}  \GRANULEsym{)}  \GRANULEsym{+}  \GRANULEsym{(}  \Delta_{{\mathrm{1}}}  \GRANULEsym{-}  \Delta_{{\mathrm{2}}}  \GRANULEsym{)}  \vdash   ( \GRANULEnt{t_{{\mathrm{1}}}} ,  \GRANULEnt{t_{{\mathrm{2}}}} )   :   \GRANULEnt{A}  \, \otimes \,  \GRANULEnt{B}}
    \]
    %
    From Lemma~\ref{lemma:contextLemma1}, we have that:
    \begin{align*}
      \GRANULEsym{(}  \Gamma  \GRANULEsym{-}  \Delta_{{\mathrm{1}}}  \GRANULEsym{)}  \GRANULEsym{+}  \GRANULEsym{(}  \Delta_{{\mathrm{1}}}  \GRANULEsym{-}  \Delta_{{\mathrm{2}}}  \GRANULEsym{)} \sqsubseteq \GRANULEsym{(}  \GRANULEsym{(}  \Gamma  \GRANULEsym{-}  \Delta_{{\mathrm{1}}}  \GRANULEsym{)}  \GRANULEsym{+}  \Delta_{{\mathrm{1}}}  \GRANULEsym{)}  \GRANULEsym{-}  \Delta_{{\mathrm{2}}}
    \end{align*}
    %
    and from Lemma~\ref{lemma:contextLemma2}, that:
    %
    \begin{align*}
      \GRANULEsym{(}  \GRANULEsym{(}  \Gamma  \GRANULEsym{-}  \Delta_{{\mathrm{1}}}  \GRANULEsym{)}  \GRANULEsym{+}  \Delta_{{\mathrm{1}}}  \GRANULEsym{)}  \GRANULEsym{-}  \Delta_{{\mathrm{2}}} \sqsubseteq \Gamma  \GRANULEsym{-}  \Delta_{{\mathrm{2}}}
    \end{align*}
    %
    From which we then apply \textsc{Approx} to the
    above derivation,
    yielding the goal $\Gamma  \GRANULEsym{-}  \Delta_{{\mathrm{2}}}  \vdash   ( \GRANULEnt{t_{{\mathrm{1}}}} ,  \GRANULEnt{t_{{\mathrm{2}}}} )   :   \GRANULEnt{A}  \, \otimes \,  \GRANULEnt{B}$.

  \item Case \subPairElimName \\
    The synthesis rule for pair elimination has the form:
    \[
      \subPairElim
    \]
    %
    By induction we get:
    %
    \begin{align*}
      \GRANULEsym{(}  \Gamma  \GRANULEsym{,}   \GRANULEmv{x_{{\mathrm{1}}}}  :  \GRANULEnt{A}   \GRANULEsym{,}   \GRANULEmv{x_{{\mathrm{2}}}}  :  \GRANULEnt{B}   \GRANULEsym{)}  \GRANULEsym{-}  \Delta  \vdash  \GRANULEnt{t_{{\mathrm{2}}}}  :  \GRANULEnt{C}
    \end{align*}
    %
    since $\GRANULEmv{x_{{\mathrm{1}}}}  \not\in | \Delta | \wedge \GRANULEmv{x_{{\mathrm{2}}}}  \not\in | \Delta |$ then
    $\GRANULEsym{(}  \Gamma  \GRANULEsym{,}   \GRANULEmv{x_{{\mathrm{1}}}}  :  \GRANULEnt{A}   \GRANULEsym{,}   \GRANULEmv{x_{{\mathrm{2}}}}  :  \GRANULEnt{B}   \GRANULEsym{)}  \GRANULEsym{-}  \Delta = \GRANULEsym{(}  \Gamma  \GRANULEsym{-}  \Delta  \GRANULEsym{)}  \GRANULEsym{,}   \GRANULEmv{x_{{\mathrm{1}}}}  :  \GRANULEnt{A}   \GRANULEsym{,}   \GRANULEmv{x_{{\mathrm{2}}}}  :  \GRANULEnt{B}$.

    From this, we can construct the following typing derivation,
    matching the conclusion:
    \[
    \inferrule*[Right=\textsc{Case}]
    {\inferrule*[right=\textsc{Var}] {\quad} {\GRANULEmv{x_{{\mathrm{3}}}}  :   \GRANULEnt{A}  \, \otimes \,  \GRANULEnt{B}    \vdash  \GRANULEmv{x_{{\mathrm{3}}}}  :   \GRANULEnt{A}  \, \otimes \,  \GRANULEnt{B}} \\ \GRANULEsym{(}  \Gamma  \GRANULEsym{-}  \Delta  \GRANULEsym{)}  \GRANULEsym{,}   \GRANULEmv{x_{{\mathrm{1}}}}  :  \GRANULEnt{A}   \GRANULEsym{,}   \GRANULEmv{x_{{\mathrm{2}}}}  :  \GRANULEnt{B}   \vdash  \GRANULEnt{t_{{\mathrm{2}}}}  :  \GRANULEnt{C}}
    {\GRANULEsym{(}  \Gamma  \GRANULEsym{-}  \Delta  \GRANULEsym{)}  \GRANULEsym{,}   \GRANULEmv{x_{{\mathrm{3}}}}  :   \GRANULEnt{A}  \, \otimes \,  \GRANULEnt{B}    \vdash   \textbf{let} \, ( \GRANULEmv{x_{{\mathrm{1}}}} ,  \GRANULEmv{x_{{\mathrm{2}}}} ) =  \GRANULEmv{x_{{\mathrm{3}}}}  \, \textbf{in} \,  \GRANULEnt{t_{{\mathrm{2}}}}   :  \GRANULEnt{C}}
    \]
    % JACK: tighten from context subset
    which matches the conclusion since $\GRANULEsym{(}  \Gamma  \GRANULEsym{-}  \Delta  \GRANULEsym{)}  \GRANULEsym{,}   \GRANULEmv{x_{{\mathrm{3}}}}  :   \GRANULEnt{A}  \, \otimes \,  \GRANULEnt{B} = \GRANULEsym{(}  \Gamma  \GRANULEsym{,}   \GRANULEmv{x_{{\mathrm{3}}}}  :   \GRANULEnt{A}  \, \otimes \,  \GRANULEnt{B}    \GRANULEsym{)}  \GRANULEsym{-}  \Delta$ since $\GRANULEmv{x_{{\mathrm{3}}}}  \not\in | \Delta |$ by its
    disjointness from $\Gamma$.

  \item Case \subSumIntroLname and \subSumIntroRname \\
    The synthesis rules for sum introduction are straightforward. For
     \subSumIntroLname we have the rule:
    \[
       \subSumIntroL
    \]
    By induction we have:
    %
      \begin{align*}
        \Gamma  \GRANULEsym{-}  \Delta  \vdash  \GRANULEnt{t}  :  \GRANULEnt{A} \tag{ih1}
       \end{align*}
    %
    from which we can construct the typing derivation, matching the
    conclusion:
    \[
    \inferrule*[Right=\subSumIntroLname]
    {\Gamma  \GRANULEsym{-}  \Delta  \vdash  \GRANULEnt{t}  :  \GRANULEnt{A}}
    {\Gamma  \GRANULEsym{-}  \Delta  \vdash  \GRANULEkw{inl} \, \GRANULEnt{t}  :   \GRANULEnt{A}  \, \oplus \,  \GRANULEnt{B}}
    \]
    Matching the goal. And likewise for \subSumIntroRname.

  \item Case \subSumElimName
      The synthesis rule for sum elimination has the form:
      \[
        \subSumElim
      \]
      %
      By induction:
        \begin{align*}
          \GRANULEsym{(}  \Gamma  \GRANULEsym{,}   \GRANULEmv{x_{{\mathrm{2}}}}  :  \GRANULEnt{A}   \GRANULEsym{)}  \GRANULEsym{-}  \Delta_{{\mathrm{1}}}  \vdash  \GRANULEnt{t_{{\mathrm{1}}}}  :  \GRANULEnt{C} \tag{ih}
\\        \GRANULEsym{(}  \Gamma  \GRANULEsym{,}   \GRANULEmv{x_{{\mathrm{3}}}}  :  \GRANULEnt{B}   \GRANULEsym{)}  \GRANULEsym{-}  \Delta_{{\mathrm{2}}}  \vdash  \GRANULEnt{t_{{\mathrm{2}}}}  :  \GRANULEnt{C} \tag{ih}
        \end{align*}
      %
      From this we can construct the typing derivation, matching the conclusion:
      \[
      \inferrule*[Right=Case]
      {\inferrule*[right=Var,leftskip=10em]{\quad}{\GRANULEmv{x_{{\mathrm{1}}}}  :   \GRANULEnt{A}  \, \oplus \,  \GRANULEnt{B}    \vdash  \GRANULEnt{t_{{\mathrm{1}}}}  :   \GRANULEnt{A}  \, \oplus \,  \GRANULEnt{B}} \\ \GRANULEsym{(}  \Gamma  \GRANULEsym{-}  \Delta_{{\mathrm{1}}}  \GRANULEsym{)}  \GRANULEsym{,}   \GRANULEmv{x_{{\mathrm{2}}}}  :  \GRANULEnt{A}   \vdash  \GRANULEnt{t_{{\mathrm{2}}}}  :  \GRANULEnt{C} \\ \GRANULEsym{(}  \Gamma  \GRANULEsym{-}  \Delta_{{\mathrm{2}}}  \GRANULEsym{)}  \GRANULEsym{,}   \GRANULEmv{x_{{\mathrm{3}}}}  :  \GRANULEnt{B}   \vdash  \GRANULEnt{t_{{\mathrm{3}}}}  :  \GRANULEnt{C}}{\GRANULEsym{(}  \Gamma  \GRANULEsym{,}   \GRANULEmv{x_{{\mathrm{1}}}}  :   \GRANULEnt{A}  \, \oplus \,  \GRANULEnt{B}    \GRANULEsym{)} - (\Delta_{{\mathrm{1}}} \sqcap \Delta_{{\mathrm{2}}}) \vdash  \textbf{case} \ x_{1}\ \textbf{of}\ \textbf{inl}\ x_{2} \rightarrow t_{1};\ \textbf{inr}\ x_{3} \rightarrow t_{2} : C}
      \]

      \item Case \subUnitIntroName

          \begin{align*}
            \subUnitIntro{}
           \end{align*}
         %
         By Lemma~\ref{lemma:context-neg} we have that $\emptyset \sqsubseteq \Gamma  \GRANULEsym{-}  \Gamma$
         then we have:
         %
           \begin{align*}
             \inferrule*[right = \textsc{Approx}]
             {\inferrule*[right = 1]
             {\quad}{\emptyset   \vdash  \GRANULEsym{()}  :   \mathsf{1}}}
             {\Gamma  \GRANULEsym{-}  \Gamma  \vdash  \GRANULEsym{()}  :   \mathsf{1}}
           \end{align*}
         %
         Matching the goal

     \item Case \subUnitElimName
          \begin{align*}
            \subUnitElim{}
           \end{align*}
         %
         By induction we have:
           \begin{align*}
             \Gamma  \GRANULEsym{-}  \Delta  \vdash  \GRANULEnt{t}  :  \GRANULEnt{C} \tag{ih}
            \end{align*}
         %
         Then we make the derivation:
         %
           \begin{align*}
             \inferrule*[right = Let$1$]
             {\inferrule*[right = Var]{\quad}{\GRANULEmv{x}  :   \mathsf{1}    \vdash  \GRANULEmv{x}  :   \mathsf{1}}
             \\ \Gamma  \GRANULEsym{-}  \Delta  \vdash  \GRANULEnt{t}  :  \GRANULEnt{C}}
             {\GRANULEsym{(}  \Gamma  \GRANULEsym{-}  \Delta  \GRANULEsym{)}  \GRANULEsym{,}   \GRANULEmv{x}  :   \mathsf{1}    \vdash  \GRANULEkw{let} \, \GRANULEsym{()}  \GRANULEsym{=}  \GRANULEmv{x} \, \GRANULEkw{in} \, \GRANULEnt{t}  :  \GRANULEnt{C}}
           \end{align*}
         where the context is equal to $\GRANULEsym{(}  \Gamma  \GRANULEsym{,}   \GRANULEmv{x}  :   \mathsf{1}    \GRANULEsym{)}  \GRANULEsym{-}  \Delta$.

  \item Case \subDerName

      \begin{align*}
        \subDer
      \end{align*}
     %
     By induction:
     %
       \begin{align*}
         \GRANULEsym{(}  \Gamma  \GRANULEsym{,}   \GRANULEmv{x}  :_{\textcolor{coeffectColor}{  \GRANULEnt{s}  } }   \GRANULEnt{A}   \GRANULEsym{,}   \GRANULEmv{y}  :  \GRANULEnt{A}   \GRANULEsym{)}  \GRANULEsym{-}  \GRANULEsym{(}  \Delta  \GRANULEsym{,}   \GRANULEmv{x}  :_{\textcolor{coeffectColor}{  \GRANULEnt{s'}  } }   \GRANULEnt{A}   \GRANULEsym{)}  \vdash  \GRANULEnt{t}  :  \GRANULEnt{B} \tag{ih}
       \end{align*}
     %
     By the definition of context subtraction we have (since also $\GRANULEmv{y}  \not\in | \Delta |$)
       \begin{align*}
     &  \GRANULEsym{(}  \Gamma  \GRANULEsym{,}   \GRANULEmv{x}  :_{\textcolor{coeffectColor}{  \GRANULEnt{s}  } }   \GRANULEnt{A}   \GRANULEsym{,}   \GRANULEmv{y}  :  \GRANULEnt{A}   \GRANULEsym{)}  \GRANULEsym{-}  \GRANULEsym{(}  \Delta  \GRANULEsym{,}   \GRANULEmv{x}  :_{\textcolor{coeffectColor}{  \GRANULEnt{s'}  } }   \GRANULEnt{A}   \GRANULEsym{)} \\
   =\ &  \GRANULEsym{(}  \Gamma  \GRANULEsym{-}  \Delta  \GRANULEsym{)}  \GRANULEsym{,}   \GRANULEmv{x}  :_{\textcolor{coeffectColor}{  \GRANULEnt{q}  } }   \GRANULEnt{A}   \GRANULEsym{,}   \GRANULEmv{y}  :  \GRANULEnt{A}
       \end{align*}
     where $\exists  \GRANULEnt{q}  .\,   \GRANULEnt{s}  \sqsupseteq  \GRANULEnt{q}  \GRANULEsym{+}  \GRANULEnt{s'}$ (1) and
     $\maximal{q}{\hat{q}}{s}{\hat{q} + s'}$ (2)

       The goal context is computed by:
       %
       \begin{align*}
      & \GRANULEsym{(}  \Gamma  \GRANULEsym{,}   \GRANULEmv{x}  :_{\textcolor{coeffectColor}{  \GRANULEnt{r}  } }   \GRANULEnt{A}   \GRANULEsym{)}  \GRANULEsym{-}  \GRANULEsym{(}  \Delta  \GRANULEsym{,}   \GRANULEmv{x}  :_{\textcolor{coeffectColor}{  \GRANULEnt{s'}  } }   \GRANULEnt{A}   \GRANULEsym{)} \\
    =\ & \GRANULEsym{(}  \Gamma  \GRANULEsym{-}  \Delta  \GRANULEsym{)}  \GRANULEsym{,}   \GRANULEmv{x}  :_{\textcolor{coeffectColor}{  \GRANULEnt{q'}  } }   \GRANULEnt{A}
       \end{align*}
       where $\GRANULEnt{r}  \sqsupseteq   \GRANULEnt{q'}  \GRANULEsym{+}  \GRANULEnt{s'}$ (3)
       and $\maximal{q'}{\hat{q'}}{r}{\hat{q'} + s'}$ (4)

       From the premise of \subDerName we have
       $\GRANULEnt{r}  \sqsupseteq  \GRANULEsym{(}  \GRANULEnt{s}  \GRANULEsym{+}   1   \GRANULEsym{)}$.

       \begin{align*}
      \begin{array}{rll}
        \text{congruence of + and (1)} & \implies \GRANULEnt{s}  \GRANULEsym{+}   1    \sqsupseteq    \GRANULEnt{q}  \GRANULEsym{+}  \GRANULEnt{s'}   \GRANULEsym{+}   1 & (5) \\
        \text{transitivity with \subDerName premise and (5)} & \implies
                 \GRANULEnt{r}  \sqsupseteq    \GRANULEnt{q}  \GRANULEsym{+}  \GRANULEnt{s'}   \GRANULEsym{+}   1 & (6) \\
        \text{+ assoc./comm. on (6)} & \implies \GRANULEnt{r}  \sqsupseteq    \GRANULEnt{q}  \GRANULEsym{+}   1    \GRANULEsym{+}  \GRANULEnt{s'} & (7) \\
       \text{apply (8) to (4) with $\hat{q'} = q + 1$} & \implies
                                                    \GRANULEnt{q'}  \sqsupseteq  \GRANULEnt{q}  \GRANULEsym{+}   1
                                                     & (8)
      \end{array}
       \end{align*}
       %
       Using this last result we derive:
       \begin{align*}
         \inferrule*[right = approx]
        {
         \inferrule*[right = contraction]
         {
         \inferrule*[right = Der]
         {\GRANULEsym{(}  \Gamma  \GRANULEsym{-}  \Delta  \GRANULEsym{)}  \GRANULEsym{,}   \GRANULEmv{x}  :_{\textcolor{coeffectColor}{  \GRANULEnt{q}  } }   \GRANULEnt{A}   \GRANULEsym{,}   \GRANULEmv{y}  :  \GRANULEnt{A}   \vdash  \GRANULEnt{t}  :  \GRANULEnt{B}}
         {\GRANULEsym{(}  \Gamma  \GRANULEsym{-}  \Delta  \GRANULEsym{)}  \GRANULEsym{,}   \GRANULEmv{x}  :_{\textcolor{coeffectColor}{  \GRANULEnt{q}  } }   \GRANULEnt{A}   \GRANULEsym{,}   \GRANULEmv{y}  :_{\textcolor{coeffectColor}{   1   } }   \GRANULEnt{A}   \vdash  \GRANULEnt{t}  :  \GRANULEnt{B}}
         }
         {\GRANULEsym{(}  \Gamma  \GRANULEsym{-}  \Delta  \GRANULEsym{)}  \GRANULEsym{,}   \GRANULEmv{x}  :_{\textcolor{coeffectColor}{   \GRANULEnt{q}  \GRANULEsym{+}   1    } }   \GRANULEnt{A}   \vdash   [  \GRANULEmv{x}  /  \GRANULEmv{y}  ]  \GRANULEnt{t}   :  \GRANULEnt{B}}
         \quad (8)}
        {\GRANULEsym{(}  \Gamma  \GRANULEsym{-}  \Delta  \GRANULEsym{)}  \GRANULEsym{,}   \GRANULEmv{x}  :_{\textcolor{coeffectColor}{  \GRANULEnt{q'}  } }   \GRANULEnt{A}   \vdash   [  \GRANULEmv{x}  /  \GRANULEmv{y}  ]  \GRANULEnt{t}   :  \GRANULEnt{B}}
         \end{align*}
       Which matches the goal.

\end{enumerate}
\end{proof}

\addSynthSound*
\begin{proof}

  \begin{enumerate}
    \item Case \addLinVarName \\
  In the case of linear variable synthesis, we have the derivation:
  \[
    \addLinVar
  \]
  Therefore we can construct the following typing derivation, matching the conclusion:
  \[
    \inferrule*[Right=\textsc{var}]
    {\quad}
    {\GRANULEmv{x}  :  \GRANULEnt{A}   \vdash  \GRANULEmv{x}  :  \GRANULEnt{A}}
  \]
    \item Case \addGrVarName\\
    Matching the form of the lemma, we have the derivation:
    \[
      \addGrVar
    \]
    From this we can construct the typing derivation, matching the conclusion:
    \[
      \inferrule*[Right=\textsc{Der}]
        {\inferrule*[Right=\textsc{Var}]
          {\quad}
          {\GRANULEmv{x}  :  \GRANULEnt{A}   \vdash  \GRANULEmv{x}  :  \GRANULEnt{A}}
        }
        {\GRANULEmv{x}  :_{\textcolor{coeffectColor}{   1   } }   \GRANULEnt{A}   \vdash  \GRANULEmv{x}  :  \GRANULEnt{A}}
    \]
    \item Case \addAbsName\\
    We thus have the derivation:
    \[
      \addAbs
    \]
    %
    By induction on the premise we then have:
    \[
      \Delta  \GRANULEsym{,}   \GRANULEmv{x}  :  \GRANULEnt{A}   \vdash  \GRANULEnt{t}  :  \GRANULEnt{B}
    \]
    From this, we can construct the typing derivation, matching the conclusion:
    \[
    \inferrule*[Right=\textsc{abs}]
    {\Delta  \GRANULEsym{,}   \GRANULEmv{x}  :  \GRANULEnt{A}   \vdash  \GRANULEnt{t}  :  \GRANULEnt{B}}{\Delta  \vdash   \lambda  \GRANULEmv{x}  .  \GRANULEnt{t}   :   \GRANULEnt{A}  \multimap  \GRANULEnt{B}}
    \]
    \item Case \addAppName\\
    Matching the form of the lemma, the application derivation can be
    constructed as:
    \[
      \addApp
    \]
    %
    By induction on the premises we then have the following typing
    judgments:
    %
    \begin{align*}
      \Delta_{{\mathrm{1}}}  \GRANULEsym{,}   \GRANULEmv{x_{{\mathrm{2}}}}  :  \GRANULEnt{B}   \vdash  \GRANULEnt{t_{{\mathrm{1}}}}  :  \GRANULEnt{C} \\
      \Delta_{{\mathrm{2}}}  \vdash  \GRANULEnt{t_{{\mathrm{2}}}}  :  \GRANULEnt{A}
    \end{align*}
    %
    We can thus construct the following typing derivation, making use
    of the admissibility of linear substitution
    (Lemma~\ref{lemma:substitution}):
    \[
    \inferrule*[Right=(L.~\ref{lemma:substitution})]
    {\inferrule*[right=\textsc{app}, leftskip=2em]
    {\inferrule*[right=\textsc{var}]
    {\quad}
    {\GRANULEmv{x_{{\mathrm{1}}}}  :   \GRANULEnt{A}  \multimap  \GRANULEnt{B}    \vdash  \GRANULEmv{x_{{\mathrm{1}}}}  :   \GRANULEnt{A}  \multimap  \GRANULEnt{B}} \\ {\Delta_{{\mathrm{2}}}  \vdash  \GRANULEnt{t_{{\mathrm{2}}}}  :  \GRANULEnt{A}}}
  {\Delta_{{\mathrm{2}}}  \GRANULEsym{,}   \GRANULEmv{x_{{\mathrm{1}}}}  :   \GRANULEnt{A}  \multimap  \GRANULEnt{B}    \vdash  \GRANULEmv{x_{{\mathrm{1}}}} \, \GRANULEnt{t_{{\mathrm{2}}}}  :  \GRANULEnt{B}}
    \\ \Delta_{{\mathrm{1}}}  \GRANULEsym{,}   \GRANULEmv{x_{{\mathrm{2}}}}  :  \GRANULEnt{B}   \vdash  \GRANULEnt{t_{{\mathrm{1}}}}  :  \GRANULEnt{C}}
    {\GRANULEsym{(}  \Delta_{{\mathrm{1}}}  \GRANULEsym{+}  \Delta_{{\mathrm{2}}}  \GRANULEsym{)}  \GRANULEsym{,}   \GRANULEmv{x_{{\mathrm{1}}}}  :   \GRANULEnt{A}  \multimap  \GRANULEnt{B}    \vdash   [  \GRANULEsym{(}  \GRANULEmv{x_{{\mathrm{1}}}} \, \GRANULEnt{t_{{\mathrm{2}}}}  \GRANULEsym{)}  /  \GRANULEmv{x_{{\mathrm{2}}}}  ]  \GRANULEnt{t_{{\mathrm{1}}}}   :  \GRANULEnt{C}}
    \]
    \item Case \addBoxName\\
    The synthesis rule for boxing can be constructed as:
    \[
      \addBox
    \]
    By induction we then have:
    %
    \[
      \Delta  \vdash  \GRANULEnt{t}  :  \GRANULEnt{A}
    \]
    %
    In the conclusion of the above derivation we know that $\textcolor{coeffectColor}{ \GRANULEnt{r}   \textcolor{coeffectColor}{\,\cdot\,} }  \Delta$ is defined, therefore it must be that all of $\Delta$ are
    graded assumptions, i.e., we have that $\GRANULEsym{[}  \Delta  \GRANULEsym{]}$ holds.
    We can thus construct the following typing derivation, matching the conclusion:
    \[
    \inferrule*[Right=\textsc{Pr}]
    {\GRANULEsym{[}  \Delta  \GRANULEsym{]}  \vdash  \GRANULEnt{t}  :  \GRANULEnt{A}}{\textcolor{coeffectColor}{ \GRANULEnt{r}   \textcolor{coeffectColor}{\,\cdot\,} }  \GRANULEsym{[}  \Delta  \GRANULEsym{]}   \vdash  \GRANULEsym{[}  \GRANULEnt{t}  \GRANULEsym{]}  :   \Box_{  \GRANULEnt{r}  }  \GRANULEnt{A}}
    \]
    \item Case \addDerName\\
    From the dereliction rule we have:
    \[
      \addDer
    \]
    %
    By induction we get:
    %
      \begin{align*}
        \Delta  \GRANULEsym{,}   \GRANULEmv{y}  :  \GRANULEnt{A}   \vdash  \GRANULEnt{t}  :  \GRANULEnt{B} \tag{ih}
      \end{align*}
    %
    Case on $x \in \Delta$
    \begin{itemize}
      \item $x \in \Delta$, i.e., $\Delta = \Delta'  \GRANULEsym{,}   \GRANULEmv{x}  :_{\textcolor{coeffectColor}{  \GRANULEnt{s'}  } }   \GRANULEnt{A}$.

        Then by admissibility of contraction we can derive:
        %
        \begin{align*}
          \inferrule*{
            \inferrule*[Right=\textsc{Der}]
             {\Delta'  \GRANULEsym{,}   \GRANULEmv{x}  :_{\textcolor{coeffectColor}{  \GRANULEnt{s'}  } }   \GRANULEnt{A}   \GRANULEsym{,}   \GRANULEmv{y}  :  \GRANULEnt{A}   \vdash  \GRANULEnt{t}  :  \GRANULEnt{B}}{\Delta'  \GRANULEsym{,}   \GRANULEmv{x}  :_{\textcolor{coeffectColor}{  \GRANULEnt{s'}  } }   \GRANULEnt{A}   \GRANULEsym{,}   \GRANULEmv{y}  :_{\textcolor{coeffectColor}{   1   } }   \GRANULEnt{A}   \vdash  \GRANULEnt{t}  :  \GRANULEnt{B}}
            }
            { \GRANULEsym{(}  \Delta'  \GRANULEsym{,}   \GRANULEmv{x}  :_{\textcolor{coeffectColor}{  \GRANULEnt{s'}  } }   \GRANULEnt{A}   \GRANULEsym{)}  \GRANULEsym{+}   \GRANULEmv{x}  :_{\textcolor{coeffectColor}{   1   } }   \GRANULEnt{A}   \vdash   [  \GRANULEmv{x}  /  \GRANULEmv{y}  ]  \GRANULEnt{t}   :  \GRANULEnt{B} }
        \end{align*}
        %
        Satisfying the lemma statment.

     \item $x \not\in \Delta$. Then
      again from the admissiblity of contraction, we derive the
      typing:
      %
        \begin{align*}
          \inferrule*{
            \inferrule*[Right=\textsc{Der}]
             {\Delta  \GRANULEsym{,}   \GRANULEmv{y}  :  \GRANULEnt{A}   \vdash  \GRANULEnt{t}  :  \GRANULEnt{B}}{\Delta  \GRANULEsym{,}   \GRANULEmv{y}  :_{\textcolor{coeffectColor}{   1   } }   \GRANULEnt{A}   \vdash  \GRANULEnt{t}  :  \GRANULEnt{B}}
            }
            { \Delta  \GRANULEsym{+}   \GRANULEmv{x}  :_{\textcolor{coeffectColor}{   1   } }   \GRANULEnt{A}   \vdash   [  \GRANULEmv{x}  /  \GRANULEmv{y}  ]  \GRANULEnt{t}   :  \GRANULEnt{B} }
        \end{align*}
        %
        which is well defined as $x \not\in \Delta$ and gives the
        lemma conclusion.
        \end{itemize}

    \item Case \addUnboxName\\
    The synthesis rule for unboxing has the form:
    \[
      \addUnbox
    \]
    %
    By induction we have that:
    %
    \[
      \Delta  \vdash  \GRANULEnt{t}  :  \GRANULEnt{B} \tag{ih}
    \]
    %
    Case on $\GRANULEmv{x_{{\mathrm{2}}}}  :_{\textcolor{coeffectColor}{  \GRANULEnt{s}  } }   \GRANULEnt{A} \in \Delta$
    \begin{itemize}
        \item $\GRANULEmv{x_{{\mathrm{2}}}}  :_{\textcolor{coeffectColor}{  \GRANULEnt{s}  } }   \GRANULEnt{A} \in \Delta$, i.e., $\GRANULEnt{s} \sqsubseteq \GRANULEnt{r}$. \\
        From this, we can construct the typing derivation, matching the conclusion:
          \[
            \inferrule*[Right=\textsc{let}$\square$]
            {\inferrule*[right=\textsc{var}]{\quad}{\GRANULEmv{x_{{\mathrm{1}}}}  :   \Box_{  \GRANULEnt{r}  }  \GRANULEnt{A}    \vdash  \GRANULEmv{x_{{\mathrm{1}}}}  :   \Box_{  \GRANULEnt{r}  }  \GRANULEnt{A}} \\ \Delta  \GRANULEsym{,}   \GRANULEmv{x_{{\mathrm{2}}}}  :_{\textcolor{coeffectColor}{  \GRANULEnt{r}  } }   \GRANULEnt{A}   \vdash  \GRANULEnt{t}  :  \GRANULEnt{B}}{\Delta  \GRANULEsym{,}   \GRANULEmv{x_{{\mathrm{1}}}}  :   \Box_{  \GRANULEnt{r}  }  \GRANULEnt{A}    \vdash   \textbf{let} \, [  \GRANULEmv{x_{{\mathrm{2}}}}  ] =  \GRANULEmv{x_{{\mathrm{1}}}}  \, \textbf{in} \,  \GRANULEnt{t}   :  \GRANULEnt{B}}
          \]
        \item $\GRANULEmv{x_{{\mathrm{2}}}}  :_{\textcolor{coeffectColor}{  \GRANULEnt{s}  } }   \GRANULEnt{A} \notin \Delta$, i.e., $0 \sqsubseteq \GRANULEnt{r}$. \\
        From this, we can construct the typing derivation, matching the conclusion:
          \[
            \inferrule*[right=\textsc{let}$\square$]
            {\inferrule*[right=\textsc{var}]{\quad}{\GRANULEmv{x_{{\mathrm{1}}}}  :   \Box_{  \GRANULEnt{r}  }  \GRANULEnt{A}    \vdash  \GRANULEmv{x_{{\mathrm{1}}}}  :   \Box_{  \GRANULEnt{r}  }  \GRANULEnt{A}} \\ \inferrule*[right=\textsc{Approx}, rightskip=5em]{\inferrule*[right=Weak]{\Delta  \vdash  \GRANULEnt{t}  :  \GRANULEnt{B}}{\Delta  \GRANULEsym{,}   \GRANULEmv{x_{{\mathrm{2}}}}  :_{\textcolor{coeffectColor}{   0   } }   \GRANULEnt{A}   \vdash  \GRANULEnt{t}  :  \GRANULEnt{B}} \\ 0 \sqsubseteq \GRANULEnt{r}}{\Delta  \GRANULEsym{,}   \GRANULEmv{x_{{\mathrm{2}}}}  :_{\textcolor{coeffectColor}{  \GRANULEnt{r}  } }   \GRANULEnt{A}   \vdash  \GRANULEnt{t}  :  \GRANULEnt{B}}}{\Delta  \GRANULEsym{,}   \GRANULEmv{x_{{\mathrm{1}}}}  :   \Box_{  \GRANULEnt{r}  }  \GRANULEnt{A}    \vdash   \textbf{let} \, [  \GRANULEmv{x_{{\mathrm{2}}}}  ] =  \GRANULEmv{x_{{\mathrm{1}}}}  \, \textbf{in} \,  \GRANULEnt{t}   :  \GRANULEnt{B}}
          \]
    \end{itemize}
  \item Case \addPairIntroName\\

    The synthesis rule for pair introduction has the form:

    \[
      \addPairIntro
    \]
    %
    By induction on the premises we have that:
    %
    \begin{align*}
      \Delta_{{\mathrm{1}}}  \vdash  \GRANULEnt{t_{{\mathrm{1}}}}  :  \GRANULEnt{A} \tag{ih1}\\
      \Delta_{{\mathrm{2}}}  \vdash  \GRANULEnt{t_{{\mathrm{2}}}}  :  \GRANULEnt{B} \tag{ih2}
    \end{align*}
    %
    From this, we can construct the typing derivation, matching the conclusion:
    %
    \[
    \inferrule*[Right=\textsc{pair}]
    {\Delta_{{\mathrm{1}}}  \vdash  \GRANULEnt{t_{{\mathrm{1}}}}  :  \GRANULEnt{A} \\ \Delta_{{\mathrm{2}}}  \vdash  \GRANULEnt{t_{{\mathrm{2}}}}  :  \GRANULEnt{B}}
    {\Delta_{{\mathrm{1}}}  \GRANULEsym{+}  \Delta_{{\mathrm{2}}}  \vdash   ( \GRANULEnt{t_{{\mathrm{1}}}} ,  \GRANULEnt{t_{{\mathrm{2}}}} )   :   \GRANULEnt{A}  \, \otimes \,  \GRANULEnt{B}}
    \]

  \item Case \addPairElimName\\
    The synthesis rule for pair elimination has the form:
    \[
      \addPairElim
    \]
      By induction on the premises we have that:
    \begin{align*}
      \Delta_{{\mathrm{1}}}  \vdash  \GRANULEnt{t_{{\mathrm{1}}}}  :  \GRANULEnt{A} \tag{ih1} \\
      \Delta_{{\mathrm{2}}}  \vdash  \GRANULEnt{t_{{\mathrm{2}}}}  :  \GRANULEnt{B} \tag{ih2}
    \end{align*}
    From this, we can construct the typing derivation, matching the conclusion:
    \[
    \inferrule*[Right=\textsc{LetPair}]
    {\inferrule*[right=\textsc{Var}] {\quad} {\GRANULEmv{x_{{\mathrm{3}}}}  :   \GRANULEnt{A}  \, \otimes \,  \GRANULEnt{B}    \vdash  \GRANULEmv{x_{{\mathrm{3}}}}  :   \GRANULEnt{A}  \, \otimes \,  \GRANULEnt{B}} \\ \Delta  \GRANULEsym{,}   \GRANULEmv{x_{{\mathrm{1}}}}  :  \GRANULEnt{A}   \GRANULEsym{,}   \GRANULEmv{x_{{\mathrm{2}}}}  :  \GRANULEnt{B}   \vdash  \GRANULEnt{t_{{\mathrm{2}}}}  :  \GRANULEnt{C}}
    {\Delta  \GRANULEsym{,}   \GRANULEmv{x_{{\mathrm{3}}}}  :   \GRANULEnt{A}  \, \otimes \,  \GRANULEnt{B}    \vdash   \textbf{let} \, ( \GRANULEmv{x_{{\mathrm{1}}}} ,  \GRANULEmv{x_{{\mathrm{2}}}} ) =  \GRANULEmv{x_{{\mathrm{3}}}}  \, \textbf{in} \,  \GRANULEnt{t_{{\mathrm{2}}}}   :  \GRANULEnt{C}}
    \]

  \item Case \addSumIntroLName and \addSumIntroRName\\
    The synthesis rules for sum introduction are straightforward. For
    \addSumIntroLName we have the rule:
    \[
      \addSumIntroL
    \]
    By induction on the premises we have that:
    \begin{align*}
      \Delta  \vdash  \GRANULEnt{t}  :  \GRANULEnt{A} \tag{ih}
    \end{align*}
    From this, we can construct the typing derivation, matching the conclusion:
    \[
    \inferrule*[Right=\textsc{Inl}]
    {\Delta  \vdash  \GRANULEnt{t}  :  \GRANULEnt{A}}
    {\Delta  \vdash  \GRANULEkw{inl} \, \GRANULEnt{t}  :   \GRANULEnt{A}  \, \oplus \,  \GRANULEnt{B}}
    \]
    Likewise, for the \addSumIntroRName we have the
    synthesis rule:
    \[
      \addSumIntroR
    \]
    By induction on the premises we have that:
    \begin{align*}
      \Delta  \vdash  \GRANULEnt{t}  :  \GRANULEnt{B} \tag{ih}
    \end{align*}
    From this, we can construct the typing derivation, matching the conclusion:
    \[
    \inferrule*[Right=\textsc{Inr}]
    {\Delta  \vdash  \GRANULEnt{t}  :  \GRANULEnt{B}}
    {\Delta  \vdash  \GRANULEkw{inl} \, \GRANULEnt{t}  :   \GRANULEnt{A}  \, \oplus \,  \GRANULEnt{B}}
    \]

    \item Case \addSumElimName\\
      The synthesis rule for sum elimination has the form:
      \[
      \addSumElim
      \]
    By induction on the premises we have that:
    \begin{align*}
      \Delta_{{\mathrm{1}}}  \GRANULEsym{,}   \GRANULEmv{x_{{\mathrm{2}}}}  :  \GRANULEnt{A}   \vdash  \GRANULEnt{t_{{\mathrm{1}}}}  :  \GRANULEnt{C} \tag{ih1}\\
      \Delta_{{\mathrm{2}}}  \GRANULEsym{,}   \GRANULEmv{x_{{\mathrm{3}}}}  :  \GRANULEnt{B}   \vdash  \GRANULEnt{t_{{\mathrm{2}}}}  :  \GRANULEnt{C} \tag{ih2}
    \end{align*}
      From this, we can construct the typing derivation, matching the
      conclusion:
      \[
      \inferrule*[Right=\textsc{Case}]
      {\inferrule*[right=\textsc{Var}, leftskip=5em]{\quad}{\GRANULEmv{x_{{\mathrm{1}}}}  :   \GRANULEnt{A}  \, \oplus \,  \GRANULEnt{B}    \vdash  \GRANULEmv{x_{{\mathrm{1}}}}  :   \GRANULEnt{A}  \, \oplus \,  \GRANULEnt{B}} \\ \Delta_{{\mathrm{1}}}  \GRANULEsym{,}   \GRANULEmv{x_{{\mathrm{2}}}}  :  \GRANULEnt{A}   \vdash  \GRANULEnt{t_{{\mathrm{1}}}}  :  \GRANULEnt{C} \\ \Delta_{{\mathrm{2}}}  \GRANULEsym{,}   \GRANULEmv{x_{{\mathrm{3}}}}  :  \GRANULEnt{B}   \vdash  \GRANULEnt{t_{{\mathrm{2}}}}  :  \GRANULEnt{C}}{(\Delta_{{\mathrm{1}}} \sqcup \Delta_{{\mathrm{2}}} ), \GRANULEmv{x_{{\mathrm{1}}}}  :   \GRANULEnt{A}  \, \oplus \,  \GRANULEnt{B} \vdash \textbf{case} \ x_{1}\ \textbf{of}\ \textbf{inl}\ x_{2} \rightarrow t_{1};\ \textbf{inr}\ x_{3} \rightarrow t_{2} : C}
      \]

  \item Case \addUnitIntroName\\
    The synthesis rule for unit introduction has the form:
    \[
      \addUnitIntro
    \]
    From this, we can construct the typing derivation, matching the conclusion:
    \[
    \inferrule*[Right=\textsc{1}]
    {\quad}
    {\emptyset   \vdash  \GRANULEsym{()}  :   \mathsf{1}}
    \]

  \item Case \addUnitElimName\\
    The synthesis rule for unit elimination has the form:
    \[
      \addUnitElim
    \]
    By induction on the premises we have that:
    \begin{align*}
      \Delta  \vdash  \GRANULEnt{t}  :  \GRANULEnt{C} \tag{ih}
    \end{align*}
    From this, we can construct the typing derivation, matching the
    conclusion:
    \[
    \inferrule*[Right=\textsc{Let1}]
    { \inferrule*[Right=\textsc{Var}]{\quad}{\GRANULEmv{x}  :   \mathsf{1}    \vdash  \GRANULEmv{x}  :   \mathsf{1}} \\ \Delta  \vdash  \GRANULEnt{t}  :  \GRANULEnt{C}}
    {\Delta  \GRANULEsym{,}   \GRANULEmv{x}  :   \mathsf{1}    \vdash  \GRANULEkw{let} \, \GRANULEsym{()}  \GRANULEsym{=}  \GRANULEmv{x} \, \GRANULEkw{in} \, \GRANULEnt{t}  :  \GRANULEnt{C}}
    \]
    %

  \end{enumerate}
\end{proof}
\addPruningSynthSound*
\begin{proof}
  The cases for the rules in the additive pruning synthesis calculus are equivalent to lemma \eqref{lemma:addSynthSound}, except for the cases of the \addPruningAppName and \addPruningPairIntroName rules which we consider here:
  \begin{enumerate}
    \item Case \addPruningAppName\\
    Matching the form of the lemma, the application derivation can be
    constructed as:
    \[
      \addPruneApp
    \]
    %
    By induction on the premises we then have the following typing
    judgments:
    %
    \begin{align*}
      \Delta_{{\mathrm{1}}}  \GRANULEsym{,}   \GRANULEmv{x_{{\mathrm{2}}}}  :  \GRANULEnt{B}   \vdash  \GRANULEnt{t_{{\mathrm{1}}}}  :  \GRANULEnt{C} \\
      \Delta_{{\mathrm{2}}}  \vdash  \GRANULEnt{t_{{\mathrm{2}}}}  :  \GRANULEnt{A}
    \end{align*}
    %
    We can thus construct the following typing derivation, making use
    of the admissibility of linear substitution
    (Lemma~\ref{lemma:substitution}):
    \[
    \inferrule*[Right=(L.~\ref{lemma:substitution})]
    {\inferrule*[right=\textsc{app}, leftskip=2em]
    {\inferrule*[right=\textsc{var}]
    {\quad}
    {\GRANULEmv{x_{{\mathrm{1}}}}  :   \GRANULEnt{A}  \multimap  \GRANULEnt{B}    \vdash  \GRANULEmv{x_{{\mathrm{1}}}}  :   \GRANULEnt{A}  \multimap  \GRANULEnt{B}} \\ {\Delta_{{\mathrm{2}}}  \vdash  \GRANULEnt{t_{{\mathrm{2}}}}  :  \GRANULEnt{A}}}
  {\Delta_{{\mathrm{2}}}  \GRANULEsym{,}   \GRANULEmv{x_{{\mathrm{1}}}}  :   \GRANULEnt{A}  \multimap  \GRANULEnt{B}    \vdash  \GRANULEmv{x_{{\mathrm{1}}}} \, \GRANULEnt{t_{{\mathrm{2}}}}  :  \GRANULEnt{B}}
    \\ \Delta_{{\mathrm{1}}}  \GRANULEsym{,}   \GRANULEmv{x_{{\mathrm{2}}}}  :  \GRANULEnt{B}   \vdash  \GRANULEnt{t_{{\mathrm{1}}}}  :  \GRANULEnt{C}}
    {\GRANULEsym{(}  \Delta_{{\mathrm{1}}}  \GRANULEsym{+}  \Delta_{{\mathrm{2}}}  \GRANULEsym{)}  \GRANULEsym{,}   \GRANULEmv{x_{{\mathrm{1}}}}  :   \GRANULEnt{A}  \multimap  \GRANULEnt{B}    \vdash   [  \GRANULEsym{(}  \GRANULEmv{x_{{\mathrm{1}}}} \, \GRANULEnt{t_{{\mathrm{2}}}}  \GRANULEsym{)}  /  \GRANULEmv{x_{{\mathrm{2}}}}  ]  \GRANULEnt{t_{{\mathrm{1}}}}   :  \GRANULEnt{C}}
    \]

  \item Case \addPruningPairIntroName\\

    The synthesis rule for the pruning alternative for pair introduction has the form:

    \[
      \addPrunePairIntro
    \]
    By induction on the premises we have that:
    \begin{align*}
      \Delta_{{\mathrm{1}}}  \vdash  \GRANULEnt{t_{{\mathrm{1}}}}  :  \GRANULEnt{A} \tag{ih1} \\
      \Delta_{{\mathrm{2}}}  \vdash  \GRANULEnt{t_{{\mathrm{2}}}}  :  \GRANULEnt{B} \tag{ih2}
    \end{align*}

    From this, we can construct the typing derivation, matching the conclusion:

    \[
    \inferrule*[Right=\textsc{pair}]
    {\Delta_{{\mathrm{1}}}  \vdash  \GRANULEnt{t_{{\mathrm{1}}}}  :  \GRANULEnt{A} \\ \Delta_{{\mathrm{2}}}  \vdash  \GRANULEnt{t_{{\mathrm{2}}}}  :  \GRANULEnt{B}}
    {\Delta_{{\mathrm{1}}}  \GRANULEsym{+}  \Delta_{{\mathrm{2}}}  \vdash   ( \GRANULEnt{t_{{\mathrm{1}}}} ,  \GRANULEnt{t_{{\mathrm{2}}}} )   :   \GRANULEnt{A}  \, \otimes \,  \GRANULEnt{B}}
    \]


  \end{enumerate}
\end{proof}

%\focusSoundSub*
\begin{restatable}[Soundness of focusing for subtractive synthesis]{lemma}{focusSoundSub}
For all contexts $\Gamma$, $\Omega$ and types $\GRANULEnt{A}$
then:
\begin{align*}
\begin{array}{lll}
 1.\ Right\ Async: & \Gamma  ;  \Omega  \vdash   \GRANULEnt{A}  \Uparrow\   \Rightarrow^{-}  \GRANULEnt{t}  \ |\  \Delta \quad &\implies \quad \Gamma  \GRANULEsym{,}  \Omega  \vdash  \GRANULEnt{A}  \Rightarrow^-  \GRANULEnt{t} \ |\  \Delta\\
 2.\ Left\ Async: & \Gamma  ;   \Omega  \Uparrow\   \vdash  \GRANULEnt{C}  \Rightarrow^{-}  \GRANULEnt{t}  \ |\  \Delta \quad &\implies \quad \Gamma  \GRANULEsym{,}  \Omega  \vdash  \GRANULEnt{C}  \Rightarrow^-  \GRANULEnt{t} \ |\  \Delta\\
 3.\ Right\ Sync: & \Gamma  ;   \emptyset   \vdash   \GRANULEnt{A}  \Downarrow\   \Rightarrow^{-}  \GRANULEnt{t}  \ |\  \Delta \quad &\implies \quad \Gamma  \vdash  \GRANULEnt{A}  \Rightarrow^-  \GRANULEnt{t} \ |\  \Delta\\
 4.\ Left\ Sync: & \Gamma  ;     \GRANULEmv{x}  :  \GRANULEnt{A}    \Downarrow\   \vdash  \GRANULEnt{C}  \Rightarrow^{-}  \GRANULEnt{t}  \ |\  \Delta \quad &\implies \quad \Gamma  \GRANULEsym{,}   \GRANULEmv{x}  :  \GRANULEnt{A}   \vdash  \GRANULEnt{C}  \Rightarrow^-  \GRANULEnt{t} \ |\  \Delta\\
 5.\ Focus\ Right: & \Gamma  ;   \Omega  \Uparrow\   \vdash  \GRANULEnt{C}  \Rightarrow^{-}  \GRANULEnt{t}  \ |\  \Delta \quad &\implies \quad \Gamma  \vdash  \GRANULEnt{C}  \Rightarrow^-  \GRANULEnt{t} \ |\  \Delta\\
 6.\ Focus\ Left: & \Gamma  \GRANULEsym{,}   \GRANULEmv{x}  :  \GRANULEnt{A}   ;   \Omega  \Uparrow\   \vdash  \GRANULEnt{C}  \Rightarrow^{-}  \GRANULEnt{t}  \ |\  \Delta \quad &\implies \quad \Gamma  \vdash  \GRANULEnt{C}  \Rightarrow^-  \GRANULEnt{t} \ |\  \Delta
\end{array}
\end{align*}
\end{restatable}
\begin{proof}
  \begin{enumerate}
      \item Case 1. Right Async: \\
      \begin{enumerate}
        \item Case \subAbsName \\
          In the case of the right asynchronous rule for abstraction introduction, the synthesis rule has the form:
          \[
          \fSubAbsRuleNoLabel
          \]
          By induction on the first premise, we have that:
          \[
            \GRANULEsym{(}  \Gamma  \GRANULEsym{,}  \Omega  \GRANULEsym{)}  \GRANULEsym{,}   \GRANULEmv{x}  :  \GRANULEnt{A}   \vdash  \GRANULEnt{A}  \Rightarrow^-  \GRANULEnt{t} \ |\  \Delta \tag{ih}
          \]
          from case 1 of the lemma. From which, we can construct the following instantiation of the \subAbsName synthesis rule in the non-focusing calculus:
          \[
          \inferrule*[right=\subAbsName]
          {\GRANULEsym{(}  \Gamma  \GRANULEsym{,}  \Omega  \GRANULEsym{)}  \GRANULEsym{,}   \GRANULEmv{x}  :  \GRANULEnt{A}   \vdash  \GRANULEnt{B}  \Rightarrow^-  \GRANULEnt{t} \ |\  \Delta \quad\; \GRANULEmv{x}  \not\in | \Delta |}{\Gamma  \GRANULEsym{,}  \Omega  \vdash   \GRANULEnt{A}  \multimap  \GRANULEnt{B}   \Rightarrow^-   \lambda  \GRANULEmv{x}  .  \GRANULEnt{t}  \ |\  \Delta}
          \]
    \item Case \fSubRAsyncTransitionName \\
          In the case of the right asynchronous rule for transition to a left asynchronous judgement, the synthesis rule has the form:
          \[
            \fSubRAsyncTransitionRule
          \]
          By induction on the first premise, we have that:
          \[
            \Gamma  \GRANULEsym{,}  \Omega  \vdash  \GRANULEnt{C}  \Rightarrow^-  \GRANULEnt{t} \ |\  \Delta
          \]
          from case 2 of the lemma.
    \end{enumerate}
    \item Case 2. Left Async: \\
      \begin{enumerate}
        \item Case \subPairElimName \\
          In the case of the left asynchronous rule for pair elimination, the synthesis rule has the form:
          \[
          \fSubPairElimRuleNoLabel
          \]
          By induction on the first premise, we have that:
            \[
            \GRANULEsym{(}  \Gamma  \GRANULEsym{,}  \Omega  \GRANULEsym{)}  \GRANULEsym{,}   \GRANULEmv{x_{{\mathrm{1}}}}  :  \GRANULEnt{A}   \GRANULEsym{,}   \GRANULEmv{x_{{\mathrm{2}}}}  :  \GRANULEnt{B}   \vdash  \GRANULEnt{C}  \Rightarrow^-  \GRANULEnt{t} \ |\  \Delta \tag{ih}
            \]
          from From which, we can construct the following instantiation of the \subPairIntroName\ synthesis rule in the non-focusing calculus:
          \[
          \inferrule*[right=\subPairElimName]
          {\GRANULEsym{(}  \Gamma  \GRANULEsym{,}  \Omega  \GRANULEsym{)}  \GRANULEsym{,}   \GRANULEmv{x_{{\mathrm{1}}}}  :  \GRANULEnt{A}   \GRANULEsym{,}   \GRANULEmv{x_{{\mathrm{2}}}}  :  \GRANULEnt{B}   \vdash  \GRANULEnt{C}  \Rightarrow^-  \GRANULEnt{t} \ |\  \Delta \\ \GRANULEmv{x_{{\mathrm{1}}}}  \not\in | \Delta | \\ \GRANULEmv{x_{{\mathrm{2}}}}  \not\in | \Delta |}{\Gamma  \GRANULEsym{,}  \GRANULEsym{(}  \Omega  \GRANULEsym{,}   \GRANULEmv{x_{{\mathrm{3}}}}  :   \GRANULEnt{A}  \, \otimes \,  \GRANULEnt{B}    \GRANULEsym{)}  \vdash  \GRANULEnt{C}  \Rightarrow^-   \textbf{let} \, ( \GRANULEmv{x_{{\mathrm{1}}}} ,  \GRANULEmv{x_{{\mathrm{2}}}} ) =  \GRANULEmv{x_{{\mathrm{3}}}}  \, \textbf{in} \,  \GRANULEnt{t}  \ |\  \Delta_{{\mathrm{2}}}}
          \]
        \item Case \subSumElimName \\
          In the case of the left asynchronous rule for sum elimination, the synthesis rule has the form:
          \[
          \fSubSumElimRule
          \]
          By induction on the first and second premises, we have that:
          \[
            \GRANULEsym{(}  \Gamma  \GRANULEsym{,}  \Omega  \GRANULEsym{)}  \GRANULEsym{,}   \GRANULEmv{x_{{\mathrm{2}}}}  :  \GRANULEnt{A}   \vdash  \GRANULEnt{C}  \Rightarrow^-  \GRANULEnt{t_{{\mathrm{1}}}} \ |\  \Delta_{{\mathrm{1}}} \tag{ih1}\\
          \]
          \[
            \GRANULEsym{(}  \Gamma  \GRANULEsym{,}  \Omega  \GRANULEsym{)}  \GRANULEsym{,}   \GRANULEmv{x_{{\mathrm{3}}}}  :  \GRANULEnt{B}   \vdash  \GRANULEnt{C}  \Rightarrow^-  \GRANULEnt{t_{{\mathrm{2}}}} \ |\  \Delta_{{\mathrm{2}}} \tag{ih2}
          \]
          from case 2 of the lemma. From which, we can construct the following instantiation of the \subSumElimName\ synthesis rule in the non-focusing calculus:
          \[
    \inferrule*[right=\subSumElimName]
      {\GRANULEsym{(}  \Gamma  \GRANULEsym{,}  \Omega  \GRANULEsym{)}  \GRANULEsym{,}   \GRANULEmv{x_{{\mathrm{2}}}}  :  \GRANULEnt{A}   \vdash  \GRANULEnt{C}  \Rightarrow^-  \GRANULEnt{t_{{\mathrm{1}}}} \ |\  \Delta_{{\mathrm{1}}} \quad\,
       \GRANULEsym{(}  \Gamma  \GRANULEsym{,}  \Omega  \GRANULEsym{)}  \GRANULEsym{,}   \GRANULEmv{x_{{\mathrm{3}}}}  :  \GRANULEnt{B}   \vdash  \GRANULEnt{C}  \Rightarrow^-  \GRANULEnt{t_{{\mathrm{2}}}} \ |\  \Delta_{{\mathrm{2}}} \quad\, \GRANULEmv{x_{{\mathrm{2}}}}  \not\in | \Delta_{{\mathrm{1}}} | \quad \GRANULEmv{x_{{\mathrm{3}}}}  \not\in | \Delta_{{\mathrm{2}}} |}
     {\Gamma  \GRANULEsym{,}  \GRANULEsym{(}  \Omega  \GRANULEsym{,}   \GRANULEmv{x_{{\mathrm{1}}}}  :   \GRANULEnt{A}  \, \oplus \,  \GRANULEnt{B}    \GRANULEsym{)} \vdash C \Rightarrow^{-}  \textbf{case} \ x_{1}\ \textbf{of}\ \textbf{inl}\ x_{2} \rightarrow t_{1};\ \textbf{inr}\ x_{3} \rightarrow t_{2} \Delta_{{\mathrm{1}}} \sqcap \Delta_{{\mathrm{2}}}}
          \]
        \item Case \subUnitElimName \\
          In the case of the left asynchronous rule for unit elimination, the synthesis rule has the form:
          \[
          \fSubUnitElimRule
          \]
          By induction on the premise, we have that:
          \[
            \Gamma  \vdash  \GRANULEnt{C}  \Rightarrow^-  \GRANULEnt{t} \ |\  \Delta \tag{ih}
          \]
          from case 2 of the lemma. From which, we can construct the following instantiation of the \subUnitElimName\ synthesis rule in the non-focusing calculus matching the conclusion:
          \[
    \inferrule*[right=\subUnitElimName]
    {\Gamma  \vdash  \GRANULEnt{C}  \Rightarrow^-  \GRANULEnt{t} \ |\  \Delta}
    {\Gamma  \GRANULEsym{,}   \GRANULEmv{x}  :   \mathsf{1}    \vdash  \GRANULEnt{C}  \Rightarrow^-  \GRANULEkw{let} \, \GRANULEsym{()}  \GRANULEsym{=}  \GRANULEmv{x} \, \GRANULEkw{in} \, \GRANULEnt{t} \ |\  \Delta}
          \]
        \item Case \subUnboxName \\
          In the case of the left asynchronous rule for graded modality elimination, the synthesis rule has the form:
          \[
          \fSubUnboxRule
          \]
          By induction on the first premise, we have that:
          \[
            \GRANULEsym{(}  \Gamma  \GRANULEsym{,}  \Omega  \GRANULEsym{)}  \GRANULEsym{,}   \GRANULEmv{x_{{\mathrm{2}}}}  :_{\textcolor{coeffectColor}{  \GRANULEnt{r}  } }   \GRANULEnt{A}   \vdash  \GRANULEnt{B}  \Rightarrow^-  \GRANULEnt{t} \ |\  \Delta  \GRANULEsym{,}   \GRANULEmv{x_{{\mathrm{2}}}}  :_{\textcolor{coeffectColor}{  \GRANULEnt{s}  } }   \GRANULEnt{A}  \tag{ih}\\
          \]
          from case 2 of the lemma. From which, we can construct the following instatiation of the \subUnboxName synthesis rule in the non-focusing calculus:
          \[
  \inferrule*[right=\subUnboxName]
    {\GRANULEsym{(}  \Gamma  \GRANULEsym{,}  \Omega  \GRANULEsym{)}  \GRANULEsym{,}   \GRANULEmv{x_{{\mathrm{2}}}}  :_{\textcolor{coeffectColor}{  \GRANULEnt{r}  } }   \GRANULEnt{A}   \vdash  \GRANULEnt{B}  \Rightarrow^-  \GRANULEnt{t} \ |\  \Delta  \GRANULEsym{,}   \GRANULEmv{x_{{\mathrm{2}}}}  :_{\textcolor{coeffectColor}{  \GRANULEnt{s}  } }   \GRANULEnt{A}  \\ 0 \sqsubseteq \GRANULEnt{s}}{\Gamma  \GRANULEsym{,}  \GRANULEsym{(}  \Omega  \GRANULEsym{,}   \GRANULEmv{x_{{\mathrm{1}}}}  :   \Box_{  \GRANULEnt{r}  }  \GRANULEnt{A}    \GRANULEsym{)}  \vdash  \GRANULEnt{B}  \Rightarrow^-   \textbf{let} \, [  \GRANULEmv{x_{{\mathrm{2}}}}  ] =  \GRANULEmv{x_{{\mathrm{1}}}}  \, \textbf{in} \,  \GRANULEnt{t}  \ |\  \Delta}
          \]
        \item Case \subDerName \\
          In the case of the left asynchronous rule for dereliction, the synthesis rule has the form:
          \[
          \fSubDerRule
          \]
          By induction on the first premise, we have that:
          \[
            \Gamma  \GRANULEsym{,}   \GRANULEmv{x}  :_{\textcolor{coeffectColor}{  \GRANULEnt{s}  } }   \GRANULEnt{A}   \GRANULEsym{,}   \GRANULEmv{y}  :  \GRANULEnt{A}   \vdash  \GRANULEnt{B}  \Rightarrow^-  \GRANULEnt{t} \ |\  \Delta  \GRANULEsym{,}   \GRANULEmv{x}  :_{\textcolor{coeffectColor}{  \GRANULEnt{s'}  } }   \GRANULEnt{A}  \tag{ih}\\
          \]
          from case 2 of the lemma. From which, we can construct the following instatiation of the \subDerName synthesis rule in the non-focusing calculus:
          \[
      \inferrule*[right=\subDerName]
{\Gamma  \GRANULEsym{,}   \GRANULEmv{x}  :_{\textcolor{coeffectColor}{  \GRANULEnt{s}  } }   \GRANULEnt{A}   \GRANULEsym{,}   \GRANULEmv{y}  :  \GRANULEnt{A}   \vdash  \GRANULEnt{B}  \Rightarrow^-  \GRANULEnt{t} \ |\  \Delta  \GRANULEsym{,}   \GRANULEmv{x}  :_{\textcolor{coeffectColor}{  \GRANULEnt{s'}  } }   \GRANULEnt{A} \\
\GRANULEmv{y}  \not\in | \Delta | \\
\exists  \GRANULEnt{s}  .\,   \GRANULEnt{r}  \sqsupseteq  \GRANULEnt{s}  \GRANULEsym{+}   1
}
{\Gamma  \GRANULEsym{,}   \GRANULEmv{x}  :_{\textcolor{coeffectColor}{  \GRANULEnt{r}  } }   \GRANULEnt{A}   \vdash  \GRANULEnt{B}  \Rightarrow^-   [  \GRANULEmv{x}  /  \GRANULEmv{y}  ]  \GRANULEnt{t}  \ |\  \Delta  \GRANULEsym{,}   \GRANULEmv{x}  :_{\textcolor{coeffectColor}{  \GRANULEnt{s'}  } }   \GRANULEnt{A}}
          \]

        \item Case \fSubLAsyncTransitionName \\
          In the case of the left asynchronous rule for transitioning an assumption from the focusing context $\Omega$ to the non-focusing context $\Gamma$, the synthesis rule has the form:
          \[
            \fSubLAsyncTransitionRule
          \]
          By induction on the first premise, we have that:
          \[
            \Gamma  \GRANULEsym{,}   \GRANULEmv{x}  :  \GRANULEnt{A}    \GRANULEsym{,}  \Omega  \vdash  \GRANULEnt{C}  \Rightarrow^-  \GRANULEnt{t} \ |\  \Delta \tag{ih}
          \]
          from case 2 of the lemma.
      \end{enumerate}
    \item Case 3. Right Sync: \\
      \begin{enumerate}
        \item Case \subPairIntroName \\
          In the case of the right synchronous rule for pair introduction, the synthesis rule has the form:
          \[
          \fSubPairIntroRuleNoLabel
          \]
          By induction on the first and second premises, we have that:
          \[
            \Gamma  \vdash  \GRANULEnt{A}  \Rightarrow^-  \GRANULEnt{t_{{\mathrm{1}}}} \ |\  \Delta_{{\mathrm{1}}}  \tag{ih1}
          \]
          \[
            \Delta_{{\mathrm{1}}}  \vdash  \GRANULEnt{B}  \Rightarrow^-  \GRANULEnt{t_{{\mathrm{2}}}} \ |\  \Delta_{{\mathrm{2}}} \tag{ih2}
          \]
          from case 3 of the lemma. From which, we can construct the following instatiation of the \subPairIntroName\ synthesis rule in the non-focusing calculus:
          \[
    \inferrule*[right=\subPairIntroName]
    {\Gamma  \vdash  \GRANULEnt{A}  \Rightarrow^-  \GRANULEnt{t_{{\mathrm{1}}}} \ |\  \Delta_{{\mathrm{1}}} \\ \Delta_{{\mathrm{1}}}  \vdash  \GRANULEnt{B}  \Rightarrow^-  \GRANULEnt{t_{{\mathrm{2}}}} \ |\  \Delta_{{\mathrm{2}}}}{\Gamma  \vdash   \GRANULEnt{A}  \, \otimes \,  \GRANULEnt{B}   \Rightarrow^-   ( \GRANULEnt{t_{{\mathrm{1}}}} ,  \GRANULEnt{t_{{\mathrm{2}}}} )  \ |\  \Delta_{{\mathrm{2}}}}
          \]
        \item Case \subSumIntroLname\ and \subSumIntroRname\\
          In the case of the right synchronous rules for sum introduction, the synthesis rules has the form:
          \[
          \fSubSumIntroRuleL
          \]
          \[
          \fSubSumIntroRuleR
          \]
          By induction on the premises of these rules, we have that:
          \[
            \Gamma  \vdash  \GRANULEnt{A}  \Rightarrow^-  \GRANULEnt{t} \ |\  \Delta  \tag{ih1}
          \]
          \[
            \Gamma  \vdash  \GRANULEnt{B}  \Rightarrow^-  \GRANULEnt{t} \ |\  \Delta \tag{ih2}
          \]
          from case 3 of the lemma. From which, we can construct the following instatiations of the \subSumIntroLname\ and \subSumIntroRname\ rule in the non-focusing calculus, respectively:
          \[
    \inferrule*[right=\subSumIntroLname]
    {\Gamma  \vdash  \GRANULEnt{A}  \Rightarrow^-  \GRANULEnt{t} \ |\  \Delta}
    {\Gamma  \vdash   \GRANULEnt{A}  \, \oplus \,  \GRANULEnt{B}   \Rightarrow^-  \GRANULEkw{inl} \, \GRANULEnt{t} \ |\  \Delta}
          \]
          \[
    \inferrule*[right=\subSumIntroRname]
    {\Gamma  \vdash  \GRANULEnt{B}  \Rightarrow^-  \GRANULEnt{t} \ |\  \Delta}
    {\Gamma  \vdash   \GRANULEnt{A}  \, \oplus \,  \GRANULEnt{B}   \Rightarrow^-  \GRANULEkw{inr} \, \GRANULEnt{t} \ |\  \Delta}
          \]
        \item Case \subUnitIntroName \\
          In the case of the right synchronous rule for unit introduction, the synthesis rule has the form:
          \[
          \fSubUnitIntroRule
          \]
          From which, we can construct the following instatiation of the \subUnitIntroName\ synthesis rule in the non-focusing calculus:
          \[
    \inferrule*[right=\subUnitIntroName]
    {\quad}
    {\Gamma  \GRANULEsym{,}  \Omega  \vdash   \mathsf{1}   \Rightarrow^-  \GRANULEsym{()} \ |\  \Gamma}
          \]
        \item Case \subBoxName \\
          In the case of the right synchronous rule for graded modality introduction, the synthesis rule has the form:
          \[
          \fSubBoxRule
          \]
          By induction on the premise, we have that:
          \[
            \Gamma  \vdash  \GRANULEnt{A}  \Rightarrow^-  \GRANULEnt{t} \ |\  \Delta \tag{ih}
          \]
          from case 1 of the lemma. From which, we can construct the following instatiation of the \subBoxName synthesis rule in the non-focusing calculus:
          \[
  \inferrule*[right=\subBoxName]
  {\Gamma  \vdash  \GRANULEnt{A}  \Rightarrow^-  \GRANULEnt{t} \ |\  \Delta}{\Gamma  \vdash   \Box_{  \GRANULEnt{r}  }  \GRANULEnt{A}   \Rightarrow^-  \GRANULEsym{[}  \GRANULEnt{t}  \GRANULEsym{]} \ |\  \Gamma  \GRANULEsym{-}   \textcolor{coeffectColor}{ \GRANULEnt{r}   \textcolor{coeffectColor}{\,\cdot\,} }  \GRANULEsym{(}  \Gamma  \GRANULEsym{-}  \Delta  \GRANULEsym{)}}
          \]
      \item Case \fSubRSyncTransitionName \\
          In the case of the right synchronous rule for transitioning back to an asynchronous judgement, the synthesis rule has the form:
          \[
            \fSubRSyncTransitionRule
          \]
          By induction on the premise, we have that:
          \[
            \Gamma  \vdash  \GRANULEnt{A}  \Rightarrow^-  \GRANULEnt{t} \ |\  \Delta \tag{ih}
          \]
          from case 1 of the lemma.
      \end{enumerate}
    \item Case 4. Left Sync \\
      \begin{enumerate}
          \item Case \subAppName \\
          In the case of the left synchronous rule for application, the synthesis rule has the form:
          \[
          \fSubAppRuleNoLabel
          \]
          By induction on the first premise, we have that:
          \[
            \Gamma  \GRANULEsym{,}   \GRANULEmv{x_{{\mathrm{2}}}}  :  \GRANULEnt{B}   \vdash  \GRANULEnt{C}  \Rightarrow^-  \GRANULEnt{t_{{\mathrm{1}}}} \ |\  \Delta_{{\mathrm{1}}} \tag{ih1}
          \]
          from case 4 of the lemma. By induction on the third premise, we have that:
          \[
            \Delta_{{\mathrm{1}}}  \vdash  \GRANULEnt{A}  \Rightarrow^-  \GRANULEnt{t_{{\mathrm{2}}}} \ |\  \Delta_{{\mathrm{2}}} \tag{ih2}
          \]
          from case 3 of the lemma. From which, we can construct the following instatiation of the \subAppName synthesis rule in the non-focusing calculus:
          \[
  \inferrule*[right=\subAppName]
  {\Gamma  \GRANULEsym{,}   \GRANULEmv{x_{{\mathrm{2}}}}  :  \GRANULEnt{B}   \vdash  \GRANULEnt{C}  \Rightarrow^-  \GRANULEnt{t_{{\mathrm{1}}}} \ |\  \Delta_{{\mathrm{1}}} \qquad \GRANULEmv{x_{{\mathrm{2}}}}  \not\in | \Delta_{{\mathrm{1}}} | \qquad \Delta_{{\mathrm{1}}}  \vdash  \GRANULEnt{A}  \Rightarrow^-  \GRANULEnt{t_{{\mathrm{2}}}} \ |\  \Delta_{{\mathrm{2}}}}{\Gamma  \GRANULEsym{,}   \GRANULEmv{x_{{\mathrm{1}}}}  :   \GRANULEnt{A}  \multimap  \GRANULEnt{B}    \vdash  \GRANULEnt{C}  \Rightarrow^-   [  \GRANULEsym{(}  \GRANULEmv{x_{{\mathrm{1}}}} \, \GRANULEnt{t_{{\mathrm{2}}}}  \GRANULEsym{)}  /  \GRANULEmv{x_{{\mathrm{2}}}}  ]  \GRANULEnt{t_{{\mathrm{1}}}}  \ |\  \Delta_{{\mathrm{2}}}}
          \]
          \item Case \subLinVarName \\
          In the case of the left synchronous rule for linear variable synthesis, the synthesis rule has the form:
          \[
          \fSubLinVarRule
          \]
          From which, we can construct the following instatiation of the \subLinVarName\  synthesis rule in the non-focusing calculus:
          \[
                             \inferrule*[right=\subLinVarName]
                             {\quad}{\Gamma  \GRANULEsym{,}   \GRANULEmv{x}  :  \GRANULEnt{A}   \vdash  \GRANULEnt{A}  \Rightarrow^-  \GRANULEmv{x} \ |\  \Gamma}
          \]
          \item Case \subGrVarName \\
          In the case of the left synchronous rule for graded variable synthesis, the synthesis rule has the form:
          \[
          \fSubGrVarRule
          \]
          From which, we can construct the following instatiation of the \subGrVarName\  synthesis rule in the non-focusing calculus:
          \[
      \inferrule*[right=\subGrVarName]
  {\exists  \GRANULEnt{s}  .\,   \GRANULEnt{r}  \sqsubseteq   \GRANULEnt{s}  \GRANULEsym{+}   1}{\Gamma  \GRANULEsym{,}   \GRANULEmv{x}  :_{\textcolor{coeffectColor}{  \GRANULEnt{r}  } }   \GRANULEnt{A}   \vdash  \GRANULEnt{A}  \Rightarrow^-  \GRANULEmv{x} \ |\  \Gamma  \GRANULEsym{,}   \GRANULEmv{x}  :_{\textcolor{coeffectColor}{  \GRANULEnt{s}  } }   \GRANULEnt{A}}
          \]
      \item Case \fSubLSyncTransitionName \\
          In the case of the left synchronous rule for transitioning back to an asynchronous judgement, the synthesis rule has the form:
          \[
            \fSubLSyncTransitionRule
          \]
          By induction on the premise, we have that:
          \[
            \Gamma  \GRANULEsym{,}   \GRANULEmv{x}  :  \GRANULEnt{A}   \vdash  \GRANULEnt{C}  \Rightarrow^-  \GRANULEnt{t} \ |\  \Delta \tag{ih}
          \]
          from case 2 of the lemma.
      \end{enumerate}
        \item Case 5. Focus Right: \fSubFocusRName \\
          In the case of the focusing rule for transitioning from a left asynchronous judgement to a right synchronous judgement, the synthesis rule has the form:
          \[
            \fSubFocusRRuleNoLabel
          \]
          By induction on the first premise, we have that:
          \[
            \Gamma  \vdash  \GRANULEnt{C}  \Rightarrow^-  \GRANULEnt{t} \ |\  \Delta \tag{ih}
          \]
          from case 2 of the lemma.
        \item Case 6. Focus Left \fSubFocusLName \\
          In the case of the focusing rule for transitioning from a left asynchronous judgement to a left synchronous judgement, the synthesis rule has the form:
          \[
            \fSubFocusLRule
          \]
          By induction on the first premise, we have that:
          \[
            \Gamma  \GRANULEsym{,}   \GRANULEmv{x}  :  \GRANULEnt{A}   \vdash  \GRANULEnt{C}  \Rightarrow^-  \GRANULEnt{t} \ |\  \Delta \tag{ih}
          \]
          from case 2 of the lemma.
  \end{enumerate}
\end{proof}

%\focusSoundAdd*
\begin{restatable}[Soundness of focusing for additive synthesis]{lemma}{focusSoundAdd}
  \label{lemma:fAddSynthSound}
For all contexts $\Gamma$, $\Omega$ and types $\GRANULEnt{A}$
then:
\begin{align*}
\begin{array}{lll}
 1.\ Right\ Async: & \Gamma  ;  \Omega  \vdash   \GRANULEnt{A}  \Uparrow\   \Rightarrow^{+}  \GRANULEnt{t}  \ | \  \Delta \quad &\implies \quad \Gamma  \GRANULEsym{,}  \,  \GRANULEsym{,}  \Omega  \vdash  \GRANULEnt{A}  \Rightarrow^+  \GRANULEnt{t}  ;\,  \Delta\\
 2.\ Left\ Async: & \Gamma  ;   \Omega  \Uparrow\   \vdash  \GRANULEnt{C}  \Rightarrow^{+}  \GRANULEnt{t}  \ | \  \Delta \quad &\implies \quad \Gamma  \GRANULEsym{,}  \,  \GRANULEsym{,}  \Omega  \vdash  \GRANULEnt{C}  \Rightarrow^+  \GRANULEnt{t}  ;\,  \Delta\\
 3.\ Right\ Sync: & \Gamma  ;   \emptyset   \vdash   \GRANULEnt{A}  \Downarrow\   \Rightarrow^{+}  \GRANULEnt{t}  \ | \  \Delta \quad &\implies \quad \Gamma  \vdash  \GRANULEnt{A}  \Rightarrow^+  \GRANULEnt{t}  ;\,  \Delta\\
 4.\ Left\ Sync: & \Gamma  ;     \GRANULEmv{x}  :  \GRANULEnt{A}    \Downarrow\   \vdash  \GRANULEnt{C}  \Rightarrow^{+}  \GRANULEnt{t}  \ | \  \Delta \quad &\implies \quad \Gamma  \GRANULEsym{,}   \GRANULEmv{x}  :  \GRANULEnt{A}   \vdash  \GRANULEnt{C}  \Rightarrow^+  \GRANULEnt{t}  ;\,  \Delta\\
 5.\ Focus\ Right: & \Gamma  ;   \Omega  \Uparrow\   \vdash  \GRANULEnt{C}  \Rightarrow^{+}  \GRANULEnt{t}  \ | \  \Delta \quad &\implies \quad \Gamma  \vdash  \GRANULEnt{C}  \Rightarrow^+  \GRANULEnt{t}  ;\,  \Delta\\
 6.\ Focus\ Left: & \Gamma  \GRANULEsym{,}   \GRANULEmv{x}  :  \GRANULEnt{A}   ;   \Omega  \Uparrow\   \vdash  \GRANULEnt{C}  \Rightarrow^{+}  \GRANULEnt{t}  \ | \  \Delta \quad &\implies \quad \Gamma  \vdash  \GRANULEnt{C}  \Rightarrow^+  \GRANULEnt{t}  ;\,  \Delta
\end{array}
\end{align*}
\end{restatable}
\begin{proof}
  \begin{enumerate}
      \item Case 1. Right Async: \\
      \begin{enumerate}
        \item Case \addAbsName \\
          In the case of the right asynchronous rule for abstraction introduction, the synthesis rule has the form:
          \[
          \fAddAbsRuleNoLabel
          \]
          By induction on the premise, we have that:
          \[
            \GRANULEsym{(}  \Gamma  \GRANULEsym{,}  \Omega  \GRANULEsym{)}  \GRANULEsym{,}   \GRANULEmv{x}  :  \GRANULEnt{A}   \vdash  \GRANULEnt{B}  \Rightarrow^+  \GRANULEnt{t}  ;\,  \Delta  \GRANULEsym{,}   \GRANULEmv{x}  :  \GRANULEnt{A}   \tag{ih}
          \]
          from case 1 of the lemma. From which, we can construct the following instatiation of the \addAbsName\ synthesis rule in the non-focusing calculus:
          \[
    \inferrule*[right=R$\multimap^{+}$]
    {\GRANULEsym{(}  \Gamma  \GRANULEsym{,}  \Omega  \GRANULEsym{)}  \GRANULEsym{,}   \GRANULEmv{x}  :  \GRANULEnt{A}   \vdash  \GRANULEnt{B}  \Rightarrow^+  \GRANULEnt{t}  ;\,  \Delta  \GRANULEsym{,}   \GRANULEmv{x}  :  \GRANULEnt{A}}{\Gamma  \GRANULEsym{,}  \Omega  \vdash   \GRANULEnt{A}  \multimap  \GRANULEnt{B}   \Rightarrow^+   \lambda  \GRANULEmv{x}  .  \GRANULEnt{t}   ;\,  \Delta}
          \]
          \item Case \fAddRAsyncTransitionName
          In the case of the right asynchronous rule for transition to a left asynchronous judgement, the synthesis rule has the form:
          \[
            \fAddRAsyncTransitionRule
          \]
          By induction on the first premise, we have that:
          \[
            \Gamma  \GRANULEsym{,}  \Omega  \vdash  \GRANULEnt{C}  \Rightarrow^+  \GRANULEnt{t}  ;\,  \Delta
          \]
          from case 2 of the lemma.
      \end{enumerate}
    \item Case 2. Left Async: \\
      \begin{enumerate}
        \item Case \addPairElimName \\
          In the case of the left asynchronous rule for pair elimination, the synthesis rule has the form:
          \[
          \fAddPairElimRuleNoLabel
          \]
          By induction on the premise, we have that:
          \[
            \GRANULEsym{(}  \Gamma  \GRANULEsym{,}  \Omega  \GRANULEsym{)}  \GRANULEsym{,}   \GRANULEmv{x_{{\mathrm{1}}}}  :  \GRANULEnt{A}   \GRANULEsym{,}   \GRANULEmv{x_{{\mathrm{2}}}}  :  \GRANULEnt{B}   \vdash  \GRANULEnt{C}  \Rightarrow^+  \GRANULEnt{t_{{\mathrm{2}}}}  ;\,  \Delta  \GRANULEsym{,}   \GRANULEmv{x_{{\mathrm{1}}}}  :  \GRANULEnt{A}   \GRANULEsym{,}   \GRANULEmv{x_{{\mathrm{2}}}}  :  \GRANULEnt{B}   \tag{ih}
          \]
          from case 2 of the lemma. From which, we can construct the following instatiation of the \addPairElimName\ synthesis rule in the non-focusing calculus:
          \[
    \inferrule*[right=L$\otimes^{+}$]
    {\GRANULEsym{(}  \Gamma  \GRANULEsym{,}  \Omega  \GRANULEsym{)}  \GRANULEsym{,}   \GRANULEmv{x_{{\mathrm{1}}}}  :  \GRANULEnt{A}   \GRANULEsym{,}   \GRANULEmv{x_{{\mathrm{2}}}}  :  \GRANULEnt{B}   \vdash  \GRANULEnt{C}  \Rightarrow^+  \GRANULEnt{t_{{\mathrm{2}}}}  ;\,  \Delta  \GRANULEsym{,}   \GRANULEmv{x_{{\mathrm{1}}}}  :  \GRANULEnt{A}   \GRANULEsym{,}   \GRANULEmv{x_{{\mathrm{2}}}}  :  \GRANULEnt{B}}
    {\Gamma  \GRANULEsym{,}  \GRANULEsym{(}  \Omega  \GRANULEsym{,}   \GRANULEmv{x_{{\mathrm{3}}}}  :   \GRANULEnt{A}  \, \otimes \,  \GRANULEnt{B}    \GRANULEsym{)}  \vdash  \GRANULEnt{C}  \Rightarrow^+   \textbf{let} \, ( \GRANULEmv{x_{{\mathrm{1}}}} ,  \GRANULEmv{x_{{\mathrm{2}}}} ) =  \GRANULEmv{x_{{\mathrm{3}}}}  \, \textbf{in} \,  \GRANULEnt{t_{{\mathrm{2}}}}   ;\,  \Delta  \GRANULEsym{,}   \GRANULEmv{x_{{\mathrm{3}}}}  :   \GRANULEnt{A}  \, \otimes \,  \GRANULEnt{B}}
          \]
        \item Case \addSumElimName \\
          In the case of the left asynchronous rule for sum elimination, the synthesis rule has the form:
          \[
          \fAddSumElimRule
          \]
          By induction on the premises, we have that:
          \[
           \GRANULEsym{(}  \Gamma  \GRANULEsym{,}  \Omega  \GRANULEsym{)}  \GRANULEsym{,}   \GRANULEmv{x_{{\mathrm{2}}}}  :  \GRANULEnt{A}   \vdash  \GRANULEnt{C}  \Rightarrow^+  \GRANULEnt{t_{{\mathrm{1}}}}  ;\,  \Delta_{{\mathrm{1}}}  \GRANULEsym{,}   \GRANULEmv{x_{{\mathrm{2}}}}  :  \GRANULEnt{A}   \tag{ih1}
          \]
          \[
           \GRANULEsym{(}  \Gamma  \GRANULEsym{,}  \Omega  \GRANULEsym{)}  \GRANULEsym{,}   \GRANULEmv{x_{{\mathrm{3}}}}  :  \GRANULEnt{B}   \vdash  \GRANULEnt{C}  \Rightarrow^+  \GRANULEnt{t_{{\mathrm{2}}}}  ;\,  \Delta_{{\mathrm{2}}}  \GRANULEsym{,}   \GRANULEmv{x_{{\mathrm{3}}}}  :  \GRANULEnt{B}   \tag{ih2}
          \]
          from case 2 of the lemma. From which, we can construct the following instatiation of the \addSumElimName\ synthesis rule in the non-focusing calculus:
          \[
    \inferrule*[right=L$\oplus^{+}$]
    {\GRANULEsym{(}  \Gamma  \GRANULEsym{,}  \Omega  \GRANULEsym{)}  \GRANULEsym{,}   \GRANULEmv{x_{{\mathrm{2}}}}  :  \GRANULEnt{A}   \vdash  \GRANULEnt{C}  \Rightarrow^+  \GRANULEnt{t_{{\mathrm{1}}}}  ;\,  \Delta_{{\mathrm{1}}}  \GRANULEsym{,}   \GRANULEmv{x_{{\mathrm{2}}}}  :  \GRANULEnt{A} \\ \GRANULEsym{(}  \Gamma  \GRANULEsym{,}  \Omega  \GRANULEsym{)}  \GRANULEsym{,}   \GRANULEmv{x_{{\mathrm{3}}}}  :  \GRANULEnt{B}   \vdash  \GRANULEnt{C}  \Rightarrow^+  \GRANULEnt{t_{{\mathrm{2}}}}  ;\,  \Delta_{{\mathrm{2}}}  \GRANULEsym{,}   \GRANULEmv{x_{{\mathrm{3}}}}  :  \GRANULEnt{B}}{\Gamma  \GRANULEsym{,}  \GRANULEsym{(}  \Omega  \GRANULEsym{,}   \GRANULEmv{x_{{\mathrm{1}}}}  :   \GRANULEnt{A}  \, \oplus \,  \GRANULEnt{B}    \GRANULEsym{)} \vdash C \Rightarrow^{+} \textbf{case} \ x_{1}\ \textbf{of}\ \textbf{inl}\ x_{2} \rightarrow t_{1};\ \textbf{inr}\ x_{3} \rightarrow t_{2}\ |\  (\Delta_{{\mathrm{1}}} \sqcup \Delta_{{\mathrm{2}}}), \GRANULEmv{x_{{\mathrm{1}}}}  :   \GRANULEnt{A}  \, \oplus \,  \GRANULEnt{B}}
          \]
        \item Case \addUnitElimName \\
          In the case of the left asynchronous rule for unit elimination, the synthesis rule has the form:
          \[
          \fAddUnitElimRule
          \]
          By induction on the premise, we have that:
          \[
           \Gamma  \vdash  \GRANULEnt{C}  \Rightarrow^+  \GRANULEnt{t}  ;\,  \Delta   \tag{ih}
          \]
          from case 2 of the lemma. From which, we can construct the following instatiation of the \addUnitElimName\ synthesis rule in the non-focusing calculus:
          \[
    \inferrule*[right=L1$^{+}$]
    {\Gamma  \vdash  \GRANULEnt{C}  \Rightarrow^+  \GRANULEnt{t}  ;\,  \Delta}
    {\Gamma  \GRANULEsym{,}   \GRANULEmv{x}  :   \mathsf{1}    \vdash  \GRANULEnt{C}  \Rightarrow^+  \GRANULEkw{let} \, \GRANULEsym{()}  \GRANULEsym{=}  \GRANULEmv{x} \, \GRANULEkw{in} \, \GRANULEnt{t}  ;\,  \Delta  \GRANULEsym{,}   \GRANULEmv{x}  :   \mathsf{1}}
          \]
        \item Case \addUnboxName \\
          In the case of the left asynchronous rule for graded modality elimination, the synthesis rule has the form:
          \[
          \fAddUnboxRule
          \]
          By induction on the first premise, we have that:
          \[
            \GRANULEsym{(}  \Gamma  \GRANULEsym{,}  \Omega  \GRANULEsym{)}  \GRANULEsym{,}   \GRANULEmv{x_{{\mathrm{2}}}}  :_{\textcolor{coeffectColor}{  \GRANULEnt{r}  } }   \GRANULEnt{A}   \vdash  \GRANULEnt{B}  \Rightarrow^+  \GRANULEnt{t}  ;\,  \Delta \tag{ih}
          \]
          from case 2 of the lemma. From which, we can construct the following instatiation of the \addUnboxName\ synthesis rule in the non-focusing calculus:
          \[
    \inferrule*[right=L$\square^{+}$]
    {\GRANULEsym{(}  \Gamma  \GRANULEsym{,}  \Omega  \GRANULEsym{)}  \GRANULEsym{,}   \GRANULEmv{x_{{\mathrm{2}}}}  :_{\textcolor{coeffectColor}{  \GRANULEnt{r}  } }   \GRANULEnt{A}   \vdash  \GRANULEnt{B}  \Rightarrow^+  \GRANULEnt{t}  ;\,  \Delta \\ \textit{if}\ \GRANULEmv{x_{{\mathrm{2}}}}  :_{\textcolor{coeffectColor}{  \GRANULEnt{s}  } }   \GRANULEnt{A} \in
      \Delta\ \textit{then}\ \GRANULEnt{s} \sqsubseteq \GRANULEnt{r}\ \textit{else}\ 0 \sqsubseteq \GRANULEnt{r}}{\Gamma  \GRANULEsym{,}  \GRANULEsym{(}  \Omega  \GRANULEsym{,}   \GRANULEmv{x_{{\mathrm{1}}}}  :   \Box_{  \GRANULEnt{r}  }  \GRANULEnt{A}    \GRANULEsym{)}  \vdash  \GRANULEnt{B}  \Rightarrow^+   \textbf{let} \, [  \GRANULEmv{x_{{\mathrm{2}}}}  ] =  \GRANULEmv{x_{{\mathrm{1}}}}  \, \textbf{in} \,  \GRANULEnt{t}   ;\,  \GRANULEsym{(}   \Delta \!\setminus\!  \GRANULEmv{x_{{\mathrm{2}}}}   \GRANULEsym{)}  \GRANULEsym{,}   \GRANULEmv{x_{{\mathrm{1}}}}  :   \Box_{  \GRANULEnt{r}  }  \GRANULEnt{A}}
          \]
        \item Case \addDerName \\
          In the case of the left asynchronous rule for dereliction, the synthesis rule has the form:
          \[
          \fAddDerRule
          \]
          By induction on the premise, we have that:
          \[
           \Gamma  \GRANULEsym{,}   \GRANULEmv{x}  :_{\textcolor{coeffectColor}{  \GRANULEnt{s}  } }   \GRANULEnt{A}   \GRANULEsym{,}   \GRANULEmv{y}  :  \GRANULEnt{A}   \vdash  \GRANULEnt{B}  \Rightarrow^+  \GRANULEnt{t}  ;\,  \Delta  \GRANULEsym{,}   \GRANULEmv{y}  :  \GRANULEnt{A}   \tag{ih}
          \]
          from case 2 of the lemma. From which, we can construct the following instantiation of the \addDerName\ synthesis rule in the non-focusing calculus:
          \[
\inferrule*[right=der$^{+}$]
{ \Gamma  \GRANULEsym{,}   \GRANULEmv{x}  :_{\textcolor{coeffectColor}{  \GRANULEnt{s}  } }   \GRANULEnt{A}   \GRANULEsym{,}   \GRANULEmv{y}  :  \GRANULEnt{A}   \vdash  \GRANULEnt{B}  \Rightarrow^+  \GRANULEnt{t}  ;\,  \Delta  \GRANULEsym{,}   \GRANULEmv{y}  :  \GRANULEnt{A} }
{ \Gamma  \GRANULEsym{,}   \GRANULEmv{x}  :_{\textcolor{coeffectColor}{  \GRANULEnt{s}  } }   \GRANULEnt{A}   \vdash  \GRANULEnt{B}  \Rightarrow^+   [  \GRANULEmv{x}  /  \GRANULEmv{y}  ]  \GRANULEnt{t}   ;\,  \Delta  \GRANULEsym{+}   \GRANULEmv{x}  :_{\textcolor{coeffectColor}{   1   } }   \GRANULEnt{A} }
          \]
        \item Case \fAddLAsyncTransitionName \\
          In the case of the left asynchronous rule for transitioning an assumption from the focusing context $\Omega$ to the non-focusing context $\Gamma$, the synthesis rule has the form:
          \[
            \fAddLAsyncTransitionRule
          \]
          By induction on the first premise, we have that:
          \[
            \Gamma  \GRANULEsym{,}   \GRANULEmv{x}  :  \GRANULEnt{A}    \GRANULEsym{,}  \Omega  \vdash  \GRANULEnt{C}  \Rightarrow^+  \GRANULEnt{t}  ;\,  \Delta \tag{ih}
          \]
          from case 2 of the lemma.
      \end{enumerate}
    \item Case 3. Right Sync: \\
      \begin{enumerate}
        \item Case \addPairIntroName \\
          In the case of the right synchronous rule for pair introduction, the synthesis rule has the form:
          \[
          \fAddPairIntroRuleNoLabel
          \]
          By induction on the premises, we have that:
          \[
           \Gamma  \vdash  \GRANULEnt{A}  \Rightarrow^+  \GRANULEnt{t_{{\mathrm{1}}}}  ;\,  \Delta_{{\mathrm{1}}}   \tag{ih1}
          \]
          \[
           \Gamma  \vdash  \GRANULEnt{B}  \Rightarrow^+  \GRANULEnt{t_{{\mathrm{2}}}}  ;\,  \Delta_{{\mathrm{2}}}  \tag{ih2}
          \]
          from case 3 of the lemma. From which, we can construct the following instantiation of the \addPairIntroName\ synthesis rule in the non-focusing calculus:
          \[
    \inferrule*[right=R$\otimes^{+}$]
    {\Gamma  \vdash  \GRANULEnt{A}  \Rightarrow^+  \GRANULEnt{t_{{\mathrm{1}}}}  ;\,  \Delta_{{\mathrm{1}}} \\ \Gamma  \vdash  \GRANULEnt{B}  \Rightarrow^+  \GRANULEnt{t_{{\mathrm{2}}}}  ;\,  \Delta_{{\mathrm{2}}}}
    {\Gamma  \vdash   \GRANULEnt{A}  \, \otimes \,  \GRANULEnt{B}   \Rightarrow^+   ( \GRANULEnt{t_{{\mathrm{1}}}} ,  \GRANULEnt{t_{{\mathrm{2}}}} )   ;\,  \Delta_{{\mathrm{1}}}  \GRANULEsym{+}  \Delta_{{\mathrm{2}}}}
          \]
        \item Case \addSumIntroLName\ and \addSumIntroRName\\
          In the case of the right synchronous rules for sum introduction, the synthesis rules have the form:
          \[
          \fAddSumIntroRuleL
          \]
          \[
          \fAddSumIntroRuleR
          \]
          By induction on the premises of the rules, we have that:
          \[
           \Gamma  \vdash  \GRANULEnt{A}  \Rightarrow^+  \GRANULEnt{t}  ;\,  \Delta   \tag{ih1}
          \]
          \[
           \Gamma  \vdash  \GRANULEnt{B}  \Rightarrow^+  \GRANULEnt{t}  ;\,  \Delta  \tag{ih2}
          \]
          from case 3 of the lemma. From which, we can construct the following instantiations of the \addSumIntroLName\ and \addSumIntroRName\ synthesis rules in the non-focusing calculus, respectively:
          \[
    \inferrule*[right=R$\oplus_{1}^{+}$]
    {\Gamma  \vdash  \GRANULEnt{A}  \Rightarrow^+  \GRANULEnt{t}  ;\,  \Delta}
    {\Gamma  \vdash   \GRANULEnt{A}  \, \oplus \,  \GRANULEnt{B}   \Rightarrow^+  \GRANULEkw{inl} \, \GRANULEnt{t}  ;\,  \Delta}
          \]
          \[
    \inferrule*[right=R$\oplus_{2}^{+}$]
    {\Gamma  \vdash  \GRANULEnt{B}  \Rightarrow^+  \GRANULEnt{t}  ;\,  \Delta}
    {\Gamma  \vdash   \GRANULEnt{A}  \, \oplus \,  \GRANULEnt{B}   \Rightarrow^+  \GRANULEkw{inr} \, \GRANULEnt{t}  ;\,  \Delta}
          \]
        \item Case \addUnitIntroName \\
          In the case of the right synchronous rule for unit introduction, the synthesis rule has the form:
          \[
          \fAddUnitIntroRule
          \]
          From which, we can construct the following instantiation of the \addUnitIntroName\  synthesis rule in the non-focusing calculus:
          \[
    \inferrule*[right=R1$^{+}$]
    {\quad}
    {\Gamma  \vdash   \mathsf{1}   \Rightarrow^+  \GRANULEsym{()}  ;\,   \emptyset}
          \]
        \item Case \addBoxName \\
          In the case of the right synchronous rule for graded modality introduction, the synthesis rule has the form:
          \[
          \fAddBoxRule
          \]
          By induction on the premise, we have that:
          \[
           \Gamma  \vdash  \GRANULEnt{A}  \Rightarrow^+  \GRANULEnt{t}  ;\,  \Delta   \tag{ih}
          \]
          from case 1 of the lemma. From which, we can construct the following instantiation of the \addBoxName\ synthesis rule in the non-focusing calculus:
          \[
    \inferrule*[right=R$\square^{+}$]
    {\Gamma  \vdash  \GRANULEnt{A}  \Rightarrow^+  \GRANULEnt{t}  ;\,  \Delta}{\Gamma  \vdash   \Box_{  \GRANULEnt{r}  }  \GRANULEnt{A}   \Rightarrow^+  \GRANULEsym{[}  \GRANULEnt{t}  \GRANULEsym{]}  ;\,   \textcolor{coeffectColor}{ \GRANULEnt{r}   \textcolor{coeffectColor}{\,\cdot\,} }  \Delta}
          \]
      \item Case \fAddRSyncTransitionName \\
          In the case of the right synchronous rule for transitioning back to an asynchronous judgement, the synthesis rule has the form:
          \[
            \fAddRSyncTransitionRule
          \]
          By induction on the premise, we have that:
          \[
            \Gamma  \vdash  \GRANULEnt{A}  \Rightarrow^+  \GRANULEnt{t}  ;\,  \Delta \tag{ih}
          \]
          from case 1 of the lemma.
      \end{enumerate}
    \item Case 4. Left Sync \\
      \begin{enumerate}
          \item Case \addAppName \\
          In the case of the left synchronous rule for application, the synthesis rule has the form:
          \[
          \fAddAppRuleNoLabel
          \]
          By induction on the first premise, we have that:
          \[
            \Gamma  \GRANULEsym{,}   \GRANULEmv{x_{{\mathrm{2}}}}  :  \GRANULEnt{B}   \vdash  \GRANULEnt{C}  \Rightarrow^+  \GRANULEnt{t_{{\mathrm{1}}}}  ;\,  \Delta_{{\mathrm{1}}}  \GRANULEsym{,}   \GRANULEmv{x_{{\mathrm{2}}}}  :  \GRANULEnt{B} \tag{ih1}
          \]
          from case 4 of the lemma. By induction on the second premise, we have that:
          \[
            \Gamma  \vdash  \GRANULEnt{A}  \Rightarrow^+  \GRANULEnt{t_{{\mathrm{2}}}}  ;\,  \Delta_{{\mathrm{2}}} \tag{ih2}
          \]
          from case 3 of the lemma. From which, we can construct the following instantiation of the \addAppName synthesis rule in the non-focusing calculus:
          \[
    \inferrule*[right=L$\multimap^{+}$]
    {\Gamma  \GRANULEsym{,}   \GRANULEmv{x_{{\mathrm{2}}}}  :  \GRANULEnt{B}   \vdash  \GRANULEnt{C}  \Rightarrow^+  \GRANULEnt{t_{{\mathrm{1}}}}  ;\,  \Delta_{{\mathrm{1}}}  \GRANULEsym{,}   \GRANULEmv{x_{{\mathrm{2}}}}  :  \GRANULEnt{B} \\ \Gamma  \vdash  \GRANULEnt{A}  \Rightarrow^+  \GRANULEnt{t_{{\mathrm{2}}}}  ;\,  \Delta_{{\mathrm{2}}}}{\Gamma  \GRANULEsym{,}   \GRANULEmv{x_{{\mathrm{1}}}}  :   \GRANULEnt{A}  \multimap  \GRANULEnt{B}    \vdash  \GRANULEnt{C}  \Rightarrow^+   [  \GRANULEsym{(}  \GRANULEmv{x_{{\mathrm{1}}}} \, \GRANULEnt{t_{{\mathrm{2}}}}  \GRANULEsym{)}  /  \GRANULEmv{x_{{\mathrm{2}}}}  ]  \GRANULEnt{t_{{\mathrm{1}}}}   ;\,  \GRANULEsym{(}  \Delta_{{\mathrm{1}}}  \GRANULEsym{+}  \Delta_{{\mathrm{2}}}  \GRANULEsym{)}  \GRANULEsym{,}   \GRANULEmv{x_{{\mathrm{1}}}}  :   \GRANULEnt{A}  \multimap  \GRANULEnt{B} }
          \]
          \item Case \addLinVarName \\
          In the case of the left synchronous rule for linear variable synthesis, the synthesis rule has the form:
          \[
          \fAddLinVarRule
          \]
          From which, we can construct the following instantiation of the \addLinVarName\ in the non-focusing calculus:
          \[
    \inferrule*[right=LinVar$^{+}$]
    {\quad}
    {\Gamma  \GRANULEsym{,}   \GRANULEmv{x}  :  \GRANULEnt{A}   \vdash  \GRANULEnt{A}  \Rightarrow^+  \GRANULEmv{x}  ;\,   \GRANULEmv{x}  :  \GRANULEnt{A}}
          \]
          \item Case \addGrVarName \\
          In the case of the left synchronous rule for graded variable synthesis, the synthesis rule has the form:
          \[
          \fAddGrVarRule
          \]
          From which, we can construct the following instantiation of the \addGrVarName\ synthesis rule in the non-focusing calculus:
          \[
    \inferrule*[right=GrVar$^{+}$]
    {\quad}
      {\Gamma  \GRANULEsym{,}   \GRANULEmv{x}  :_{\textcolor{coeffectColor}{  \GRANULEnt{r}  } }   \GRANULEnt{A}   \vdash  \GRANULEnt{A}  \Rightarrow^+  \GRANULEmv{x}  ;\,   \GRANULEmv{x}  :_{\textcolor{coeffectColor}{   1   } }   \GRANULEnt{A} }
          \]
      \item Case \fAddLSyncTransitionName \\
          In the case of the left synchronous rule for transitioning back to an asynchronous judgement, the synthesis rule has the form:
          \[
            \fAddLSyncTransitionRule
          \]
          By induction on the premise, we have that:
          \[
            \Gamma  \GRANULEsym{,}   \GRANULEmv{x}  :  \GRANULEnt{A}   \vdash  \GRANULEnt{C}  \Rightarrow^+  \GRANULEnt{t}  ;\,  \Delta \tag{ih}
          \]
          from case 2 of the lemma.
      \end{enumerate}
    \item Case 5. Focus Right: \fAddFocusRName \\
          In the case of the focusing rule for transitioning from a left asynchronous judgement to a right synchronous judgement, the synthesis rule has the form:
          \[
            \fAddFocusRRuleNoLabel
          \]
          By induction on the first premise, we have that:
          \[
            \Gamma  \vdash  \GRANULEnt{C}  \Rightarrow^+  \GRANULEnt{t}  ;\,  \Delta \tag{ih}
          \]
          from case 2 of the lemma.
    \item Case 6. Focus Left: \fAddFocusLName \\
          In the case of the focusing rule for transitioning from a left asynchronous judgement to a left synchronous judgement, the synthesis rule has the form:
          \[
            \fAddFocusLRule
          \]
          By induction on the first premise, we have that:
          \[
            \Gamma  \GRANULEsym{,}   \GRANULEmv{x}  :  \GRANULEnt{A}   \vdash  \GRANULEnt{C}  \Rightarrow^+  \GRANULEnt{t}  ;\,  \Delta \tag{ih}
          \]
          from case 2 of the lemma.

  \end{enumerate}
\end{proof}

%\focusSoundAddPruning*
\begin{restatable}[Soundness of focusing for additive pruning synthesis]{lemma}{focusSoundAddPruning}
For all contexts $\Gamma$, $\Omega$ and types $\GRANULEnt{A}$
then:
\begin{align*}
\begin{array}{lll}
 1.\ Right\ Async: & \Gamma  ;  \Omega  \vdash   \GRANULEnt{A}  \Uparrow\   \Rightarrow^{+}  \GRANULEnt{t}  \ | \  \Delta \quad &\implies \quad \Gamma  \GRANULEsym{,}  \,  \GRANULEsym{,}  \Omega  \vdash  \GRANULEnt{A}  \Rightarrow^+  \GRANULEnt{t}  ;\,  \Delta\\
 2.\ Left\ Async: & \Gamma  ;   \Omega  \Uparrow\   \vdash  \GRANULEnt{C}  \Rightarrow^{+}  \GRANULEnt{t}  \ | \  \Delta \quad &\implies \quad \Gamma  \GRANULEsym{,}  \,  \GRANULEsym{,}  \Omega  \vdash  \GRANULEnt{C}  \Rightarrow^+  \GRANULEnt{t}  ;\,  \Delta\\
 3.\ Right\ Sync: & \Gamma  ;   \emptyset   \vdash   \GRANULEnt{A}  \Downarrow\   \Rightarrow^{+}  \GRANULEnt{t}  \ | \  \Delta \quad &\implies \quad \Gamma  \vdash  \GRANULEnt{A}  \Rightarrow^+  \GRANULEnt{t}  ;\,  \Delta\\
 4.\ Left\ Sync: & \Gamma  ;     \GRANULEmv{x}  :  \GRANULEnt{A}    \Downarrow\   \vdash  \GRANULEnt{C}  \Rightarrow^{+}  \GRANULEnt{t}  \ | \  \Delta \quad &\implies \quad \Gamma  \GRANULEsym{,}   \GRANULEmv{x}  :  \GRANULEnt{A}   \vdash  \GRANULEnt{C}  \Rightarrow^+  \GRANULEnt{t}  ;\,  \Delta\\
 5.\ Focus\ Right: & \Gamma  ;   \Omega  \Uparrow\   \vdash  \GRANULEnt{C}  \Rightarrow^{+}  \GRANULEnt{t}  \ | \  \Delta \quad &\implies \quad \Gamma  \vdash  \GRANULEnt{C}  \Rightarrow^+  \GRANULEnt{t}  ;\,  \Delta\\
 6.\ Focus\ Left: & \Gamma  \GRANULEsym{,}   \GRANULEmv{x}  :  \GRANULEnt{A}   ;   \Omega  \Uparrow\   \vdash  \GRANULEnt{C}  \Rightarrow^{+}  \GRANULEnt{t}  \ | \  \Delta \quad &\implies \quad \Gamma  \vdash  \GRANULEnt{C}  \Rightarrow^+  \GRANULEnt{t}  ;\,  \Delta
\end{array}
\end{align*}
\end{restatable}
\begin{proof}
  \begin{enumerate}
      \item Case: 1. Right Async: The proofs for right asynchronous rules are equivalent to those of lemma  \eqref{lemma:fAddSynthSound}\\
    \item Case 2. Left Async: The proofs for left asynchronous rules are equivalent to those of lemma \eqref{lemma:fAddSynthSound}\\
    \item Case 3. Right Sync: The proofs for right synchronous rules are equivalent to those of lemma \eqref{lemma:fAddSynthSound}, except for the case of the \addPruningPairIntroName rule:\\
      \begin{enumerate}
        \item Case \addPruningPairIntroName \\
          In the case of the right synchronous rule for pair introduction, the synthesis rule has the form:
          \[
          \fAddAltPairIntroRule
          \]
          By induction on the premises, we have that:
          \[
           \Gamma  \vdash  \GRANULEnt{A}  \Rightarrow^+  \GRANULEnt{t_{{\mathrm{1}}}}  ;\,  \Delta_{{\mathrm{1}}}   \tag{ih1}
          \]
          \[
           \Gamma  \GRANULEsym{-}  \Delta_{{\mathrm{1}}}  \vdash  \GRANULEnt{B}  \Rightarrow^+  \GRANULEnt{t_{{\mathrm{2}}}}  ;\,  \Delta_{{\mathrm{2}}}   \tag{ih2}
          \]
          from case 3 of the lemma. From which, we can construct the following instantiation of the \addPruningPairIntroName\ synthesis rule in the non-focusing calculus:
          \[
    \inferrule*[right=R$^{\prime}{\otimes^{+}}$]
    {\Gamma  \vdash  \GRANULEnt{A}  \Rightarrow^+  \GRANULEnt{t_{{\mathrm{1}}}}  ;\,  \Delta_{{\mathrm{1}}} \\ \Gamma  \GRANULEsym{-}  \Delta_{{\mathrm{1}}}  \vdash  \GRANULEnt{B}  \Rightarrow^+  \GRANULEnt{t_{{\mathrm{2}}}}  ;\,  \Delta_{{\mathrm{2}}}}
    {\Gamma  \vdash   \GRANULEnt{A}  \, \otimes \,  \GRANULEnt{B}   \Rightarrow^+   ( \GRANULEnt{t_{{\mathrm{1}}}} ,  \GRANULEnt{t_{{\mathrm{2}}}} )   ;\,  \Delta_{{\mathrm{1}}}  \GRANULEsym{+}  \Delta_{{\mathrm{2}}}}
          \]
      \end{enumerate}
    \item Case 4. Left Sync: The proofs for left synchronous rules are equivalent to those of lemma  \eqref{lemma:fAddSynthSound}, except for the case of the \addPruningAppName\  rule:\\\\
      \begin{enumerate}
          \item Case \addPruningAppName \\
          In the case of the left synchronous rule for application, the synthesis rule has the form:
          \[
          \fAddAltAppRule
          \]
          By induction on the first premise, we have that:
          \[
            \Gamma  \GRANULEsym{,}   \GRANULEmv{x_{{\mathrm{2}}}}  :  \GRANULEnt{B}   \vdash  \GRANULEnt{C}  \Rightarrow^+  \GRANULEnt{t_{{\mathrm{1}}}}  ;\,  \Delta_{{\mathrm{1}}}  \GRANULEsym{,}   \GRANULEmv{x_{{\mathrm{2}}}}  :  \GRANULEnt{B} \tag{ih1}
          \]
          from case 4 of the lemma. By induction on the second premise, we have that:
          \[
            \Gamma  \vdash  \GRANULEnt{A}  \Rightarrow^+  \GRANULEnt{t_{{\mathrm{2}}}}  ;\,  \Delta_{{\mathrm{2}}} \tag{ih2}
          \]
          from case 3 of the lemma. From which, we can construct the following instantiation of the \addPruningAppName\ synthesis rule in the non-focusing calculus:
          \[
\inferrule*[right=L$^{\prime}\multimap^{+}$]
    {\Gamma  \GRANULEsym{,}   \GRANULEmv{x_{{\mathrm{2}}}}  :  \GRANULEnt{B}   \vdash  \GRANULEnt{C}  \Rightarrow^+  \GRANULEnt{t_{{\mathrm{1}}}}  ;\,  \Delta_{{\mathrm{1}}}  \GRANULEsym{,}   \GRANULEmv{x_{{\mathrm{2}}}}  :  \GRANULEnt{B} \\ \Gamma  \GRANULEsym{-}  \Delta_{{\mathrm{1}}}  \vdash  \GRANULEnt{A}  \Rightarrow^+  \GRANULEnt{t_{{\mathrm{2}}}}  ;\,  \Delta_{{\mathrm{2}}}}{\Gamma  \GRANULEsym{,}   \GRANULEmv{x_{{\mathrm{1}}}}  :   \GRANULEnt{A}  \multimap  \GRANULEnt{B}    \vdash  \GRANULEnt{C}  \Rightarrow^+   [  \GRANULEsym{(}  \GRANULEmv{x_{{\mathrm{1}}}} \, \GRANULEnt{t_{{\mathrm{2}}}}  \GRANULEsym{)}  /  \GRANULEmv{x_{{\mathrm{2}}}}  ]  \GRANULEnt{t_{{\mathrm{1}}}}   ;\,  \GRANULEsym{(}  \Delta_{{\mathrm{1}}}  \GRANULEsym{+}  \Delta_{{\mathrm{2}}}  \GRANULEsym{)}  \GRANULEsym{,}   \GRANULEmv{x_{{\mathrm{1}}}}  :   \GRANULEnt{A}  \multimap  \GRANULEnt{B} }
          \]
      \end{enumerate}
    \item Case 5. Right Focus: \fAddFocusRName\ - The proof for right focusing rule is equivalent to that of lemma \eqref{lemma:fAddSynthSound}\\
    \item Case 6. Left Focus: \fAddFocusLName\ - The proof for left focusing rule is equivalent to that of lemma \eqref{lemma:fAddSynthSound}\\
  \end{enumerate}
\end{proof}

% \synthSound*
\begin{proof}
Induction on the synthesis rules
\begin{enumerate}
\item Case $\textsc{Var}$ \\
        For synthesis of a variable term, we have the derivation:
        \begin{align*}
          \synVar
        \end{align*}
        from which we can construct the following typing derivation, matching the above conclusion:
        \begin{align*}
          \tyVar
        \end{align*}
\item Case $\multimap_{R}$\\
        For synthesis of an abstraction term, we have the derivation:
        \begin{align*}
          \synAbs
        \end{align*}
        By induction on the premise, we have:
        \begin{align*}
          \Delta  \GRANULEsym{,}   \GRANULEmv{x}  :_{\textcolor{coeffectColor}{  \GRANULEnt{r}  } }   \GRANULEnt{A}   \vdash  \GRANULEnt{t}  :  \GRANULEnt{B}
        \end{align*}
        and that:
        \begin{align*}
          \GRANULEnt{r}  \, \textcolor{coeffectColor}{\sqsubseteq} \,  \GRANULEnt{q}
        \end{align*}
        from which we can construct the following typing derivation, matching the conclusion:
        \begin{align*}
          \inferrule*[Right=Abs]
            {\inferrule*[Right=Approx]{\Delta  \GRANULEsym{,}   \GRANULEmv{x}  :_{\textcolor{coeffectColor}{  \GRANULEnt{r}  } }   \GRANULEnt{A}   \vdash  \GRANULEnt{t}  :  \GRANULEnt{B} \\ \GRANULEnt{r}  \, \textcolor{coeffectColor}{\sqsubseteq} \,  \GRANULEnt{q}}{\Delta  \GRANULEsym{,}   \GRANULEmv{x}  :_{\textcolor{coeffectColor}{  \GRANULEnt{q}  } }   \GRANULEnt{A}   \vdash  \GRANULEnt{t}  :  \GRANULEnt{B}}}
            {\Delta  \vdash   \lambda  \GRANULEmv{x}  .  \GRANULEnt{t}   :   \GRANULEnt{A} ^ \GRANULEnt{q}  \rightarrow  \GRANULEnt{B}}
        \end{align*}
\item Case $\multimap_{L}$\\
        For synthesising an application, we have the derivation:
        \begin{align*}
          \synApp
        \end{align*}
        By induction on the premises, we obtain the following typing judgements:
        \begin{align*}
          \Delta_{{\mathrm{1}}}  \GRANULEsym{,}   \GRANULEmv{x_{{\mathrm{1}}}}  :_{\textcolor{coeffectColor}{  \GRANULEnt{s_{{\mathrm{1}}}}  } }    \GRANULEnt{A} ^ \GRANULEnt{q}  \rightarrow  \GRANULEnt{B}    \GRANULEsym{,}   \GRANULEmv{x_{{\mathrm{2}}}}  :_{\textcolor{coeffectColor}{  \GRANULEnt{s_{{\mathrm{2}}}}  } }   \GRANULEnt{B}   \vdash  \GRANULEnt{t_{{\mathrm{1}}}}  :  \GRANULEnt{C} \\
          \Delta_{{\mathrm{2}}}  \GRANULEsym{,}   \GRANULEmv{x_{{\mathrm{1}}}}  :_{\textcolor{coeffectColor}{  \GRANULEnt{s_{{\mathrm{3}}}}  } }    \GRANULEnt{A} ^ \GRANULEnt{q}  \rightarrow  \GRANULEnt{B}    \vdash  \GRANULEnt{t_{{\mathrm{2}}}}  :  \GRANULEnt{A}
        \end{align*}
        from which we can construct the following derivation, making use of the admissibility of substitution:
        \begin{align*}
          \inferrule*[Right=Subst]{
          {\inferrule*[Right=App]{\inferrule*[Right=Var]{\quad}{\GRANULEmv{x_{{\mathrm{1}}}}  :_{\textcolor{coeffectColor}{   1   } }    \GRANULEnt{A} ^ \GRANULEnt{q}  \rightarrow  \GRANULEnt{B}    \vdash  \GRANULEmv{x_{{\mathrm{1}}}}  :   \GRANULEnt{A} ^ \GRANULEnt{q}  \rightarrow  \GRANULEnt{B}} \\ \Delta_{{\mathrm{2}}}  \GRANULEsym{,}   \GRANULEmv{x_{{\mathrm{1}}}}  :_{\textcolor{coeffectColor}{  \GRANULEnt{s_{{\mathrm{3}}}}  } }    \GRANULEnt{A} ^ \GRANULEnt{q}  \rightarrow  \GRANULEnt{B}    \vdash  \GRANULEnt{t_{{\mathrm{2}}}}  :  \GRANULEnt{A}}{ \textcolor{coeffectColor}{ \GRANULEnt{q}   \textcolor{coeffectColor}{\,\cdot\,} }  \Delta_{{\mathrm{2}}}    \GRANULEsym{,}   \GRANULEmv{x_{{\mathrm{1}}}}  :_{\textcolor{coeffectColor}{    1   \GRANULEsym{+}  \GRANULEsym{(}  \GRANULEnt{q}  \textcolor{coeffectColor}{\,\cdot\,}  \GRANULEnt{s_{{\mathrm{3}}}}  \GRANULEsym{)}   } }    \GRANULEnt{A} ^ \GRANULEnt{q}  \rightarrow  \GRANULEnt{B}    \vdash  \GRANULEmv{x_{{\mathrm{1}}}} \, \GRANULEnt{t_{{\mathrm{2}}}}  :  \GRANULEnt{B}}} \\ \Delta_{{\mathrm{1}}}  \GRANULEsym{,}   \GRANULEmv{x_{{\mathrm{1}}}}  :_{\textcolor{coeffectColor}{  \GRANULEnt{s_{{\mathrm{1}}}}  } }    \GRANULEnt{A} ^ \GRANULEnt{q}  \rightarrow  \GRANULEnt{B}    \GRANULEsym{,}   \GRANULEmv{x_{{\mathrm{2}}}}  :_{\textcolor{coeffectColor}{  \GRANULEnt{s_{{\mathrm{2}}}}  } }   \GRANULEnt{B}   \vdash  \GRANULEnt{t_{{\mathrm{1}}}}  :  \GRANULEnt{C}}
          {\GRANULEsym{(}  \Delta_{{\mathrm{1}}}  \GRANULEsym{+}   \textcolor{coeffectColor}{ \GRANULEnt{s_{{\mathrm{2}}}}   \textcolor{coeffectColor}{\,\cdot\,} }    \textcolor{coeffectColor}{ \GRANULEnt{q}   \textcolor{coeffectColor}{\,\cdot\,} }  \Delta_{{\mathrm{2}}}     \GRANULEsym{)}  \GRANULEsym{,}   \GRANULEmv{x_{{\mathrm{1}}}}  :_{\textcolor{coeffectColor}{   \GRANULEnt{s_{{\mathrm{2}}}}  \GRANULEsym{+}   \GRANULEnt{s_{{\mathrm{1}}}}  \GRANULEsym{+}  \GRANULEsym{(}  \GRANULEnt{s_{{\mathrm{2}}}}  \textcolor{coeffectColor}{\,\cdot\,}   \GRANULEnt{q}  \textcolor{coeffectColor}{\,\cdot\,}  \GRANULEnt{s_{{\mathrm{3}}}}   \GRANULEsym{)}    } }    \GRANULEnt{A} ^ \GRANULEnt{q}  \rightarrow  \GRANULEnt{B}    \vdash    [  \GRANULEsym{(}  \GRANULEmv{x_{{\mathrm{1}}}} \, \GRANULEnt{t_{{\mathrm{2}}}}  \GRANULEsym{)}  /  \GRANULEmv{x_{{\mathrm{2}}}}  ]  \GRANULEnt{t_{{\mathrm{1}}}}    :  \GRANULEnt{C}}
        \end{align*}
        making use of the distributivity property of semirings, along with unitality of $1$
        and commutativity of $+$, such that
        $\GRANULEnt{s_{{\mathrm{1}}}}  \GRANULEsym{+}  \GRANULEnt{s_{{\mathrm{2}}}}  \textcolor{coeffectColor}{\,\cdot\,}  \GRANULEsym{(}   1   \GRANULEsym{+}  \GRANULEsym{(}  \GRANULEnt{q}  \textcolor{coeffectColor}{\,\cdot\,}  \GRANULEnt{s_{{\mathrm{3}}}}  \GRANULEsym{)}  \GRANULEsym{)} =  \GRANULEnt{s_{{\mathrm{1}}}}  \GRANULEsym{+}  \GRANULEsym{(}  \GRANULEnt{s_{{\mathrm{2}}}}  \textcolor{coeffectColor}{\,\cdot\,}   1   \GRANULEsym{)}  \GRANULEsym{+}  \GRANULEsym{(}  \GRANULEnt{s_{{\mathrm{2}}}}  \textcolor{coeffectColor}{\,\cdot\,}  \GRANULEnt{q}  \textcolor{coeffectColor}{\,\cdot\,}  \GRANULEnt{s_{{\mathrm{3}}}}  \GRANULEsym{)} = \GRANULEnt{s_{{\mathrm{2}}}}  \GRANULEsym{+}  \GRANULEnt{s_{{\mathrm{1}}}}  \GRANULEsym{+}  \GRANULEsym{(}  \GRANULEnt{s_{{\mathrm{2}}}}  \textcolor{coeffectColor}{\,\cdot\,}  \GRANULEnt{q}  \textcolor{coeffectColor}{\,\cdot\,}  \GRANULEnt{s_{{\mathrm{3}}}}  \GRANULEsym{)}$.

\item Case $\textsc{Con}_{R}$\\
        For synthesising a constructor introduction, we have the derivation:
        \begin{align*}
          \synCon
        \end{align*}
        By induction on the premises, we obtain the following typing judgements:
        \begin{align*}
          \Delta_{{\mathrm{1}}}  \vdash  \GRANULEnt{t_{{\mathrm{1}}}}  :  \GRANULEnt{B_{{\mathrm{1}}}} \quad , ..., \quad \Delta_{\GRANULEmv{n}}  \vdash  \GRANULEnt{t_{\GRANULEmv{n}}}  :  \GRANULEnt{B_{\GRANULEmv{n}}}
        \end{align*}
        from which we can construct the following derivation, matching the above conclusion:
        \begin{align*}
          \inferrule*[Right=App]{\inferrule*[Right=App]{\inferrule*[right=App, vdots=1em]{\inferrule*[Right=Con]{(  C  :  \GRANULEnt{B_{{\mathrm{1}}}} ^ \GRANULEnt{q_{{\mathrm{1}}}}  \rightarrow ... \rightarrow  \GRANULEnt{B_{\GRANULEmv{n}}} ^ \GRANULEnt{q_{\GRANULEmv{n}}}  \rightarrow     K  \, \GRANULEnt{A}   ) \in  D}{\textcolor{coeffectColor}{  0    \textcolor{coeffectColor}{\,\cdot\,} }  \Gamma   \vdash   C   :   \GRANULEnt{B_{{\mathrm{1}}}} ^{q_1}  \rightarrow \! ... \! \rightarrow   \GRANULEnt{B_{\GRANULEmv{n}}} ^{q_n}  \rightarrow     K  \, \GRANULEnt{A}} \\ \Delta_{{\mathrm{1}}}  \vdash  \GRANULEnt{t_{{\mathrm{1}}}}  :  \GRANULEnt{B_{{\mathrm{1}}}}}{{\textcolor{coeffectColor}{  0    \textcolor{coeffectColor}{\,\cdot\,} }  \Gamma    \GRANULEsym{+}   \textcolor{coeffectColor}{ \GRANULEnt{q_{{\mathrm{1}}}}   \textcolor{coeffectColor}{\,\cdot\,} }  \Delta_{{\mathrm{1}}}   \vdash  C \, \GRANULEnt{t_{{\mathrm{1}}}}  :    \GRANULEnt{B_{{\mathrm{2}}}} ^{q_1}  \rightarrow \! ... \! \rightarrow   \GRANULEnt{B_{\GRANULEmv{n}}} ^{q_n}  \rightarrow     K  \, \GRANULEnt{A}}}}{\textcolor{coeffectColor}{  0    \textcolor{coeffectColor}{\,\cdot\,} }  \Gamma    \GRANULEsym{+}   \textcolor{coeffectColor}{ \GRANULEnt{q_{{\mathrm{1}}}}   \textcolor{coeffectColor}{\,\cdot\,} }  \Delta_{{\mathrm{1}}}   \GRANULEsym{+} \, ... \, \GRANULEsym{+}   \textcolor{coeffectColor}{ \GRANULEnt{q_{{\GRANULEmv{n}-1}}}   \textcolor{coeffectColor}{\,\cdot\,} }  \Delta_{{\GRANULEmv{n}-1}}   \vdash  C \, \GRANULEnt{t_{{\mathrm{1}}}} \, ... \, \GRANULEnt{t_{{\GRANULEmv{n}-1}}}  :   \GRANULEnt{B_{\GRANULEmv{n}}} ^ \GRANULEnt{q_{\GRANULEmv{n}}}  \rightarrow    K  \, \GRANULEnt{A}} \\ \Delta_{\GRANULEmv{n}}  \vdash  \GRANULEnt{t_{\GRANULEmv{n}}}  :  \GRANULEnt{B_{\GRANULEmv{n}}}}{{\textcolor{coeffectColor}{  0    \textcolor{coeffectColor}{\,\cdot\,} }  \Gamma    \GRANULEsym{+}   \textcolor{coeffectColor}{ \GRANULEnt{q_{{\mathrm{1}}}}   \textcolor{coeffectColor}{\,\cdot\,} }  \Delta_{{\mathrm{1}}}+ ...+  \textcolor{coeffectColor}{ \GRANULEnt{q_{\GRANULEmv{n}}}   \textcolor{coeffectColor}{\,\cdot\,} }  \Delta_{\GRANULEmv{n}}   \vdash  C \, \GRANULEnt{t_{{\mathrm{1}}}} \, ... \, \GRANULEnt{t_{\GRANULEmv{n}}}  :   K  \, \GRANULEnt{A}}}
        \end{align*}

\item Case $\textsc{Con}_{L}$ \\
        For synthesising a case statement, we have the derivation:
        \begin{align*}
          \synCaseAlt
        \end{align*}
        By induction on the premises we obtain the following typing judgements:
        \begin{align*}
          \Delta_{\GRANULEmv{i}}  \GRANULEsym{,}   \GRANULEmv{x}  :_{\textcolor{coeffectColor}{   \GRANULEnt{r_{\GRANULEmv{i}}}   } }    K  \, \GRANULEnt{A}    \GRANULEsym{,}   {  \GRANULEmv{y} ^  i  _  1   }:_{\textcolor{coeffectColor}{    \GRANULEnt{s} ^  i  _  1     } }   \GRANULEnt{B_{{\mathrm{1}}}}    , ... ,   {  \GRANULEmv{y} ^  i  _  n   }:_{\textcolor{coeffectColor}{    \GRANULEnt{s} ^  i  _  n     } }   \GRANULEnt{B_{\GRANULEmv{n}}}    \vdash  \GRANULEnt{t_{\GRANULEmv{i}}}  :  \GRANULEnt{B}
        \end{align*}
        % By monotonicty of $\sqcup$ we have that $ \GRANULEnt{s} _  i     \, \textcolor{coeffectColor}{\sqsubseteq} \,      \GRANULEnt{s} _  1     \sqcup ... \sqcup    \GRANULEnt{s} _  n$
        % and by $*_{1}$ that $\GRANULEnt{s} _  i     =      \GRANULEnt{s} ^  i  _  1     \sqcup ... \sqcup    \GRANULEnt{s} ^  i  _  n$. Therefore, by monotonicty of $\sqcup$
        % we have that $\GRANULEnt{s} ^  i  _  j     \, \textcolor{coeffectColor}{\sqsubseteq} \,      \GRANULEnt{s} _  1     \sqcup ... \sqcup    \GRANULEnt{s} _  n$
        % from which we can construct the following
        % derivation, matching the above conclusion:
        We have by the definition of $\sqcup$:
        %
        \begin{enumerate}
                \item $ \Delta_{\GRANULEmv{i}} \sqsubseteq \GRANULEsym{(}   \Delta_{{\mathrm{1}}}  \sqcup ...  \sqcup  \Delta_{\GRANULEmv{m}}   \GRANULEsym{)}$
                \item $ \GRANULEnt{r} _  i     \, \textcolor{coeffectColor}{\sqsubseteq} \,      \GRANULEnt{r} _  1     \sqcup ... \sqcup    \GRANULEnt{r} _  m$
                % \item $\GRANULEnt{s} _  i     =      \GRANULEnt{s} ^  i  _  1     \sqcup ... \sqcup    \GRANULEnt{s} ^  i  _  n$
                % \item $\GRANULEnt{s} ^  i  _  j     \, \textcolor{coeffectColor}{\sqsubseteq} \,      \GRANULEnt{s} _  1     \sqcup ... \sqcup    \GRANULEnt{s} _  n$
        \end{enumerate}

        and from the premises of the synthesis rule:
        \begin{enumerate}[resume]
                \item $\GRANULEnt{s'} ^  i  _  j     \, \textcolor{coeffectColor}{\sqsubseteq} \,      \GRANULEnt{s} _  1     \sqcup ... \sqcup    \GRANULEnt{s} _  m$
                \item $ \GRANULEnt{s} ^  i  _  j     \, \textcolor{coeffectColor}{\sqsubseteq} \,    \GRANULEnt{s'} ^  i  _  j     \textcolor{coeffectColor}{\,\cdot\,}    \GRANULEnt{q} ^  i  _  j$
                \item $ |    K  \, \GRANULEnt{A}   | > 1 \Rightarrow 1 \sqsubseteq      \GRANULEnt{s} _  1     \sqcup ... \sqcup    \GRANULEnt{s} _  m $
        \end{enumerate}

        We then construct the following two derivations towards the goal:
        %
        \begin{equation}
          \label{soundConLSub1}
          \inferrule*[Right=PCon]{\inferrule*[Right=PVar]{\quad}{ \GRANULEnt{q} ^  i  _  j     \textcolor{coeffectColor}{\,\cdot\,}      \GRANULEnt{s} _  1     \sqcup ... \sqcup    \GRANULEnt{s} _  m       \vdash \,   \GRANULEmv{y} ^i_j   :  \GRANULEnt{B_{\GRANULEmv{j}}}  \, \rhd \,   {  \GRANULEmv{y} ^  i  _  j   }:_{\textcolor{coeffectColor}{    \GRANULEnt{q} ^  i  _  j     \textcolor{coeffectColor}{\,\cdot\,}      \GRANULEnt{s} _  1     \sqcup ... \sqcup    \GRANULEnt{s} _  m       } }   \GRANULEnt{B_{\GRANULEmv{j}}} } \\  (e) }{\GRANULEnt{s} _  1     \sqcup ... \sqcup    \GRANULEnt{s} _  m       \vdash \,   C_{\GRANULEmv{i}}  \  \GRANULEmv{y} ^  i  _  1   ...  \GRANULEmv{y} ^  i  _  n    :   K  \, \GRANULEnt{A}  \, \rhd \,    {  \GRANULEmv{y} ^  i  _  j   }:_{\textcolor{coeffectColor}{    \GRANULEnt{q} ^  i  _  j     \textcolor{coeffectColor}{\,\cdot\,}      \GRANULEnt{s} _  1     \sqcup ... \sqcup    \GRANULEnt{s} _  m       } }   \GRANULEnt{B_{\GRANULEmv{j}}}   , ... ,   {  \GRANULEmv{y} ^  i  _  n   }:_{\textcolor{coeffectColor}{    \GRANULEnt{q} ^  i  _  n     \textcolor{coeffectColor}{\,\cdot\,}      \GRANULEnt{s} _  1     \sqcup ... \sqcup    \GRANULEnt{s} _  m       } }   \GRANULEnt{B_{\GRANULEmv{n}}}}\\
        \end{equation}
        and
        \begin{equation}
          \label{soundConLSub2}
          \hspace{-7em}\inferrule*[Right=Approx, rightskip=5em]{ \inferrule*[Right=Approx]{ \inferrule*[Right=induction]{\quad}{\Delta_{\GRANULEmv{i}}  \GRANULEsym{,}   \GRANULEmv{x}  :_{\textcolor{coeffectColor}{   \GRANULEnt{r_{\GRANULEmv{i}}}   } }    K  \, \GRANULEnt{A}    \GRANULEsym{,}   {  \GRANULEmv{y} ^  i  _  1   }:_{\textcolor{coeffectColor}{    \GRANULEnt{s} ^  i  _  1     } }   \GRANULEnt{B_{{\mathrm{1}}}}    , ... ,   {  \GRANULEmv{y} ^  i  _  n   }:_{\textcolor{coeffectColor}{    \GRANULEnt{s} ^  i  _  n     } }   \GRANULEnt{B_{\GRANULEmv{n}}}    \vdash  \GRANULEnt{t_{\GRANULEmv{i}}}  :  \GRANULEnt{B}} \qquad \\ (d) }{\inferrule*[Right=Approx]{ \Delta_{\GRANULEmv{i}}  \GRANULEsym{,}   \GRANULEmv{x}  :_{\textcolor{coeffectColor}{    \GRANULEnt{r} _  i     } }     K  \, \GRANULEnt{A}    \GRANULEsym{,}   {  \GRANULEmv{y} ^  i  _  1   }:_{\textcolor{coeffectColor}{    \GRANULEnt{q} ^  i  _  1     \textcolor{coeffectColor}{\,\cdot\,}    \GRANULEnt{s'} ^  i  _  j     } }   \GRANULEnt{B_{{\mathrm{1}}}}    , ... ,   {  \GRANULEmv{y} ^  i  _  n   }:_{\textcolor{coeffectColor}{    \GRANULEnt{q} ^  i  _  n     \textcolor{coeffectColor}{\,\cdot\,}    \GRANULEnt{s'} ^  i  _  j     } }   \GRANULEnt{B_{\GRANULEmv{n}}}    \vdash  \GRANULEnt{t_{\GRANULEmv{i}}}  :  \GRANULEnt{B} \\ (c)}{\inferrule*[Right=Approx]{ \Delta_{\GRANULEmv{i}}  \GRANULEsym{,}   \GRANULEmv{x}  :_{\textcolor{coeffectColor}{    \GRANULEnt{r} _  i     } }     K  \, \GRANULEnt{A}    \GRANULEsym{,}   {  \GRANULEmv{y} ^  i  _  1   }:_{\textcolor{coeffectColor}{    \GRANULEnt{q} ^  i  _  1     \textcolor{coeffectColor}{\,\cdot\,}      \GRANULEnt{s} _  1     \sqcup ... \sqcup    \GRANULEnt{s} _  m       } }   \GRANULEnt{B_{{\mathrm{1}}}}    , ... ,   {  \GRANULEmv{y} ^  i  _  n   }:_{\textcolor{coeffectColor}{    \GRANULEnt{q} ^  i  _  n     \textcolor{coeffectColor}{\,\cdot\,}      \GRANULEnt{s} _  1     \sqcup ... \sqcup    \GRANULEnt{s} _  m       } }   \GRANULEnt{B_{\GRANULEmv{n}}}    \vdash  \GRANULEnt{t_{\GRANULEmv{i}}}  :  \GRANULEnt{B} \\ (b)}{\Delta_{\GRANULEmv{i}}  \GRANULEsym{,}   \GRANULEmv{x}  :_{\textcolor{coeffectColor}{   \GRANULEsym{(}     \GRANULEnt{r} _  1     \sqcup ... \sqcup    \GRANULEnt{r} _  m      \GRANULEsym{)}   } }     K  \, \GRANULEnt{A}    \GRANULEsym{,}   {  \GRANULEmv{y} ^  i  _  1   }:_{\textcolor{coeffectColor}{    \GRANULEnt{q} ^  i  _  1     \textcolor{coeffectColor}{\,\cdot\,}      \GRANULEnt{s} _  1     \sqcup ... \sqcup    \GRANULEnt{s} _  m       } }   \GRANULEnt{B_{{\mathrm{1}}}}    , ... ,   {  \GRANULEmv{y} ^  i  _  n   }:_{\textcolor{coeffectColor}{    \GRANULEnt{q} ^  i  _  n     \textcolor{coeffectColor}{\,\cdot\,}      \GRANULEnt{s} _  1     \sqcup ... \sqcup    \GRANULEnt{s} _  m       } }   \GRANULEnt{B_{\GRANULEmv{n}}}    \vdash  \GRANULEnt{t_{\GRANULEmv{i}}}  :  \GRANULEnt{B} \\ (a)} }}}{ \GRANULEsym{(}   \Delta_{{\mathrm{1}}}  \sqcup ...  \sqcup  \Delta_{\GRANULEmv{m}}   \GRANULEsym{)}  \GRANULEsym{,}   \GRANULEmv{x}  :_{\textcolor{coeffectColor}{   \GRANULEsym{(}     \GRANULEnt{r} _  1     \sqcup ... \sqcup    \GRANULEnt{r} _  m      \GRANULEsym{)}   } }     K  \, \GRANULEnt{A}    \GRANULEsym{,}   {  \GRANULEmv{y} ^  i  _  1   }:_{\textcolor{coeffectColor}{    \GRANULEnt{q} ^  i  _  1     \textcolor{coeffectColor}{\,\cdot\,}      \GRANULEnt{s} _  1     \sqcup ... \sqcup    \GRANULEnt{s} _  m       } }   \GRANULEnt{B_{{\mathrm{1}}}}    , ... ,   {  \GRANULEmv{y} ^  i  _  n   }:_{\textcolor{coeffectColor}{    \GRANULEnt{q} ^  i  _  n     \textcolor{coeffectColor}{\,\cdot\,}      \GRANULEnt{s} _  1     \sqcup ... \sqcup    \GRANULEnt{s} _  m       } }   \GRANULEnt{B_{\GRANULEmv{n}}}    \vdash  \GRANULEnt{t_{\GRANULEmv{i}}}  :  \GRANULEnt{B} }
        \end{equation}


        \begin{align*}
          \inferrule*[Right=Case]{ \inferrule*[Right=Var,leftskip=5em]{\quad}{\GRANULEmv{x}  :_{\textcolor{coeffectColor}{   1   } }     K  \, \GRANULEnt{A}    \vdash  \GRANULEmv{x}  :    K  \, \GRANULEnt{A}} \quad \\ \eqref{soundConLSub1} \\ \eqref{soundConLSub2} }
          { \inferrule*[Right=$\equiv$] {\GRANULEsym{(}  \GRANULEsym{(}   \Delta_{{\mathrm{1}}}  \sqcup ...  \sqcup  \Delta_{\GRANULEmv{m}}   \GRANULEsym{)}  \GRANULEsym{,}    \GRANULEmv{x}  :_{\textcolor{coeffectColor}{   \GRANULEsym{(}     \GRANULEnt{r} _  1     \sqcup ... \sqcup    \GRANULEnt{r} _  m      \GRANULEsym{)}   } }     K  \, \GRANULEnt{A}     \GRANULEsym{)}  \GRANULEsym{+}   \GRANULEmv{x}  :_{\textcolor{coeffectColor}{   \GRANULEsym{(}      \GRANULEnt{s} _  1     \sqcup ... \sqcup    \GRANULEnt{s} _  m       \textcolor{coeffectColor}{\,\cdot\,}   1   \GRANULEsym{)}   } }     K  \, \GRANULEnt{A}    \vdash   \textbf{case} \  \GRANULEmv{x}  \ \textbf{of} \   \overline{   C_{\GRANULEmv{i}}  \  \GRANULEmv{y} ^  i  _  1   ...  \GRANULEmv{y} ^  i  _  n    \mapsto  \GRANULEnt{t_{\GRANULEmv{i}}} }    :  \GRANULEnt{B}} {\GRANULEsym{(}   \Delta_{{\mathrm{1}}}  \sqcup ...  \sqcup  \Delta_{\GRANULEmv{m}}   \GRANULEsym{)}  \GRANULEsym{,}    \GRANULEmv{x}  :_{\textcolor{coeffectColor}{   \GRANULEsym{(}     \GRANULEnt{r} _  1     \sqcup ... \sqcup    \GRANULEnt{r} _  m      \GRANULEsym{)}   \GRANULEsym{+}   \GRANULEsym{(}      \GRANULEnt{s} _  1     \sqcup ... \sqcup    \GRANULEnt{s} _  m       \GRANULEsym{)}   } }     K  \, \GRANULEnt{A}     \vdash   \textbf{case} \  \GRANULEmv{x}  \ \textbf{of} \   \overline{   C_{\GRANULEmv{i}}  \  \GRANULEmv{y} ^  i  _  1   ...  \GRANULEmv{y} ^  i  _  n    \mapsto  \GRANULEnt{t_{\GRANULEmv{i}}} }    :  \GRANULEnt{B}}}
        \end{align*}



% \item Case $\textsc{Con}_{L}$ \\
%         For synthesising a case statement, we have the derivation:
%         \begin{align*}
%           \synCaseAltAlt
%         \end{align*}
%         By induction on the premises we obtain the following typing judgements:
%         \begin{align*}
%           \Delta_{\GRANULEmv{i}}  \GRANULEsym{,}   \GRANULEmv{x}  :_{\textcolor{coeffectColor}{   \GRANULEnt{r_{\GRANULEmv{i}}}   } }    K  \, \GRANULEnt{A}    \GRANULEsym{,}   {  \GRANULEmv{y} ^  i  _  1   }:_{\textcolor{coeffectColor}{    \GRANULEnt{s} ^  i  _  1     } }   \GRANULEnt{B_{{\mathrm{1}}}}    , ... ,   {  \GRANULEmv{y} ^  i  _  n   }:_{\textcolor{coeffectColor}{    \GRANULEnt{s} ^  i  _  n     } }   \GRANULEnt{B_{\GRANULEmv{n}}}    \vdash  \GRANULEnt{t_{\GRANULEmv{i}}}  :  \GRANULEnt{B}
%         \end{align*}
%         % By monotonicty of $\sqcup$ we have that $ \GRANULEnt{s} _  i     \, \textcolor{coeffectColor}{\sqsubseteq} \,      \GRANULEnt{s} _  1     \sqcup ... \sqcup    \GRANULEnt{s} _  n$
%         % and by $*_{1}$ that $\GRANULEnt{s} _  i     =      \GRANULEnt{s} ^  i  _  1     \sqcup ... \sqcup    \GRANULEnt{s} ^  i  _  n$. Therefore, by monotonicty of $\sqcup$
%         % we have that $\GRANULEnt{s} ^  i  _  j     \, \textcolor{coeffectColor}{\sqsubseteq} \,      \GRANULEnt{s} _  1     \sqcup ... \sqcup    \GRANULEnt{s} _  n$
%         % from which we can construct the following
%         % derivation, matching the above conclusion:
%         We have:

%         \begin{enumerate}
%                 \item $ \Delta_{\GRANULEmv{i}} \sqsubseteq \GRANULEsym{(}   \Delta_{{\mathrm{1}}}  \sqcup ...  \sqcup  \Delta_{\GRANULEmv{n}}   \GRANULEsym{)}$
%                 \item $ \GRANULEnt{r} _  i     \, \textcolor{coeffectColor}{\sqsubseteq} \,      \GRANULEnt{r} _  1     \sqcup ... \sqcup    \GRANULEnt{r} _  n$
%                 % \item $\GRANULEnt{s} _  i     =      \GRANULEnt{s} ^  i  _  1     \sqcup ... \sqcup    \GRANULEnt{s} ^  i  _  n$
%                 % \item $\GRANULEnt{s} ^  i  _  j     \, \textcolor{coeffectColor}{\sqsubseteq} \,      \GRANULEnt{s} _  1     \sqcup ... \sqcup    \GRANULEnt{s} _  n$
%                 \item $\GRANULEnt{s'} _  i     \, \textcolor{coeffectColor}{\sqsubseteq} \,      \GRANULEnt{s'} _  1     \sqcup ... \sqcup    \GRANULEnt{s'} _  n$
%                 \item $ \GRANULEnt{s} _  i     \, \textcolor{coeffectColor}{\sqsubseteq} \,    \GRANULEnt{s'} _  i     \textcolor{coeffectColor}{\,\cdot\,}    \GRANULEnt{q} ^  i  _  j$
%                 \item $\GRANULEnt{s} ^  i  _  j     \, \textcolor{coeffectColor}{\sqsubseteq} \,    \GRANULEnt{s} _  i$
%         \end{enumerate}
%         \begin{equation}
%           \label{soundConLAltSub1}
%           \inferrule*[Right=PCon]{\inferrule*[Right=PVar]{\quad}{ \GRANULEnt{q} ^  i  _  j     \textcolor{coeffectColor}{\,\cdot\,}      \GRANULEnt{s'} _  1     \sqcup ... \sqcup    \GRANULEnt{s'} _  n       \vdash \,   \GRANULEmv{y} ^i_j   :  \GRANULEnt{B_{\GRANULEmv{j}}}  \, \rhd \,   {  \GRANULEmv{y} ^  i  _  j   }:_{\textcolor{coeffectColor}{    \GRANULEnt{q} ^  i  _  j     \textcolor{coeffectColor}{\,\cdot\,}      \GRANULEnt{s'} _  1     \sqcup ... \sqcup    \GRANULEnt{s'} _  n       } }   \GRANULEnt{B_{\GRANULEmv{j}}}  }}{\GRANULEnt{s'} _  1     \sqcup ... \sqcup    \GRANULEnt{s'} _  n       \vdash \,   C_{\GRANULEmv{i}}  \  \GRANULEmv{y} ^  i  _  1   ...  \GRANULEmv{y} ^  i  _  n    :   K  \, \GRANULEnt{A}  \, \rhd \,    {  \GRANULEmv{y} ^  i  _  1   }:_{\textcolor{coeffectColor}{    \GRANULEnt{q} ^  i  _  1     \textcolor{coeffectColor}{\,\cdot\,}      \GRANULEnt{s'} _  1     \sqcup ... \sqcup    \GRANULEnt{s'} _  n       } }   \GRANULEnt{B_{{\mathrm{1}}}}   , ... ,   {  \GRANULEmv{y} ^  i  _  n   }:_{\textcolor{coeffectColor}{    \GRANULEnt{q} ^  i  _  n     \textcolor{coeffectColor}{\,\cdot\,}      \GRANULEnt{s'} _  1     \sqcup ... \sqcup    \GRANULEnt{s'} _  n       } }   \GRANULEnt{B_{\GRANULEmv{n}}}}\\
%         \end{equation}

%         \begin{equation}
%           \label{soundConLAltSub2}
%           \inferrule*[Right=Approx, rightskip=5em]{ \inferrule*[Right=Approx]{ \Delta_{\GRANULEmv{i}}  \GRANULEsym{,}   \GRANULEmv{x}  :_{\textcolor{coeffectColor}{   \GRANULEnt{r_{\GRANULEmv{i}}}   } }    K  \, \GRANULEnt{A}    \GRANULEsym{,}   {  \GRANULEmv{y} ^  i  _  1   }:_{\textcolor{coeffectColor}{    \GRANULEnt{s} ^  i  _  1     } }   \GRANULEnt{B_{{\mathrm{1}}}}    , ... ,   {  \GRANULEmv{y} ^  i  _  n   }:_{\textcolor{coeffectColor}{    \GRANULEnt{s} ^  i  _  n     } }   \GRANULEnt{B_{\GRANULEmv{n}}}    \vdash  \GRANULEnt{t_{\GRANULEmv{i}}}  :  \GRANULEnt{B} \\ (e)}{\inferrule*[Right=Approx]{\Delta_{\GRANULEmv{i}}  \GRANULEsym{,}   \GRANULEmv{x}  :_{\textcolor{coeffectColor}{   \GRANULEnt{r_{\GRANULEmv{i}}}   } }    K  \, \GRANULEnt{A}    \GRANULEsym{,}   {  \GRANULEmv{y} ^  i  _  1   }:_{\textcolor{coeffectColor}{    \GRANULEnt{s} ^  i  _  1     } }   \GRANULEnt{B_{{\mathrm{1}}}}    , ... ,   {  \GRANULEmv{y} ^  i  _  n   }:_{\textcolor{coeffectColor}{    \GRANULEnt{s} _  i     } }   \GRANULEnt{B_{\GRANULEmv{n}}}    \vdash  \GRANULEnt{t_{\GRANULEmv{i}}}  :  \GRANULEnt{B} \\ (e)}{\inferrule*[Right=Approx]{\Delta_{\GRANULEmv{i}}  \GRANULEsym{,}   \GRANULEmv{x}  :_{\textcolor{coeffectColor}{    \GRANULEnt{r} _  i     } }     K  \, \GRANULEnt{A}    \GRANULEsym{,}   {  \GRANULEmv{y} ^  i  _  1   }:_{\textcolor{coeffectColor}{    \GRANULEnt{s} _  i     } }   \GRANULEnt{B_{{\mathrm{1}}}}    , ... ,   {  \GRANULEmv{y} ^  i  _  n   }:_{\textcolor{coeffectColor}{    \GRANULEnt{s} _  i     } }   \GRANULEnt{B_{\GRANULEmv{n}}}    \vdash  \GRANULEnt{t_{\GRANULEmv{i}}}  :  \GRANULEnt{B}  \\ (d)  }{\inferrule*[Right=Approx]{ \Delta_{\GRANULEmv{i}}  \GRANULEsym{,}   \GRANULEmv{x}  :_{\textcolor{coeffectColor}{    \GRANULEnt{r} _  i     } }     K  \, \GRANULEnt{A}    \GRANULEsym{,}   {  \GRANULEmv{y} ^  i  _  1   }:_{\textcolor{coeffectColor}{    \GRANULEnt{q} ^  i  _  1     \textcolor{coeffectColor}{\,\cdot\,}    \GRANULEnt{s'} _  i     } }   \GRANULEnt{B_{{\mathrm{1}}}}    , ... ,   {  \GRANULEmv{y} ^  i  _  n   }:_{\textcolor{coeffectColor}{    \GRANULEnt{q} ^  i  _  n     \textcolor{coeffectColor}{\,\cdot\,}    \GRANULEnt{s'} _  i     } }   \GRANULEnt{B_{\GRANULEmv{n}}}    \vdash  \GRANULEnt{t_{\GRANULEmv{i}}}  :  \GRANULEnt{B} \\ (c)}{\inferrule*[Right=Approx]{ \Delta_{\GRANULEmv{i}}  \GRANULEsym{,}   \GRANULEmv{x}  :_{\textcolor{coeffectColor}{    \GRANULEnt{r} _  i     } }     K  \, \GRANULEnt{A}    \GRANULEsym{,}   {  \GRANULEmv{y} ^  i  _  1   }:_{\textcolor{coeffectColor}{    \GRANULEnt{q} ^  i  _  1     \textcolor{coeffectColor}{\,\cdot\,}      \GRANULEnt{s} _  1     \sqcup ... \sqcup    \GRANULEnt{s} _  n       } }   \GRANULEnt{B_{{\mathrm{1}}}}    , ... ,   {  \GRANULEmv{y} ^  i  _  n   }:_{\textcolor{coeffectColor}{    \GRANULEnt{q} ^  i  _  n     \textcolor{coeffectColor}{\,\cdot\,}      \GRANULEnt{s'} _  1     \sqcup ... \sqcup    \GRANULEnt{s'} _  n       } }   \GRANULEnt{B_{\GRANULEmv{n}}}    \vdash  \GRANULEnt{t_{\GRANULEmv{i}}}  :  \GRANULEnt{B} \\ (b)}{\Delta_{\GRANULEmv{i}}  \GRANULEsym{,}   \GRANULEmv{x}  :_{\textcolor{coeffectColor}{   \GRANULEsym{(}     \GRANULEnt{r} _  1     \sqcup ... \sqcup    \GRANULEnt{r} _  n      \GRANULEsym{)}   } }     K  \, \GRANULEnt{A}    \GRANULEsym{,}   {  \GRANULEmv{y} ^  i  _  1   }:_{\textcolor{coeffectColor}{    \GRANULEnt{q} ^  i  _  1     \textcolor{coeffectColor}{\,\cdot\,}      \GRANULEnt{s'} _  1     \sqcup ... \sqcup    \GRANULEnt{s'} _  n       } }   \GRANULEnt{B_{{\mathrm{1}}}}    , ... ,   {  \GRANULEmv{y} ^  i  _  n   }:_{\textcolor{coeffectColor}{    \GRANULEnt{q} ^  i  _  n     \textcolor{coeffectColor}{\,\cdot\,}      \GRANULEnt{s'} _  1     \sqcup ... \sqcup    \GRANULEnt{s'} _  n       } }   \GRANULEnt{B_{\GRANULEmv{n}}}    \vdash  \GRANULEnt{t_{\GRANULEmv{i}}}  :  \GRANULEnt{B} \\ (a)} }}}}}{ \GRANULEsym{(}   \Delta_{{\mathrm{1}}}  \sqcup ...  \sqcup  \Delta_{\GRANULEmv{n}}   \GRANULEsym{)}  \GRANULEsym{,}   \GRANULEmv{x}  :_{\textcolor{coeffectColor}{   \GRANULEsym{(}     \GRANULEnt{r} _  1     \sqcup ... \sqcup    \GRANULEnt{r} _  n      \GRANULEsym{)}   } }     K  \, \GRANULEnt{A}    \GRANULEsym{,}   {  \GRANULEmv{y} ^  i  _  1   }:_{\textcolor{coeffectColor}{    \GRANULEnt{q} ^  i  _  1     \textcolor{coeffectColor}{\,\cdot\,}      \GRANULEnt{s'} _  1     \sqcup ... \sqcup    \GRANULEnt{s'} _  n       } }   \GRANULEnt{B_{{\mathrm{1}}}}    , ... ,   {  \GRANULEmv{y} ^  i  _  n   }:_{\textcolor{coeffectColor}{    \GRANULEnt{q} ^  i  _  n     \textcolor{coeffectColor}{\,\cdot\,}      \GRANULEnt{s'} _  1     \sqcup ... \sqcup    \GRANULEnt{s'} _  n       } }   \GRANULEnt{B_{\GRANULEmv{n}}}    \vdash  \GRANULEnt{t_{\GRANULEmv{i}}}  :  \GRANULEnt{B} }
%         \end{equation}


%         \begin{align*}
%           \inferrule*[Right=Case]{ \inferrule*[Right=Var,leftskip=5em]{\quad}{\GRANULEmv{x}  :_{\textcolor{coeffectColor}{   1   } }     K  \, \GRANULEnt{A}    \vdash  \GRANULEmv{x}  :    K  \, \GRANULEnt{A}} \\ \eqref{soundConLAltSub1} \\ \eqref{soundConLAltSub2} }
%           {\GRANULEsym{(}   \Delta_{{\mathrm{1}}}  \sqcup ...  \sqcup  \Delta_{\GRANULEmv{n}}   \GRANULEsym{)}  \GRANULEsym{,}    \GRANULEmv{x}  :_{\textcolor{coeffectColor}{   \GRANULEsym{(}     \GRANULEnt{r} _  1     \sqcup ... \sqcup    \GRANULEnt{r} _  n      \GRANULEsym{)}   \GRANULEsym{+}   \GRANULEsym{(}      \GRANULEnt{s'} _  1     \sqcup ... \sqcup    \GRANULEnt{s'} _  n       \textcolor{coeffectColor}{\,\cdot\,}   1   \GRANULEsym{)}   } }     K  \, \GRANULEnt{A}     \vdash   \textbf{case} \  \GRANULEmv{x}  \ \textbf{of} \   \overline{   C_{\GRANULEmv{i}}  \  \GRANULEmv{y} ^  i  _  1   ...  \GRANULEmv{y} ^  i  _  n    \mapsto  \GRANULEnt{t_{\GRANULEmv{i}}} }    :  \GRANULEnt{B}}
        % \end{align*}

\item Case $\Box_{R}$ \\
        For synthesising a promotion, we have the derivation:
        \begin{align*}
          \synBox
        \end{align*}
        By induction on the premise we have:
        \begin{align*}
          \Delta  \vdash  \GRANULEnt{t}  :  \GRANULEnt{A}
        \end{align*}
        From which we can construct the following derivation, matching the above conclusion:
        \begin{align*}
          \inferrule*[Right=Pr]{\Delta  \vdash  \GRANULEnt{t}  :  \GRANULEnt{A}}{ \textcolor{coeffectColor}{ \GRANULEnt{r}   \textcolor{coeffectColor}{\,\cdot\,} }  \Delta   \vdash  \GRANULEsym{[}  \GRANULEnt{t}  \GRANULEsym{]}  :   \Box_{  \GRANULEnt{r}  }  \GRANULEnt{A}}
        \end{align*}
  \item Case $\Box_{L}$ \\
        For synthesising an unboxing, we have the derivation:
        \begin{align*}
          \synUnbox
        \end{align*}
        By induction on the premise we have:
        \begin{align*}
          \Delta  \GRANULEsym{,}   \GRANULEmv{y}  :_{\textcolor{coeffectColor}{  \GRANULEnt{s_{{\mathrm{1}}}}  } }   \GRANULEnt{A}    \GRANULEsym{,}   \GRANULEmv{x}  :_{\textcolor{coeffectColor}{  \GRANULEnt{s_{{\mathrm{2}}}}  } }    \Box_{  \GRANULEnt{q}  }  \GRANULEnt{A}    \vdash  \GRANULEnt{t}  :  \GRANULEnt{B}
        \end{align*}
        and that:
        \begin{align*}
          \GRANULEnt{s_{{\mathrm{1}}}}  \, \textcolor{coeffectColor}{\sqsubseteq} \,   \GRANULEnt{s_{{\mathrm{3}}}}  \textcolor{coeffectColor}{\,\cdot\,}  \GRANULEnt{q}
        \end{align*}
        From this we can construct the following derivation, matching the above conclusion:
        \begin{align*}
          \inferrule*[Right=Case]{\inferrule*[Right=Var, leftskip=5em]{\quad}{\GRANULEmv{x}  :_{\textcolor{coeffectColor}{   1   } }    \Box_{  \GRANULEnt{q}  }  \GRANULEnt{A}    \vdash  \GRANULEmv{x}  :   \Box_{  \GRANULEnt{q}  }  \GRANULEnt{A}} \\ \inferrule*[Right=PBox, leftskip=0em]{\inferrule*[Right=PVar]{\quad}{\GRANULEnt{s_{{\mathrm{3}}}}  \textcolor{coeffectColor}{\,\cdot\,}  \GRANULEnt{q}  \vdash \,  \GRANULEmv{y}  :  \GRANULEnt{A}  \, \rhd \,   \GRANULEmv{y}  :_{\textcolor{coeffectColor}{   \GRANULEnt{s_{{\mathrm{3}}}}  \textcolor{coeffectColor}{\,\cdot\,}  \GRANULEnt{q}   } }   \GRANULEnt{A}}}{\GRANULEnt{s_{{\mathrm{3}}}}  \vdash \,  \GRANULEsym{[}  \GRANULEmv{y}  \GRANULEsym{]}  :   \Box_{  \GRANULEnt{q}  }  \GRANULEnt{A}   \, \rhd \,   \GRANULEmv{y}  :_{\textcolor{coeffectColor}{   \GRANULEnt{s_{{\mathrm{3}}}}  \textcolor{coeffectColor}{\,\cdot\,}  \GRANULEnt{q}   } }   \GRANULEnt{A}} \\ \inferrule*[Right=Approx, rightskip=4.5em] {\Delta  \GRANULEsym{,}   \GRANULEmv{y}  :_{\textcolor{coeffectColor}{  \GRANULEnt{s_{{\mathrm{1}}}}  } }   \GRANULEnt{A}   \GRANULEsym{,}   \GRANULEmv{x}  :_{\textcolor{coeffectColor}{  \GRANULEnt{s_{{\mathrm{2}}}}  } }    \Box_{  \GRANULEnt{q}  }  \GRANULEnt{A}    \vdash  \GRANULEnt{t}  :  \GRANULEnt{B} \\ \GRANULEnt{s_{{\mathrm{1}}}}  \, \textcolor{coeffectColor}{\sqsubseteq} \,   \GRANULEnt{s_{{\mathrm{3}}}}  \textcolor{coeffectColor}{\,\cdot\,}  \GRANULEnt{q}} { \Delta  \GRANULEsym{,}   \GRANULEmv{y}  :_{\textcolor{coeffectColor}{   \GRANULEnt{s_{{\mathrm{3}}}}  \textcolor{coeffectColor}{\,\cdot\,}  \GRANULEnt{q}   } }   \GRANULEnt{A}   \GRANULEsym{,}   \GRANULEmv{x}  :_{\textcolor{coeffectColor}{  \GRANULEnt{s_{{\mathrm{2}}}}  } }    \Box_{  \GRANULEnt{q}  }  \GRANULEnt{A}    \vdash  \GRANULEnt{t}  :  \GRANULEnt{B}} }{\Delta  \GRANULEsym{,}   \GRANULEmv{x}  :_{\textcolor{coeffectColor}{   \GRANULEnt{s_{{\mathrm{3}}}}  \GRANULEsym{+}  \GRANULEnt{s_{{\mathrm{2}}}}   } }    \Box_{  \GRANULEnt{q}  }  \GRANULEnt{A}    \vdash   \textbf{case} \  \GRANULEmv{x}  \ \textbf{of} \  \GRANULEsym{[}  \GRANULEmv{y}  \GRANULEsym{]}  \rightarrow  \GRANULEnt{t}   :  \GRANULEnt{B}}
        \end{align*}

\end{enumerate}

\end{proof}

% \focusSynthSound*
% \begin{proof}

% \end{proof}
%

\begin{restatable}[Soundness of focusing for graded-base synthesis]{lemma}{gradedBaseFocusingSoundness}
For all contexts $\Gamma$, $\Omega$ and types $\GRANULEnt{A}$
then:
\begin{align*}
\begin{array}{lll}
 1.\ Right\ Async: & \Gamma  ;  \Omega  \vdash  \GRANULEnt{A}  \Uparrow \Rightarrow  \GRANULEnt{t}  \mid  \Delta \quad &\implies \quad \Gamma  \GRANULEsym{,}  \,  \GRANULEsym{,}  \Omega  \vdash  \GRANULEnt{A}  \Rightarrow  \GRANULEnt{t}  \mid  \Delta\\
 2.\ Left\ Async: & \Gamma  ;  \Omega  \Uparrow \vdash  \GRANULEnt{B}  \Rightarrow  \GRANULEnt{t}  \mid  \Delta \quad &\implies \quad \Gamma  \GRANULEsym{,}  \,  \GRANULEsym{,}  \Omega  \vdash  \GRANULEnt{B}  \Rightarrow  \GRANULEnt{t}  \mid  \Delta\\
 3.\ Right\ Sync: & \Gamma ; \emptyset \vdash \GRANULEnt{A} \Downarrow\ \Rightarrow \GRANULEnt{t} \mid\  \Delta \quad &\implies \quad \Gamma  \vdash  \GRANULEnt{A}  \Rightarrow  \GRANULEnt{t}  \mid  \Delta\\
 4.\ Left\ Sync: & \Gamma  ;    \GRANULEmv{x}  :  \GRANULEnt{A}    \Downarrow \vdash  \GRANULEnt{B}  \Rightarrow  \GRANULEnt{t}  \mid  \Delta \quad &\implies \quad \Gamma  \GRANULEsym{,}   \GRANULEmv{x}  :  \GRANULEnt{A}   \vdash  \GRANULEnt{B}  \Rightarrow  \GRANULEnt{t}  \mid  \Delta\\
 5.\ Focus\ Right: & \Gamma  ;  \Omega  \Uparrow \vdash  \GRANULEnt{B}  \Rightarrow  \GRANULEnt{t}  \mid  \Delta \quad &\implies \quad \Gamma  \vdash  \GRANULEnt{B}  \Rightarrow  \GRANULEnt{t}  \mid  \Delta\\
 6.\ Focus\ Left: & \Gamma  \GRANULEsym{,}   \GRANULEmv{x}  :  \GRANULEnt{A}   ;  \Omega  \Uparrow \vdash  \GRANULEnt{B}  \Rightarrow  \GRANULEnt{t}  \mid  \Delta \quad &\implies \quad \Gamma  \vdash  \GRANULEnt{B}  \Rightarrow  \GRANULEnt{t}  \mid  \Delta
\end{array}
\end{align*}
\end{restatable}
\begin{proof}
\begin{enumerate}
\item Case: 1. Right Async: \\
    \begin{enumerate}
      \item Case $\multimap_{R}$\\
          In the case of the right asynchronous rule for abstraction introduction, the synthesis rule has the form:
          \[
            \fsynAbs
          \]
          By induction on the premise, we have that:
          \[
            \GRANULEsym{(}  \Gamma  \GRANULEsym{,}  \Omega  \GRANULEsym{)}  \GRANULEsym{,}   \GRANULEmv{x}  :_{\textcolor{coeffectColor}{  \GRANULEnt{q}  } }   \GRANULEnt{A}   \vdash  \GRANULEnt{B}  \Rightarrow  \GRANULEnt{t}  \mid  \Delta  \GRANULEsym{,}   \GRANULEmv{x}  :_{\textcolor{coeffectColor}{  \GRANULEnt{r}  } }   \GRANULEnt{A}   \tag{ih}
          \]
          from case 1 of the lemma. From which, we can construct the following instatiation of the $\multimap_{\textsc{R}}$ synthesis rule in the non-focusing calculus:
          \[
    \inferrule*[right=$\multimap_{R}$]
    {\GRANULEsym{(}  \Gamma  \GRANULEsym{,}  \Omega  \GRANULEsym{)}  \GRANULEsym{,}   \GRANULEmv{x}  :_{\textcolor{coeffectColor}{  \GRANULEnt{q}  } }   \GRANULEnt{A}   \vdash  \GRANULEnt{B}  \Rightarrow  \GRANULEnt{t}  \mid  \Delta  \GRANULEsym{,}   \GRANULEmv{x}  :_{\textcolor{coeffectColor}{  \GRANULEnt{r}  } }   \GRANULEnt{A}    \quad\;\;   \GRANULEnt{r}  \, \textcolor{coeffectColor}{\sqsubseteq} \,  \GRANULEnt{q}}{\Gamma  \GRANULEsym{,}  \Omega  \vdash  \GRANULEnt{A}  \rightarrow  \GRANULEnt{B}  \Rightarrow   \lambda  \GRANULEmv{x}  .  \GRANULEnt{t}   \mid  \Delta}
          \]
          \item Case \fsynRAsyncTransName\ \\
          In the case of the right asynchronous rule for transition to a left asynchronous judgement, the synthesis rule has the form:
          \[
            \fsynRAsyncTrans
          \]
          By induction on the first premise, we have that:
          \[
            \Gamma  \GRANULEsym{,}  \Omega  \vdash  \GRANULEnt{B}  \Rightarrow^+  \GRANULEnt{t}  ;\,  \Delta
          \]
          from case 2 of the lemma.
    \end{enumerate}
\item Case 2. Left Async: \\
    \begin{enumerate}
      \item Case $\textsc{Con}_{L}$\\
        In the case of the left asynchronous rule for constructor elimination, the synthesis rule has the form:
            \[
            \fsynCase
            \]
            By induction on the second premise, we have that:
            \[
\GRANULEsym{(}  \Gamma  \GRANULEsym{,}  \Omega  \GRANULEsym{)}  \GRANULEsym{,}       \GRANULEmv{x}  :_{\textcolor{coeffectColor}{   \GRANULEnt{r}   } }    K  \,   \vec{ \GRANULEnt{A} }      \GRANULEsym{,}   {  \GRANULEmv{y} ^  i  _  1   }:_{\textcolor{coeffectColor}{   \GRANULEnt{r}  \textcolor{coeffectColor}{\,\cdot\,}    \GRANULEnt{q} ^  i  _  1      } }   \GRANULEnt{B_{{\mathrm{1}}}}    , ... ,   {  \GRANULEmv{y} ^  i  _  n   }:_{\textcolor{coeffectColor}{   \GRANULEnt{r}  \textcolor{coeffectColor}{\,\cdot\,}    \GRANULEnt{q} ^  i  _  1      } }   \GRANULEnt{B_{\GRANULEmv{n}}}      \Uparrow\   \vdash  \GRANULEnt{B}  \Rightarrow  \GRANULEnt{t_{\GRANULEmv{i}}}  \mid     \Delta_{\GRANULEmv{i}}  \GRANULEsym{,}   \GRANULEmv{x}  :_{\textcolor{coeffectColor}{   \GRANULEnt{r_{\GRANULEmv{i}}}   } }    K  \,   \vec{ \GRANULEnt{A} }      \GRANULEsym{,}   {  \GRANULEmv{y} ^  i  _  1   }:_{\textcolor{coeffectColor}{    \GRANULEnt{s} ^  i  _  1     } }   \GRANULEnt{B_{{\mathrm{1}}}}    , ... ,   {  \GRANULEmv{y} ^  i  _  n   }:_{\textcolor{coeffectColor}{    \GRANULEnt{s} ^  i  _  n     } }   \GRANULEnt{B_{\GRANULEmv{n}}}
            \]
            from case 2 of the lemma. From which we can construct the following instantiation of the $\textsc{Con}_{L}$ rule in the non-focusing calculus:
            \begin{align*}
            \inferrule*[right=$\textsc{Con}_{\textsc{L}}$]
            {
            (  C_{\GRANULEmv{i}}  :  \GRANULEnt{B_{{\mathrm{1}}}} ^{q_1^i} \rightarrow ... \rightarrow  \GRANULEnt{B_{\GRANULEmv{n}}} ^{q_n^i} \rightarrow     K  \,   \vec{ \GRANULEnt{A} }     ) \in  D\\\\
            \GRANULEsym{(}  \Gamma  \GRANULEsym{,}  \Omega  \GRANULEsym{)}  \GRANULEsym{,}   \GRANULEmv{x}  :_{\textcolor{coeffectColor}{   \GRANULEnt{r}   } }    K  \,   \vec{ \GRANULEnt{A} }      \GRANULEsym{,}   {  \GRANULEmv{y} ^  i  _  1   }:_{\textcolor{coeffectColor}{   \GRANULEnt{r}  \textcolor{coeffectColor}{\,\cdot\,}    \GRANULEnt{q} ^  i  _  1      } }   \GRANULEnt{B_{{\mathrm{1}}}}    , ... ,   {  \GRANULEmv{y} ^  i  _  n   }:_{\textcolor{coeffectColor}{   \GRANULEnt{r}  \textcolor{coeffectColor}{\,\cdot\,}    \GRANULEnt{q} ^  i  _  1      } }   \GRANULEnt{B_{\GRANULEmv{n}}}    \vdash  \GRANULEnt{B}  \Rightarrow  \GRANULEnt{t_{\GRANULEmv{i}}}  \mid     \Delta_{\GRANULEmv{i}}  \GRANULEsym{,}   \GRANULEmv{x}  :_{\textcolor{coeffectColor}{   \GRANULEnt{r_{\GRANULEmv{i}}}   } }    K  \,   \vec{ \GRANULEnt{A} }      \GRANULEsym{,}   {  \GRANULEmv{y} ^  i  _  1   }:_{\textcolor{coeffectColor}{    \GRANULEnt{s} ^  i  _  1     } }   \GRANULEnt{B_{{\mathrm{1}}}}    , ... ,   {  \GRANULEmv{y} ^  i  _  n   }:_{\textcolor{coeffectColor}{    \GRANULEnt{s} ^  i  _  n     } }   \GRANULEnt{B_{\GRANULEmv{n}}}\\\\
            \exists    \GRANULEnt{s'} ^  i  _  j     .\,     \GRANULEnt{s} ^  i  _  j     \sqsubseteq    \GRANULEnt{s'} ^  i  _  j     \textcolor{coeffectColor}{\,\cdot\,}    \GRANULEnt{q} ^  i  _  j     \sqsubseteq  \GRANULEnt{r}  \textcolor{coeffectColor}{\,\cdot\,}    \GRANULEnt{q} ^  i  _  j\\\\
            \GRANULEnt{s} _  i     =      \GRANULEnt{s'} ^  i  _  1     \sqcup ... \sqcup    \GRANULEnt{s'} ^  i  _  n\\\\
            |    K  \,   \vec{ \GRANULEnt{A} }     | > 1 \Rightarrow 1 \sqsubseteq      \GRANULEnt{s} _  1     \sqcup ... \sqcup    \GRANULEnt{s} _  m
            }
            {\GRANULEsym{(}  \Gamma  \GRANULEsym{,}  \Omega  \GRANULEsym{)}  \GRANULEsym{,}   \GRANULEmv{x}  :_{\textcolor{coeffectColor}{  \GRANULEnt{r}  } }     K  \,   \vec{ \GRANULEnt{A} }      \vdash  \GRANULEnt{B}  \Rightarrow   \textbf{case} \  \GRANULEmv{x}  \ \textbf{of} \   \overline{   C_{\GRANULEmv{i}}  \  \GRANULEmv{y} ^  i  _  1   ...  \GRANULEmv{y} ^  i  _  n    \mapsto  \GRANULEnt{t_{\GRANULEmv{i}}} }    \mid  \GRANULEsym{(}   \Delta_{{\mathrm{1}}}  \sqcup ...  \sqcup  \Delta_{\GRANULEmv{m}}   \GRANULEsym{)}  \GRANULEsym{,}    \GRANULEmv{x}  :_{\textcolor{coeffectColor}{   \GRANULEsym{(}     \GRANULEnt{r} _  1     \sqcup ... \sqcup    \GRANULEnt{r} _  m      \GRANULEsym{)}   \GRANULEsym{+}   \GRANULEsym{(}     \GRANULEnt{s} _  1     \sqcup ... \sqcup    \GRANULEnt{s} _  m      \GRANULEsym{)}   } }     K  \,   \vec{ \GRANULEnt{A} }
            }
            \end{align*}


      \item Case $\Box_{L}$ \\
            In the case of the left asynchronous rule for graded modality elimination, the synthesis rule has the form:
            \[
            \fsynUnbox
            \]
            By induction on the first premise, we have that:
            \[
            \GRANULEsym{(}  \Gamma  \GRANULEsym{,}  \Omega  \GRANULEsym{)}  \GRANULEsym{,}   \GRANULEmv{y}  :_{\textcolor{coeffectColor}{   \GRANULEnt{r}  \textcolor{coeffectColor}{\,\cdot\,}  \GRANULEnt{q}   } }   \GRANULEnt{A}    \GRANULEsym{,}   \GRANULEmv{x}  :_{\textcolor{coeffectColor}{  \GRANULEnt{r}  } }    \Box_{  \GRANULEnt{q}  }  \GRANULEnt{A}     \vdash  \GRANULEnt{B}  \Rightarrow  \GRANULEnt{t}  \mid  \Delta  \GRANULEsym{,}   \GRANULEmv{y}  :_{\textcolor{coeffectColor}{   \GRANULEnt{s_{{\mathrm{1}}}}   } }   \GRANULEnt{A}   \GRANULEsym{,}   \GRANULEmv{x}  :_{\textcolor{coeffectColor}{  \GRANULEnt{s_{{\mathrm{2}}}}  } }    \Box_{  \GRANULEnt{q}  }  \GRANULEnt{A}
            \]
            from case 2 of the lemma. From which, we can construct the following instantiation of the $\Box_{L}$ synthesis rule in the non focusing calculus:
            \[
            \inferrule*[right=$\Box_{\textsc{L}}$]
            {
            \GRANULEsym{(}  \Gamma  \GRANULEsym{,}  \Omega  \GRANULEsym{)}  \GRANULEsym{,}   \GRANULEmv{y}  :_{\textcolor{coeffectColor}{   \GRANULEnt{r}  \textcolor{coeffectColor}{\,\cdot\,}  \GRANULEnt{q}   } }   \GRANULEnt{A}    \GRANULEsym{,}   \GRANULEmv{x}  :_{\textcolor{coeffectColor}{  \GRANULEnt{r}  } }    \Box_{  \GRANULEnt{q}  }  \GRANULEnt{A}     \vdash  \GRANULEnt{B}  \Rightarrow  \GRANULEnt{t}  \mid  \Delta  \GRANULEsym{,}   \GRANULEmv{y}  :_{\textcolor{coeffectColor}{   \GRANULEnt{s_{{\mathrm{1}}}}   } }   \GRANULEnt{A}   \GRANULEsym{,}   \GRANULEmv{x}  :_{\textcolor{coeffectColor}{  \GRANULEnt{s_{{\mathrm{2}}}}  } }    \Box_{  \GRANULEnt{q}  }  \GRANULEnt{A} \\
            \exists  \GRANULEnt{s_{{\mathrm{3}}}}  .\,   \GRANULEnt{s_{{\mathrm{1}}}}  \sqsubseteq   \GRANULEnt{s_{{\mathrm{3}}}}  \textcolor{coeffectColor}{\,\cdot\,}  \GRANULEnt{q}   \sqsubseteq   \GRANULEnt{r}  \textcolor{coeffectColor}{\,\cdot\,}  \GRANULEnt{q}
            }
            {
            \GRANULEsym{(}  \Gamma  \GRANULEsym{,}  \Omega  \GRANULEsym{)}  \GRANULEsym{,}   \GRANULEmv{x}  :_{\textcolor{coeffectColor}{  \GRANULEnt{r}  } }    \Box_{  \GRANULEnt{q}  }  \GRANULEnt{A}     \vdash  \GRANULEnt{B}  \Rightarrow   \textbf{case} \  \GRANULEmv{x}  \ \textbf{of} \  \GRANULEsym{[}  \GRANULEmv{y}  \GRANULEsym{]}  \rightarrow  \GRANULEnt{t}   \mid   \Delta   \GRANULEsym{,}   \GRANULEmv{x}  :_{\textcolor{coeffectColor}{   \GRANULEnt{s_{{\mathrm{3}}}}  \GRANULEsym{+}  \GRANULEnt{s_{{\mathrm{2}}}}   } }    \Box_{  \GRANULEnt{q}  }  \GRANULEnt{A}
            }
            \]

        \item Case \fsynLAsyncTransName \\
          In the case of the left asynchronous rule for transitioning an assumption from the focusing context $\Omega$ to the non-focusing context $\Gamma$, the synthesis rule has the form:
          \[
            \fsynLAsyncTrans
          \]
          By induction on the first premise, we have that:
          \[
            \Gamma  \GRANULEsym{,}   \GRANULEmv{x}  :  \GRANULEnt{A}    \GRANULEsym{,}  \Omega  \vdash  \GRANULEnt{C}  \Rightarrow  \GRANULEnt{t}  \mid  \Delta \tag{ih}
          \]
          from case 2 of the lemma.
    \end{enumerate}
\item Case 3. Right Sync: \\
    \begin{enumerate}
      \item Case $\textsc{Con}_{\textsc{R}}$\\
          In the case of the right synchronous rule for constructor introduction, the synthesis rule has the form:
            \[
            \inferrule*[Right=$\textsc{Con}_{\textsc{R}}$]
            { (  C  :  \GRANULEnt{B_{{\mathrm{1}}}} ^1 \rightarrow ... \rightarrow  \GRANULEnt{B_{\GRANULEmv{n}}} ^1 \rightarrow     K  \,   \vec{ \GRANULEnt{A} }     ) \in  D \\
             \Gamma ; \emptyset \vdash \GRANULEnt{B_{\GRANULEmv{i}}} \Downarrow\ \Rightarrow \GRANULEnt{t_{\GRANULEmv{i}}} \mid\ \Delta_{\GRANULEmv{i}}}
            {\Gamma ; \emptyset \vdash K  \,   \vec{ \GRANULEnt{A} } \Downarrow\ \Rightarrow C \, \GRANULEnt{t_{{\mathrm{1}}}} \, ... \, \GRANULEnt{t_{\GRANULEmv{n}}} \mid\ \Delta_{{\mathrm{1}}}  \GRANULEsym{+} \, ... \, \GRANULEsym{+}  \Delta_{\GRANULEmv{n}}}
            \]
          By induction on the second premise, we have that:
            \[
            \Gamma  \vdash  \GRANULEnt{B_{\GRANULEmv{i}}}  \Rightarrow  \GRANULEnt{t_{\GRANULEmv{i}}}  \mid  \Delta_{\GRANULEmv{i}}
            \]
          from case 3 of the lemma. From which, we can construct the following instantiation of the $\textsc{Con}_{\textsc{R}}$\ synthesis rule in the non-focusing calculus:
            \[
            \inferrule*[]
            {
            (  C  :  \GRANULEnt{B_{{\mathrm{1}}}} ^ \GRANULEnt{q_{{\mathrm{1}}}}  \rightarrow ... \rightarrow  \GRANULEnt{B_{\GRANULEmv{n}}} ^ \GRANULEnt{q_{\GRANULEmv{n}}}  \rightarrow     K  \,   \vec{ \GRANULEnt{A} }     ) \in  D \\
            \Gamma  \vdash  \GRANULEnt{B_{\GRANULEmv{i}}}  \Rightarrow  \GRANULEnt{t_{\GRANULEmv{i}}}  \mid  \Delta_{\GRANULEmv{i}}
            }
            {\Gamma  \vdash   K  \,   \vec{ \GRANULEnt{A} }    \Rightarrow  C \, \GRANULEnt{t_{{\mathrm{1}}}} \, ... \, \GRANULEnt{t_{\GRANULEmv{n}}}  \mid    \textcolor{coeffectColor}{  0    \textcolor{coeffectColor}{\,\cdot\,} }  \Gamma    \GRANULEsym{+}   \GRANULEsym{(}   \textcolor{coeffectColor}{ \GRANULEnt{q_{{\mathrm{1}}}}   \textcolor{coeffectColor}{\,\cdot\,} }  \Delta_{{\mathrm{1}}}   \GRANULEsym{)}  \GRANULEsym{+} \, ... \, \GRANULEsym{+}  \GRANULEsym{(}   \textcolor{coeffectColor}{ \GRANULEnt{q_{\GRANULEmv{n}}}   \textcolor{coeffectColor}{\,\cdot\,} }  \Delta_{\GRANULEmv{n}}   \GRANULEsym{)}}
            \]
      \item Case $\Box_{R}$ \\

          In the case of the right synchronous rule for graded modality introduction, the synthesis rule has the form:
            \[
            \inferrule*[Right=$\Box_{\textsc{R}}$]
            { \Gamma ; \emptyset \vdash \GRANULEnt{A} \Downarrow\ \Rightarrow \GRANULEnt{t} \mid\ \Delta}
            { \Gamma ; \emptyset \vdash \Box_{  \GRANULEnt{r}  }  \GRANULEnt{A} \Downarrow\ \Rightarrow \GRANULEsym{[}  \GRANULEnt{t}  \GRANULEsym{]} \mid\ \textcolor{coeffectColor}{ \GRANULEnt{r}   \textcolor{coeffectColor}{\,\cdot\,} }  \Delta}
            \]
          By induction on the premises, we have that:
            \[
            \Gamma  \vdash  \GRANULEnt{A}  \Rightarrow  \GRANULEnt{t}  \mid  \Delta
            \]
          from case 3 of the lemma. From which, we can construct the following instantiation of the $\Box_{\textsc{R}}$\ synthesis rule in the non-focusing calculus:
            \[
            \synBox
            \]
    \end{enumerate}
\item Case 4. Left Sync: \\
    \begin{enumerate}
      \item Case $\multimap_{L}$\\
            In the case of the left synchronous rule for application, the synthesis rule has the form:
            \[
            \inferrule*[]{
            \Gamma  \GRANULEsym{,}   \GRANULEmv{x_{{\mathrm{1}}}}  :_{\textcolor{coeffectColor}{  \GRANULEnt{r_{{\mathrm{1}}}}  } }    \GRANULEnt{A} ^ \GRANULEnt{q}  \rightarrow  \GRANULEnt{B}    ;    \GRANULEmv{x_{{\mathrm{2}}}}  :_{\textcolor{coeffectColor}{  \GRANULEnt{r_{{\mathrm{1}}}}  } }   \GRANULEnt{B}    \Downarrow \vdash  \GRANULEnt{C}  \Rightarrow  \GRANULEnt{t_{{\mathrm{1}}}}  \mid  \Delta_{{\mathrm{1}}}  \GRANULEsym{,}   \GRANULEmv{x_{{\mathrm{1}}}}  :_{\textcolor{coeffectColor}{  \GRANULEnt{s_{{\mathrm{1}}}}  } }    \GRANULEnt{A} ^ \GRANULEnt{q}  \rightarrow  \GRANULEnt{B}    \GRANULEsym{,}   \GRANULEmv{x_{{\mathrm{2}}}}  :_{\textcolor{coeffectColor}{  \GRANULEnt{s_{{\mathrm{2}}}}  } }   \GRANULEnt{B} \\
            \Gamma  \GRANULEsym{,}   \GRANULEmv{x_{{\mathrm{1}}}}  :_{\textcolor{coeffectColor}{  \GRANULEnt{r_{{\mathrm{1}}}}  } }    \GRANULEnt{A} ^ \GRANULEnt{q}  \rightarrow  \GRANULEnt{B} ; \emptyset \vdash \GRANULEnt{A} \Downarrow\ \Rightarrow \GRANULEnt{t_{{\mathrm{2}}}} \mid\ \Delta_{{\mathrm{2}}}  \GRANULEsym{,}   \GRANULEmv{x_{{\mathrm{1}}}}  :_{\textcolor{coeffectColor}{  \GRANULEnt{s_{{\mathrm{3}}}}  } }    \GRANULEnt{A} ^ \GRANULEnt{q}  \rightarrow  \GRANULEnt{B}
            }
            {
            \Gamma  ;    \GRANULEmv{x_{{\mathrm{1}}}}  :_{\textcolor{coeffectColor}{  \GRANULEnt{r_{{\mathrm{1}}}}  } }    \GRANULEnt{A} ^ \GRANULEnt{q}  \rightarrow  \GRANULEnt{B}     \Downarrow \vdash  \GRANULEnt{C}  \Rightarrow    [  \GRANULEsym{(}  \GRANULEmv{x_{{\mathrm{1}}}} \, \GRANULEnt{t_{{\mathrm{2}}}}  \GRANULEsym{)}  /  \GRANULEmv{x_{{\mathrm{2}}}}  ]  \GRANULEnt{t_{{\mathrm{1}}}}    \mid  \GRANULEsym{(}  \Delta_{{\mathrm{1}}}  \GRANULEsym{+}   \textcolor{coeffectColor}{ \GRANULEnt{s_{{\mathrm{2}}}}   \textcolor{coeffectColor}{\,\cdot\,} }    \textcolor{coeffectColor}{ \GRANULEnt{q}   \textcolor{coeffectColor}{\,\cdot\,} }  \Delta_{{\mathrm{2}}}     \GRANULEsym{)}  \GRANULEsym{,}   \GRANULEmv{x_{{\mathrm{1}}}}  :_{\textcolor{coeffectColor}{   \GRANULEnt{s_{{\mathrm{2}}}}  \GRANULEsym{+}   \GRANULEnt{s_{{\mathrm{1}}}}  \GRANULEsym{+}   \GRANULEsym{(}  \GRANULEnt{s_{{\mathrm{2}}}}  \textcolor{coeffectColor}{\,\cdot\,}   \GRANULEnt{q}  \textcolor{coeffectColor}{\,\cdot\,}  \GRANULEnt{s_{{\mathrm{3}}}}   \GRANULEsym{)}     } }    \GRANULEnt{A} ^ \GRANULEnt{q}  \rightarrow  \GRANULEnt{B}
            }
            \]
            By induction on the first premise, we have that:
            \[
\Gamma  \GRANULEsym{,}   \GRANULEmv{x_{{\mathrm{1}}}}  :_{\textcolor{coeffectColor}{  \GRANULEnt{r_{{\mathrm{1}}}}  } }    \GRANULEnt{A} ^ \GRANULEnt{q}  \rightarrow  \GRANULEnt{B}    \GRANULEsym{,}    \GRANULEmv{x_{{\mathrm{2}}}}  :_{\textcolor{coeffectColor}{  \GRANULEnt{r_{{\mathrm{1}}}}  } }   \GRANULEnt{B}    \vdash  \GRANULEnt{C}  \Rightarrow  \GRANULEnt{t_{{\mathrm{1}}}}  \mid  \Delta_{{\mathrm{1}}}  \GRANULEsym{,}   \GRANULEmv{x_{{\mathrm{1}}}}  :_{\textcolor{coeffectColor}{  \GRANULEnt{s_{{\mathrm{1}}}}  } }    \GRANULEnt{A} ^ \GRANULEnt{q}  \rightarrow  \GRANULEnt{B}    \GRANULEsym{,}   \GRANULEmv{x_{{\mathrm{2}}}}  :_{\textcolor{coeffectColor}{  \GRANULEnt{s_{{\mathrm{2}}}}  } }   \GRANULEnt{B}
            \]
            from case 4 of the lemma. By induction on the second premise, we have that:
            \[
            \Gamma  \GRANULEsym{,}   \GRANULEmv{x_{{\mathrm{1}}}}  :_{\textcolor{coeffectColor}{  \GRANULEnt{r_{{\mathrm{1}}}}  } }    \GRANULEnt{A} ^ \GRANULEnt{q}  \rightarrow  \GRANULEnt{B} \vdash \GRANULEnt{A} \Rightarrow \GRANULEnt{t_{{\mathrm{2}}}} \mid\ \Delta_{{\mathrm{2}}}  \GRANULEsym{,}   \GRANULEmv{x_{{\mathrm{1}}}}  :_{\textcolor{coeffectColor}{  \GRANULEnt{s_{{\mathrm{3}}}}  } }    \GRANULEnt{A} ^ \GRANULEnt{q}  \rightarrow  \GRANULEnt{B}
            \]
            from case 3 of the lemma. From which, we can construc the following instantiation of the $\multimap_{\textsc{L}}$ synthesis rule in the non-focusing calculus:
            \[
            \inferrule*[right=$\multimap_{\textsc{L}}$]
            {
            \Gamma  \GRANULEsym{,}   \GRANULEmv{x_{{\mathrm{1}}}}  :_{\textcolor{coeffectColor}{  \GRANULEnt{r_{{\mathrm{1}}}}  } }    \GRANULEnt{A} ^ \GRANULEnt{q}  \rightarrow  \GRANULEnt{B}    \GRANULEsym{,}   \GRANULEmv{x_{{\mathrm{2}}}}  :_{\textcolor{coeffectColor}{  \GRANULEnt{r_{{\mathrm{1}}}}  } }   \GRANULEnt{B}   \vdash  \GRANULEnt{C}  \Rightarrow  \GRANULEnt{t_{{\mathrm{1}}}}  \mid  \Delta_{{\mathrm{1}}}  \GRANULEsym{,}   \GRANULEmv{x_{{\mathrm{1}}}}  :_{\textcolor{coeffectColor}{  \GRANULEnt{s_{{\mathrm{1}}}}  } }    \GRANULEnt{A} ^ \GRANULEnt{q}  \rightarrow  \GRANULEnt{B}    \GRANULEsym{,}   \GRANULEmv{x_{{\mathrm{2}}}}  :_{\textcolor{coeffectColor}{  \GRANULEnt{s_{{\mathrm{2}}}}  } }   \GRANULEnt{B}\\
            \Gamma  \GRANULEsym{,}   \GRANULEmv{x_{{\mathrm{1}}}}  :_{\textcolor{coeffectColor}{  \GRANULEnt{r_{{\mathrm{1}}}}  } }    \GRANULEnt{A} ^ \GRANULEnt{q}  \rightarrow  \GRANULEnt{B}    \vdash  \GRANULEnt{A}  \Rightarrow  \GRANULEnt{t_{{\mathrm{2}}}}  \mid  \Delta_{{\mathrm{2}}}  \GRANULEsym{,}   \GRANULEmv{x_{{\mathrm{1}}}}  :_{\textcolor{coeffectColor}{  \GRANULEnt{s_{{\mathrm{3}}}}  } }    \GRANULEnt{A} ^ \GRANULEnt{q}  \rightarrow  \GRANULEnt{B}
            }
            {\Gamma  \GRANULEsym{,}   \GRANULEmv{x_{{\mathrm{1}}}}  :_{\textcolor{coeffectColor}{  \GRANULEnt{r_{{\mathrm{1}}}}  } }    \GRANULEnt{A} ^ \GRANULEnt{q}  \rightarrow  \GRANULEnt{B}    \vdash  \GRANULEnt{C}  \Rightarrow    [  \GRANULEsym{(}  \GRANULEmv{x_{{\mathrm{1}}}} \, \GRANULEnt{t_{{\mathrm{2}}}}  \GRANULEsym{)}  /  \GRANULEmv{x_{{\mathrm{2}}}}  ]  \GRANULEnt{t_{{\mathrm{1}}}}    \mid  \GRANULEsym{(}  \Delta_{{\mathrm{1}}}  \GRANULEsym{+}   \textcolor{coeffectColor}{ \GRANULEnt{s_{{\mathrm{2}}}}   \textcolor{coeffectColor}{\,\cdot\,} }    \textcolor{coeffectColor}{ \GRANULEnt{q}   \textcolor{coeffectColor}{\,\cdot\,} }  \Delta_{{\mathrm{2}}}     \GRANULEsym{)}  \GRANULEsym{,}   \GRANULEmv{x_{{\mathrm{1}}}}  :_{\textcolor{coeffectColor}{   \GRANULEnt{s_{{\mathrm{2}}}}  \GRANULEsym{+}   \GRANULEnt{s_{{\mathrm{1}}}}  \GRANULEsym{+}   \GRANULEsym{(}  \GRANULEnt{s_{{\mathrm{2}}}}  \textcolor{coeffectColor}{\,\cdot\,}   \GRANULEnt{q}  \textcolor{coeffectColor}{\,\cdot\,}  \GRANULEnt{s_{{\mathrm{3}}}}   \GRANULEsym{)}     } }    \GRANULEnt{A} ^ \GRANULEnt{q}  \rightarrow  \GRANULEnt{B}}
            \]

      \item Case $\textsc{Var}$ \\
        In the case of the left synchronous rule for variable synthesis, the synthesis rule has the form:
          \[
          \fsynVar
          \]
          From which, we can construct the following instantiation of the \textsc{Var}\ synthesis rule in the non-focusing calculus:
          \[
            \synVar
          \]
    \end{enumerate}
\item Case 5. Right Focus: \\
          In the case of the focusing rule for transitioning from a left asynchronous judgement to a right synchronous judgement, the synthesis rule has the form:
          \[
            \fsynFocusRNoLabel
          \]
          By induction on the first premise, we have that:
          \[
            \Gamma  \vdash  \GRANULEnt{C}  \Rightarrow  \GRANULEnt{t}  \mid  \Delta \tag{ih}
          \]
          from case 2 of the lemma.
\item Case 6. Left Focus: \\
          In the case of the focusing rule for transitioning from a left asynchronous judgement to a left synchronous judgement, the synthesis rule has the form:
          \[
            \fsynFocusL
          \]
          By induction on the first premise, we have that:
          \[
            \Gamma  \GRANULEsym{,}   \GRANULEmv{x}  :  \GRANULEnt{A}   \vdash  \GRANULEnt{C}  \Rightarrow  \GRANULEnt{t}  \mid  \Delta \tag{ih}
          \]
          from case 2 of the lemma.
\end{enumerate}
\end{proof}
