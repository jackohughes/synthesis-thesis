\jnote{Try to outline the approach of the background - the focus is on
  introducing the two languages: graded and linear base. Why do we need two languages? What is the
  difference between them? What do we use each language for in synthesis - i.e.
  why do we decide to do the initial calculus in the linear base and then shift
  to the graded base for the more advanced stuff?}
This chapter provides the background material necessary for understanding the
design of synthesis calculi. This is done primarily by formally introducing two programming languages
which include resourceful types. Language features, and concepts, are explained
as they arise. We begin with an outline of linear and graded
types themselves, briefly charting their origins from linear logic and bounded
linear logic. Following this, two languages are presented which make use of
these resourceful types in different ways. These two languages will become the
bases of the target languages of two synthesis calculi in subsequent chapters.



\jnote{How much detail should be provided here? It's very likely that we will
  cover a brief history of graded type systems (from linear logic to graded
  modal types) in the intro. So should we use this space to go into further
  detail? Or cut it here and leave it in the intro}
\paragraph{Linear types}



\paragraph{Graded types}


\jnote{possibly doesn't need to be its own section}
\section{Two target languages}
When designing a type-directed synthesis tool, a natural first consideration would be
the \textit{target} language of synthesis: what programming language
will synthesised programs be written in? What types will be used to specify a
desired program?
If we strip away all but the most essential language elements that we wish to
include in our target language, a good starting point would be the simply typed
linear $\lambda$-calculus, extended with a graded modal type. The linear
$\lambda$-calculus has the same syntax as the standard simply typed $\lambda$
calculus only with the structural rules of weakening and contraction
prohibited, making terms which express non-linear behaviour ill-typed.


This calculus is equivalent to the core calculus Granule,
\textsc{GrMini}\cite{}. Granule's full type system is an extension of this
graded linear core, introducing (G)ADTs and recurison.

For this chapter we concern ourselves only with the $\lambda$-calculus core.
In subsequent chapters these calculi will be extended further to become more
full-fledged programming languages.\jnote{How?}



An alternative approach is to replace this underlying linear system  with a
\textit{graded} $\lambda$-calculus.

In this system, function types are augmented with a grade: $\GRANULEnt{A} ^ \GRANULEnt{r}  \rightarrow  \GRANULEnt{B}$
which desrcibes how the value of the argument type may be used in the body of
the function. This approach more closely resembles the graded systems of
\jnote{list some systems here, linear haskell, QTT, etc.}.

The different approaches to graded types here gives rise to the question: how might the
features of the target language influence the design of a synthesis tool?

\jnote{It's still not clear why we choose these two languages - or why we choose
two languages at all instead of focusing on one, possibly need to come up with a
better justification}

The two approaches are now outlined formally in turn, beginning with the
linear-base calculus.

\subsection{Linear-base}
\label{sec:linear-base}
The types of the graded linear $\lambda$-calculus are defined as:
\begin{align*}
\hspace{-0.9em}\GRANULEnt{A} , \GRANULEnt{B} & ::=
       \GRANULEnt{A} \multimap \GRANULEnt{B}
  \mid \Box_{  \GRANULEnt{r}  }  \GRANULEnt{A}
{\small{\tag{types}}}
\end{align*}
The type $\Box_{r} A$ is an indexed family of type operators where $r$ is a
\textit{grade} ranging over the elements of a pre-ordered semiring
$({\mathcal{R}}, {*}, {1}, {+}, {0},
{\sqsubseteq})$ parameterising the calculus (where $\ast$ and $+$
are monotonic with respect to the pre-order $\sqsubseteq$).

The syntax of terms provides the elimination and introduction forms:
\begin{align*}
\hspace{-0.8em} \GRANULEnt{t} ::= \;
       & \GRANULEmv{x}
  \mid \lambda  \GRANULEmv{x}  .  \GRANULEnt{t}
  \mid \GRANULEnt{t_{{\mathrm{1}}}} \, \GRANULEnt{t_{{\mathrm{2}}}}
  \mid \GRANULEsym{[}  \GRANULEnt{t}  \GRANULEsym{]}
  \mid \textbf{let} \, [  \GRANULEmv{x}  ] =  \GRANULEnt{t_{{\mathrm{1}}}}  \, \textbf{in} \,  \GRANULEnt{t_{{\mathrm{2}}}}
{\small{\tag{terms}}}
\end{align*}
In addition the the terms of the linear $\lambda$-calculus, we also have the
construct $\GRANULEsym{[}  \GRANULEnt{t}  \GRANULEsym{]}$ which introduces a graded modal type $\Box_{  \GRANULEnt{r}  }  \GRANULEnt{A}$ by `promoting' a term $t$ to the graded modality, and it's dual $\textbf{let} \, [  \GRANULEmv{x}  ] =  \GRANULEnt{t_{{\mathrm{1}}}}  \, \textbf{in} \,  \GRANULEnt{t_{{\mathrm{2}}}}$ eliminates a graded modal value $\GRANULEnt{t_{{\mathrm{1}}}}$, binding a graded variable $x$
in scope of $\GRANULEnt{t_{{\mathrm{2}}}}$. The typing rules provide further understanding of the
behaviour of these terms.

Typing judgments are of the form $\Gamma  \vdash  \GRANULEnt{t}  :  \GRANULEnt{A}$, where $\Gamma$ ranges over contexts:
\begin{equation*}
  \Gamma ::= \emptyset
  \mid \Gamma  \GRANULEsym{,}   \GRANULEmv{x}  :  \GRANULEnt{A}
  \mid \Gamma  \GRANULEsym{,}   \GRANULEmv{x}  :_{\textcolor{coeffectColor}{  \GRANULEnt{r}  } }   \GRANULEnt{A}
\tag{contexts}
\end{equation*}

Thus, a context may be empty $\emptyset$, extended with a
linear assumption $\GRANULEmv{x}  :  \GRANULEnt{A}$ or extended with a graded assumption $\GRANULEmv{x}  :_{\textcolor{coeffectColor}{  \GRANULEnt{r}  } }   \GRANULEnt{A}$. For linear assumptions, structural rules of weakening
and contraction are disallowed. Graded assumptions may be used
non-linearly according to the constraints given by their grade, the
semiring element $r$. Throughout, comma denotes disjoint context
concatenation.

Various operations on contexts are used to capture non-linear data flow
via grading. Firstly, \emph{context addition} provides an
analogue to contraction, combining contexts that have come from
typing multiple subterms in a rule.
Context addition, written $\Gamma_{{\mathrm{1}}}  \GRANULEsym{+}  \Gamma_{{\mathrm{2}}}$, is undefined if $\Gamma_{{\mathrm{1}}}$ and $\Gamma_{{\mathrm{2}}}$
overlap in their linear assumptions. Otherwise graded assumptions appearing
in both contexts are combined via the semiring $+$ of their grades.


\begin{definition}[Context addition]\label{def:contextAdd}
  For all $\Gamma_{{\mathrm{1}}}, \Gamma_{{\mathrm{2}}}$
  \emph{context addition} is defined
as follows by ordered cases matching inductively on the structure of
$\Gamma_{{\mathrm{2}}}$:
\begin{align*}
\Gamma_{{\mathrm{1}}}  \GRANULEsym{+}  \Gamma_{{\mathrm{2}}} = \left\{\begin{matrix}
    \begin{array}{ll}
    \Gamma_{{\mathrm{1}}} &
     \Gamma_{{\mathrm{2}}} = \emptyset
             \\
      ((\Gamma'_{{\mathrm{1}}}, \Gamma''_{{\mathrm{1}}}) + \Gamma'_{{\mathrm{2}}}), \GRANULEmv{x}  :_{\textcolor{coeffectColor}{  \GRANULEsym{(}  \GRANULEnt{r}  \GRANULEsym{+}  \GRANULEnt{s}  \GRANULEsym{)}  } }   \GRANULEnt{A} \; &
\Gamma_{{\mathrm{2}}} = \Gamma'_{{\mathrm{2}}}  \GRANULEsym{,}   \GRANULEmv{x}  :_{\textcolor{coeffectColor}{  \GRANULEnt{s}  } }   \GRANULEnt{A} \wedge \Gamma_{{\mathrm{1}}} = \Gamma'_{{\mathrm{1}}}  \GRANULEsym{,}   \GRANULEmv{x}  :_{\textcolor{coeffectColor}{  \GRANULEnt{r}  } }   \GRANULEnt{A},\Gamma''_{{\mathrm{1}}}  \\
 (\Gamma_{{\mathrm{1}}} + \Gamma'_{{\mathrm{2}}}), \GRANULEmv{x}  :  \GRANULEnt{A} & \Gamma_{{\mathrm{2}}} = \Gamma'_{{\mathrm{2}}}  \GRANULEsym{,}   \GRANULEmv{x}  :  \GRANULEnt{A}\ \wedge\  \GRANULEmv{x}  :  \GRANULEnt{A} \notin \Gamma_{{\mathrm{1}}}
    \end{array}
  \end{matrix}\right.
\end{align*}
\end{definition}


\jnote{Possibly some example programs in this section? The problem might be that
they'll have to be purely lambda calc programs}


\begin{figure}[t]
\hspace{-0.5em}
\begin{align*}
\hspace{-0.5em}
  \begin{array}{c}
  \inferrule*[right = Var]
  {\;}
  {\GRANULEmv{x}  :  \GRANULEnt{A}   \vdash  \GRANULEmv{x}  :  \GRANULEnt{A}}
\;\;
  \inferrule*[right = Abs]
  {\Gamma  \GRANULEsym{,}   \GRANULEmv{x}  :  \GRANULEnt{A}   \vdash  \GRANULEnt{t}  :  \GRANULEnt{B}}
  {\Gamma  \vdash   \lambda  \GRANULEmv{x}  .  \GRANULEnt{t}   :  \GRANULEnt{A}  \rightarrow  \GRANULEnt{B}}
\;\;
  \inferrule*[right = App]
  {\Gamma_{{\mathrm{1}}}  \vdash  \GRANULEnt{t_{{\mathrm{1}}}}  :  \GRANULEnt{A}  \rightarrow  \GRANULEnt{B} \;\;\;
   \Gamma_{{\mathrm{2}}}  \vdash  \GRANULEnt{t_{{\mathrm{2}}}}  :  \GRANULEnt{A} }
  {\Gamma_{{\mathrm{1}}}  \GRANULEsym{+}  \Gamma_{{\mathrm{2}}}  \vdash  \GRANULEnt{t_{{\mathrm{1}}}} \, \GRANULEnt{t_{{\mathrm{2}}}}  :  \GRANULEnt{B}}
\\[0.75em]
 \inferrule*[right = Weak]
  {\Gamma  \vdash  \GRANULEnt{t}  :  \GRANULEnt{A}}
  {\Gamma  \GRANULEsym{,}  \,  \GRANULEsym{,}    [  \Delta  ]_{  0  }    \vdash  \GRANULEnt{t}  :  \GRANULEnt{A}}
\;\;\;
\inferrule*[right = Der]
  {\Gamma  \GRANULEsym{,}   \GRANULEmv{x}  :  \GRANULEnt{A}   \vdash  \GRANULEnt{t}  :  \GRANULEnt{B}}
  {\Gamma  \GRANULEsym{,}   \GRANULEmv{x}  :_{\textcolor{coeffectColor}{   1   } }   \GRANULEnt{A}   \vdash  \GRANULEnt{t}  :  \GRANULEnt{B}}
\;\;\;
\inferrule*[right = Approx]
{\Gamma  \GRANULEsym{,}   \GRANULEmv{x}  :_{\textcolor{coeffectColor}{  \GRANULEnt{r}  } }   \GRANULEnt{A}    \GRANULEsym{,}  \,  \GRANULEsym{,}  \Gamma'  \vdash  \GRANULEnt{t}  :  \GRANULEnt{A} \quad r \sqsubseteq s }
{\Gamma  \GRANULEsym{,}   \GRANULEmv{x}  :_{\textcolor{coeffectColor}{  \GRANULEnt{s}  } }   \GRANULEnt{A}    \GRANULEsym{,}  \,  \GRANULEsym{,}  \Gamma'  \vdash  \GRANULEnt{t}  :  \GRANULEnt{A}}
\\[0.75em]
\inferrule*[right = Pr]
  {\GRANULEsym{[}  \Gamma  \GRANULEsym{]}  \vdash  \GRANULEnt{t}  :  \GRANULEnt{A}}
  {\textcolor{coeffectColor}{ \GRANULEnt{r}   \textcolor{coeffectColor}{\,\cdot\,} }   \GRANULEsym{[}  \Gamma  \GRANULEsym{]}    \vdash  \GRANULEsym{[}  \GRANULEnt{t}  \GRANULEsym{]}  :   \Box_{  \GRANULEnt{r}  }  \GRANULEnt{A}}
\;\;\;
\inferrule*[right = Let$\Box$]
  {\Gamma_{{\mathrm{1}}}  \vdash  \GRANULEnt{t_{{\mathrm{1}}}}  :   \Box_{  \GRANULEnt{r}  }  \GRANULEnt{A} \;\;\;
   \Gamma_{{\mathrm{2}}}  \GRANULEsym{,}   \GRANULEmv{x}  :_{\textcolor{coeffectColor}{  \GRANULEnt{r}  } }   \GRANULEnt{A}   \vdash  \GRANULEnt{t_{{\mathrm{2}}}}  :  \GRANULEnt{B} }
    {\Gamma_{{\mathrm{1}}}  \GRANULEsym{+}  \Gamma_{{\mathrm{2}}}  \vdash   \textbf{let} \, [  \GRANULEmv{x}  ] =  \GRANULEnt{t_{{\mathrm{1}}}}  \, \textbf{in} \,  \GRANULEnt{t_{{\mathrm{2}}}}   :  \GRANULEnt{B}}
\end{array}
\end{align*}
\vspace{-1.25em}
  \caption{Typing rules of the graded linear $\lambda$-calculus}
\label{fig:typing}
\vspace{-0.65em}
 \end{figure}


Figure~\ref{fig:typing} defines the typing rules.
Linear variables are typed in a singleton context
(\textsc{Var}). Abstraction (\textsc{Abs}) and application (\textsc{App})
follow the rules of the linear $\lambda$-calculus.
The $\textsc{Weak}$ rule captures
weakening of assumptions graded by $0$ (where $[  \Delta  ]_{  0  }$ denotes a context
containing only graded assumptions graded by $0$). Dereliction ($\textsc{Der}$),
allows a linear assumption to be converted to a graded assumption with grade
$1$. Grade approximation is captured by the $\textsc{Approx}$
rule, which allows a grade $s$ to be converted to another grade $r$,
providing that $r$ \textit{approximates} $s$, where the relation
$\sqsubseteq$ is the pre-order provided
with the semiring.
Introduction and elimination of the graded modality is provided by the
$\textsc{Pr}$ and $\textsc{Let}$ rules
respectively. The $\textsc{Pr}$ rule propagates the grade $r$ to the
assumptions through \emph{scalar multiplication} of $\Gamma$ by $r$ where
every assumption in $\Gamma$ must already be graded (written $\GRANULEsym{[}  \Gamma  \GRANULEsym{]}$ in the rule), defined:
%
%
\begin{definition}[Scalar context multiplication]
  \label{def:scalar}
\begin{align*}
   \textcolor{coeffectColor}{ \GRANULEnt{r}   \textcolor{coeffectColor}{\,\cdot\,} }   \emptyset = \emptyset
    \qquad\qquad
    \textcolor{coeffectColor}{ \GRANULEnt{r}   \textcolor{coeffectColor}{\,\cdot\,} }  \GRANULEsym{(}  \Gamma  \GRANULEsym{,}   \GRANULEmv{x}  :_{\textcolor{coeffectColor}{  \GRANULEnt{s}  } }   \GRANULEnt{A}   \GRANULEsym{)} = \GRANULEsym{(}   \textcolor{coeffectColor}{ \GRANULEnt{r}   \textcolor{coeffectColor}{\,\cdot\,} }  \Gamma   \GRANULEsym{)}  \GRANULEsym{,}   \GRANULEmv{x}  :_{\textcolor{coeffectColor}{  \GRANULEsym{(}  \GRANULEnt{r}  \textcolor{coeffectColor}{\,\cdot\,}  \GRANULEnt{s}  \GRANULEsym{)}  } }   \GRANULEnt{A}
\end{align*}
\end{definition}

The $\textsc{Let}$ rule eliminates a graded modal value $\Box_{  \GRANULEnt{r}  }  \GRANULEnt{A}$
into a graded assumption $\GRANULEmv{x}  :_{\textcolor{coeffectColor}{  \GRANULEnt{r}  } }   \GRANULEnt{A}$ with a matching
grade in the scope of the \textbf{let} body.


\paragraph{Metatheory}
The admissibility of substitution is a key result that holds
for this language~\cite{DBLP:journals/pacmpl/OrchardLE19}, which is
leveraged in soundness of the synthesis calculi.
%
\begin{restatable}[Admissibility of substitution]{lemma}{linearSubst}
Let $\Delta  \vdash  \GRANULEnt{t'}  :  \GRANULEnt{A}$, then:
\label{lemma:substitution}
\begin{itemize}[leftmargin=1em]
\item (Linear) \hspace{0.04em} If $\Gamma  \GRANULEsym{,}   \GRANULEmv{x}  :  \GRANULEnt{A}    \GRANULEsym{,}  \,  \GRANULEsym{,}   \Gamma'   \vdash  \GRANULEnt{t}  :  \GRANULEnt{B}$ then $\Gamma  \GRANULEsym{+}  \Delta  \GRANULEsym{+}  \Gamma'  \vdash   [  \GRANULEnt{t'}  /  \GRANULEmv{x}  ]  \GRANULEnt{t}   :  \GRANULEnt{B}$
\item (Graded) If $\Gamma  \GRANULEsym{,}   \GRANULEmv{x}  :_{\textcolor{coeffectColor}{  \GRANULEnt{r}  } }   \GRANULEnt{A}    \GRANULEsym{,}  \,  \GRANULEsym{,}   \Gamma'   \vdash  \GRANULEnt{t}  :  \GRANULEnt{B}$
then $\Gamma  \GRANULEsym{+}  \GRANULEsym{(}   \textcolor{coeffectColor}{ \GRANULEnt{r}   \textcolor{coeffectColor}{\,\cdot\,} }  \Delta   \GRANULEsym{)}  \GRANULEsym{+}  \Gamma'  \vdash   [  \GRANULEnt{t'}  /  \GRANULEmv{x}  ]  \GRANULEnt{t}   :  \GRANULEnt{B}$
\end{itemize}
\end{restatable}


\subsection{Graded-base}
\label{sec:graded-base}

We now define our second core language with graded types. This system will be familiar to
those who have encountered graded types before, drawing from the coeffect calculus of \citet{petricek2014coeffects}, Quantitative Type Theory (QTT) by \citet{McBride2016} and refined further by \citet{quantitative-type-theory} (although we omit dependent types from our language), the calculus of \citet{DBLP:journals/pacmpl/AbelB20}, and other graded dependent type theories~\cite{quantitative-type-theory,DBLP:conf/esop/MoonEO21}. Similar
systems also form the basis of the core of the linear types extension to Haskell~\cite{DBLP:journals/pacmpl/BernardyBNJS18}.
This calculus shares much in common with languages based on linear types, such
as the graded monadic-comonadic calculus of~\cite{combining2016}, generalisations of Bounded
Linear Logic~\cite{DBLP:conf/esop/BrunelGMZ14,DBLP:conf/esop/GhicaS14}


This system corresponds to Granule's ``graded base'' language extension rather than its
default linear types basis outlined above.

Again, our system is parameterised by a graded necessity modality. The syntax of types is given by:

\begin{align*}
\hspace{-0.9em}\GRANULEnt{A} , \GRANULEnt{B} & ::=
       \GRANULEnt{A} ^ \GRANULEnt{r}  \rightarrow  \GRANULEnt{B}
  \mid \Box_{  \GRANULEnt{r}  }  \GRANULEnt{A}
{\small{\tag{\textit{types}}}}
\end{align*}


where the function space $\GRANULEnt{A} ^ \GRANULEnt{r}  \rightarrow  \GRANULEnt{B}$ annotates the input type with a \emph{grade} $\GRANULEnt{r}$
drawn from a pre-ordered semiring $(\mathcal{R}, {\ast}, {1}, {+}, {0}, \sqsubseteq)$ parameterising
the calculus. Type constructors $K$ include the unit and multiplicative linear product type but
are also extended by user-defined ADTs in the implementation which can be formed by products, coproducts,
and recursive types (although we do not given an explicit syntax to the latter two type formers here).
The graded necessity modality $\Box_{  \GRANULEnt{r}  }  \GRANULEnt{A}$ is similarly annotated by the grade $\GRANULEnt{r}$ being
an element of the semiring. %The syntax $\alpha$ denotes a type variable



\jnote{Do we want to include pattern matching here? In fact, do we want to
  include it in the linear-base calculus from the start as well?}
The syntax of terms is given as:
%
\begin{align*}
\hspace{-0.8em} \GRANULEnt{t} ::= \;
       & \GRANULEmv{x}
  \mid \lambda  \GRANULEmv{x}  .  \GRANULEnt{t}
  \mid \GRANULEnt{t_{{\mathrm{1}}}} \, \GRANULEnt{t_{{\mathrm{2}}}}
  \mid \GRANULEsym{[}  \GRANULEnt{t}  \GRANULEsym{]}
  \mid C \, \GRANULEnt{t_{{\mathrm{1}}}} \, ... \, \GRANULEnt{t_{\GRANULEmv{n}}}
  \mid \textbf{case} \  \GRANULEnt{t}  \ \textbf{of} \   \GRANULEnt{p_{{\mathrm{1}}}}  \mapsto  \GRANULEnt{t_{{\mathrm{1}}}} ; ... ;  \GRANULEnt{p_{\GRANULEmv{n}}}  \mapsto  \GRANULEnt{t_{\GRANULEmv{n}}}
{\small{\tag{\textit{terms} }}}
         \\
\hspace{-0.8em} \GRANULEnt{p} ::= \;
       & \GRANULEmv{x}
  \mid \_
  \mid \GRANULEsym{[}  \GRANULEnt{p}  \GRANULEsym{]}
  \mid C \, \GRANULEnt{p_{{\mathrm{1}}}} \, ... \, \GRANULEnt{p_{\GRANULEmv{n}}}
{\small{\tag{\textit{patterns}}}}
\end{align*}
%
Terms consist of a graded $\lambda$-calculus, a \textit{promotion} construct [t] which introduces a graded modality explicitly, as well as data constructor introduction ($C \, \GRANULEnt{t_{{\mathrm{1}}}} \, ... \, \GRANULEnt{t_{\GRANULEmv{n}}}$) and elimination via
 $\textbf{case}$ expressions which are defined via the syntax of patterns $\GRANULEnt{p}$.



 \jnote{Talk about how we overload the definition from linear-base but removing
   the linear assumption case}
\begin{definition}[Context addition]\label{def:contextAdd}
  For all $\Gamma_{{\mathrm{1}}}, \Gamma_{{\mathrm{2}}}$
  \emph{context addition} is defined
as follows by ordered cases matching inductively on the structure of
$\Gamma_{{\mathrm{2}}}$:
\begin{align*}
\Gamma_{{\mathrm{1}}}  \GRANULEsym{+}  \Gamma_{{\mathrm{2}}} = \left\{\begin{matrix}
    \begin{array}{ll}
    \Gamma_{{\mathrm{1}}} &
    \Gamma_{{\mathrm{2}}} = \emptyset
             \\
      ((\Gamma'_{{\mathrm{1}}}, \Gamma''_{{\mathrm{1}}}) + \Gamma'_{{\mathrm{2}}}), \GRANULEmv{x}  :_{\textcolor{coeffectColor}{  \GRANULEsym{(}  \GRANULEnt{r}  \GRANULEsym{+}  \GRANULEnt{s}  \GRANULEsym{)}  } }   \GRANULEnt{A} \; &
\Gamma_{{\mathrm{2}}} = \Gamma'_{{\mathrm{2}}}  \GRANULEsym{,}   \GRANULEmv{x}  :_{\textcolor{coeffectColor}{  \GRANULEnt{s}  } }   \GRANULEnt{A} \wedge \Gamma_{{\mathrm{1}}} = \Gamma'_{{\mathrm{1}}}  \GRANULEsym{,}   \GRANULEmv{x}  :_{\textcolor{coeffectColor}{  \GRANULEnt{r}  } }   \GRANULEnt{A},\Gamma''_{{\mathrm{1}}} \\
 \GRANULEsym{(}  \Gamma_{{\mathrm{1}}}  \GRANULEsym{+}  \Gamma'_{{\mathrm{2}}}  \GRANULEsym{)}  \GRANULEsym{,}   \GRANULEmv{x}  :_{\textcolor{coeffectColor}{  \GRANULEnt{s}  } }   \GRANULEnt{A} & \Gamma_{{\mathrm{2}}} = \Gamma'_{{\mathrm{2}}}  \GRANULEsym{,}   \GRANULEmv{x}  :_{\textcolor{coeffectColor}{  \GRANULEnt{s}  } }   \GRANULEnt{A} \wedge \GRANULEmv{x} \not\in \mathsf{dom}(\Gamma_{{\mathrm{1}}})
\end{array}
  \end{matrix}\right.
\end{align*}
\end{definition}

\jnote{No need to redefine scalar context multiplication}

\section{Related systems}

\section{Conclusion}
This chapter introduced some of the key features of languages with resourceful
types. Linear and graded types embed usage constraints in the typing rules,
enforcing the notion that a well-typed program is also \textit{well-resourced}.


The next chapter focuses on the linear-base core calculus of
section~\ref{sec:linear-base}, extending this calculus with multiplicative and additive
types, as well as a unit type to form a more practical programming language. This
then comprises the target language of a synthesis algorithm. Likewise, the graded-base
calculus is revisited in chapter~\ref{chapter:graded-base}, where it is extended
with (G)ADTs, and recursion, providing a target language for a more in-depth and
featureful synthesis tool.
